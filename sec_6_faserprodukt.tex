\section{Faserprodukt} %Seite 105

\begin{definition}[Faserprodukt]
    \index[def]{Schema!Faserprodukt}
    Seien $f: X\to Y$ und $g:Z\to Y$ Schemamorphismen.
    Dann ist das \emph{Faserprodukt} $X\times_Y Z$ ein Schema
    zusammen mit Morphismen $X\times_Y Z \xto\alpha X$ und
    $X\times_Y Z \xto\beta Z$, so dass
    \[\begin{tikzcd}
        X \times_Y Z \rar{\alpha} \dar{\beta} & X \dar{f}\\
        Z \rar{g} & Y
    \end{tikzcd}\]
    kommutiert und $(X\times_Y Z, \alpha, \beta)$ damit universell ist,
    d.h. 
    \[\begin{tikzcd}
        T \ar[bend left]{drr} \ar[bend right]{ddr} \ar[dotted]{dr}{\exists!}&\\
        &  X \times_Y Z \rar{\alpha} \dar{\beta} & X \dar{X} \\
        & Z \rar{g} & Y.
    \end{tikzcd}\]
\end{definition}

\subsection{Anwendungen}

\subsubsection{Faser eines Morphismus}
\begin{definition}[Faser eines Morphismus]
    \label{def:faser eines morphismus}
    \index[def]{Schema!Faserprodukt!Faser eines Morphismus}
    Ist $f:X\to Y$, $\Spec k \o Y$ ein $k$-rationaler Punkt in $Y$ 
    (beispielsweise $k:= k(y) := \O_{Y,y}\big/\m_y$),
    so heißt
    \[ X \times_Y \Spec k =: X_y\]
    die \emph{Faser von $f$ über $y\in Y$}.
\end{definition}


\subsubsection{Basiswechsel}
\begin{definition}
    \label{def:basiswechsel}
    \index[def]{Schema!Basiswechsel}
    Sei $X$ ein $S$-Schema. Ist $T$ ein weiteres $S$-Schema, so heißt
    \[X \times_S T =: X_T\]
    der \emph{Basiswechsel vom $S$-Schema $X$ zum $T$-Schema $X_T$}.
\end{definition}

\begin{bemerkung}
    In der Tat ist $X\times_S T$ in natürlicher Weise ein $T$-Schema.
    Seien nämlich $f:X \to S$ und $g:T\to S$ die Strukturmorphismen,
    so haben wir
    \[\begin{tikzcd}
        X\times_S T \rar{\alpha} \dar{\beta} & X \dar{f} \\
        T \rar{g} & S.
    \end{tikzcd}\]
\end{bemerkung}

\begin{bemerkung}
    Man kann die Definition des Basiswechsels auch kategoriell lesen:
    Zu $g:T\to S$ hat man einen Funktor
    \[
        \funcdef{\Sch_S & \to & \Sch_T \\
            (X\xto f S) & \mapsto & (X\times_S T \xto\beta T).}
    \]
\end{bemerkung}

\begin{definition}[pull-back von Schemata]
    \label{def:pull-back}
    \index[def]{Schema!pull-back}
    In obiger Situation heißt ein $A$ mit 
    \[\begin{tikzcd}[remember picture]
        |[alias=mynode]| A \rar  \dar & X \dar \\
        Z \rar  & Y
    \end{tikzcd}
    \tikz[remember picture, overlay]{ 
        \draw[very thick] 
            ($(mynode) + (10pt,-10pt)$) +(5pt,0) -| +(0,-5pt);}
    \]
    \emph{pull-back}, falls $A = X \times_Y Z$.
\end{definition} 

\begin{satz}
    In $\Sch$ existiert zu jedem $X \xto f Y$, $Z \xto g Y$ ein Faserprodukt
    $X\times_Y Z$. Es ist eindeutig bis auf eindeutige Isomorphie.
    
    Für $X = \Spec A$, $Y = \Spec B$, $Z = \Spec R$ gilt sogar
    \[ X\times_Y Z = \Spec A\otimes_R B.\]
\end{satz}
\begin{proof}
    \TODO
\end{proof}

\begin{bemerkung}
    Es gilt:
    \begin{itemize}
      \item $X \times_S S = X$.
      \item $X \times_S Y = Y \times_S X$.
      \item $(X \times_S Y) \times_S Z = X \times_S (Y\times_S Z)$.
      \item Für $X \to S$ und $Z\to Y \to S $ gilt
        \[ (X \times_S Y) \times_Y Z = X \times_S Z.\]
    \end{itemize}
\end{bemerkung}

\begin{lemma}
    Sei $f:X \to Y$. $y\in Y$ mit Restklassenkörper $k(Y) = \O_{Y,y}\big/\m_y$.
    In
    \[\begin{tikzcd}
        X \times_Y \Spec k(y) =: X_y \rar{p} \dar & X \dar\\
        \Spec k(y) \rar & Y
    \end{tikzcd}\]
    ist $p$ ein Homöomorphismus auf $f\inv(y) \subseteq X$.
\end{lemma}
\begin{proof}
    \TODO
\end{proof}


\begin{beispiel}
    Sei $a \in \Z \setminus\{0\}$ und 
    $X := \Spec\Z[T_1,T_2] \big/ (T_1T_2^2-a) \xto{f} \Spec \Z$. 
    Die Fasern zu Punkten in $\Spec \Z$ sind:
    \begin{itemize}
      \item $(p) \in \Spec\Z$, $p$ Primzahl. Haben
        \[\begin{tikzcd}
            X_p \dar \rar  
                & X \dar\\
            \Spec\F_p \rar{\iota_p} 
                & \Spec \Z 
        \end{tikzcd}\]
        mit
        \[X_p = \Spec\left(\F_p[T_1,T_2 \big/ (T_1T_2^2 - \bar a\right)\]
        \begin{description}
        \item[1. Fall: $p\nmid a$.] Dann ist
            \[X_p \cong \Spec \F_p[T,T\inv].\]
        \item[2. Fall: $p \mid a$.] Dann hat $\F_p[T_1,T_2]\big/(T_1T_2^2)$
            nilpotente Elemente, also ist $X_p$ nicht reduziert und 
            nicht irreduzibel.
        \end{description}
        Man nennt $X_p$ auch oft die \emph{Reduktion von $X \mod p$}.
        \item $(0)\in \Spec\Z$, so hat man
            \[\begin{tikzcd}
                X_\Q := X_0 \dar \rar  
                    & X \dar\\
                \Spec\Q \rar 
                    & \Spec \Z 
            \end{tikzcd}\]
            und $X_\Q = \Spec(\Q[T_1,T_2]\big/(T_1T_2^2 - a)) \cong \G_{m,\Q}$.
    \end{itemize}
\end{beispiel}

\begin{definition}[multiplikative Gruppe von $k$]
    \index[def]{Körper!multiplikative Gruppe}
    Ist $k$ ein Körper, so heißt
    \[\bG_{m,k} := \Spec k[T,T\inv]\]
    die \emph{multiplikative Gruppe von $k$} als $k$-SChema.
    Man definiert auch
    \[\bG_m := \Spec \Z[T,T\inv].\]
\end{definition}

\begin{bemerkung}
    Man hat 
    \[\bG_{m,k}= \Spec k[T,T\inv] \to \Spec k[T] = \A^1_k\]
    einen Homöomorphismus auf $D(T) \osubset \A^1_k$.
\end{bemerkung}

\subsubsection{Basiswechsel und projektive Schemata}
\begin{satz}
    Sei $A$ ein Ring, $S = \oplus_{d\geq 0} S_d$ eine graduierte $A$-Algebra.
    Sei $B$ eine $A$-Algebra via $\varphi:A\to B$ und 
    \[T:= S \otimes_A B = \oplus_{d\geq 0} (S_d \otimes_A B)\]
    eine graduierte $B$-Algebra. Dann gilt:
    \[ \Proj(T) \cong \Proj(S) \times_{\Spec A} \Spec B.\]
\end{satz}
\begin{proof}
\TODO
\end{proof}

\begin{definition}[$n$-dimensionale projektive Raum über $S$]
    \index[def]{Schema!projektiver Raum über $S$}
    Ist $S$ ein Schema, so heißt
    \[\P^n_S := \P^n_{\Spec \Z} \times_{\Spec \Z} S\]
    der \emph{$n$-dimensionale projektive Raum über $S$}.
\end{definition}

\begin{bemerkung}
    Ist $S= \Spec A$, so stimmen die Definitionen
    von $\P^n_{\Spec A}$ überein.
\end{bemerkung}
% vim: set ft=tex :
