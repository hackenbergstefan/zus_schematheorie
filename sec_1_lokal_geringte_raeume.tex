\section{Lokal geringte R�ume}

\subsection{Garben}

\begin{definition}[Pr�garbe]
	Sei $X$ ein topologischer Raum. Eine \emph{Pr�garbe} $\F$ auf $X$
	ist eine Zuordnung
	$$\F: U\mapsto \F(U) \,,$$
	die jedem offenen $U\subset X$ eine abelsche Gruppe
	$\F(U)$ zuordnet, zusammen mit Homomorphismen
	$$\tikzmark{$\rho_{UV}$}: \F(U) \to \F(V)$$
	f�r jedes Paar $U\subset V$, so dass
	\[
	\begin{tikzcd}
		\F(U) \arrow{r}{\rho_{UV}}
			\arrow[bend right]{rr}{\rho_{UW}}& \F(V) \arrow{r}{\rho_{VW}}& \F(W)
	\end{tikzcd}
	\]
\end{definition}

\tikzmargin{south}{Wir nennen $\rho_{UV}$ \emph{Restriktion}, schreiben
	meist $s\mid_U := \rho_{UV}(s)$.\\
	Man nennt $s\in \F(U)$ \emph{Schnitt �ber $U$}.}


\begin{beispiel}
	$$\cal C_X^\circ: U \mapsto \cal C_X^\circ (U) := 
		\{f: U\to \R \mid \text{ $f$ stetig}\} $$
	mit $\rho_{UV}: \cal C_X^\circ(V) \mapsto \cal C_X^\circ(U)$,
	$f \mapsto f\mid_U$.
\end{beispiel}
