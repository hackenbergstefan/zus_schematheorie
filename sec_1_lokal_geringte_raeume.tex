\section{Lokal geringte R�ume}

\subsection{Garben}

\begin{definition}[Pr�garbe]
	Sei $X$ ein topologischer Raum. Eine \emph{Pr�garbe} $\F$ auf $X$
	ist eine Zuordnung
	$$\F: U\mapsto \F(U) \,,$$
	die jedem offenen $U\subset X$ eine abelsche Gruppe
	$\F(U)$ zuordnet, zusammen mit Homomorphismen
	$$\rho_{UV}: \F(U) \to \F(V)$$
	f�r jedes Paar $V\subset U$, so dass
	\[
	\begin{tikzcd}
		\F(U) \arrow{r}{\rho_{UV}}
			\arrow[bend right]{rr}{\rho_{UW}}& \F(V) \arrow{r}{\rho_{VW}}& \F(W)
	\end{tikzcd}
	\]
	kommutiert.
	
	Wir nennen $\rho_{UV}$ \emph{Restriktion}, schreiben
	meist $\tikzmark{s\rest V} := \rho_{UV}(s)$.
	
	Man nennt $s\in \F(U)$ auch \emph{Schnitt �ber $U$}.
\end{definition}
	
\tikzmargin{north, above=1cm}{\color{red}
Bei mir steht hier im Skript $s\rest U$. Offenbar ein Fehler!?}

\begin{beispiel}
	$$\cal C_X^\circ: U \mapsto \cal C_X^\circ (U) := 
		\{f: U\to \R \mid \text{ $f$ stetig}\} $$
	mit $\rho_{VU}: \cal C_X^\circ(V) \mapsto \cal C_X^\circ(U)$,
	$f \mapsto f\rest U$.
\end{beispiel}

\begin{bemerkung}
	Ist $\kat{Ab}$ die Kategorie der abelschen Gruppen und
	\[
		\kat{Top}_X := 
		\begin{cases}
		\Obj: U\subset X \text{ offen}\\
		\Morph: \Hom(U,V) = 
			\begin{cases}
				\emptyset & U\not\subset V,\\
				U\to V & U\subset V,
			\end{cases}
		\end{cases}
	\]
	dann ist eine Pr�garbe gerade ein kontravarianter Funktor
	\[
		\F: \funcdef{\kat{Top}_X & \to & \kat{Ab}\\
			U & \mapsto & \F(U)\\
			(U\to V) & \mapsto & (\F(V)\to \F(U)).}
	\]
	Oder anders ausgedr�ckt: Es ist
	\[
		\F: \funcdef{\kat{Top}_X\op & \to & \kat{Ab}\\
			U & \mapsto & \F(U)\\
			(V\to U) & \mapsto & (\F(V)\to \F(U)).}
	\]
	ein kovarianter Funktor.
\end{bemerkung}

\begin{definition}[Morphismus von Pr�garben]
	Ein \emph{Morphismus von Pr�garben} $\F \xto{\phi} \G$ auf $X$ ist
	eine nat�rliche Transformation der Funktoren $\F$ und $\G$, d.h.
	f�r alle $U\subset X$ offen gibt es einen Morphismus
	$\F(U) \xto{\phi_U} \G(U)$, so dass f�r $U\subset V$
	\[
		\begin{tikzcd}
			\F(U) \arrow{r}{\phi_U} & \G(U)\\
			\F(V) \arrow{r}{\phi_U} \arrow{u} & \G(V) \arrow{u}
		\end{tikzcd}
	\] 
	kommutiert.
\end{definition}


\begin{definition}[Garbe]
	Eine Pr�garbe $\F$ auf $X$ hei�t \emph{Garbe}, falls gilt:
	Ist $U\subset X$ offen und $U=\bigcup_{i\in I} U_i$ f�r 
	offene $U_i\subset X$, so gilt
	\begin{enumerate}
	  \item Ist $s\in \F(U)$ und $s\rest{U_i} = 0$ f�r alle $i\in I$,
	  	so ist $s=0\in \F(U)$.
	  \item Sind $s_i \in \F(U_i)$ gegeben, mit
	  	$$s_i \rest{U_i\cap U_j} = s_j \rest{U_i\cap U_j}\qquad \forall i,j,$$
	  	so existiert ein $s\in \F(U)$ mit
	  	$$s_i = s \rest{U_i}\qquad\forall i.$$
	\end{enumerate}
\end{definition}

\begin{bemerkung}
	$\F$ ist eine Garbe, genau dann, wenn die folgende Sequenz abelscher
	Gruppen exakt ist:
	\[	\everymath{\displaystyle}
		\begin{tikzcd}[row sep=tiny, column sep=small]
		0 \rar & \tikzmark[1]{\F(U)} \rar & 
			\tikzmark[2]{\prod_{i\in I} \F(U_i)} \rar
			& \prod_{(i,j)\in I^2} \F(U_i\cap U_j)\\
		& s \rar[mapsto] & \left(s\rest{U_i}\right)_{i\in I}\\
		&& (s_i)_{i\in I} \rar[mapsto] &
			\left(s_i\rest{U_i\cap U_j} - 
			s_j\rest{U_i\cap U_j}\right)_{(i,j)\in I^2}  
		\end{tikzcd}
	\]
	
	Exaktheit an \tikzarrow[1]{south, mark above}{dieser} Stelle ist �quivalent 
	zu Eigenschaft 1.
	Exaktheit \tikzarrow[2]{south, mark above}{hier} zu Eigenschaft 2.
\end{bemerkung}