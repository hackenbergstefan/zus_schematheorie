\section{Lokal geringte Räume}

\subsection{Garben}

\begin{definition}[Prägarbe]
	Sei $X$ ein topologischer Raum. Eine \emph{Prägarbe} $\F$ auf $X$
	ist eine Zuordnung
	$$\F: U\mapsto \F(U) \,,$$
	die jedem offenen $U\subset X$ eine abelsche Gruppe
	$\F(U)$ zuordnet, zusammen mit Homomorphismen
	$$\rho_{UV}: \F(U) \to \F(V)$$
	für jedes Paar $V\subset U$, so dass
	\[
	\begin{tikzcd}
		\F(U) \arrow{r}{\rho_{UV}}
			\arrow[bend right]{rr}{\rho_{UW}}& \F(V) \arrow{r}{\rho_{VW}}& \F(W)
	\end{tikzcd}
	\]
	kommutiert.
	
	Wir nennen $\rho_{UV}$ \emph{Restriktion}, schreiben
	meist $\tikzmark{s\rest V} := \rho_{UV}(s)$.
	
	Man nennt $s\in \F(U)$ auch \emph{Schnitt über $U$}.
\end{definition}
	
\tikzmargin{north, above=1cm}{\color{red}
Bei mir steht hier im Skript $s\rest U$. Offenbar ein Fehler!?}

\begin{beispiel}
	$$\cal C_X^\circ: U \mapsto \cal C_X^\circ (U) := 
		\{f: U\to \R \mid \text{ $f$ stetig}\} $$
	mit $\rho_{VU}: \cal C_X^\circ(V) \mapsto \cal C_X^\circ(U)$,
	$f \mapsto f\rest U$.
\end{beispiel}

\begin{bemerkung}
	Ist $\kat{Ab}$ die Kategorie der abelschen Gruppen und
	\[
		\kat{Top}_X := 
		\begin{cases}
		\Obj: U\subset X \text{ offen}\\
		\Morph: \Hom(U,V) = 
			\begin{cases}
				\emptyset & U\not\subset V,\\
				U\to V & U\subset V,
			\end{cases}
		\end{cases}
	\]
	dann ist eine Prägarbe gerade ein kontravarianter Funktor
	\[
		\F: \funcdef{\kat{Top}_X & \to & \kat{Ab}\\
			U & \mapsto & \F(U)\\
			(U\to V) & \mapsto & (\F(V)\to \F(U)).}
	\]
	Oder anders ausgedrückt: Es ist
	\[
		\F: \funcdef{\kat{Top}_X\op & \to & \kat{Ab}\\
			U & \mapsto & \F(U)\\
			(V\to U) & \mapsto & (\F(V)\to \F(U)).}
	\]
	ein kovarianter Funktor.
\end{bemerkung}

\begin{definition}[Morphismus von Prägarben]
	Ein \emph{Morphismus von Prägarben} $\F \xto{\phi} \G$ auf $X$ ist
	eine natürliche Transformation der Funktoren $\F$ und $\G$, d.h.
	für alle $U\subset X$ offen gibt es einen Morphismus
	$\F(U) \xto{\phi_U} \G(U)$, so dass für $U\subset V$
	\[
		\begin{tikzcd}
			\F(U) \arrow{r}{\phi_U} & \G(U)\\
			\F(V) \arrow{r}{\phi_V} \arrow{u} & \G(V) \arrow{u}
		\end{tikzcd}
	\] 
	kommutiert.
\end{definition}


\begin{definition}[Garbe]
  Eine Prägarbe $\F$ auf $X$ heißt \emph{Garbe} (engl. sheaf), falls
  gilt: Ist $U\subset X$ offen und $U=\bigcup_{i\in I} U_i$ für 
	offene $U_i\subset X$, so gilt
	\begin{enumerate}
	  \item Ist $s\in \F(U)$ und $s\rest{U_i} = 0$ für alle $i\in I$,
	  	so ist $s=0\in \F(U)$.
	  \item Sind $s_i \in \F(U_i)$ gegeben, mit
	  	$$s_i \rest{U_i\cap U_j} = s_j \rest{U_i\cap U_j}\qquad \forall i,j,$$
	  	so existiert ein $s\in \F(U)$ mit
	  	$$s_i = s \rest{U_i}\qquad\forall i.$$
	\end{enumerate}
\end{definition} 

\begin{bemerkung}
	$\F$ ist eine Garbe, genau dann, wenn die folgende Sequenz abelscher
	Gruppen exakt ist:
	\[	\everymath{\displaystyle}
		\begin{tikzcd}[row sep=tiny, column sep=small]
		0 \rar & \tikzmark[1]{\F(U)} \rar & 
			\tikzmark[2]{\prod_{i\in I} \F(U_i)} \rar
			& \prod_{(i,j)\in I^2} \F(U_i\cap U_j)\\
		& s \rar[mapsto] & \left(s\rest{U_i}\right)_{i\in I}\\
		&& (s_i)_{i\in I} \rar[mapsto] &
			\left(s_i\rest{U_i\cap U_j} - 
			s_j\rest{U_i\cap U_j}\right)_{(i,j)\in I^2}  
		\end{tikzcd}
	\]
	
	Exaktheit an \tikzarrow[1]{mark above}{dieser} Stelle ist äquivalent 
	zu Eigenschaft 1 und
	Exaktheit \tikzarrow[2]{mark above}{hier} zu Eigenschaft 2.
\end{bemerkung}

\begin{beispiel}
	Sei $M$ eine $\mathrm C^\infty$ Mannigfaltigkeit, so ist
  	\[ \cal C^\infty_M: U \mapsto
  		\cal C^\infty_M(U) := \{f:U\to \R \mid f\in \mathrm C^\infty(U)\}
  	\]
  	eine Garbe.
\end{beispiel}

\begin{beispiel}
	Sei $M$ eine $\C$ Mannigfaltigkeit, so ist
  	\[ \cal O_M: U \mapsto
  		\cal O_M(U) := \{f:U\to \C \mid f \text{ holomorph}\}
  	\]
  	eine Garbe. Für $M = \C$ haben wir zusätzlich die Garbe
  	\[ \cal O_\C^\times: U \mapsto
  		\cal O_\C^\times(U) := \{f:U\to \C^\times \mid f \text{ holomorph}\},
  	\]
  	(wobei die Gruppenverknüpfung multiplikativ zu lesen ist).
  	Dies liefert uns einen Morphismus von (Prä)garben
  	\[ \O \to \O_C^\times,\ f \mapsto \exp(f).\]
  	Betrachte nun die Prägarbe
  	\[\cal H := \tikzmark{\im^\text{naiv}}(\exp): U \mapsto \im(\exp_U) = 
  		\{\exp \circ f: U\to \C \mid f:U\to \C \text{ holomorph}\}.\]
  	Dies ist \emph{keine} Garbe:
  	Betrachte die Scheibe 
  	\[U = \{z\in \C \mid \tfrac{1}{2} < |z| < \tfrac{3}{2}\}\]
  	zerlegt in die beiden offenen Teilmengen
  	\begin{align*}
  		U_1 &= \{z \in U \mid \Re z > -\varepsilon\}\\
  		U_2 &= \{z \in U \mid \Re z < \varepsilon\}
  	\end{align*}
  	mit $U = U_1 \cup U_2$ für ein $\varepsilon > 0$ beliebig. Für $i=1,2$
  	ist 
  	$(z: U_i \to \C, z\mapsto z) \in \cal H(U_i)$,
  	da sich der komplexe Logarithmus auf beiden $U_i$ problemlos definieren
  	lässt.
  	Ferner ist auch
 	\[ (z: U_1 \to \C) \rest{U_1\cap U_2} = 
 		(z: U_2 \to \C) \rest{U_1 \cap U_2},\]
 	erfüllt, jedoch kommen diese nicht von einem gemeinsamen Schnitt
 	da
 	\[ (z: U\to \C) \notin \cal H(U). \]
\end{beispiel}

\tikzmargin{north}{Warum steht hier naiv??}


\begin{definition}
	Für einen topologischen Raum $X$ bezeichne
	\begin{align*}
		\PSh_X & := \text{die Kategorie der Prägarben auf $X$},\\
		\Sh_X & := \text{die Kategorie der Garben auf $X$, wobei
			} \Hom_{\Sh_X}(\F,\G) := \Hom_{\PSh_X}(\F,\G)
	\end{align*} 
\end{definition}

\begin{bemerkung}
	Man hat den Inklusionsfunktor
	\[ \iota: \Sh_X \to \PSh_X,\ \F \mapsto \F\]
\end{bemerkung}

\begin{definition}[Halm, Keim]
	Ist $\F$ eine (Prä)Garbe auf $X$ und $x_0 \in X$, so heißt
	\[ \F_{x_0} := \varinjlim_{x_0 \in U \subset X\text{ offen}} \F(U)
		 = \coprod_{U\subset X\text{ offen}} \F(U) \Big/ \sim\] 
	mit 
	\[ s \sim t \  :\Leftrightarrow \  
		\exists W \subset X \text{ offen}:\ x_0 \in W \subset U \cap U'
		\text{ und } s\rest W = t \rest W
	\]
	für $s \in \F(U)$, $t \in \F(U')$ der \emph{Halm von $\F$ bei $x_0$}.
	
	Die Elemente $[s] \in \F_{x_0}$ heißen \emph{Keime von Schnitten bei $x_0$}.  
\end{definition}

\begin{beispiel}
	$(\cal C^\infty_M)_{x_0} = \{ [f: U \xto{C^\infty} \R]\mid
  	f\sim g \Leftrightarrow \exists W\subset M\text{ offen}, x_0 \in W
  	\text{ mit } f\rest W = g\rest W\}$
\end{beispiel}
\begin{beispiel}
	\begin{align*}
	  	\mathcal O_{\C,x_0} &= \{[f:U \xto{\text{hol}} \C] \mid x_0 \in U\}\\
	  	&= \{\sum_{n=0}^\infty a_n(x-x_0)^n \mid \text{Reihe hat positiven 
	  	Konvergenzradius}\}\\
	  	&:= \C\{x-x_0\}
	\end{align*}
\end{beispiel}



\begin{uebung}[Übungsblatt 1 Aufgabe 3]
	\begin{enumerate}
	  \item Es sei $\F$ eine Garbe auf einen topologischen Raum $X$.
	  Es sei $U\subset X$ eine offene Teilmenge. Für $r\in \F(U)$, $x_0 \in U$
	  bezeichne $r_{x_0}$ den Keim $[r]$ von $\F$ bei $x_0$. Es seien
	  nun $s,t \in \F(U)$, für die $\forall x_0 \in U: s_{x_0} = t_{x_0}$ 
	  gelte. Zeige, dass $s=t$.
	  \item Gib ein Beispiel einer Prägarbe an, die nicht separiert ist,
	  die also nicht die erste Garbenbedingung erfüllt.
	\end{enumerate}
\end{uebung}
\begin{proof}
	\begin{enumerate}
	  \item Für alle $x_0 \in U$ existieren offene $U_{x_0}$ mit
	  	$s \rest{U_{x_0} \cap U} = t\rest{U_{x_0} \cap U}$ nach Definition
	  	der Keime. Es ist $U = \cap_{x_0 \in U} U_{x_0} \cap U$, also
	  	folgt nach erster Garbenbedingung $s=t$.
	  \item Wähle $X = \{0,1\}$ mit diskreter Topologie.
	  	Definiere die Prägarbe
	  	\[\F(X) := \Z \qquad \F(\emptyset) = \F(\{1\}) = \F(\{0\}) := 1\]
	  	Nun ist
	  	\begin{align*}
	  		2\rest{\{0\}} &= 5\rest{\{0\}} \\
	  		2\rest{\{1\}} &= 5\rest{\{1\}}
	  	\end{align*}
	  	aber $2\neq 5 \in \Z$.
	\end{enumerate}
\end{proof}

\begin{definition}[push-forward]
	Ist $f:X \to Y$ stetig und $\F$ eine Garbe auf $X$, so ist durch
	\[ f_\ast \F: V \mapsto \F(f^{-1}(V))\]
	für $V\subset Y \text{ offen}$ eine Garbe definiert, 
	der \emph{push-forward von $\F$}.
\end{definition}



\subsection{Lokal geringte Räume}
Betrachte nun 
\[\Ring := \text{ Kategorie der kommuativen Ringe mit $1$}\]
und entsprechend Garben
\[\F:\Top_X\op \to \Ring.\]

\begin{definition}[lokaler Ring]
	Sei  $R$ ein Ring. Dann heißt $R$ \emph{lokal}, wenn $R$ genau ein
	maximales Ideal besitzt.
\end{definition}

\begin{beispiel}
	$\Z_{(p)} := \left\{\frac{a}{b} \in \Q \mid p \nmid b\right\}
  \underset{\text{Unterring}}{\subset}\Q$
\end{beispiel}

\begin{bemerkung}
	Ist $R$ lokaler Ring und $\m \ideal R$ das maximale Ideal,
	so ist $R \setminus \m = R^\times$.
\end{bemerkung}

\begin{uebung}[Übungsblatt 1 Aufgabe 1]
	\begin{enumerate}
	  \item Es sei $R$ ein kommutativer Ring und $R^\times$ seine 
	  	Einheitengruppe. Zeige, dass $R$ genau dann lokal ist, wenn
	  	$R\setminus R^\times \ideal R$ gilt, d.h. wenn die Nichteinheiten
	  	$R\setminus R^\times$ ein Ideal in $R$ bilden.
	  \item Es sei $R$ ein kommutativer nullteilerfreier Ring. Den 
	  	Quotientenkörper zu $R$ bezeichen wir mit $\Quot(R)$. Lokalisieren wir
	  	$R$ nach $\p$, so erhalten wir den Ring
	  	$R_\p = \{\tfrac{a}{b} \in \Quot(R) \mid a \in R,\ b \notin \p\}$.
	  	Zeige, dass $R_\p$ ein lokaler Ring ist.
	\end{enumerate}
\end{uebung}
\begin{proof}
	\begin{enumerate}
	  \item
	  \begin{description mathquote}
	  \item[\Rightarrow]
	  	Ist $R$ lokal, so ist $R\setminus R^\times = \m$ das maximale Ideal
	  	von $R$.
	  \item[\Leftarrow]
	  	Ist $R\setminus R^\times$ ein Ideal, so ist dies maximal (klar).
	  	Sei $\m \ideal R$ ein maximales Ideal, so gilt offenbar schon
	  	$R\setminus R^\times = \m$. 
	  \end{description mathquote}
	  \item
	  Wir zeigen $\p R_\p = R_\p \setminus R_\p^\times$, dann folgt die
	  Behauptung mit 1.
	  \begin{description mathquote}
	  \item[\subseteq]
	  	Es sei 
	  	\[
	  		h = p_1 \frac{s_1}{t_1} + \ldots + p_n \frac{s_n}{t_n}.
	  	\]
	  	Setze
	  	\begin{align*}
	  		z_1 &:= p_1 s_1 t_2 \ldots t_n + 
	  			p_2 s_2 t_1 t_3 \ldots t_n +
	  			p_n s_n t_1 \ldots t_{n-1}\\
	  		z_0 &:= t_1 \ldots t_n,
	  	\end{align*}
	  	so ist $h = \frac{z_1}{z_0}$. Wäre
	  	$h \in R_\p^\times$, sagen wir $\frac{s}{t}$ sein Inverses, so müsste
	  	gelten
	  	$z_1 s = z_0 t$. Die linke Seite jedoch ist in $\p$, die rechte
	  	nicht. Damit ist
	  	$h\in R_\p \setminus R_\p^\times$.
	  \item[\supseteq]
	  	Sei $\frac s t \in R_\p \setminus R_\p^\times$,
	  	so ist $s \frac{1}{t} \in \p R_\p$.
	  \end{description mathquote}
	\end{enumerate}
\end{proof}

\begin{beispiel}
	Sei $M$ eine $C^\infty$ Mannigfaltigkeit und $x_0 \in M$.
	Dann ist $\cal C^\infty_{M,x_0}$ ein lokaler Ring, denn
	\[
		\cal C^\infty_{M,x_0} \setminus \big(\cal C^\infty_{M,x_0}\big)^\times
		= \{[f:U\xto{C^\infty} \R] \mid x_0 \in U\text{ mit } f(x_0) = 0\}
		=: \m,
	\]
	da $[f]$ eine Einheit ist, genau dann, wenn $f(x_0) \neq 0$: 
	Ist $f: U\xto{C^\infty} \R$ mit $f(x_0) \neq 0$, so existiert
	$W\subset U$ offen, $x_0\in W$ mit $f(x) \neq 0$ für alle $x\in W$.
	Damit folgt
	\[
		\left[\frac{1}{f}: W \to \R,\ x\mapsto \frac{1}{f(x)}\right]
		\in \cal C^\infty_{M,x_0}
	\]
	ist Inverses zu $[f]$.
	Zudem ist $\m$ ein Ideal.
\end{beispiel}

\begin{definition}[lokal geringter Raum]
	Ein \emph{lokal geringter Raum} ist ein Paar $(X, \O_X)$ bestehend aus:
	\begin{itemize}
	  \item einem topologischen Raum $X$ und
	  \item einer Garbe $\O_X$ auf $X$ von Ringen,
	\end{itemize}
	so dass $\O_{X,x_0}$ für alle $x_0\in X$ ein lokaler Ring ist.
	
	Man nennt $\O_X$ die \emph{Strukturgarbe von $(X,\O_X)$}. Ist
	$x_0\in X$, so hat man das maximale Ideal
	$\m_{x_0} \ideal \O_{X,x_0}$.
	
	Der Körper 
	\[\kappa(x_0) := \O_{X,x_0} \big/ \m_{x_0} \]
	heißt \emph{Restklassenkörper von $x_0$ in $(X,\O_X)$}.
\end{definition}

\begin{beispiel}
	Sei $M$ eine $C^\infty$-Mannigfaltigkeit und $x_0 \in M$,
	so ist $\kappa(x_0) = \R$.
\end{beispiel}

\begin{uebung}[Übungsblatt 1 Aufgabe 2]
	\begin{enumerate}
	  \item Zeige, dass das Tupel $(\R, C_\R^\infty)$ bestehend aus $\R$
	  	und der Garbe der $C^\infty$-Funktionen einen lokal geringten Raum
	  	bilden. Zeige also, dass $C_{\R,x_0}^\infty$ für beliebiges
	  	$x_0 \in \R$ ein lokaler Ring ist, indem Du sein maximales Ideal
	  	$\m_{x_0}$ angiebst. Warum ist es das einzige maximale Ideal?
	  \item Zeige, dass $\forall x_0 \in \R:$ 
	  	$C_{\R,x_0}^\infty \big/ \m_{x_0} \cong\R$.
	  \item Zeige nun auf gleiche Weise, dass $\C$ mit der Garbe der 
	 	holomorphen Funktionen $\O_\C$ eine lokal gerinter Raum ist
	 	und dass $\O_{\C,z_0} \big/ \m_{z_0} \cong \C$ für alle
	 	$z_0 \in \C$ gilt.
	\end{enumerate}
\end{uebung}
\begin{proof}
	\begin{enumerate}
	  \item Es gilt
	  \begin{align*}
	  	[f:U\to \R] \in (C_{\R,x_0}^\infty)^\times
		  	\quad &\Leftrightarrow\quad
		  	\exists \text{ offene Umgebung $V$ um $x_0$ mit}
		  	f\rest{U\cap V} \neq 0 \ \forall x \in U\cap V\\
	  	&\Leftrightarrow\quad
	  		f(x_0) \neq 0.
	  \end{align*}
	  Also $[f] \in C_{\R,x_0}^\infty \setminus (C_{\R,x_0}^\infty)^\times$
	  genau dann, wenn $f(x_0) = 0$.
	  Damit ist $C_{\R,x_0}^\infty \setminus (C_{\R,x_0}^\infty)^\times$
	  ein Ideal. Es ist klar, dass dies das einzige maximale ist.
	  \item 
	  Wir definieren den surjektiven Gruppenhomomorphismus
	  \[
	  	\varphi: \funcdef{ C_{\R,x_0}^\infty & \to & \R\\
	  		{[f]} & \mapsto & f(x_0),}
	  \]
	  so folgt die Aussage aus dem Homomorphiesatz.
	  \item
	  Analog zu den vorherigen beiden.
	\end{enumerate}
\end{proof}

\begin{definition}[lokale Ringhomomorphismen]
	Sind $R,S$ lokale Ringe mit den maximalen Idealen
	$\m_R \ideal R$, $\m_S \ideal S$, so heißt der Ringhomomorphismus
	$\varphi: R\to S$ \emph{lokal},
	falls
	\[\varphi\inv (\m_S) = \m_R .\]
	Äquivalent lässt sich fordern, dass
	\[\varphi(\m_R) \subset \m_S.\]
\end{definition}

\begin{definition}[Morphismus lokal geringter Räume]
	\label{def:morphismus lokal geringter raume}
	Ein \emph{Morphismus $f:(X,\O_X) \to (Y,\O_Y)$ lokal geringter Räume}
	ist ein Paar $(f,f\fis)$ bestehend aus
	\begin{align*}
		f:  X &\to Y \text{ stetig},\\
		f\fis:  \O_Y &\to f_\ast\O_X \text{ Morphismus von Garben auf $Y$},
	\end{align*}
	so dass der von $f\fis$ induzierte Ringhomomorphismus für
	$x_0 \in X$, $y_0:= f(x_0) \in Y$
	\[f_{x_0}\fis:
		\funcdef{ \O_{Y,y_0} & \to & \O_{X,x_0}\\~ 
			[s] & \mapsto & [f_U\fis(s)]
		}
	\]
	für $s\in \O_Y(U)$ und $y_0\in U$  ein lokaler Ringhomomorphismus ist.
\end{definition}

\begin{bemerkung}
	In \thref{def:morphismus lokal geringter raume} ist $f_{x_0}\fis$
	wohldefiniert:
	
	Sei $[s] = [t] \in \O_{Y,y_0}$, d.h. es existiert $W\subset Y$ offen mit
	$y_0\in W$ und $s\rest W = t\rest W \in \O_Y(W)$.
	Betrachte nun $f_U\fis (s) \in \O_X(f\inv(U))$ für 
	$s\in \O_Y(U)$, $U\subset Y$, $y_0\in U$ und analog 
	$f_V\fis (t) \in \O_X(f\inv(V))$ für 
	$t\in \O_Y(V)$, $V\subset Y$, $y_0\in V$.
	Da $f\fis$ ein Garbenmorphismus ist, kommutiert damit folgendes Diagramm:
	\[
		\begin{tikzcd}[row sep=large, column sep=large]
		s \dar[mapsto] \symb{\in}
			&[-1cm] \O_Y(U) \rar{f_U\fis} \dar[swap]{\rest W} 
			& \O_X(f\inv(U)) \dar{\rest{f\inv(W)}} \symb{\ni}
			& \dar[mapsto] f_U\fis(s)\\
		s\rest W = t\rest W \symb{\in}
			& \O_Y(W) \rar{f_W\fis}				
			& \O_X(f\inv(W)) \symb{\ni}
			& f_U\fis(s)\rest{f\inv(W)} = f_V\fis(t)\rest{f\inv(W)}\\
		t \uar[mapsto] \symb{\in}
			& \O_Y(V) \rar{f_W\fis} \uar{\rest W} 
			& \O_X(f\inv(V)) \uar[swap]{\rest{f\inv(W)}} \symb{\ni}
			& \uar[mapsto] f_V\fis(t)\\
		\end{tikzcd}
	\]
\end{bemerkung}% vim: set ft=tex :
