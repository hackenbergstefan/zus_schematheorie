\section{Eigenschaften von Schemata} %Seite 79

\subsection{Noethersch}

\begin{definition}[(lokal) noethersch]
    $X$ heißt \emph{noethersch}, wenn es eine endliche affine offene
    Überdeckung gibt, d.h. 
    \[X = \bigcup_{i=1}^r \Spec A_i\]
    mit noetherschen Ringen $A_i$.
    
    $X$ heißt \emph{lokal noethersch}, wenn jeder Punkt $x \in X$ eine
    affine offene Umgebung $\Spec A \subseteq X$ hat mit $A$ noethersch.
\end{definition}

\begin{bemerkung}
    Aus $X$ lokal noethersch folgt $\O_{X,x}$ noethersch (Übungsaufgabe).
    Die Umkehrung gilt i.A. jedoch nicht.
\end{bemerkung}

\subsection{$k$-Varietäten}

\begin{definition}[algebraische/projektive $k$-Varietät]
    Sei $k$ ein Körper. Eine \emph{algebraische $k$-Varietät} ist ein
    $k$-Schema $X$, das eine endliche offene Überdeckung
    \[ X=  \bigcup_{i=1}^r \Spec A_i\]
    mit endlich erzeugten $k$-Algebren $A_i$ besitzt.
    
    Eine \emph{projektive $k$-Varietät} ist ein projektives $k$-Schema.
\end{definition}

\begin{bemerkung}
    \begin{itemize}
      \item Eine projektive $k$-Varietät ist eine algebraische 
        $k$-Varietät, da wir die abgeschlossene Immersion
        \[ X \immersion \P_k^n = \bigcup_{i=0}^n D_+(T_i) \cong
            \Spec k[Y_0,\ldots, \cancel i, \ldots, Y_n]\]
        haben.
      \item Eine $k$-Alegbra $A$ ist \emph{endlich erzeugt}, wenn
        es $n\in \N$ gibt und surjektive $k$-Algebrenhomomorphismen
        \[  
            \funcdef{k[Y_1,\ldots,Y_n] &\twoheadrightarrow& A\\
                Y_i & \mapsto & a_i.}\]
         Die $a_i$ sind dabei die Erzeuger von $A$.
    \end{itemize}
\end{bemerkung}


\subsection{Reduzierte Schemata}

\begin{definition}[reduzierte Ringe]
    Ein Ring $A$ heißt \emph{reduziert}, wenn
    \[ \sqrt{(0)} =: \Nil(A) = (0),\]
    also wenn $A$ keine nilpotenten Elemente hat.
\end{definition}

\begin{definition}[reduzierte lokal geringte Räume]
    $X$ heißt \emph{reduziert}, wenn $\O_{X,x}$ für jedes $x \in X$
    reduziert ist.
\end{definition}

\begin{satz}
    Es ist äquivalent:
    \begin{enumerate}
      \item $X$ ist reduziert.
      \item Zu jedem $x \in X$ existiert eine affin offene Umgebung 
        $U = \Spec A$ um $x$ mit $A$ reduziert.
      \item $O_X(U)$ ist reduziert für alle offenen $U\osubset X$. 
    \end{enumerate}
\end{satz}
\begin{proof}
    \TODO
\end{proof}


\subsection{Garbifizierung}

\begin{definition}[Garbifizierung]
    \renewcommand{\P}{\cal P}
    Sei $X$ ein topologischer Raum und $\P$ eine Prägarbe auf $X$.
    Dann ist die \emph{Garbifizierung} von $\P$
    \[
        \P^\dagger := \left( U \mapsto \P^\dagger(U)
            := \left\{f : U \to \coprod_{x \in U} \P_x \left|
            \begin{array}{l} f(x) \in \P_x \ \forall x \in U \\
            \forall x \in U \exists V \text{ mit } x\in V\osubset U\\
            \text{ und }\exists s \in \P(V) \text{ mit }
            \forall z\in V:\ f(z) = s_z := [s] \in \P_z.
            \end{array}\right.\right\} \right)
    \]
\end{definition}

\begin{satz}
    \renewcommand{\P}{\cal P}
    \begin{enumerate}
      \item $\P^\dagger$ ist eine Garbe und man hat einen kanonischen
        Prägarbenmorphismus $\P \to \P^\dagger$.
      \item Ist $\F$ eine Garbe, so ist $\F^\dagger \cong \F$ kanonisch
        via 1.
      \item Für alle $x\in X$ ist $(\P^\dagger)_x \cong \P_x$ kanonisch
        via 1.
      \item $\P^\dagger$ erfüllt die offenbare universelle Eigenschaft.     
    \end{enumerate}
\end{satz}
\begin{proof}
    \TODO
\end{proof}

% vim: set ft=tex :
