\section{Eigenschaften von Schemata} %Seite 79

\subsection{Noethersch}

\begin{definition}[(lokal) noethersch]
    \index[def]{Schema!noethersch}
    \index[def]{Schema!noethersch!lokal noethersch}
    $X$ heißt \emph{noethersch}, wenn es eine endliche affine offene
    Überdeckung gibt, d.h. 
    \[X = \bigcup_{i=1}^r \Spec A_i\]
    mit noetherschen Ringen $A_i$.
    
    $X$ heißt \emph{lokal noethersch}, wenn jeder Punkt $x \in X$ eine
    affine offene Umgebung $\Spec A \subseteq X$ hat mit $A$ noethersch.
\end{definition}

\begin{bemerkung}
    Aus $X$ lokal noethersch folgt $\O_{X,x}$ noethersch (Übungsaufgabe).
    Die Umkehrung gilt i.A. jedoch nicht.
\end{bemerkung}

\subsection{$k$-Varietäten}

\begin{definition}[algebraische/projektive $k$-Varietät]
    \index[def]{Varietät!algebraische}
    \index[def]{Varietät!projektive}
    Sei $k$ ein Körper. Eine \emph{algebraische $k$-Varietät} ist ein
    $k$-Schema $X$, das eine endliche offene Überdeckung
    \[ X=  \bigcup_{i=1}^r \Spec A_i\]
    mit endlich erzeugten $k$-Algebren $A_i$ besitzt.
    
    Eine \emph{projektive $k$-Varietät} ist ein projektives $k$-Schema.
\end{definition}

\begin{bemerkung}
    \begin{itemize}
      \item Eine projektive $k$-Varietät ist eine algebraische 
        $k$-Varietät, da wir die abgeschlossene Immersion
        \[ X \immersion \P_k^n = \bigcup_{i=0}^n D_+(T_i) \cong
            \Spec k[Y_0,\ldots, \cancel i, \ldots, Y_n]\]
        haben.
      \item Eine $k$-Alegbra $A$ ist \emph{endlich erzeugt}, wenn
        es $n\in \N$ gibt und surjektive $k$-Algebrenhomomorphismen
        \[  
            \funcdef{k[Y_1,\ldots,Y_n] &\twoheadrightarrow& A\\
                Y_i & \mapsto & a_i.}\]
         Die $a_i$ sind dabei die Erzeuger von $A$.
    \end{itemize}
\end{bemerkung}


\subsection{Reduzierte Schemata}

\begin{definition}[reduzierte Ringe]
    \index[def]{Ring!reduziert}
    Ein Ring $A$ heißt \emph{reduziert}, wenn
    \[ \sqrt{(0)} =: \Nil(A) = (0),\]
    also wenn $A$ keine nilpotenten Elemente hat.
\end{definition}

\begin{definition}[reduzierte lokal geringte Räume]
    \index[def]{lokal geringter Raum!reduziert}
    $X$ heißt \emph{reduziert}, wenn $\O_{X,x}$ für jedes $x \in X$
    reduziert ist.
\end{definition}

\begin{satz}
    Es ist äquivalent:
    \begin{enumerate}
      \item $X$ ist reduziert.
      \item Zu jedem $x \in X$ existiert eine affin offene Umgebung 
        $U = \Spec A$ um $x$ mit $A$ reduziert.
      \item $O_X(U)$ ist reduziert für alle offenen $U\osubset X$. 
    \end{enumerate}
\end{satz}
\begin{proof}
    \TODO
\end{proof}


\subsection{Garbifizierung}

\begin{definition}[Garbifizierung]
    \index[def]{Garbifizierung}
    \renewcommand{\P}{\cal P}
    Sei $X$ ein topologischer Raum und $\P$ eine Prägarbe auf $X$.
    Dann ist die \emph{Garbifizierung} von $\P$
    \[
        \P^\dagger := \left( U \mapsto \P^\dagger(U)
            := \left\{f : U \to \coprod_{x \in U} \P_x \left|
            \begin{array}{l} f(x) \in \P_x \ \forall x \in U \\
            \forall x \in U \exists V \text{ mit } x\in V\osubset U\\
            \text{ und }\exists s \in \P(V) \text{ mit }
            \forall z\in V:\ f(z) = s_z := [s] \in \P_z.
            \end{array}\right.\right\} \right)
    \]
\end{definition}

\begin{satz}
    \renewcommand{\P}{\cal P}
    \begin{enumerate}
      \item $\P^\dagger$ ist eine Garbe und man hat einen kanonischen
        Prägarbenmorphismus $\P \to \P^\dagger$.
      \item Ist $\F$ eine Garbe, so ist $\F^\dagger \cong \F$ kanonisch
        via 1.
      \item Für alle $x\in X$ ist $(\P^\dagger)_x \cong \P_x$ kanonisch
        via 1.
      \item $\P^\dagger$ erfüllt die offenbare universelle Eigenschaft.     
    \end{enumerate}
\end{satz}
\begin{proof}
    \TODO
\end{proof}

\begin{bemerkung}
    Für einen Ring $A$ und $\a \ideal A$ ist
    \[ \Spec A\big/\a \to \Spec A\]
    ein Homöomorphismus auf $V(\a) = V(\sqrt\a) \subseteq \Spec A$.
\end{bemerkung}

\begin{satz}
    Sei $X$ ein Schema. Dann existiert eine eindeutig bestimmte
    abgeschlossene Immersion eines reduzierten Schemas $X^\text{red}$
    \[X^\text{red} \immersion X\]
    mit $\operatorname{topRaum}(X^\text{red}) = 
    \operatorname{topRaum}(X)$.
\end{satz}

\begin{definition}[Kern- und Bildgarbe]
    \index[def]{Garbe!Garbenmorphismus!Kerngarbe}
    \index[def]{Garbe!Garbenmorphismus!Bildgarbe}
    \label{def:kern- bildgarbe}
    Sei $\alpha: \F \to \G$ ein Garbenmorphismus. Dann heißen
    \begin{align*}
        \ker\alpha: &\big( U \mapsto \ker(\alpha(U))\big)\\
        \im\alpha: &\big( U \mapsto \im(\alpha(U))\big)^\dagger
    \end{align*}
    \emph{Kern-} und \emph{Bildgarbe} von $\alpha$.
\end{definition}

\begin{bemerkung}
    In der Tat ist $\ker\alpha$ bereits eine Garbe.
\end{bemerkung}

\subsection{Sequenzen von Garben und der Homomorphiesatz}

\begin{definition}[Exakte Sequenz von Garben]
    \index[def]{Garbe!exakte Sequenz}
    Eine Sequenz von Garben
    \[0\to \F \xto\alpha \G \xto\beta \cal H \to 0\]
    heißt exakt, falls
    \begin{itemize}
        \item $\im\alpha = \ker\beta$
        \item $\ker\alpha = 0$
        \item $\im\beta = \cal H$         
    \end{itemize}
    im Sinne von \thref{def:kern- bildgarbe} gilt.
\end{definition}

\begin{satz}
    \label{satz:garbensequenz exakt <=> halmweise exakt}
    Eine Sequenz von Garben
    \[0\to \F \xto\alpha \G \xto\beta \cal H \to 0\]
    ist exakt genau dann, wenn sie halmweise exakt ist, d.h.
    \[ 0 \to \F_x \xto{\alpha_x} \G_x \xto{\beta_x} \to 0\]
    für jedes $x\in X$ exakt ist.
\end{satz}
\begin{proof}
    Zur Übung.
\end{proof}

\begin{satz}
    \label{satz:garbeniso <=> halmweise iso}
    Ist $\alpha:\F \to \G$ ein Garbenmorphismus. Es ist äquivalent:
    \begin{enumerate}
    \item $\alpha$ ist ein Garbenisomorphismus ist, d.h.
    für alle $U\osubset X$ ist $\alpha(U)$ ein Isomorphismus (von Ringen),
    \item $\alpha_x : \F_x \to \G_x$ ist ein Isomorphismus.
    \end{enumerate}   
\end{satz}
\begin{proof}
    klar.
\end{proof}

\begin{satz}[Homomorphiesatz für Garben]
    \label{satz:homomophiesatz garben}
    Ist 
    \[0 \to \cal N \to \cal F \xto\alpha \G \to 0\]
    eine kurze exakte Sequenz von Garben,
    so induziert $\alpha$ einen Isomorphismus
    \[\bar\alpha: \F\big/\ker\alpha \xto\cong \G.\]
\end{satz}
\begin{proof}
\TODO
\end{proof}

\subsection{Reduzierte Schemata II}

\begin{satz}
    Sei $X$ ein Schema, $Z\subseteq X$ eine abgeschlossene Teilmenge. Dann
    kann man auf $Z$ eine Schemastruktur definieren, so dass
    $(Z,\O_Z) \immersion (X,\O_X)$ eine abgeschlossene Immersion ist und
    $(Z, \O_Z)$ reduziert ist. Diese ist eindeutig und heißt
    \emph{reduzierte Unterschema-Strukur}.
    \index[def]{Schema!Unterschema!reduzierte Unterschema-Struktur}
\end{satz}
\begin{proof}
\TODO
\end{proof}

\subsection{Integere Schemata}

\begin{definition}[integeres Schema]
    \index[def]{Schema!integer}
    Ein Schema $X$ heißt \emph{integer}, wenn für jedes $U\osubset X$
    offen der Ring $\O_X(U)$ nullteilerfrei ist.
\end{definition}

\begin{bemerkung}
    $X = \Spec A$ ist integer genau dann, wenn $A$ nullteilerfrei.
\end{bemerkung}

\begin{lemma}
    \label{lemma:a nullteilerfrei. kanonischer morphismus nach quot a injektiv}
    Ist $A$ nullteilerfrei, so ist für jedes $U \osubset X = \Spec A$
    der kanonische Morphismus
    \[ \O_X(U) \to \O_{X,\eta} = \Quot(A)\]
    für $\eta = (0)$ injektiv.
    Ferner ist für $V\osubset U$ die Restriktion $\O_X(U) \to \O_X(V)$
    injektiv.
\end{lemma}
\begin{proof}
\TODO
\end{proof}

\begin{satz}
    \label{satz:integer <=> reduziert und irreduzibel}
    Ein Schema $X$ ist genau dann integer,
    wenn $X$ reduziert und irreduzibel ist.
\end{satz}
\begin{proof}
\TODO
\end{proof}


% vim: set ft=tex :
