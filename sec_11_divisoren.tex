\section{Divisoren}

\subsection{Cartier-Divisoren}
\newcommand{\KX}{\cal K_X}

\begin{definition}[Garbe der Keime meromorpher Funktionen]
    Die Garbe
    \[ a(')\]
    mit 
    \[
        \KX: U \mapsto \O_X(U)[S(U)\inv]
    \]
    für 
    \[S(U) := \{\sigma \in \O_X(U) \mid [\sigma]_x \in \O_{X,x}
        \text{ regulär }\forall x\in X\}\]
    heißt \emph{Garbe der meromorphen Funktionen}.
\end{definition}

\begin{bemerkung}
    Ist $X$ ganz, dann ist $\KX$ die konstante Garbe mit Wert $K(X)$, 
    schreibe $\KX = \underline{K(X)}$.
\end{bemerkung}


\begin{definition}[Cartier-Divisor]
    Sei $X$ ein Schema.
    \begin{enumerate}[label=(\roman*)]
      \item Dann heißt $\Div(X) := \Gamma(X, \KX^\times \big/ \O_X^\times)$
        die Gruppe der \emph{Cartier-Divisoren auf $X$}.
      \item Das Bild von $\div: \KX^\times(X) \to \Div(X),\ f \mapsto \div(f)$
        sind die \emph{Hauptdivisoren}.
      \item Zwei Divisoren $D_1,D_2 \in \Div(X)$ heißen 
        \emph{linear äquivalent}, falls $D_1 - D_2$ ein Hauptdivisor ist.
      \item Die \emph{Cartier-Divisoren-Klassengruppe von $X$} ist
        \[ \CaCl(X) := \Div(X) \big/ \sim,\]
         wobei $\sim$ lineare Äquivalenz ist.
    \end{enumerate}
\end{definition}

\begin{bemerkung}
    Ein 
    \[
        D\in \Div(X) = \KX^\times \big/ \O_X^\times (X) = 
        \varinjlim_{\U} \Hv^0(\U, \KX^\times \big/^\text{pre} \O_X^\times)
    \]
    entspricht einer Familie 
    \[ D= [(U_i,f_i)_{i\in I}]\] mit
    $X = \cup_i U_i$ einer offenen Überdeckung und 
    $\KX^\times(U_i) \ni f_i = \tfrac{a_i}{b_i}$ mit 
    $a_i,b_i \in \O_X^\times (U_i)$ halmweise regulär.
\end{bemerkung}

\subsubsection{Cartier-Divisoren und Geradenbündel}

\begin{lemma}
    Es existiert eine eindeutig bestimmte Untergarbe ($\O_X$-Modul-Untergarbe)
    $\O_X(D) \subseteq \KX$
    mit \[\O_X(D) = f_i\inv \O_X\rest{U_i}.\]
    Sie ist unanbhängig von der Darstellung $D = (U_i,f_i)_{i\in I}$.
\end{lemma}
\begin{proof}
\TODO
\end{proof}

\begin{satz}
    Die Abbildung $\rho: \Div(X) \to\Pic(X),\ D \mapsto \O_X(D)$ ist additiv
    und induziert einen injektiven Gruppenhomomorphismus
    \[ \rho: \CaCl(X) \hookrightarrow \Pic(X).\]
    Es ist $\im(\rho) = \{[\L] \in \Pic(X) \mid \L \subseteq \KX\}$.
\end{satz}
\begin{proof}
\TODO
\end{proof}

\begin{satz}
    Ist $X$ ganz, so ist $\rho: \CaCl(X) \xto{\cong} \Pic(X)$ ein 
    Isomorphismus.
\end{satz}

\subsection{Weil-Divisoren}

\begin{definition}[Primzykel, Zykel, Träger eines Zykels, 
    Zykel von Kodimension 1]
    Sei $X$ noethersch.
    \begin{enumerate}[label=(\roman*)]
      \item Ein \emph{Primzykel in $X$} ist eine irreduzible, abgeschlossene
        Teilmenge.
      \item Ein \emph{Zykel in $X$} ist ein Element der abelschen Gruppe
        \[\Z^{(X)} := \{Z = \sum_{x\in X} n_x \cl x \mid
            n_x \in \Z,\ n_x = 0\text{ für fast alle }x\}.\]
      \item Für $Z = \sum n_x \cl x$ heißt
        \[ \supp Z := \bigcup_{x\in X\atop n_x\neq 0}\]
        der \emph{Träger von $Z$}.
      \item Ein Zykel $Z$ heißt von \emph{Kodimension 1}, wenn alle $x$ mit
        $n_x \neq 0$ von Kodimension 1 sind, d.h. 
        $\codim_X \cl x = 1$. Äquivalent dazu ist zu fordern, dass
        $\dim \O_{X,x} = 1$.\\
        $Z^1(X) \subseteq \Z^{(X)}$ bezeichne die Untergruppe dieser.
    \end{enumerate}
\end{definition}

\begin{definition}[Weil-Divisoren]
    Sei $X$ noethersch und integer, so heißt $Z^1(X)$ die Gruppe der
    \emph{Weil-Divisoren}.
\end{definition}

\begin{satz}
    Sei $X$ noethersch und integer, $0\neq f \in K(X) = \O_{X,\eta}$.
    Dann gilt $f \in \O_{X,x}^\times$ für fast alle $x\in X$ mit
    $\dim \O_{X,x} = 1$.
\end{satz}


\begin{einschub}{Integere Schemata und Bewertungen}
Sei $X$ integer.

\begin{definition}[normal, ganzabgeschlossen]
    $X$ heißt \emph{normal}, wenn für jedes $x \in X$ der lokale Ring
    $\O_{X,x}$ in $\Quot(\O_{X,x})$ \emph{ganzabgeschlossen} ist,
    d.h. ist $\sigma \in \Quot(\O_{X,x})$, welches einer Polynomgleichung
    $f(\sigma) = 0$ mit $f \in \O_{X,x}[X]$, $f$ normiert, genügt,
    so ist $\sigma \in \O_{X,x}$.
\end{definition}

\begin{bemerkung}
    Ist $X$ lokal noethersch, integer, normal und $x \in X$ ein Punkt mit
    Kodimension 1, so ist $\O_{X,x}$ ein lokaler Dedekindring.
\end{bemerkung}

\begin{lemma}
    Jeder lokale Dedekindring $(A,\m)$ ist ein Hauptidealring. 
\end{lemma}
\begin{proof}
\TODO
\end{proof}

\begin{folgerung}
    $\O_{X,x}$ trägt die kanonische diskrete Bewertung
    \[ \nu_x: \funcdef{ \O_{X,x} & \to & \N \cup\{\infty\} \\
        0\neq a = u \pi^r & \mapsto & r = \sup\{k \mid a \in \m^k\}\\
        0 & \mapsto & \infty  }\]
    wobei das maximale Ideal $\m = (\pi)$ sei. Fortgesetzt auf
    $\Quot(\O_{X,x})$ ergibt sich
    \[\nu_x: \Quot(\O_{X,x}) = K(X) \to \Z\cup \{\infty\}.\]
    Das bedeutet, man hat die diskrete Bewertung
    \[\mult_x: K(X) \to \Z\cup\{\infty\}\]
    mit
    \begin{itemize}
      \item $\mult_x(f) = \infty$ $\Leftrightarrow$ $f=0$.
      \item $\mult_x(fg) = \mult_x(f) + \mult_x(g)$.
      \item $\mult_x(f+g) \geq \min\{\mult_x(f), \mult_x(g)\}$.
    \end{itemize}
\end{folgerung}
\end{einschub}

\begin{definition}[Hauptdivisor]
    Sei $X$ noethersch und normal. Für $f \in K(X)\setminus\{0\}$ heißt
    \[(f) := \sum_{x\in X\atop \dim\O_{X,x} = 1} \mult_x(f) \cdot \cl x
        \quad \in Z^1(X)\]
    der \emph{Hauptdivisor zu $f$}.
\end{definition}



\pagebreak



