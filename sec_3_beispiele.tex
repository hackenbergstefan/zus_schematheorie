\section{Beispiele}

\subsection{$\Spec \Z$}

Jeder Ring $A$ hat einen eindeutigen Homomorphismus
\[
	\funcdef{
		\Z & \to & A\\
		 1 & \mapsto & 1\\
		 z & \mapsto & \begin{cases} 1 + 1 + \ldots + 1 & z > 0\\
		 	0 & z = 0\\
		 	-1 -1 - \ldots- 1 & z < 0
		 \end{cases}.}
\]
$\Z$ ist daher ein \emph{initiales Objekt} in der Kategorie $\Ring$.

Wir haben daher einen eindeutigen Morphismus $\Spec A \to \Spec \Z$ von
affinen Schemata. $\Spec \Z$ ist somit ein \emph{finales Objekt} in
der Kategorie $\affSch$.

Ferner können wir zusammenfassen
\paragraph{Offene Mengen}
	$\emptyset \neq U\subseteq \Spec \Z$ offen 
	$\Leftrightarrow$ $U = 
\begin{cases}
  \Spec \Z \setminus \{(p_1),\ldots,(p_r)\} &, r\in \N_0\\
  \emptyset
\end{cases}
$
	
\paragraph{Basisoffene Mengen}
	$D(f) = \{\p \in \Spec\Z \mid f \notin \p\} = 
	\Spec\Z \setminus \{(p_1),\ldots,(p_r)\}$ für 
	$f = p_1^{\nu_1}\ldots p_r^{\nu_r}$.
	
\paragraph{Strukturgarbe}
	\begin{align*}
		\O_{\Spec\Z} (D(f)) &= \Z_f  = 
			\left\{ \frac{a}{f^n} \mid n\in \N_0, a\in \Z\right\} \\
		\O_{\Spec\Z, (p)} &= \Z_{(p)} = 
			\left\{ \frac{a}{b} \mid p\nmid b, a\in \Z\right\} 
 	\end{align*}

\subsection{$\Spec k$ für einen Körper $k$}
\paragraph{Als topologischer Raum}
	$\Spec k = \{(0)\}$.

\paragraph{Strukturgarbe}
	$\O_{\Spec k}(\{(0)\}) = k$.

\begin{bemerkung}
   Sei $A$ ein Ring. Angenommen wir haben 
  	$\Spec A \xto{(f,f\fis)} \Spec k$ für einen Körper $k$, so haben wir
  	\[
  		f\fis_{\Spec k}: k = \O_{\Spec k} \to f_\ast\O_{\Spec A}(\Spec k)
  			\tikzmark{=} A,
  	\]
  	wobei \tikzarrow{mark above}{} aus 
  	$\O_{\Spec A}(f\inv(\{(0)\})) = \O_{\Spec A}(\Spec A)$ resultiert.
  	Insgesamt ist $A$ also eine $k$-Algebra (d.h. ein Ring zusammen mit
  	$k\to A$).
  	
  	Bemerke hierbei "`Grothendiecks Gesamtphilosophie"':
  	\begin{quote}\itshape
  		Alles relativ lesen!
  	\end{quote}
\end{bemerkung}

\begin{definition}[$S$-Schema]
	Sei $S$ ein Schema. Dann ist ein \emph{$S$-Schema} ein Schema $X$
	zusammen mit einem Strukturmorphismus $X \xto{\varphi} S$.
	Dies ergibt die Kategorie $\Sch_S$, wenn man
	\[
		\Hom( X\xto{\varphi}S, Y\xto{\varphi} S) := 
		\left\{ 
		\begin{tikzcd}
		X \arrow{rr}{f} \drar{\varphi} & & Y \dlar{\psi} \\ & S &
		\end{tikzcd}
		\right\}
	\]
	setzt.
\end{definition}

\begin{beispiel}
	$\Sch_k := \Sch_{\Spec k}$ sind die sog. \emph{$k$-Schemata}.
	Ein Beispiel hierfür ist
	$\Spec k[X_1,\ldots,X_n] \to \Spec k$ via 
	$k \hookrightarrow k[X_1,\ldots,X_n]$.
\end{beispiel}


\begin{bemerkung}
	Sei $X$ ein Schema und $x\in X$ und weiter $\m_x \ideal \O_{X,x}$ das
	maximale Ideal.
	Dann ist 
	\[
		\kappa(x) := k(x) := \O_{X,x} \big/ \m_x
	\]
	der \emph{Restklassenkörper von $x$}.
	
	Betrachte nun $(f,f\fis): \Spec k \to X$ mit
	\[
		f: \funcdef{\Spec k(x) & \to & X \\
			\eta_x & \mapsto & x,}
	\]
	wobei topologisch gesehen $\eta_x \in \Spec k(x)$ der einzige Punkt 
	dieses Schemas ist.
	Für $U\subseteq X$ offen haben wir:
	\[
		f\fis_U : \O_X \to 
			f_\ast \O_{\Spec k(x)}(U) = 
			\begin{cases} 0 & x\notin U \\ k(x) & x \in U. \end{cases}
	\]
	Im Fall $x \in U$ geht dies via
	\[
		\O_X(U) \to \O_{X,x} = \varinjlim_{x\in V} \O_X(V)
			\overset\pi\twoheadrightarrow  \O_{X,x}\big/ \m_x = k(x). 
	\]
	
	Ist umgekehrt $(f,f\fis):\Spec k \to X$ ein Schemamorphismus, so
	setze $x := f((0)) \in X$ und
	$f\fis: \O_X \to f_\ast \O_{\Spec k}$ liefert einen Ringhomomorphismus der
	Halme:
	\[
		f_x\fis: \O_{X,x} \to \O_{\Spec k, (0)} = k.
	\]
	Dieser ist lokal (also $f\fis_x (\m_x) = (0)$). Damit ist
	\[
		\begin{tikzcd}
		k(x) = \O_{X,x} \big/ \m_x \rar[hookrightarrow]{f_x\fis \mod \m_x} 
		&[7ex]  {f_x\fis \mod \m_x} k
		\end{tikzcd}
	\]
	wohldefiniert und somit ist $k \mid k(x)$ eine Körpererweiterung.
	
	Zusammengefasst haben wir:
	\[	\fbox{\parbox{5cm}{
			Einen Punkt $x\in X$ wählen mit Restklassenkörper
			$k(x)$ und eine Körpererweiterung $k\mid k(x)$.}}
		\Longleftrightarrow
		\fbox{\parbox{5cm}{
			Einen Schemamorphismus $\Spec k \to X$ wählen
			für eine Körpererweiterung $k\mid k(x)$.}}
	\]
\end{bemerkung}

\subsection{Der Affine $n$-dimensionale Raum über $k$}
Sei $k$ wieder ein Körper. Der affine $n$-dimensionale Raum über $k$ ist
$\A_k^n := \Spec k[X_1,\ldots, X_n]$.

Wir erinnern an den Hilbertschen Nullstellensatz:
\begin{satz}[Hilbertscher Nullstellensatz]
	\label{satz:hilbertscher nullstellensatz}
	Sei $k$ algebraisch abgeschlossen. Dann ist jedes maximale Ideal
	in $k[X_1,\ldots, X_n]$ von der Form
	$(X_1-a_1, \ldots, X_n - a_n)$.
\end{satz}
\begin{proof}
	ohne Beweis.
\end{proof}

Wir haben bereits gezeigt:
\[
	|\A_k^n| = k^n, \qquad\text{via } 
		(X_1-a_1,\ldots,X_n-a_n) \mapsto (a_1,\ldots,a_n).
\]
Sei $\p = (f_1, \ldots, f_r)$ ein nicht maximales Ideal in $k[X_1,\ldots,X_n]$
(die Darstellung ist nach \thref{satz:hilbertscher nullstellensatz}) möglich,
so gilt
\[
	\p \subseteq (X_1 - a_1, \ldots, X_n - a_n)
	\quad\Leftrightarrow\quad
	f_1(a_1,\ldots,a_n) = 0, \ldots,
	f_r(a_1,\ldots,a_n) = 0
\]
Wir können dies in \autoref{fig:spec k xy 2} "`sehen"'.

\begin{figure}\centering
	\caption{$\Spec k[X_1,\ldots,X_n]$}
	\label{fig:spec k xy 2}
	\begin{tikzpicture}
		\draw[very thick] 
			(-3,0) -- (3,0) node[near end, above] {$X_1$}
			(0,-2) -- (0,2) node[near end, right] {$X_2$};
		
% 		\draw[col1,thick] 
% 			(-3,1) to[out=-5, in=135]  (0.2,-0.2) 
% 			to[out=-45, in=225, looseness=2] (-1,0) 
% 			to[out=45, in=180, looseness=0.5] (3,1.5)
% 			node[pos=0.9] {$f(X,Y) = 0$};
		\draw[col1, thick]
			(-3,1) 
			.. controls (5,-2) and (-8,-2) .. 
			(3,1.5)
			node[pos=1, right, text width=2.9cm, font=\scriptsize] 
				{$\{(a_1,\ldots,a_n) \mid f_j(a_1,\ldots,a_n) = 0,$\\ 
					$j=1\ldots r\}$}
			coordinate[pos=0.05] (a);
		
		\fill[col1shade2] (a) circle[radius=2pt]
			node[above right, col1] {$(\alpha,\beta)$};
		
		\path (-3,1)
			node[generic point=10pt, fill=col1shade2] {}
			node[above left, col1] {$\p$};

		\fill[col2] (0,0) circle[radius=2pt]
      node[below right, col2] {$(0,\dots,0)$\scriptsize{ "`entspricht"'
      $(X_1,\dots,X_n)\in \A_k^n$}};
	\end{tikzpicture}
\end{figure}

\subsection{Weiteres Beispiel}
Betrachte $k\ldbrack X_1, \ldots, X_n\rdbrack = 
	k\ldbrack X_1,\ldots,X_{n-1}\rdbrack\ldbrack X_n\rdbrack$
mit $R\ldbrack X\rdbrack = \{\sum_{i=0}^\infty a_i X^i \mid a_i \in R\}$.

\begin{bemerkung}
	$g \in k\ldbrack X_1, \ldots, X_n\rdbrack \setminus (X_1,\ldots,X_n)$
	ist eine Einheit.
\end{bemerkung}
\begin{proof}
	Idee: Ansatz für eine Variable:
	$g(X) = a_0 + a_1X + a_2X^2+ \ldots$. Dann
	\[
		1 = g(X)h(X) = 
		\underbrace{a_0 b_0}{= 1} + 
		(\underbrace{a_0b_1+a_1b_0}{= 0})X + \ldots
	\] 
\end{proof}

\paragraph{Funktor $\Spec$} Wir haben den Funktor $\Spec$:
Die Ringhomomorphismen
\[\everymath{\displaystyle} \begin{tikzcd}[row sep=tiny, outer sep=5pt]
  k[X_1,\ldots,X_n] \rar & \tikzmark{k[X_1,\ldots,X_N]_{(X_1,\ldots,X_n)}} \rar
  &
 k\ldbrack X_1,\ldots,X_n \rdbrack \rar & k \\
	f \rar[mapsto] & \frac{f}{1} \\
	& \frac f g \rar[mapsto] & f g\inv \\
	&& h \rar[mapsto] & h(0)
\end{tikzcd}\]
\tikzmargin{south}{Lokalisierung an $\m=(X_1,\dots,X_n)$}
induzieren 
\[\everymath{\displaystyle} \begin{tikzcd}[row sep=tiny, outer sep=5pt]
	&\Spec k \rar & k\ldbrack X_1,\ldots,X_n \rdbrack \rar & 
	k[X_1,\ldots,X_n]_{(X_1,\ldots,X_n)} \rar &
	\Spec k[X_1,\ldots,X_n] \\
	\text{topologisch:} &  
	(0) \rar[mapsto] & (X_1,\ldots,X_n)  \rar[mapsto]& 
	(X_1,\ldots,X_n) \rar[mapsto] & (X_1 , \ldots,X_n).\\
	&&\makebox[0pt]{\parbox{3cm}{\centering\small 
		einziger abgeschlossener Punkt}} 
	&\makebox[0pt]{\parbox{3cm}{\centering\small 
		einziger abgeschlossener Punkt}}
	&\makebox[0pt]{\parbox{3cm}{\centering\small 
		entspricht dem abgeschlossenen Punkt $(0,\ldots,0) \in k^n$}}
\end{tikzcd}\]
Dies ist ein Homöomorphismus auf $\{\p \in \A_k^n \mid 
\p \subseteq (X_1,\ldots,X_n)\} = V(\p) = \overline{\{\p\}} \subseteq \A_k^n$.
\begin{comment}
Gilt das nicht nur für den letzten Pfeil?
\end{comment}

Was passiert aber auf Schemaniveau?
\begin{center}\begin{tikzcd}[column sep=large]
	\node{\tikz{
		\fill[col1shade2] circle[radius=2pt];
	}};
	\rar &
	\node{\tikz{
		\draw[->]
			(-1,0) -- (1,0) node[very near end, above] {$X_1$};
		\draw[->]
			(0,-1) -- (0,1) node[very near end, right] {$X_n$};
		\fill[col1shade2] circle[radius=2pt];
		\node[text width=2cm, font=\scriptsize, text=col1shade2, right]
			 at (0.2,-0.5)
			 (text)
			 {einziger abgeschlossener Punkt};
	}};
	\rar &
	\node{\tikz{
		\draw[->]
			(-1,0) -- (1,0) node[very near end, above] {$X_1$};
		\draw[->]
			(0,-1) -- (0,1) node[very near end, right] {$X_n$};
		\fill[col1shade2] circle[radius=2pt];
		\node[text width=2cm, font=\scriptsize, text=col1shade2, right]
			 at (0.2,-0.5)
			 (text)
			 {einziger abgeschlossener Punkt};
		\draw[col1, thick, dashed]
			(-1,1) 
			.. controls (1.5,-0.8) and (-1.5,-0.8) .. 
			(1,1);
		\node[generic point=5pt, fill=col1] at (1,1) {};
		\node[right] at (1,1) {$\p$};
	}};
	\rar &
	\node{\tikz{
		\draw[->]
			(-1,0) -- (1,0) node[very near end, above] {$X_1$};
		\draw[->]
			(0,-1) -- (0,1) node[very near end, right] {$X_n$};
		\fill[col1shade2] circle[radius=2pt];
		\draw[col1, thick]
			(-1,1) 
			.. controls (1.5,-0.8) and (-1.5,-0.8) .. 
			(1,1);
		\node[generic point=5pt, fill=col1] at (1,1) {};
		\node[right] at (1,1) {$\p$};
	}};
\end{tikzcd}\end{center}
Betrachte dazu
\[\everymath{\displaystyle} \begin{tikzcd}[row sep=tiny, outer sep=5pt]
	\Spec k \rar & k\ldbrack X_1,\ldots,X_n \rdbrack \big/ \p \rar & 
	k[X_1,\ldots,X_N]_{(X_1,\ldots,X_n)} \big/ \p \rar &
	\Spec k[X_1,\ldots,X_n]\big/\p \quad \approx\quad V(\p)
	\end{tikzcd}
\]
Nehmen wir das explizite Beispiel $\p = (Y^2 - X^2(X+1))$. Es ist $\p$ ein
Primideal und $V(\p)$ irreduzibel.

Beachte: $1+X \in k\ldbrack X \rdbrack$ hat eine Wurzel, wie man durch
folgenden Ansatz mit $h(X) = a_0 + a_1 X + \ldots$ sieht:
\[
	1+ X = (h(X))^2 = a_0^2 + 2a_0a_1 X + \ldots
\]
Setze $a_0 := 1$ oder $-1$ und löse sukzessizve auf. Demnach ist
$Y^2 - X^2(X+1) = (Y-Xh(X))(Y + X h(X))$ nicht mehr prim, also
$V(\p) \subseteq k\ldbrack X,Y \rdbrack$ nicht mehr irreduziebel!

Betrachte genauer
\[\begin{tikzcd}[row sep=tiny]
	k \ldbrack u,v\rdbrack \big/(uv) \rar{\cong} & 
	k \ldbrack z,w\rdbrack \big/(z^2-w^2) \rar{\cong} &
	k \ldbrack X,Y\rdbrack \big/(Y^2 - X^2(h(X))^2)\\
	u \rar[mapsto] & z+w \qquad z \rar[mapsto] & Y\\
	v \rar[mapsto] & z-w \qquad w \rar[mapsto] & Xh(X)\\
\end{tikzcd}\]
In Bildern:
\[\begin{tikzcd}[row sep=-15pt]
	\Spec k\ldbrack u,v\rdbrack \big/(uv) \rar & \Spec k\ldbrack X,Y\rdbrack
		\big/ (Y^2 - X^2(X+1)) \\
	\node{\tikz{
		\draw[->]
			(-1,0) -- (1,0) node[very near end, above] {$X$};
		\draw[->]
			(0,-1) -- (0,1) node[very near end, right] {$Y$};
		\draw[col1, opacity=0.4, line width=3pt]
			(-.8,0) -- (.8,0)
			(0,-.8) -- (0,.8);
	}}; \rar &
	\node{\tikz{
		\draw[->]
			(-1,0) -- (1,0) node[very near end, above] {$X$};
		\draw[->]
			(0,-1) -- (0,1) node[very near end, right] {$Y$};
		\draw[dashed]
			(1,1) 
			.. controls (-1.4,-2) and (-1.4,2) .. 
			(1,-1);
		\clip (-0.3,-0.3) rectangle (0.3,0.3);
		\draw[line width=3pt, col1, opacity=0.4]
			(1,1) 
			.. controls (-1.4,-2) and (-1.4,2) .. 
			(1,-1);
	}};
\end{tikzcd}\]

\subsection{Spezielles Beispiel $\A_\Z^1 = \Spec \Z[X]$}
Wir haben $\pi: \A_\Z^1 \to \Spec \Z$. Topologisch ist
\[
	\A_\Z^1 = \bigcup_{p \text{ prim}} \pi\inv((p)) \cup \pi\inv((0)).
\]
\autoref{fig:A 1 Z to Spec Z} verdeutlicht dies.

\begin{figure}
	\caption{Veranschaulichung von $\A_\Z^1 \to \Spec\Z$}
	\label{fig:A 1 Z to Spec Z}
	\centering
	\begin{tikzpicture}
		\draw[fill=col1shade1, draw=col1]
			(-0.5,0) rectangle (6,4);
		\node[right, text=col1shade2] at (6,2) {$\A_\Z^1$};
		
		\draw[col1, thick]
			(0,0) -- (0,4)
			node[near end, below, sloped] {$\pi\inv((0))$};
		\draw[col1, thick]
			(2,0) -- (2,4)
			node[near end, below, sloped] {$\pi\inv((2))$};
		\draw[col1, thick]
			(3,0) -- (3,4)
			node[near end, below, sloped] {$\pi\inv((p))$};
		
		\draw[->,thick] 
			(3,-0.5) -- (3,-1.5);
		
		\draw[col1shade2, thick]
			(-0.5,-2) -- (6,-2)
			node[right, text=col1shade2]{$\Spec \Z$};
		\draw[col1, thick]
			(0,-1.8) -- +(0,-0.4)
			node[below] {$(0)$};
		\draw[col1, thick]
			(2,-1.8) -- +(0,-0.4)
			node[below] {$(2)$};
		\draw[col1, thick]
			(3,-1.8) -- +(0,-0.4)
			node[below] {$(p)$};
	\end{tikzpicture}
\end{figure}

\paragraph{Zu $\pi\inv((0))$}
Betrachte nun $\p \in \Spec \Z[X]$, so gilt
$\p \in \pi\inv((0))$ $\Leftrightarrow$ $\p \cap \Z = (0)$.

Betrachte $S:= \Z \setminus \{0\} \subseteq \Z[X]$ und die Lokalisierung
$g: \Z[X] \hookrightarrow \Z[X]_S$. Es ist klar: $\Z[X]_S = \Q[X]$

Ferner gilt $\Spec \Q[X] \to \Spec \Z[X]$ ist ein Homöomorphismus auf sein 
Bild:
\[
	\{\p \in \Spec\Z[X] \mid \p \cap S = \emptyset \} = 
	\{\p \in \A_\Z^1 \mid \p \cap \Z = (0) \} = \pi\inv(0),
\]

\paragraph{Zu $\pi\inv((p))$}
Es ist $\p \in \pi\inv((p))$ $\Leftrightarrow$ $p \in \p$.
Dann betrachte
$\rho: \Z[X] \twoheadrightarrow \bb F_p[X]$ und
$\rho^\ast: \Spec \bb F_p[X] \to \A_\Z^1$.
Wegen $\bb F_p[X] \cong \Z[X] \big/ \ker\rho$ ist $\rho^\ast$ ein Homöomorphismus
auf 
\[
	V(\ker \rho) = \{\p \in \Spec\Z[X] \mid \ker \rho \subseteq \p\} = 
	\pi\inv((p)) \subseteq \A_\Z^1.
\] 

Zusammengefasst ist:
\begin{align*}
	\pi\inv((0)) &= \A_\Q^1\\
	\pi\inv((p)) &= \A_{\bb F_p}^1,
\end{align*}
wobei die Gleichheiten topologisch zu lesen sind.

\paragraph{Betrachte $\p\in \Spec\Z[X]$}
\begin{description}
\item[1. Fall.]
	$\p\in \pi\inv((0))\ \Leftrightarrow\ \p\cap \Z = (0)$, also
	\[
		\p = (\mu(X))
	\]
	mit $\mu(X) \in \Z[X]$ einem primitiven, irreduziblen Polynom.
\item[2. Fall.]
	$\p\in\pi\inv((p))$, so ist $\p = \rho\inv(\q)$ für ein 
	$\q\in \Spec\bb F_p[X]$, also
	$\p = \rho\inv((q(X)))$ für ein irreduzibles $q(X)\in \bb F_p[X]$
	oder $(0)$. Dann ist
	\[
		\p = (r(X), p)
	\]
	mit $r(X) \in \Z[X]$ und $r(X) \equiv q(X) \bmod p$.
\end{description}
Es stellt sich die Frage, wie für $f\in \Z[X]$ die $D(f) \subseteq \A_\Z^1$
aussehen. Dazu
\begin{description}
\item[1. Fall $\p\in \pi\inv((0))$.] Sei $f(X) \in \Q[X]$. Dann
	$f(X) = \xi q_1(X)^{\nu_1} \ldots q_r(X)^{\nu_r}$ und es gilt
	\[
		f\notin \p \ \Leftrightarrow\ \p = (q(X))
	\]
	mit $q \neq q_1, \ldots, q_r$.
\item[2. Fall $\p \in \pi\inv((p))$.] $f(X) \notin (r(X), p)$
	mit $r(X) \mod p \in \bb F_p[X]$ irreduzibel. Für eine Primzahl $p$,
	betrachte $\bar f(X) \in \bb F_p[X]$.
	Ist $\bar f(X) = 0$, so ist $f(X) \in (r(X),p)$ für alle $r(X)$.
	Für $\bar f(X) = \bar q_1(X)^{\nu_1} \ldots \bar q_s(X)^{\nu_s}$, ist
	$f(X) \in (q_i(X), p)$ für diese $i$.
\end{description}
Dargestellt ist dies wieder in \autoref{fig:A 1 Z to Spec Z 2}.

\begin{figure}
	\caption{Veranschaulichung von $D(f) \subseteq \A_\Z^1$}
	\label{fig:A 1 Z to Spec Z 2}
	\centering
	\begin{tikzpicture}
		\draw[fill=col1shade1, draw=col1]
			(-0.5,0) rectangle (6,4);
		\node[right, text=col1shade2, font=\scriptsize] at (6,2) {$\A_\Z^1$};
		
		\draw[col1, thick]
			(0,0) -- (0,4)
			(2,0) -- (2,4)
			(3,0) -- (3,4)
			(3.5,0) -- (3.5,4)
			(4,0) -- (4,4)
			(4.5,0) -- (4.5,4);
		
		\draw[->,thick] 
			(3,-0.5) -- (3,-1.5);
		
		\draw[col1shade2, thick]
			(-0.5,-2) -- (6,-2)
			node[right, text=col1shade2]{$\Spec \Z$};
		\draw[col1, thick]
			(0,-1.8) -- +(0,-0.4)
			(3,-1.8) -- +(0,-0.4)
			(3.5,-1.8) -- +(0,-0.4)
			(4,-1.8) -- +(0,-0.4)
			(4.5,-1.8) -- +(0,-0.4);
		\draw[thick, col2]
			(2,-1.8) -- +(0,-0.4);
			
		\foreach \x in {0, 3}{	
			\foreach \y in {0.5,1,...,3}{
				\fill[col2] (\x,\y) circle[radius=2pt];
			}
		}
		\fill[col2] (4,3) circle[radius=2pt];
		\fill[col2] (4.5,2) circle[radius=2pt];
		\draw[col2] (2,0) -- (2,4);
		
		\node[right, col2, text width=3cm, font=\scriptsize] at (3.5,-0.5) 
			{irreduzible Teiler von $\bar f\in \bb F_p[X]$};
			
		\node[below, col2, text width=2cm, font=\scriptsize] at (0,0) 
			{irreduzible Faktoren von $f$};
			
		\node[below, col2, text width=4cm, font=\scriptsize] at (2,-2.5)
			{Primteiler aller Koeffizienten von $f$};
			
		\fill[col2] (7.5,3) circle[radius=3pt] 
			node[right] {\ $\notin D(f)$};
		\fill[col1] (7.5,2) circle[radius=3pt] 
			node[right] {\ $\in D(f)$}; 
	\end{tikzpicture}
\end{figure}


\subsection{Diskrete Bewertungsringe}

\begin{definition}[Diskrete Bewertung]
	\label{def:diskrete Bewertung}
	Eine \emph{diskrete Bewertung} auf einem Körper $k$ ist eine Abbildung
	\[ v: k\to \Z \cup \{\infty\}, \]
	so dass
	\begin{enumerate}
	  \item $v(0) = \infty$, $v(x)\in \Z$ für $x \neq 0$,
	  \item $v(xy) = v(x) + v(y)$ für alle $x,y$ und
	  \item $v(x+y) \geq \min\{v(x), v(y)\}$ für alle $x,y$.
	\end{enumerate}
\end{definition}

\begin{bemerkung}
	Wählt man $q > 1$ (in $\R$), so ist
	\[
		|\cdot |: k \to \R,\ x\mapsto |x| := q^{-v(x)}
	\]
	eine Betragsfunktion mit
	\begin{enumerate}
	  \item $|x| = 0 \ \Leftrightarrow\ x = 0$,
	  \item $|xy| = |x||y|$.
	  \item $|x+y| \leq \max \{ |x|,|y|\} \leq |x| + |y|$.
	   Die erste Ungleichung wird auch nicht-archimedische
	   Dreiecksungleichung genannt.
	\end{enumerate}
\end{bemerkung}


\begin{definition}[Bewertungsring]
	\label{def:bewertungsring}
	Ist $(k,v)$ ein diskret bewerteter Körper, so ist
	\[
		\O:= \{x\in k\mid v(x) \geq 0 \} = \{ x \in k \mid |x|\leq 1\}
	\]
	ein lokaler Ring mit maximalem Ideal
	\[
		\m := \{x\in k\mid v(x) > 0\} = \{ x \in k\mid |x|< 1\} \ideal \O,
	\]
	der \emph{Bewertungsring} zu $k$.
	
	Ein \emph{diskreter Bewertungsring (dvr)} ist ein Integritätsbereich $R$,
	zusammen mit diskreter Bewertung $v: K = \Quot(R) \to \Z \cup \{\infty\}$,
	so dass $R = \O$ gilt.
	
	Ferner gilt $\O$ ist ein Hauptidealbereich (PID), $k = \Quot(\O)$.
	
	Ist $\pi\in \O$ mit $v(\pi) = 1$, so ist 
	$\m = (\pi)$ und $\O$ hat genau die Ideale $(\pi^k)$ für $k\in \N_0$.
\end{definition}


\begin{bemerkung}
	Der Wertebereich $v(k\setminus \{0\}) \subseteq \Z$ ist eine Untergruppe,
	also $v(k\setminus \{0\}) = d \Z$ für ein $d$. 
	Wir können meistens \obda $d = 1$ annehmen.	
\end{bemerkung}

\begin{bemerkung}
	Beachte: Für $x\in \O$ gilt
	\[ v(x) = n \ \Leftrightarrow \ x \in \m^n \setminus \m^{n+1}. \]
	
	Für $\xi = \frac x y \in K = \Quot(\O)$ ist $v(\xi) = v(x) - v(y)$.
\end{bemerkung}


\begin{proof}[von \thref{def:bewertungsring}]
	TODO.
\end{proof}

\begin{bemerkung}
	\[\Spec \O = \{(0), (\pi) = \m \},\]
	da in Hauptidealbereichen jedes Primideal $\neq (0)$ auch maximal ist.
\end{bemerkung}

\begin{definition}
	Ist $\O$ ein diskreter Bewertungsring, so heißt 
	\[ \O\big/ \m =: k\]
	der \emph{Restklassenkörper} von $\O$.
	
	$\O$ heißt 
  \begin{itemize}
    \item von \emph{verschiedener Charakteristik}, wenn
      für $K = \Quot(\O)$, $\charak K = 0$ und $\charak k \neq 0$ ist und
    \item von \emph{gleicher Charakteristik}, wenn $\charak K = \charak k$.
  \end{itemize}
\end{definition}

\subsubsection{Beispiele}

\paragraph{1.} Sei $k$ ein Körper, 
\[K := k \dr t := \Quot k \dk t = \left\{f(t) = \sum_{l=-N}^\infty a_t t^l \mid 
	a_l \in k \right\}\]
und
\[
	v: \funcdef{ k\dk t &\to& \N_0 \cup\{\infty\} \\
		f(t) = \sum a_l t^l  & \mapsto & 
			\max\{ k\in \N_0 \mid t^{-k} f(t) \in k\dk t\}
			= \min\{ l\in \N_0 \mid a_l \neq 0 \}.}
\]
Auf $k\dr t$ Dies ist eine diskrete Bewertung mit $\O = k\dk t$:
\[
	v: \funcdef{ k\dr t &\to& \Z_0 \cup\{\infty\} \\
		f(t) = \sum a_l t^l  & \mapsto & 
			= \min\{ l\in \Z_0 \mid a_l \neq 0 \}.}
\]
$k\dr t$ trägt damit $|\cdot | := q^{-v(\cdot)}$, also ist $k\dr t$ ein
metrischer Raum mit $d(x,y) := |x-y|$, dieser ist vollständig.

Für den Restklassenkörper gilt
\[ \O \big/ \m = k\dk t \big/ t k \dr t \cong k, \]
da $\m = t k\dk t = (t)$. $t$ heißt dabei \emph{Uniformierende}.


\paragraph{2.}
Betrachte
\[ \nu_p: \funcdef{ \Q & \to & \Z \cup \{\infty\}\\
	\frac a b & \mapsto & v(a) - v(b)}
\]
mit $v(a) = \max\{ k:\ p^k \mid a \}$ für eine Primzahl $p$. 

$\nu_p$ ist eine diskrete Bewertung, die \emph{$p$-adische Bewertung}.
Ferner ist
\[
	\O = \left\{ \frac a b \in \Q \mid \nu_p\left(\frac a b\right) > 0
		\right\}
		=
		\left\{ \frac a b \in \Q \text{ in gekürzter Form} \mid 
			p\nmid b \right\}
		= \Z_{(p)}
\]
und $\m = p \Z_{(p)}$ und 
\[
	\O\big/ \m = \Z_{(p)} \big/ p\Z_{(p)} \cong
		\Z \big/ p\Z = \fr F_p.
\]

$|\cdot|_p := p^{-\nu_p(\cdot)}: \Q \to \R_{\geq 0}$ heißt
\emph{$p$-adischer Betrag}.
$(\Q, |\cdot|_p)$ ist jedoch nicht vollständig, da
z.B. $\sum_{n=0}^\infty p^n$ ein Cauchyfolge bildet.

Man erhält die Vervollständigungen
\begin{align*}
	(\Q, |\cdot|) &\leadsto \R\\
	(\Q, |\cdot|_p) & \leadsto \Q_p.
\end{align*}

\paragraph{Zurück zu Schemata}
Sei $\O$ ein dvr, so ist $\Spec \O = \{(0), (\pi)\}$. Dabei ist
$(0)$ der generische Punkt mit $\overline{\{(0)\}} = V((0)) = \Spec \O$
und $(\pi)$ ein abgeschlossener Punkt, genannt der \emph{spezielle Punkt}
in $\Spec \O$.

\begin{beispiel}
	Sei $k$ ein Körper mit $\charak k \neq 2,3$ und $k$ algebraisch 
	abgeschlossen. Wir betrachten
	\[
		E := \Spec A \quad \text{mit } 
			A := k[X,Y] \big/ (Y^2 - (X^3 + aX + b)).
	\]
	Dies ist der affine Teil einer \emph{elliptischen Kurve}, wenn
	$4a^3 + 27b^2 \neq 0 \in k$.
	
	Wir haben
	\[ |E| \cong \{ (x_0,y_0) \in k^2 \mid y_0^2 - (x_0^3 + ax_0 + b) = 0\}.
	\]
	Sei $(x_0,y_0) \in |E|$, oder besser $\p := (X-x_0, Y - y_0) \in E$.
	Es ist $\O_{E,\p}$ ein dvr.
	
	Dazu:
	\begin{description}
	\item[1. Fall $y_0 \neq 0$.], so ist
		$\O_{E,y} = A_\p$. Betrachten wir
		$\frac{\bar f(X,Y)}{\bar g(X,Y)} \in A_\p$, also
		$\bar f, \bar g \in A$ und $\bar g \notin (X-x_0, Y-y_0)$, d.h.
		$\bar g(x_0, y_0) \neq 0$. Ferner ist
		\[
			Y^2 - (X^3 + aX + b) = 
			(Y + y_0)(Y-y_0) + (X^2 x_0X + (x+x_0^2))(X-x_0)
		\]
		und wenn $y_0 \neq 0$, so ist
		$(Y+y_0) \notin (X-x_0, Y-y_0)$. Demnach ist
		$Y+y_0 \in A_\p^\times$, also gilt in $A_\p$:
		\[
			Y-y_0 = \frac{X^2 + x_0 X + (a+x_0^2)}{Y + y_0}(X-x_0)
		\]
		und
		$(X-x_0, Y-y_0) A_\p = \p A_\p = (X-x_0)A_\p$ ist 
		ein Hauptideal.
		
		Also ist
		\[
			v: \funcdef{ A_\p & \to & \N_0 \cup \{\infty\}\\
				a & \mapsto & \max\{k \in \N_0 \mid a \in (X-x_0)^k \}}
		\]
		eine diskrete Bewertung!
	\item[2. Fall $y_0 = 0$.] Dies geht analog und man sieht, dass
		\[
			X^2 + x_0 X + (a + x_0^2) \notin (X-x_0,Y),
		\]
		da nach Voraussetzung $4a^2 + 27b^2 \neq 0$. Also ist
		$\p A_\p = (Y - y_0) A_\p$.
	\end{description} 
\end{beispiel}

\begin{bemerkung}
	Sei $K(E) := \O_{E,(0)} = \Quot(A) = A_{(0)}$ der 
	\emph{Funktionenkörper} von $E$.
	Für $\p \in E$ hat man die \emph{Null-/Polstellenordnung}
	\[
		v_\p: K(E) \to \Quot(A_\p) = \Quot(\O_{E,\p}) \xto v \Z\cup\{\infty\}.
	\]
\end{bemerkung}
\pagebreak

% vim: set ft=tex :
