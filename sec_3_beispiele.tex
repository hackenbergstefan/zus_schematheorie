\section{Beispiele}

\subsection{$\Spec \Z$}

Jeder Ring $A$ hat einen eindeutigen Homomorphismus
\[
	\funcdef{
		\Z & \to & A\\
		 1 & \mapsto & 1\\
		 z & \mapsto & \begin{cases} 1 + 1 + \ldots + 1 & z > 0\\
		 	0 & z = 0\\
		 	-1 -1 - \ldots- 1 & z < 0
		 \end{cases}.}
\]
$\Z$ ist daher ein \emph{initiales Objekt} in der Kategorie $\Ring$.

Wir haben daher einen eindeutigen Morphismus $\Spec A \to \Spec \Z$ von
affinen Schemata. $\Spec \Z$ ist ein \emph{finales Objekt} in
der Kategorie $\affSch$.

Ferner können wir zusammenfassen
\paragraph{Offene Mengen}
	$\emptyset \neq U\subseteq \Spec \Z$ offen 
	$\Leftrightarrow$ $U = \Spec \Z \setminus \{(p_1),\ldots,(p_r)\}$
	
\paragraph{Basisoffene Mengen}
	$D(f) = \{\p \in \Spec\Z \mid f \notin \p\} = 
	\Spec\Z \setminus \{(p_1),\ldots,(p_r)\}$ für 
	$f = p_1^{\nu_1}\ldots p_r^{\nu_r}$.
	
\paragraph{Strukturgarbe}
	\begin{align*}
		\O_{\Spec\Z} (D(f)) &= \Z_f  = 
			\left\{ \frac{a}{f^n} \mid n\in \N_0, a\in \Z\right\} \\
		\O_{\Spec\Z, (p)} &= \Z_{(p)} = 
			\left\{ \frac{a}{b} \mid p\nmid b, a\in \Z\right\} 
 	\end{align*}

\subsection{$\Spec k$ für einen Körper $k$}
\paragraph{Als topologischer Raum}
	$\Spec k = \{(0)\}$.

\paragraph{Strukturgarbe}
	$\O_{\Spec k}(\{(0)\}) = k$.

\begin{bemerkung}
   Sei $A$ ein Ring. Angenommen wir haben 
  	$\Spec A \xto{(f,f\fis)} \Spec k$ für einen Körper $k$, so haben wir
  	\[
  		f\fis_{\Spec k}: k = \O_{\Spec k} \to f_\ast\O_{\Spec A}(\Spec k)
  			\tikzmark{=} A,
  	\]
  	wobei \tikzarrow{mark above}{} aus 
  	$\O_{\Spec A}(f\inv(\{(0)\})) = \O_{\Spec A}(\Spec A)$ resultiert.
  	Insgesamt ist $A$ also eine $k$-Algebra (d.h. ein Ring zusammen mit
  	$k\to A$).
  	
  	Bemerke hierbei "`Grothendiecks Gesamtphilosophie"':
  	\begin{quote}\itshape
  		Alles relativ lesen!
  	\end{quote}
\end{bemerkung}

\begin{definition}[$S$-Schema]
	Sei $S$ ein Schema. Dann ist ein \emph{$S$-Schema} ein Schema $X$
	zusammen mit einem Strukturmorphismus $X \xto{\varphi} S$.
	Dies ergibt die Kategorie $\Sch_S$, wenn man
	\[
		\Hom( X\xto{\varphi}S, Y\xto{\varphi} S) := 
		\left\{ 
		\begin{tikzcd}
		X \arrow{rr}{f} \drar{\varphi} & & Y \dlar{\psi} \\ & S &
		\end{tikzcd}
		\right\}
	\]
	setzt.
\end{definition}

\begin{beispiel}
	$\Sch_k := \Sch_{\Spec k}$ sind die sog. \emph{$k$-Schemata}.
	Ein Beispiel hierfür ist
	$\Spec k[X_1,\ldots,X_n] \to \Spec k$ via 
	$k \hookrightarrow k[X_1,\ldots,X_n]$.
\end{beispiel}


\begin{bemerkung}
	Sei $X$ ein Schema und $x\in X$ und weiter $\m_x \ideal \O_{X,x}$ das
	maximale Ideal.
	Dann ist 
	\[
		\kappa(x) := k(x) := \O_{X,x} \big/ \m_x
	\]
	der \emph{Restklassenkörper von $x$}.
	
	Betrachte nun $(f,f\fis): \Spec k \to X$ mit
	\[
		f: \funcdef{\Spec k(x) & \to & X \\
			\eta_x & \mapsto & x,}
	\]
	wobei topologisch gesehen $\eta_x \in \Spec k(x)$ der einzige Punkt 
	dieses Schemas ist.
	Für $U\subseteq X$ offen haben wir:
	\[
		f\fis_U : \O_X \to 
			f_\ast \O_{\Spec k(x)}(U) = 
			\begin{cases} 0 & x\notin U \\ k(x) & x \in U. \end{cases}
	\]
	Im Fall $x \in U$ geht dies via
	\[
		\O_X(U) \to \O_{X,x} = \varinjlim_{x\in V} \O_X(V)
			\overset\pi\twoheadrightarrow  \O_{X,x}\big/ \m_x = k(x). 
	\]
	
	Ist umgekehrt $(f,f\fis):\Spec k \to X$ ein Schemamorphismus, so
	setze $x := f((0)) \in X$ und
	$f\fis: \O_X \to f_\ast \O_{\Spec k}$ liefert einen Ringhomomorphismus der
	Halme:
	\[
		f_x\fis: \O_{X,x} \to \O_{\Spec k, (0)} = k.
	\]
	Dieser ist lokal (also $f\fis_x (\m_x) = (0)$). Damit ist
	\[
		\begin{tikzcd}
		k(x) = \O_{X,x} \big/ \m_x \rar[hookrightarrow]{f_x\fis \mod \m_x} 
		&[7ex]  {f_x\fis \mod \m_x} k
		\end{tikzcd}
	\]
	wohldefiniert und somit ist $k \mid k(x)$ eine Körpererweiterung.
	
	Zusammengefasst haben wir:
	\[	\fbox{\parbox{5cm}{
			Einen Punkt $x\in X$ wählen mit Restklassenkörper
			$k(x)$ und eine Körpererweiterung $k\mid k(x)$.}}
		\Longleftrightarrow
		\fbox{\parbox{5cm}{
			Einen Schemamorphismus $\Spec k \to X$ wählen
			für eine Körpererweiterung $k\mid k(x)$.}}
	\]
\end{bemerkung}

\subsection{Der Affine $n$-dimensionale Raum über $k$}
Sei $k$ wieder ein Körper. Der affine $n$-dimensionale Raum über $k$ ist
$\A_k^n := \Spec k[X_1,\ldots, X_n]$.

Wir erinnern an den Hilbertschen Nullstellensatz:
\begin{satz}[Hilbertscher Nullstellensatz]
	\label{satz:hilbertscher nullstellensatz}
	Sei $k$ algebraisch abgeschlossen. Dann ist jedes maximale Ideal
	in $k[X_1,\ldots, X_n]$ von der Form
	$(X_1-a_1, \ldots, X_n - a_n)$.
\end{satz}
\begin{proof}
	ohne Beweis.
\end{proof}

Wir haben bereits gezeigt:
\[
	|\A_k^n| = k^n, \qquad\text{via } 
		(X_1-a_1,\ldots,X_n-a_n) \mapsto (a_1,\ldots,a_n).
\]
Sei $\p = (f_1, \ldots, f_r)$ ein nicht maximales Ideal in $k[X_1,\ldots,X_n]$
(die Darstellung ist nach \thref{satz:hilbertscher nullstellensatz}) möglich,
so gilt
\[
	\p \subseteq (X_1 - a_1, \ldots, X_n - a_n)
	\quad\Leftrightarrow\quad
	f_1(a_1,\ldots,a_n) = 0, \ldots,
	f_r(a_1,\ldots,a_n) = 0
\]
Wir können dies in \autoref{fig:spec k xy 2} "`sehen"'.

\begin{figure}\centering
	\caption{$\Spec k[X_1,\ldots,X_n]$}
	\label{fig:spec k xy 2}
	\begin{tikzpicture}
		\draw[very thick] 
			(-3,0) -- (3,0) node[near end, above] {$X_1$}
			(0,-2) -- (0,2) node[near end, right] {$X_2$};
		
% 		\draw[col1,thick] 
% 			(-3,1) to[out=-5, in=135]  (0.2,-0.2) 
% 			to[out=-45, in=225, looseness=2] (-1,0) 
% 			to[out=45, in=180, looseness=0.5] (3,1.5)
% 			node[pos=0.9] {$f(X,Y) = 0$};
		\draw[col1, thick]
			(-3,1) 
			.. controls (5,-2) and (-8,-2) .. 
			(3,1.5)
			node[pos=1, right, text width=2.9cm, font=\scriptsize] 
				{$\{(a_1,\ldots,a_n) \mid f_j(a_1,\ldots,a_n) = 0,$\\ 
					$j=1\ldots r\}$}
			coordinate[pos=0.05] (a);
		
		\fill[col1shade2] (a) circle[radius=2pt]
			node[above right, col1] {$(\alpha,\beta)$};
		
		\path (-3,1)
			node[generic point=10pt, fill=col1shade2] {}
			node[above left, col1] {$\p$};
	\end{tikzpicture}
\end{figure}

\section{}


\pagebreak

% vim: set ft=tex :