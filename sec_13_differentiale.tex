\section{Differentiale} %Seite 139

\begin{defn}
Eine \emph{$A$-Derivation von $B$ nach $M$} ist eine Abb
\[
  D:B\rightarrow M
\]
die
\begin{itemize}
\item $A$-linear ist
\item die Leibnitz-Regel $D(b\cdot\tilde b)=bD(\tilde b)+\tilde bD(b)$
erfüllt
\end{itemize}
\end{defn}

\begin{defn}
Der \emph{$B$-Modul der $1$-Formen} $\Omega_{B\mid A}^1$ ist ein $B$-Modul
zusammen mit einer Derivation
\[
  d:B\rightarrow \Omega_{B\mid A}^1
\]
die universell unter allen $A$-Derivationen von $B$ nach $M$ ist.
	\[\begin{tikzcd}
		d:B \rar{\text{univ.}} \arrow{dr}{\text{Deriv.}} & \Omega_{B\mid A}^1 \dar{\exists!} \\
		                                                 & M
	\end{tikzcd}\]
\end{defn}
\begin{bem}
Auch bezeichnet als \emph{Kähler-1-Formen}
\end{bem}

\paragraph{Nachtrag}\\
\begin{thm}[Thm B von Serre]
$X$ affin, $\cF$ qua.koh. Modulgarbe $\Rightarrow$ $H^q(X,\cF)=0$ für $q\geq 1$
\end{thm}
