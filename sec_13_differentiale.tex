\section{Differentiale} %Seite 139

\begin{definition}
Eine \emph{$A$-Derivation von $B$ nach $M$} ist eine Abb
\[
  D:B\rightarrow M
\]
die
\begin{itemize}
\item $A$-linear ist
\item die Leibnitz-Regel $D(b\cdot\tilde b)=bD(\tilde b)+\tilde bD(b)$
erfüllt
\end{itemize}
\end{definition}

\begin{definition}
Der \emph{$B$-Modul der $1$-Formen} $\Omega_{B\mid A}^1$ ist ein $B$-Modul
zusammen mit einer Derivation
\[
  d:B\rightarrow \Omega_{B\mid A}^1
\]
die universell unter allen $A$-Derivationen von $B$ nach $M$ ist.
	\[\begin{tikzcd}
		d:B \rar{\text{univ.}} \arrow{dr}{\text{Deriv.}} & \Omega_{B\mid A}^1 \dar{\exists!} \\
		                                                 & M
	\end{tikzcd}\]
\end{definition}
\begin{bemerkung}
Auch bezeichnet als \emph{Kähler-1-Formen}
\end{bemerkung}

\begin{einschub}{Nachtrag}
\begin{satz}[satz B von Serre]
$X$ affin, $\cF$ qua.koh. Modulgarbe $\Rightarrow$ $H^q(X,\cF)=0$ für $q\geq 1$
\end{satz}
\TODO
\end{einschub}
\paragraph{Kähler-Differentialformen}
$A\rightarrow B$, $M$ ein $B$-Modul:
\[
Der_A(M):=\{
D:B\rightarrow M\mid A\text{ linear}, D(b\tilde b)=D(b)\tilde b + bD(\tilde b)
\}
\]
\begin{definition}
\emph{Kähler-1-Formen} $:=$ univ. Derivationen $=:$ $(\Omega_{B|A}^1,d)$
	\[\begin{tikzcd}
		d:B \rar{\text{d}} \arrow{dr}{\text{D}} & \Omega_{B\mid A}^1
    \dar{\exists! \text{$B$-lin}} \\
		                                                 & M
	\end{tikzcd}\]
\end{definition}
\begin{satz}
$(\Omega_{B|A}^1,d)$ existiert immer.
\end{satz}
\begin{bemerkung}
ist $D$ eine $A$-Derivation, so ist:
\begin{align*}
D(a)=D(a\cdot 1)=a D(1) && \text{sowie} && D(a)=1D(a)+aD(1)
\end{align*}
und damit $D(a)=0$
\end{bemerkung}
\begin{lemma}
ist $B=A[X_i]_{i\in\{1,\dots,n\}}$, dann ist
$\Omega_{B|A}^1=\bigoplus_{i=1}^nB\cdot dX_i$
\end{lemma}
Betrachte $\mu:B\otimes_A B\rightarrow B,b\otimes b'\mapsto bb'$, welches
$B$-linear ist. $B\otimes_A B$ ein $B$-Modul (links-Multiplikation).\\
Sei $I:=\ker\mu\vartriangleleft B\otimes_A B$ und $I^2\vartriangleleft
B\otimes_A B$.\\
Wir haben damit: $d:B\rightarrow I/I^2,b\mapsto 1\otimes b-b\otimes 1\mod I^2$.
\begin{bemerkung}
$d$ ist eine $A$-Derivation. (Beweis leicht)
\end{bemerkung}
\begin{satz}
$(I/I^2,d)\cong(\Omega_{B|A}^1,d)$
\end{satz}
\begin{satz}{Erste und zweite Fundamentale-Sequenz}
  \begin{itemize}
    \item \textbf{1. fund-Sequenz:}
      für $A\rightarrow B\rightarrow C$ ex. Sequenz von $C$-Moduln
      \[\begin{tikzcd}
        \Omega_{B|A}^1\otimes_BC\rar{\alpha} & \Omega_{C|A}^1 \rar{\beta}
          & \Omega_{C|B}^1 \rar & 0\\
          db\otimes c \rar[mapsto] & c\cdot db\\
                                   & dc \rar[mapsto] & dc
      \end{tikzcd}\]
    \item \textbf{2. fund-Sequenz:} Ist $J\vartriangleleft B$, so hat man die
    ex. Sequenz von $C:=B/J$-Moduln
      \[\begin{tikzcd}
        J/J^2 \rar{\delta} & \Omega_{B|A}^1\otimes_BC\rar{\alpha} 
          & \Omega_{C|A}^1 \rar & 0\\
          b\mod J^2 \rar[mapsto] & db\otimes 1 \\
      \end{tikzcd}\]
  \end{itemize}
\end{satz}
\begin{lemma}
Ist $A\rightarrow B$ und $S\subset B$ multiplikative Teilmenge in $B$, so gilt
\[
\Omega_{S^{-1}B|A}^1=S^{-1}\Omega_{B|A}^1
\]
(Beweis leicht)
\end{lemma}
Ü.Aufg: $L=K(\alpha)|K$ und damit: $\Omega_{L|K}^1=\begin{cases}
L\cdot d\alpha & \text{$\alpha$ transzendent}\\
0 & \text{$\alpha$ alg. und sep.}\\
L\cdot d\alpha & \text{$\alpha$ alg. und nicht sep.}\\
\end{cases}$\\
Idee: \TODO
\begin{bemerkung}
$X\rightarrow \spec k$ von endlichem Typ:
\[
X\text{ glatt}\Leftrightarrow\Omega_{X|k}^1\text{ lokal frei von Rang $\dim X$}
\]
\end{bemerkung}
