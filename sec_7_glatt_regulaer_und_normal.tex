\section{Glatt, regulär \& normal} %Seite 125

\begin{definition}[Zariski-Tangentialraum]
    \index[def]{Schema!Zariski-Tangentialraum}
    Der \emph{Zariski-Tangentialraum von $X$ bei $x_0$} ist
    \[T_{x_0} X := \Hom_{k(x_0)}(\m\big/\m^2, k(x_0)).\]
\end{definition}

\subsection{Dimensionsbegriff}

\begin{definition}[Krull-Dimension, lokale Dimension, Kodimension]
\index[def]{Krull-Dimension}
\index[def]{Krull-Dimension!lokale Dimension}
\index[def]{Krull-Dimension!Kodimension}
\begin{enumerate}[label=(\roman*)]
  \item Sei $X$ ein topologischer Raum. Die \emph{Krull-Dimension von $X$} ist
    \[\dim X := \sup\{n\in\N_0 \mid
        \exists Z_0 \subsetneq Z_1 \subsetneq \ldots \subsetneq Z_n
        \text{ von irreduziblen Teilmengen von $X$}\}\]
  \item Sei $X$ ein topologischer Raum. $x_0 \in X$. Die 
    \emph{lokale Dimension bei $x_0$} ist
    \[\dim_{x_0} X := \inf\{\dim U \mid x_0\in U \osubset X\}.\]
  \item Sei $X$ ein topologischer Raum. $Y\subset X$ irreduzibel und
    abgeschlossen. 
    \[\codim(Y,X) := \sup\{n\in\N_0 \mid
        \exists Y = Z_0 \subsetneq Z_1 \subsetneq \ldots \subsetneq Z_n
        \text{ von irreduziblen Teilmengen von $X$}\}\]
    heißt \emph{Kodimension von $Y$ in $X$}.
  \item Sei $X$ ein topologischer Raum. Sei $Y\subset X$ abgeschlossen.
    \[\codim(Y,X) := \inf\{ \codim(Z,X) \mid Z\subset Y 
        \text{ abgeschlossen, irreduzibel}\}\]
    heißt \emph{Kodimension von $Y$ in $X$}. 
\end{enumerate}
\end{definition}

\begin{bemerkung}
    Speziell für $X = \Spec A$ hat man
    für $\p \in \Spec A$
    \[\dim_\p A := \inf \{\dim A_f \mid f \in A, f\notin\p\}.\]
\end{bemerkung}

\begin{lemma}
    Sei $X$ ein topologischer Raum. 
    Ist $Z\subseteq X$ abgeschlossener Teilraum. Dann gilt
    \[\codim(Z,X) + \dim Z \leq \dim X.\]
\end{lemma}
\begin{proof}
\TODO
\end{proof}

\begin{definition}[Höhe, Krull-Dimension von Ringen]
    \index[def]{Ring!Höhe eines (Prim-)Ideals}
    \index[def]{Ring!Krull-Dimension}
    Sei $A$ ein Ring.
    \begin{enumerate}[label=(\roman*)]
      \item Für $\p \in \Spec A$ heißt
        \[ \height(\p) := \sup\{n \in \N_0 \mid 
            \exists \p_0 \subsetneq \p_1 \subsetneq \ldots \subsetneq \p_n = \p,
             \p_i \ideal A \text{ Primideale}\}\]
        die \emph{Höhe von $\p$}.
      \item Für $\a \ideal A$ heißt 
        \[\height(\a) := \inf\{\height(\p) \mid \a \subseteq \p \in \Spec A\}\]
        die \emph{Höhe von $\a$}.
      \item Die \emph{Krull-Dimension von $A$} ist
        \begin{align*}
            \dim A &:=\sup\{ n\in \N_0 \mid \exists \p_0 \subsetneq \ldots 
                \subsetneq \p_n,\ \p_i\in\Spec A\}\\
                &= \sup\{\height(\p)\mid \p \in \Spec A\}
        \end{align*} 
    \end{enumerate}
\end{definition}

\begin{satz}
    Für einen Ring $A$ und $\p \in \Spec A$ gilt:
    \begin{align*}
        \dim A &= \dim\Spec A\\
        \codim(V(\p), \Spec A) &= \height(\p)
    \end{align*}
\end{satz}
\begin{proof}
    Klar, Da $Z\subsetneq \Spec A$ irreduzibel abgeschlossen, genau dann, wenn
    $ Z = V(\p)$.
\end{proof}

\begin{lemma}
    \begin{enumerate}[label=(\roman*)]
      \item Für $\p\in\Spec A$ gilt
        \[\height(\p) = \dim A_\p.\]
      \item 
        \[\dim A = \sup\{ \dim A_\m \mid \m \ideal A\text{ maximal}\}.\] 
    \end{enumerate}
\end{lemma}
\begin{proof}
\TODO
\end{proof}


\begin{bemerkung}
    Ist $A$ nullteilerfrei, so beginnt eine aufsteigende Kette von Primidealen
    bei $\p_0 = (0)$. Hat $A$ Nullteiler, so ist $(0)$ kein Primideal. 
\end{bemerkung}

\begin{beispiel}
    \begin{itemize}
      \item Ist $k$ ein Körper, so ist $\dim k = 0$.
      \item Ist $A$ ein nullteilerfreier Hauptidealring, so ist 
        $\dim A = 1$.
      \item Ist $K \mid \Q$ ein Zahlkörper, $\O_K \subseteq K$ der Ring
        der ganzen Zahlen, so ist
        $\dim \O_K = 1$.
    \end{itemize}
\end{beispiel}

\begin{bemerkung}
    In der Tat gilt für einen nullteilerfreien Ring $A$:
    \[ \dim A = 1 \quad\Leftrightarrow\quad
        \text{Jedes Primideal } \p\neq (0) \text{ ist maximal}.\]
\end{bemerkung}

\begin{satz}[Krulls-Hauptidealsatz]
    Sei $A$ noethersch und $f\in A \setminus A^\times$. Ferner sei
    $\p\in\Spec A$ mit $f\in \p$ und $\p$ minimal mit dieser Eigenschaft.
    Dann gilt
    \[\height(\p) \leq 1.\]
\end{satz}
\begin{proof}
\TODO
\end{proof}

Zum Beweis benötigt man:

\begin{lemma}[Nakajomas Lemma]
    Sei $(A,\m)$ ein lokaler Ring und $M$ ein endlich erzeugter $A$-Modul
    mit $M = \m M$. Dann gilt
    \[M = (0)\] 
\end{lemma}
\begin{proof}
\TODO
\end{proof}

\begin{korollar}
    Sei $(A,\m)$ ein lokaler Ring und $M$ ein endlich erzeugter $A$-Modul.
    Sei $N\subseteq M$ ein $A$-Untermodul mit $M \subseteq N + \m M$.
    Dann gilt
    \[N = M\]
\end{korollar}
\begin{proof}
    Wende Nakajomas Lemma auf $M/N$ an.
\end{proof}

\begin{lemma}
    Sei $A$ noethersch und nullteilerfrei und $M$ ein endlich erzeugter
    $A$-Modul. Ist ferner $\q \ideal A$ ein echtes Ideal, so gilt
    \[\bigcap_{n\in\N} \q^n M = 0.\]
\end{lemma}

Ein Beispiel zu Krulls-Hauptidealsatz:

\begin{beispiel}
    Sei $A = k[X_1,\ldots,X_n]$, $f  = f(X_1,\ldots,X_n)$ und
    $\p$ wie in Krulls-Hauptidealsatz, so ist $V((f)) \supseteq V(\p)$, 
    d.h. $V(\p)$ ist maximal unter den abgeschlossenen Teilmengen von $V((f))$. 
\end{beispiel}


Ein Korollar zu Krulls-Hauptidealsatz:
\begin{korollar}
    Sei $A$ noethersch, $f\in A$ und
    \[\p_0 \subsetneq \p_1 \subsetneq \ldots \subsetneq \p_n = \p\]
    eine Primidealkette mit $f \in \p$. Dann existiert eine Primidealkette
    \[\q_1 \subsetneq \q_2 \subsetneq \ldots \subsetneq \q_n = \p\]
    mit $f \in \q_1$.
\end{korollar}
\begin{proof}
\TODO
\end{proof}

\begin{korollar}
    Sei $A$ noethersch, $\a = (a_1,\ldots,a_r) \ideal A$. Dann gilt:
    Ist $\p \in \Spec A$ minimal mit $a\subseteq \p$, so ist
    \[\height(\p) \leq r.\]
    Insbesondere ist also
    \[\height(\p) \leq r.\]
\end{korollar}
\begin{proof}
\TODO
\end{proof}

\begin{korollar}
    Ist $(A,\m)$ ein noetherscher, lokaler Ring, so ist $\dim A < \infty$ und
    \[\dim A \leq \dim_{A/\m} \m\big/\m^2.\]
\end{korollar}
\begin{proof}
\TODO
\end{proof}

\begin{bemerkung}
    Erinnern wir an den Tangentialraum, so haben wir in obigem Fall
    \[T_\m\Spec A = (\m\big/\m^2)^\vee \]
    wobei $^\vee$ den Dualraum als $A/\m$-Vektorraum meint.
\end{bemerkung}

\begin{folgerung}
    Ist $X$ ein lokal, noethersches Schema und $x\in X$. Dann gilt
    \[\dim \O_{X,x} \leq \dim_{k(x)} T_x X.\]
\end{folgerung}


\subsection{Regularität}

\begin{definition}[regulär]
    \label{def:regular}
    \index[def]{Schema!regulär}
    \index[def]{Ring!lokal!regulär}
    \begin{enumerate}[label=(\roman*)]
      \item Ein lokaler noetherscher Ring $(A,\m)$ heißt \emph{regulär},
        wenn $\dim A = \dim_{A/\m} \m\big/\m^2$.
      \item Ein lokal noethersches Schema $X$ heißt 
        \emph{regulär bei $x \in X$}, wenn $\O_{X,x}$ regulär ist. 
    \end{enumerate}
\end{definition}


\begin{lemma}
    Sei $(A,\m)$ ein noetherscher, lokaler Ring. Dann gilt
    \[A \text{ regulär} \quad\Leftrightarrow\quad
        \m \text{ wird von } \dim A \text{-vielen Elementen erzeugt}.\]
\end{lemma}
\begin{proof}
\TODO
\end{proof}

\begin{bemerkung}
    Im $C^\infty$ Fall haben wir
    \[ C^\infty(\R^n)_0 = \{[f]\mid f: U \ni 0 \xto{C^\infty} \R^n\}\]
    mit dem maximalen Ideal
    \[ \m = \{[f] \mid f(0) = 0\}.\]
    Dann kann man $g\in \m$ darstellen durch
    \[ g(y) = \sum_{i=1}^n g_i(y) y_i\]
    mit $y_i:(x_1,\ldots,x_n) \mapsto x_i$.
\end{bemerkung}

\begin{satz}
    Sei $(A,\m)$ ein lokaler noetherscher Ring und $f\in \m$. Dann gilt:
    \begin{enumerate}[label=(\roman*)]
      \item $\dim A\big/(f) \geq \dim A -1$.
      \item Ist $f$ nicht in einem minimalen Primideal enthalten, so gilt
        \[\dim A\big/(f) = \dim A -1. \] 
    \end{enumerate}
\end{satz}
\begin{proof}
\TODO
\end{proof}

\begin{bemerkung}
    Ist in obigem Fall $A$ nullteilerfrei, so gilt insbesondere
    \[f \text{ nicht in einem Primideal enthalten }\quad\Leftrightarrow\quad 
        f \neq 0\]
\end{bemerkung}

\begin{lemma}
    Sei $(A,\m)$ ein lokaler noetherscher Ring und $\n \ideal A[T]$ maximal
    mit $\n\cap A = \m$, so gilt
    \[ \height^{A[T]} (\n) = \dim A + 1\]
\end{lemma}
\begin{proof}
\TODO
\end{proof}

\begin{korollar}
    Ist $A$ noetherscher Ring, so folgt 
    \[\dim A[T_1,\ldots,T_n] = \dim A + n.\]
\end{korollar}
\begin{proof}
\TODO
\end{proof}

\begin{korollar}
    Es gilt
    \[\dim \A^n_k = \dim k[T_1,\ldots,T_n] = n.\]
\end{korollar}

\begin{satz}
    Jeder regulärer, lokaler, noetherscher Ring ist nullteilerfrei.
\end{satz}
\begin{proof}
\TODO
\end{proof}

\begin{definition}[Koordinatensystem]
    \label{def:koordinatensystem}
    \index[def]{Ring!lokal!Koordinatensystem}
    \index[def]{Ring!regulär!Koordinatensystem}
    Ist $(A,\m)$ ein regulärer, lokaler, noetherscher Ring und $\dim A = d$,
    dann nennt man ein Erzeugendensystem $\m = (x_1,\ldots,x_d)$
    ein \emph{Koordinatensystem von $(A,\m)$} oder
    \emph{System von Parametern}
\end{definition}

\begin{satz}
    Sei $k$ ein Körper und 
    \[X = \Spec k[X_1,\ldots,X_n]\big/\a \hookrightarrow \A^n_k\]
    mit $\a = (f_1,\ldots,f_r)$, so gilt:
    \[ X \text{ regulär bei }x\in X(k) \quad\Leftrightarrow\quad
        \rk\left(\left.\frac{\partial f_i}{\partial x_j}\right|_a\right) = 
        n - \dim \O_{X,x}\]
    wobei $x = (X_1 - a_1,\ldots,X_n-a_n)$ und $a = (a_1,\ldots,a_n)$ ist.
\end{satz}
\begin{proof}
\TODO
\end{proof}

\begin{bemerkung}
    In obiger Situation gilt:
    \begin{align*}
        x \in V(\a) \quad&\Leftrightarrow\quad (X_1-a_1,\ldots,X_n-a_n) 
            \in V(\a)\\
        &\Leftrightarrow\quad
        f_i(a_1,\ldots,a_n) = 0\ \forall i=1,\ldots,r.
    \end{align*}
\end{bemerkung}

\subsection{Glattheit}

\begin{definition}[glatt]
    Eine $k$-Varietät $X \in \Var_k$ heißt \emph{glatt bei $x\in X$}, wenn
    die Punkte $\bar x \in X_{\bar k}$ regulär sind.
\end{definition}

\begin{bemerkung}
    In obiger Situation ist dabei $\bar k \mid k$ ein algebraischer Abschluss
    und 
    \[\begin{tikzcd}
        \bar x \rar[mapsto] 
        & x\\[-15pt]
        X_{\bar k} := X \times_{\Spec k} \Spec \bar k \rar \dar 
        & X \dar
        \\
        \Spec \bar k \rar
        & \Spec k
    \end{tikzcd}\]
    $\bar x$ ist dabei nicht eindeutig.
\end{bemerkung}




% vim: set ft=tex :
