\section{$k$-Varietät} %Seite 154

\begin{beispiel}
    Ein einführendes Beispiel einer $k$-Varietät ist gegeben durch
    abgeschlossene Unterschemata, wie beispielsweise
    \[ \Spec k[X_1,\ldots,X_n] \big/\a \to \Spec k[X_1,\ldots,X_n] = \A^n_k.\]
\end{beispiel}


\begin{definition}[endlich]
    \label{def:ringhom endlich}
    \index[def]{Ringhomormophismus!endlich}
    Ein Ringhomomorphismus $\varphi: B \to A$ heißt \emph{endlich},
    wenn $A$ dadurch zu einem endlich erzeugten $B$-Modul wird.
\end{definition}

\begin{satz}[Noether-Normalisierung]
    Sei $A$ eine endlich erzeugte $k$-Algebr. Dann existiert $d\geq 0$ und
    ein endlicher injektiver Ringhomomorphismus
    \[k[T_1,\ldots,T_n] \hookrightarrow A.\]
\end{satz}
\begin{proof}
\TODO
\end{proof}

\begin{korollar}
    Ist $A$ eine endlich erzeugte $k$-Algebra und $\m \ideal A$ maximal,
    dann ist $A\big/\m$ eine endliche Körpererweiterung von $k$.
\end{korollar}
\begin{proof}
\TODO
\end{proof}

\begin{korollar}
    Sei $X$ eine $k$-Varietät und $x\in X$ ein abgeschlossener Punkt, so ist
    $k(x)\mid k$ endlich.
\end{korollar}
\begin{proof}
klar.
\end{proof}

\begin{satz}[(schwacher) Hilbertscher Nullstellensatz]
    Sei $k$ algebraisch abgeschlossen, $\m\ideal k[X_1,\ldots,X_n]$ ein
    maximales Ideal, so gilt
    \[\m = (X_1-a_1,\ldots,X_n-a_n)\]
    für geeignete $a_1,\ldots,a_n \in k$.
\end{satz}
\begin{proof}
\TODO
\end{proof}


\begin{lemma}
    Sei $X$ eine irreduzible algebraische $k$-Varietät, dann gilt für
    $x\in |X|$:
    \[\dim \O_{X,x} = \dim X.\]
\end{lemma}
\begin{proof}
\TODO
\end{proof}

\begin{lemma}
    Ist $\m \ideal k[X_1,\ldots,X_n]$ ein maximales Ideal, so existieren
    Polynome
    \[f_1(X_1), f_2(X_1,X_2), \ldots, f_n(X_1,\ldots,X_n)\]
    mit
    \[\m = (f_1,\ldots,f_r).\]
\end{lemma}
\begin{proof}
Induktion mit Noethernormalisierung.
\end{proof}

\begin{folgerung}
    $\A^n_k$ ist regulär bei allen $x \in |\A^n_k|$.
\end{folgerung}


% vim: set ft=tex :
