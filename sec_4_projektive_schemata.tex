\section{Projektive Schemata}

\subsection{Eine kurze Einführung in klassische projektive Geometrie}
\ldots

\subsection{$\P^n(k)$ als Schema}
Statt einem Körper $k$ können wir einen Ring $A$ betrachten.

\subsubsection{1. Variante}
Betrachte $U_i := \Spec A[x_0,\ldots,\cancel i,\ldots,x_n] = \A_A^n$.

In $\R\P^n$ würden wir diese mit dem Kartenwechsel verkleben:
\[\everymath{\displaystyle}\begin{tikzcd}
	& [] [y_0,\ldots,\underset{i\text{-te}}{1},\ldots,y_n] & U_i \cap U_j & \\
	(y_0,\ldots,\cancel i,\ldots,y_n) \urar[mapsto] & h_i(U_i \cap U_j)
		\ar{rr}{\text{Kartenwechsel}} \urar 
		&& h_j(U_i \cap U_j) \ular \\
	&\{(y_0,\ldots,\cancel i,\ldots,y_n) \mid y_j \neq 0\} \ar{rr}
	\uar[empty]{\rotatebox{90}{=}}
	&&
	\{(z_0,\ldots,\cancel j,\ldots,z_n) \mid z_i \neq 0\}
	\uar[empty]{\rotatebox{90}{=}}\\
	& (y_0,\ldots,\cancel i,\ldots,y_n) \ar[mapsto]{rr} &&
		\left(\frac{y_0}{y_j},\ldots,\underset{i\text{-te}}{\frac{1}{y_j}},
		\ldots,\cancel j,\ldots,\frac{y_n}{y_j}\right)
\end{tikzcd}\]
Betrachte also
\begin{align*}
	U_{ij} := \Spec A[x_0,\ldots,\cancel i,\ldots,x_n][x_j\inv]
		&\hookrightarrow \Spec A[x_0,\ldots,\cancel i,\ldots,x_n] = U_i\\
	U_{ji} := \Spec A[x_0,\ldots,\cancel j,\ldots,x_n][x_i\inv]
		&\hookrightarrow \Spec A[x_0,\ldots,\cancel j,\ldots,x_n] = U_j\\
\end{align*}
und wähle einen Isomorphismus
\[
	\phi_{ij}: \funcdef{U_{ij} & \to & U_{ji}\\
		x_k &\mapsto& \frac{x_k}{x_j} \quad \text{für $k\neq i$}\\
		x_i & \mapsto& \frac{1}{x_j}.}
\]
Es gilt nun
$\phi_{ij}(U_{ij} \cap U_{ik}) = U_{ji} \cap U_{jk}$, denn
\begin{align*}
	U_{ij} \cap U_{ik} &= D(x_j x_k) \subseteq U_i\\
	U_{ji} \cap U_{jk} &= D(x_i x_k) \subseteq U_j
\end{align*}
sowie
\[
	\phi_{ik} \rest{U_{ij} \cap U_{ik}} = 
	\phi_{jk} \circ \phi_{ij} \rest{U_{ij} \cap U_{ik}}
\]

\paragraph{Wir haben also}
eine Familie $(U_i)_{i=0,\ldots,n}$ von (affinen) Schemata. Für jedes Paar
$(i,j)$ eine offene Imersion $U_{ij} \hookrightarrow U_i$ mit
(affinen) Schemata
und Isomorphismen
$\phi_{ij}: U_{ij} \xto{\cong} U_{ji}$, so dass
$\phi_{ik} \rest{U_{ij} \cap U_{ik}} = 
	\phi_{jk} \circ \phi_{ij} \rest{U_{ij} \cap U_{ik}}$.

Bleibt zur Übung lediglich zu zeigen, dass ein (bist auf Isomorphie) 
eindeutiges Schema
$\P_A^n$ mit Überdeckung $\P_A^n = \bigcup_{i=0}^n V_i$ für 
$V_i \subseteq \P_A^n$ offen und Isomorphismen
$V_i \xto\cong U_i$ von (affinen) Schemata existiert.
% vim: set ft=tex :
