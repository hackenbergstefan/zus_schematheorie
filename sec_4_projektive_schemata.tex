\section{Projektive Schemata}

\subsection{Eine kurze Einführung in klassische projektive Geometrie}
\ldots

\subsection{$\P^n(k)$ als Schema}
Statt einem Körper $k$ können wir einen Ring $A$ betrachten.

\subsubsection{1. Variante}
Betrachte $U_i := \Spec A[x_0,\ldots,\cancel i,\ldots,x_n] = \A_A^n$.

In $\R\P^n$ würden wir diese mit dem Kartenwechsel verkleben:
\[\everymath{\displaystyle}\begin{tikzcd}
	& [] [y_0,\ldots,\underset{i\text{-te}}{1},\ldots,y_n] & U_i \cap U_j & \\
	(y_0,\ldots,\cancel i,\ldots,y_n) \urar[mapsto] & h_i(U_i \cap U_j)
		\ar{rr}{\text{Kartenwechsel}} \urar 
		&& h_j(U_i \cap U_j) \ular \\
	&\{(y_0,\ldots,\cancel i,\ldots,y_n) \mid y_j \neq 0\} \ar{rr}
	\uar[empty]{\rotatebox{90}{=}}
	&&
	\{(z_0,\ldots,\cancel j,\ldots,z_n) \mid z_i \neq 0\}
	\uar[empty]{\rotatebox{90}{=}}\\
	& (y_0,\ldots,\cancel i,\ldots,y_n) \ar[mapsto]{rr} &&
		\left(\frac{y_0}{y_j},\ldots,\underset{i\text{-te}}{\frac{1}{y_j}},
		\ldots,\cancel j,\ldots,\frac{y_n}{y_j}\right)
\end{tikzcd}\]
Betrachte also
\begin{align*}
	U_{ij} := \Spec A[x_0,\ldots,\cancel i,\ldots,x_n][x_j\inv]
		&\hookrightarrow \Spec A[x_0,\ldots,\cancel i,\ldots,x_n] = U_i\\
	U_{ji} := \Spec A[x_0,\ldots,\cancel j,\ldots,x_n][x_i\inv]
		&\hookrightarrow \Spec A[x_0,\ldots,\cancel j,\ldots,x_n] = U_j\\
\end{align*}
und wähle einen Isomorphismus
\[
	\phi_{ij}: \funcdef{U_{ij} & \to & U_{ji}\\
		x_k &\mapsto& \frac{x_k}{x_j} \quad \text{für $k\neq i$}\\
		x_i & \mapsto& \frac{1}{x_j}.}
\]
Es gilt nun
$\phi_{ij}(U_{ij} \cap U_{ik}) = U_{ji} \cap U_{jk}$, denn
\begin{align*}
	U_{ij} \cap U_{ik} &= D(x_j x_k) \subseteq U_i\\
	U_{ji} \cap U_{jk} &= D(x_i x_k) \subseteq U_j
\end{align*}
sowie
\[
	\phi_{ik} \rest{U_{ij} \cap U_{ik}} = 
	\phi_{jk} \circ \phi_{ij} \rest{U_{ij} \cap U_{ik}}
\]

\paragraph{Wir haben also}
eine Familie $(U_i)_{i=0,\ldots,n}$ von (affinen) Schemata. Für jedes Paar
$(i,j)$ eine offene Imersion $U_{ij} \hookrightarrow U_i$ mit
(affinen) Schemata
und Isomorphismen
$\phi_{ij}: U_{ij} \xto{\cong} U_{ji}$, so dass
$\phi_{ik} \rest{U_{ij} \cap U_{ik}} = 
	\phi_{jk} \circ \phi_{ij} \rest{U_{ij} \cap U_{ik}}$.

Bleibt zur Übung lediglich zu zeigen, dass ein (bist auf Isomorphie) 
eindeutiges Schema
$\P_A^n$ mit Überdeckung $\P_A^n = \bigcup_{i=0}^n V_i$ für 
$V_i \subseteq \P_A^n$ offen und Isomorphismen
$V_i \xto\cong U_i$ von (affinen) Schemata existiert.


\subsubsection{2. Variante (Die $\Proj$-Konstruktion)}

\begin{definition}[graduierte $A$-Algebra]
	Sei $A$ ein Ring, dann heißt
	\[ S:= \oplus_{n\in\N_0} S_n\]
	eine \emph{graduierte $A$-Algebra}, wenn
	\begin{itemize}
	  \item $S$ ein Ring,
	  \item $S_n \subset S$ ein $\Z$-Untermodul,
	  \item $S_n S_m \subseteq S_{n+m}$ ist,
	  \item wir einen Ringhomomorphismus $A \xto \varphi S$ haben und
	  \item die $S_n$ $A$-Untermoduln sind.
	\end{itemize}
	
	Ein $s \in S_n$ heißt \emph{homogen vom Grad $n$}.
\end{definition}

\begin{definition}[homogenes Ideal]
	\label{def:homogenes ideal}
	Ein Ideal $\a \ideal S$ heißt \emph{homogen}, wenn
	\[
		\a = \oplus_{n\in\N_0} \a \cap S_n.
	\]
\end{definition}

\begin{lemma}
	\label{lemma:ideal homogen <=> von homogenen elementen erzeugt}
	Es ist äquivalent
	\begin{itemize}
		\item $\a$ homogen,
		\item $\a$ wird von homogenen Elementen erzeugt
		\item Aus $a \in \a$ mit $a = \sum_{n\in \N_0} a_n$ für
			$a_n\in S_n$ folgt $a_n \in \a$.
	\end{itemize}
\end{lemma}
\begin{proof}
	leicht.
\end{proof}


\begin{beispiel}
	$S = A[x_0,\ldots,x_n] = \oplus_{m\geq 0} S_m$ mit
	\[
		S_m = \{f(x_0,\ldots,x_n) \mid f\text{ homogen von Grad $m$}\},
	\]
	d.h. 
	\[
		f\in S_m \quad\Leftrightarrow\quad
			f = \sum_{\nu \in \N_0^{n+1}} \alpha_\nu X_0^{\nu_0} 
				\ldots X_n^{\nu_n} \quad\text{mit }
				\nu_0 + \ldots+\nu_n = m.
	\]
\end{beispiel}


\begin{definition}[$\Proj(S)$]
	Setze $S_+ := \oplus_{n\geq 1} S_n$, dann ist das
	\emph{projektive Spektrum $\Proj S$ von $S$} definiert als
	\[
		\Proj(S) := \{ \p \in \Spec S\text{ homogen} \mid
			S_+ \subsetneq \p\}.
	\]
\end{definition}

\begin{definition}[Zariski Topologie auf $\Proj(S)$]
	Für ein homogenes Ideal $\a \ideal S$ setze
	\[
		V_+(\a) := \{ \p \in \Proj(S)\mid \a\subseteq \p\} \subseteq 
			\Proj(S).
	\]
	Dann bilden diese $V_+(\a)$ die abgeschlossenen Mengen einer Topologie,
	der \emph{Zariski-Topologie auf $\Proj(S)$}.
\end{definition}
\begin{proof}
	Wie im inhomogenen Fall.
\end{proof}

\begin{bemerkung}
	Ein homogenes $\a \ideal S$, $\a\neq S$, ist prim genau dann, wenn
	gilt:
	\[ xy \in \a \quad \Rightarrow\quad x\in\a \text{ oder } y\in\a\]
	für alle homogenen $x,y$.
\end{bemerkung}

\begin{definition}[basisoffenen Mengen auf $\Proj(S)$]
	Analog zu $\Spec A$ bilden für $f\in S$ 
	die \emph{basisoffenen Mengen in $\Proj(S)$}
	\[
		D_+(f) := \{ \p \in \Proj(S) \mid f\notin \p\}\subseteq \Proj(S)
	\]
	eine Basis der Topologie auf $\Proj(S)$.
\end{definition}

\begin{definition}[homogene Lokalisierung]
	\begin{itemize}
	  \item Für $\p\in \Proj(S)$ heißt
		  \[
		  	S_{(\p)} := \left\{ \frac s t \mid s,t \in S,\ t\notin \p,\ 
		  		s,t \text{ homogen von gleichem Grad}\right\}
		  \]
		  \emph{homogene Lokalisierung von $\p$}.
	  \item Für $f \in S $ homogen von Grad $m$ heißt
	  	\[ 
	  		S_{(f)} := \left\{ \frac{s}{f^k} \mid s\in S,\ k\in \N_0,\ 
	  			s\text{ homogen von Grad } k\deg f\right\}
	  	\]
	  	\emph{homogene Lokalisierung bezüglich $f$}.	  	
	\end{itemize}
\end{definition}

\begin{lemma}
	Es gilt:
	$S_{(\p)}$ ist ein lokaler Ring mit maximalem Ideal 
	\[
		\p_{(\p)} := \left\{\frac s t \mid s\in \p\right\}.
	\]
\end{lemma}
\begin{proof}
\end{proof}

\begin{satz}
	Auf $\Proj(S)$ gibt es eine (bis auf Isomorphie) eindeutige Ringgarbe
	$\O_{\Proj(S)}$ mit:
	\begin{enumerate}
	  \item Für alle homogenen $f\in S_+$ hat man den Isomorphismus
	  	\[
	  		(\varphi, \varphi\fis): 
	  			\left(D_+(f), \O_{\Proj(S)}\rest{D_+(f)}\right)
	  			\to 
	  			\Spec(S_{(f)}, \O_{S_{(f)}})
	  	\]
	  \item
	  	Diese induzieren Isomorphismen
	  	\[
	  		\O_{\Proj(S), \p} \xto\cong S_{(\p)}.
	  	\]
	\end{enumerate}
	Damit wird $(\Proj(S), \O_{\Proj(S)})$ zu einem Schema.
\end{satz}
\begin{proof}
"`analog"' zum Beweis für $\Spec$ mit nachfolgendem Lemma.
\end{proof}

\begin{lemma}
	Ist $f\in S_+$ homogen, so ist
	\[
		\phi: \funcdef{D_+(f) & \to & \Spec(S_{(f)}) \\
			\p & \mapsto & \p S_f \cap S_{(f)}}
	\]
	ein Homöomorphismus.
\end{lemma}
\begin{proof}
	\newcommand{\Sf}{S_{(f)}}
	Sei $S \xto \lambda S_f \xhookleftarrow\iota \Sf$, so haben wir
	\[\begin{tikzcd}
		\Spec S & \Spec S_f \lar{\lambda^\ast} \ar{dd}{\iota^\ast}
		%\arrow[shift left=2pt]{dl}{(\lambda^\ast)\inv}
		\\
		D(f) \arrow{ur}{\lambda^\ast}[swap]{\approx}& \\
		D_+(f) \uar[hook]{\text{stetig}} \rar{\phi} & \Spec(\Sf)
	\end{tikzcd}\qquad
	\begin{tikzcd}
		& \p S_f \ar{dd}\\
		\p \urar[mapsto] & \\
		\p \uar[mapsto] \rar[mapsto] 
		& \p S_f \cap \Sf
	\end{tikzcd}
	\]
	Die Stetigkeit im linken Diagramm folgt aus der Tatsache, dass
	$V_+(\a) = V(\a) \cap \Proj(S)$ und $\Proj(S)$ trägt die Teilraumtopologie
	von $\Spec S$.
	Damit ist $\phi$ stetig.
	
	Wir wollen die Umkehrabbildung von $\phi$ angeben:
	\[
		\funcdef{D_+(f) & \xto\phi & \Spec(\Sf) \\
			\lambda\inv(\sqrt{\q S_f}) & \mapsfrom & \q.}
	\]
	Den Rest zeigen nachstehende Hilfslemmata. 
\end{proof}

\begin{hilfslemma}
	$\p := \lambda\inv(\sqrt{\q S_f})$ ist homogenes Primideal in $S$.
\end{hilfslemma}
\begin{proof}
	\[
		\q S_f = \left\{ \frac{b}{f^l} \frac{c}{f^n} \in S_f \left|
			\begin{array}{l}
				b\text{ homogen, } \deg b = l\deg f\\
				\frac{b}{f^n} \in \q,\ c\in S,n\in \N_0
			\end{array}\right. \right\}
	\]
	Bemerke, dass $\p$ ein homogenes Ideal ist, weil $\q S_f$ es ist.
	Genauer:
	$S_f = \oplus_{n\geq 0} S_{f,n}$ mit
	\[
		S_{f,n} := \left\{\frac{c}{f^m} \mid c\text{ homogen, }
			\deg c - m\deg f = n\right\}.
	\]
	Es bleibt also zu zeigen: Sind $a,a' \in S$ homogen und 
	$aa' \in \p$, so folgt $a\in \p$ oder $a'\in \p$.
	
	Sei dazu  $r = \deg a$, $s = \deg a'$.
	Aus $aa'\in \p$ folgt $\lambda(aa') = \frac{aa'}{1} \in \sqrt{\q S_f}$.
	Also existiert ein $k\in \N$ mit
	$\left(\frac{aa'}{1}\right)^k \in \q S_f$, also
	$\left( \frac{aa'}{1}\right)^k = \frac{b}{f^l} \frac{c}{f^n}$
	wie oben. Potenzieren mit $\deg f$ ergibt
	\[
		\frac{a^{k\deg f} a'^{k\deg f}}{f^{kr} f^{ks}} = 
		\frac{b^{\deg g}}{f^{l\deg f}} \frac{c^{\deg f}}{f^{n\deg f}}
		\frac{1}{f^{kr} f^{ks}} \in S_f.
	\]
\end{proof}

% vim: set ft=tex :
