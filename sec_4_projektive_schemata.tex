\section{Projektive Schemata}

\subsection{Eine kurze Einführung in klassische projektive Geometrie}
Sei $k$ ein Körper. So ist
\[
\P^n(k):=\P(k^{n+1}):=\{L\subset k^{n+1} \text{UVR}\mid \dim_k L=1\}
\]
der n-dimensionale projektive Raum.
\paragraph{Homogene Koordinaten} $[x_0:\dots:x_n]\in\P^n(k)$ mit
$0\neq(x_0,\dots,x_n)\in k^{n+1}$ definiert als
\[
[x_0:\dots:x_n]:=\Span_k \begin{pmatrix}x_{0}\\ \vdots\\ x_{n} \end{pmatrix} 
\]
mit $[x_0:\dots:x_n]=[y_0,\dots,y_n]$ $\Leftrightarrow$ $\exists \lambda \in
k^\times$ mit $x_i=\lambda y_i \forall i$.
Damit gilt dann, dass $\P^n(k)=k^{n+1}/\sim$, wobei $\sim$ die gerade eben
definierte Äquivalenzrelation bezeichnet.
\paragraph{Überdeckung}
$\P^n(k)=\bigcup_{i=0}^nU_i$ mit
\begin{center}
\begin{tikzpicture}
\matrix (m) [ matrix of math nodes , row sep=3em ] {
  U_i & = \{[x_0:\dots:x_n]\in\P^n(k)\mid x_i\neq 0\} & \ni &
    {[}x_0:\dots:x_0{]}\\
  k^n   & & \ni & \Big( \frac{x_0}{x_i},\dots,\frac{x_{i-1}}{x_i},
  \frac{x_{i+1}}{x_i},\dots,\frac{x_n}{x_i} \Big) \\
};
\path[->,font=\scriptsize,>=angle 90]
(m-1-1) edge node[left]{$h_i$} node[right]{$b_{ij}$} (m-2-1) ;
\path[>=stealth,|->] (m-1-4) edge (m-2-4) ;
\end{tikzpicture}
\end{center}
als "'Karten"'.
\paragraph{Beachte}
\begin{tikzcd}
\P^n(k)\backslash U_i=\{[x_0:\dots:0:\dots:x_n]\mid (x_0,\dots,\not
i,\dots,x_n)\neq0\}  \arrow{r}{1-1} & \P^{n-1}(k)
\end{tikzcd}
\begin{bemerkung}
\begin{itemize}
    \item $\R\P^n:=\P^n(\R)$
    \item $\C\P^n:=\P^n(\C)$
    \item $\C\P^1\approx S^2$
\end{itemize}
\end{bemerkung}

\subsection{$\P^n(k)$ als Schema}
Statt einem Körper $k$ können wir einen Ring $A$ betrachten.

\subsubsection{1. Variante}
Betrachte $U_i := \Spec A[x_0,\ldots,\cancel i,\ldots,x_n] = \A_A^n$.

In $\R\P^n$ würden wir diese mit dem Kartenwechsel verkleben:
\[\everymath{\displaystyle}\begin{tikzcd}[]
    & [] [y_0,\ldots,\underset{i\text{-te}}{1},\ldots,y_n] 
        & U_i \cap U_j 
        & 
        \\
    (y_0,\ldots,\cancel i,\ldots,y_n) \urar[mapsto] 
        & h_i(U_i \cap U_j) \ar{rr}{\text{Kartenwechsel}} \urar 
        &
        & h_j(U_i \cap U_j) \ular 
        \\
    & \{(y_0,\ldots,\cancel i,\ldots,y_n) \mid y_j \neq 0\} \ar{rr}
            \uar[empty]{\rotatebox{90}{=}}
        &
        & \{(z_0,\ldots,\cancel j,\ldots,z_n) \mid z_i \neq 0\}
            \uar[empty]{\rotatebox{90}{=}}
        \\
    & (y_0,\ldots,\cancel i,\ldots,y_n) \ar[mapsto]{rr} 
        &
        & \left(\frac{y_0}{y_j},\ldots,\underset{i\text{-te}}{\frac{1}{y_j}},
            \ldots,\cancel j,\ldots,\frac{y_n}{y_j}\right)
\end{tikzcd}\]
Betrachte also
\begin{align*}
	U_{ij} := \Spec A[x_0,\ldots,\cancel i,\ldots,x_n][x_j\inv]
		&\hookrightarrow \Spec A[x_0,\ldots,\cancel i,\ldots,x_n] = U_i\\
	U_{ji} := \Spec A[x_0,\ldots,\cancel j,\ldots,x_n][x_i\inv]
		&\hookrightarrow \Spec A[x_0,\ldots,\cancel j,\ldots,x_n] = U_j\\
\end{align*}
und wähle einen Isomorphismus
\[
	\phi_{ij}: \funcdef{U_{ij} & \to & U_{ji}\\
		x_k &\mapsto& \frac{x_k}{x_i} \quad \text{für $k\neq j$}\\
		x_j & \mapsto& \frac{1}{x_i}.}
\]
Es gilt nun
$\phi_{ij}(U_{ij} \cap U_{ik}) = U_{ji} \cap U_{jk}$, denn
\begin{align*}
	U_{ij} \cap U_{ik} &= D(x_j x_k) \subseteq U_i\\
	U_{ji} \cap U_{jk} &= D(x_i x_k) \subseteq U_j
\end{align*}
sowie
\[
	\phi_{ik} \rest{U_{ij} \cap U_{ik}} = 
	\phi_{jk} \circ \phi_{ij} \rest{U_{ij} \cap U_{ik}}
\]

\paragraph{Wir haben also}
eine Familie $(U_i)_{i=0,\ldots,n}$ von (affinen) Schemata. Für jedes Paar
$(i,j)$ eine offene Imersion $U_{ij} \hookrightarrow U_i$ mit
(affinen) Schemata
und Isomorphismen
$\phi_{ij}: U_{ij} \xto{\cong} U_{ji}$, so dass
$\phi_{ik} \rest{U_{ij} \cap U_{ik}} = 
	\phi_{jk} \circ \phi_{ij} \rest{U_{ij} \cap U_{ik}}$.

Bleibt zur Übung lediglich zu zeigen, dass ein (bist auf Isomorphie) 
eindeutiges Schema
$\P_A^n$ mit Überdeckung $\P_A^n = \bigcup_{i=0}^n V_i$ für 
$V_i \subseteq \P_A^n$ offen und Isomorphismen
$V_i \xto\cong U_i$ von (affinen) Schemata existiert.


\subsubsection{2. Variante (Die $\Proj$-Konstruktion)}

\begin{definition}[graduierte $A$-Algebra]
    \index[def]{Graduierte Algebra}
	Sei $A$ ein Ring, dann heißt
	\[ S:= \oplus_{n\in\N_0} S_n\]
	eine \emph{graduierte $A$-Algebra}, wenn
	\begin{itemize}
	  \item $S$ ein Ring,
	  \item $S_n \subset S$ ein $\Z$-Untermodul,
	  \item $S_n S_m \subseteq S_{n+m}$ ist,
	  \item wir einen Ringhomomorphismus $A \xto \varphi S$ haben und
	  \item die $S_n$ $A$-Untermoduln sind.
	\end{itemize}
	
	Ein $s \in S_n$ heißt \emph{homogen vom Grad $n$}.
\end{definition}

\begin{definition}[homogenes Ideal]
    \index[def]{Graduierte Algebra!homogenes Ideal}
	\label{def:homogenes ideal}
	Ein Ideal $\a \ideal S$ heißt \emph{homogen}, wenn
	\[
		\a = \oplus_{n\in\N_0} \a \cap S_n.
	\]
\end{definition}

\begin{lemma}
	\label{lemma:ideal homogen <=> von homogenen elementen erzeugt}
	Es ist äquivalent
	\begin{itemize}
		\item $\a$ homogen,
		\item $\a$ wird von homogenen Elementen erzeugt
		\item Aus $a \in \a$ mit $a = \sum_{n\in \N_0} a_n$ für
			$a_n\in S_n$ folgt $a_n \in \a$.
	\end{itemize}
\end{lemma}
\begin{proof}
	leicht.
\end{proof}


\begin{beispiel}
	$S = A[x_0,\ldots,x_n] = \oplus_{m\geq 0} S_m$ mit
	\[
		S_m = \{f(x_0,\ldots,x_n) \mid f\text{ homogen von Grad $m$}\},
	\]
	d.h. 
	\[
		f\in S_m \quad\Leftrightarrow\quad
			f = \sum_{\nu \in \N_0^{n+1}} \alpha_\nu X_0^{\nu_0} 
				\ldots X_n^{\nu_n} \quad\text{mit }
				\nu_0 + \ldots+\nu_n = m.
	\]
\end{beispiel}


\begin{definition}[$\Proj(S)$]
	Setze $S_+ := \oplus_{n\geq 1} S_n$, dann ist das
	\emph{projektive Spektrum $\Proj S$ von $S$} definiert als
	\[
		\Proj(S) := \{ \p \in \Spec S\text{ homogen} \mid
			S_+ \subsetneq \p\}.
	\]
\end{definition}

\begin{definition}[Zariski Topologie auf $\Proj(S)$]
    \index[def]{Zariski Topologie!auf $\Proj$}
	Für ein homogenes Ideal $\a \ideal S$ setze
	\[
		V_+(\a) := \{ \p \in \Proj(S)\mid \a\subseteq \p\} \subseteq 
			\Proj(S).
	\]
	Dann bilden diese $V_+(\a)$ die abgeschlossenen Mengen einer Topologie,
	der \emph{Zariski-Topologie auf $\Proj(S)$}.
\end{definition}
\begin{proof}
	Wie im inhomogenen Fall.
\end{proof}

\begin{bemerkung}
	Ein homogenes $\a \ideal S$, $\a\neq S$, ist prim genau dann, wenn
	gilt:
	\[ xy \in \a \quad \Rightarrow\quad x\in\a \text{ oder } y\in\a\]
	für alle homogenen $x,y$.
\end{bemerkung}

\begin{definition}[basisoffenen Mengen auf $\Proj(S)$]
    \index[def]{Basisoffene Menge!auf $\Proj$}
	Analog zu $\Spec A$ bilden für $f\in S$ 
	die \emph{basisoffenen Mengen in $\Proj(S)$}
	\[
		D_+(f) := \{ \p \in \Proj(S) \mid f\notin \p\}\subseteq \Proj(S)
	\]
	eine Basis der Topologie auf $\Proj(S)$.
\end{definition}

\begin{definition}[homogene Lokalisierung]
    \index[def]{Lokalisierung!homogene}
	\begin{itemize}
	  \item Für $\p\in \Proj(S)$ heißt
		  \[
		  	S_{(\p)} := \left\{ \frac s t \mid s,t \in S,\ t\notin \p,\ 
		  		s,t \text{ homogen von gleichem Grad}\right\}
		  \]
		  \emph{homogene Lokalisierung von $\p$}.
	  \item Für $f \in S $ homogen von Grad $m$ heißt
	  	\[ 
	  		S_{(f)} := \left\{ \frac{s}{f^k} \mid s\in S,\ k\in \N_0,\ 
	  			s\text{ homogen von Grad } k\deg f\right\}
	  	\]
	  	\emph{homogene Lokalisierung bezüglich $f$}.	  	
	\end{itemize}
\end{definition}

\begin{lemma}
	Es gilt:
	$S_{(\p)}$ ist ein lokaler Ring mit maximalem Ideal 
	\[
		\p_{(\p)} := \left\{\frac s t \mid s\in \p\right\}.
	\]
\end{lemma}
\begin{proof}
    \TODO
\end{proof}

\begin{satz}
    \label{satz:proj s eindeutige ringgarbe}
	Auf $\Proj(S)$ gibt es eine (bis auf Isomorphie) eindeutige Ringgarbe
	$\O_{\Proj(S)}$ mit:
	\begin{enumerate}
	  \item Für alle homogenen $f\in S_+$ hat man den Isomorphismus
	  	\[
	  		(\varphi, \varphi\fis): 
	  			\left(D_+(f), \O_{\Proj(S)}\rest{D_+(f)}\right)
	  			\to 
	  			\Spec(S_{(f)}, \O_{S_{(f)}})
	  	\]
	  \item
	  	Diese induzieren Isomorphismen
	  	\[
	  		\O_{\Proj(S), \p} \xto\cong S_{(\p)}.
	  	\]
	\end{enumerate}
	Damit wird $(\Proj(S), \O_{\Proj(S)})$ zu einem Schema.
\end{satz}
\begin{proof}
"`analog"' zum Beweis für $\Spec$ mit nachfolgendem Lemma.
\end{proof}

\begin{lemma}
	Ist $f\in S_+$ homogen, so ist
	\[
		\phi: \funcdef{D_+(f) & \to & \Spec(S_{(f)}) \\
			\p & \mapsto & \p S_f \cap S_{(f)}}
	\]
	ein Homöomorphismus.
\end{lemma}
\begin{proof}
	\newcommand{\Sf}{S_{(f)}}
	Sei $S \xto \lambda S_f \xhookleftarrow\iota \Sf$, so haben wir
	\[\begin{tikzcd}
		\Spec S & \Spec S_f \lar{\lambda^\ast} \ar{dd}{\iota^\ast}
		%\arrow[shift left=2pt]{dl}{(\lambda^\ast)\inv}
		\\
		D(f) \arrow{ur}{\lambda^\ast}[swap]{\approx}& \\
		D_+(f) \uar[hook]{\text{stetig}} \rar{\phi} & \Spec(\Sf)
	\end{tikzcd}\qquad
	\begin{tikzcd}[]
		& \p S_f \ar{dd}\\
		\p \urar[mapsto] & \\
		\p \uar[mapsto] \rar[mapsto] 
		& \p S_f \cap \Sf
	\end{tikzcd}
	\]
	Die Stetigkeit im linken Diagramm folgt aus der Tatsache, dass
	$V_+(\a) = V(\a) \cap \Proj(S)$ und $\Proj(S)$ trägt die Teilraumtopologie
	von $\Spec S$.
	Damit ist $\phi$ stetig.
	
	Wir wollen die Umkehrabbildung von $\phi$ angeben:
	\[
		\funcdef{D_+(f) & \xto\phi & \Spec(\Sf) \\
			\lambda\inv(\sqrt{\q S_f}) & \mapsfrom & \q.}
	\]
	Den Rest zeigen nachstehende Hilfslemmata. 
\end{proof}

\begin{hilfslemma}
	$\p := \lambda\inv(\sqrt{\q S_f})$ ist homogenes Primideal in $S$.
\end{hilfslemma}
\begin{proof}
	\[
		\q S_f = \left\{ \frac{b}{f^l} \frac{c}{f^n} \in S_f \left|
			\begin{array}{l}
				b\text{ homogen, } \deg b = l\deg f\\
				\frac{b}{f^n} \in \q,\ c\in S,n\in \N_0
			\end{array}\right. \right\}
	\]
	Bemerke, dass $\p$ ein homogenes Ideal ist, weil $\q S_f$ es ist.
	Genauer:
	$S_f = \oplus_{n\geq 0} S_{f,n}$ mit
	\[
		S_{f,n} := \left\{\frac{c}{f^m} \mid c\text{ homogen, }
			\deg c - m\deg f = n\right\}.
	\]
	Es bleibt also zu zeigen: Sind $a,a' \in S$ homogen und 
	$aa' \in \p$, so folgt $a\in \p$ oder $a'\in \p$.
	
	Sei dazu  $r = \deg a$, $s = \deg a'$.
	Aus $aa'\in \p$ folgt $\lambda(aa') = \frac{aa'}{1} \in \sqrt{\q S_f}$.
	Also existiert ein $k\in \N$ mit
	$\left(\frac{aa'}{1}\right)^k \in \q S_f$, also
	$\left( \frac{aa'}{1}\right)^k = \frac{b}{f^l} \frac{c}{f^n}$
	wie oben. Potenzieren mit $\deg f$ ergibt
	\[
		\frac{a^{k\deg f} a'^{k\deg f}}{f^{kr} f^{ks}} = 
		\frac{b^{\deg g}}{f^{l\deg f}} \frac{c^{\deg f}}{f^{n\deg f}}
		\frac{1}{f^{kr} f^{ks}} \in S_f.
	\]
	\TODO
\end{proof}

wir definieren $\P_A^n := \Proj(A[X_0,\ldots,X_n])$ als Schema. Dabei stellen
sich aber die Fragen, was dabei $D_+(X_i)$ sein soll und ob die beiden 
Varianten übereinstimmen.

\begin{lemma}
    Die beiden Varianten der Definition von $\P_A^n$ stimmen überein
    und es gilt
    \[ D_+(X_i) \cong \Spec S_{(X_i)} \cong \A_A^n.\]
\end{lemma}
\begin{proof}
    \TODO
\end{proof}

\subsection{Immersionen und projektive $A$-Schemata}

\begin{definition}[offene und abgeschlossene Immersion]
    \index[def]{Schema!Schemamorphismus!offene Immersion}
    \index[def]{Schema!Schemamorphismus!abgeschlossene Immersion}
    Ein Morphismus $f: Y \to X$ von Schemata heißt
    \begin{enumerate}
      \item \emph{offene Immersion}, wenn es $U\osubset X$ gibt, so dass
        \[ f: (Y,\O_Y) \xto\cong (U,\O_X\rest U) 
            \xhookrightarrow{(\iota,\iota\fis)} (X,\O_X)\]
      \item \emph{abgeschlossene Immerson}, wenn gilt:
      \begin{itemize}
        \item $f$ ist topologisch ein Homöomorphismus auf $\im f:= Z\subset X$
            abgeschlossen,
        \item $f\fis: \O_X \to f_\ast \O_Y$ ist ein surjektiver 
            Garbenmorphismus, d.h. für alle $y \in Y$ ist
            \[f_{(f(y))}\fis: \O_{X,f(y)} \to \O_{Y,y}\]
            surjektiv.
      \end{itemize}
        Wir schreiben dann auch $Y \immersion X \rightarrow Y$. 
    \end{enumerate}
\end{definition}


\begin{beispiel}
    Ist $A$ ein Ring, $a\ideal A$, so induziert
    \[A\xto{\pi} A\big/\a\]
    eine abgeschlossene Immersion
    \[ f:\Spec A\big/\a \to \Spec A\] 
\end{beispiel}
\begin{proof}
    \TODO
\end{proof}

\begin{bemerkung}
    Es ist $V(\a) = V(\sqrt\a) = V(\b)$ genau dann, wenn $\sqrt a = \sqrt b$.
    Aber es folgt nicht notwendigerweise $A\big/\a \overset?\cong A\big/\b$!
    
    Dazu betrachte einen Ring $A$ mit nilpotenten Elementen, d.h.
    $\Nil A := \sqrt{(0)} \neq (0)$ und
    \[f:\Spec A\big/\Nil(A) \hookrightarrow \Spec A\]
    ist eine abgeschlossene Immersion mit
    \[\im f = V(\Nil(A)) = \{\p\in \Spec A \mid \Nil(A) \subseteq \p\}
        = \Spec A.\]
    Jedoch ist dies \emph{kein} Isomorphismus.  
\end{bemerkung}


\begin{definition}[abgeschlossenes Unterschema]
    \index[def]{Schema!abgeschlossenes Unterschema}
    Ist $f:Y\to X$ eine abgeschlossene Immersion, so nennen wir $Y$ 
    ein \emph{(bzgl. $f$) abgeschlossenes Unterschema von $X$}.
\end{definition}

\begin{definition}[projektives Schema über $A$]
    \index[def]{Schema!projektives Schema über $A$}
    Sei $A$ ein Ring. Ein \emph{projektives Schema über $A$} ist ein
    $A$-Schema $X$ mit einer abgeschlossenen Immersion, so dass
    \[\begin{tikzcd}
        \iota: \ X \ar[immersion]{rr} \drar && \P_A^n \dlar\\
        & \Spec A & 
    \end{tikzcd}\]
    für ein $n \in\N_0$ kommutiert.
\end{definition}

\begin{bemerkung}
    \TODO
\end{bemerkung}

\subsubsection{Beispiele}
Zunächst ein etwas abstrakteres Beispiel.
\begin{satz}
    Sei $S := A[X_0,\ldots, X_n]$. Ist $\b\ideal S$ ein homogenes Ideal, so ist
    $B := S \big/\b$ in natürlicher Weise eine graduierte $A$-Algebra
    und $\Proj(B)$ ein projektives $A$-Schema.
\end{satz}
\begin{proof}
    \TODO
\end{proof}

Und nun einige konkrete!

\newcommand{\Pnklass}{\P_{\text{klass}}^n}
\newcommand{\Pn}{\P^n}
\paragraph{1. $\Pnklass(k)$ und $\Pn_k$.} Sei $k$ ein Körper.
Wir haben $\Pnklass(k) := k^{n+1}\setminus\{0\} \big/ \sim$ und
dagegen $\Pn_k := \Proj k[T_0,\ldots, T_n]$.

Eine algebraische Menge in $\Pnklass(k)$ ist per definitionem
\[Z:= \{[x_0:\ldots:x_n] \in \Pnklass(k) \mid f_i(x_0,\ldots,x_n) = 0\}\]
für $f_1(T_0,\ldots,T_n),\ldots,f_r(T_0,\ldots,T_n) \in k[T_0,\ldots,T_n]$ 
homogen.

\begin{satz}
    Die Abbildung
    \[ \rho: \funcdef{ \Pnklass(k) & \to & \Pn_k \\ {}
        [x_0:\ldots:x_n] & \mapsto & 
        \langle x_i T_j - x_j T_i \mid i,j \rangle
        }
     \]
    ist eine Bijektion auf 
    \[ \Pn_k(k) = \{\p \in \Pn_k \mid \p \text{ ist $k$-rational}\} = 
        \Hom_{\Sch_k}(\Spec k, \Pn_k).\]
\end{satz} 
\begin{proof}
    \TODO
\end{proof}


\begin{bemerkung}
    Wir haben dies auch schon affin gesehen:
    \[ \funcdef{ k^n = \A_{\text{klass}}^n(k) & \to & 
        \A_k^n = \Spec k[X_1,\ldots,X_n]\\
        (\alpha_1,\ldots,\alpha_n) & \mapsto & 
        (X_1 - \alpha_1, \ldots, X_n - \alpha_n)}.\]  
\end{bemerkung}


\begin{bemerkung}
    Sei $X$ ein Schema. Wir erinnern daran, dass
    \[X(K) := \Hom_{\Sch}(\Spec k, X) = 
        \{ (\varphi,\varphi\fis): \Spec k \to X\}\]
    mit 
    \[\varphi_\eta: \O_{X,x} \to \O_{\Spec k, \eta} = k\]
    mit $x = \varphi(\eta)$, wobei topologisch $\Spec k = \{\eta\}$.
    Damit haben wir
    \[ \overline{\varphi_\eta^n}: \O_{X,x}\big/\m_x = k(x) 
        \tikzmark{\hookrightarrow}
        k \]
    \tikzmargin{south}{\color{red} 
        Wir folgern eine Seite später dass $k(x) \cong k$ 
        kanonisch. Das ist mir nicht klar :-(}
    (Körperhomomorphismen sind immer injektiv) und wir
    erhalten folgende 1-1 Beziehung:
    \[ X(k) \overset{\text{1-1}}{=} \{x\in X \text{ zusammen mit Inklusionen }
        \iota: k(x) \hookrightarrow k\}. \]
        
    Beachte dabei:
    \begin{align*}
        X \in \Obj(\Sch) \quad&\leadsto\quad X(k) := \Hom_{\Sch}(\Spec k, X)\\
        Y \in \Obj(\Sch\rest k) \quad&\leadsto\quad 
            Y(k) := \Hom_{\Sch\rest k}(\Spec k, X) = 
            \left\{\begin{tikzcd}[ampersand replacement=\&]
                \varphi: \Spec k \ar{rr} \drar{\id} \&\& Y \dlar \\
                \& \Spec k
            \end{tikzcd}\right\}\\
    \end{align*} 
    In diesem Sinne ist $\P_k^n$ als $k$-Schema zu lesen mit 
    $\P^n_k \to \Spec k$. 
\end{bemerkung}

\paragraph{2. Projektiver Abschluss}
Sei $\a \ideal k[Y_1,\ldots, Y_n]$, so hat man die abgeschlossene Immersion
\[ \Spec k[Y_1,\ldots,Y_n]\big/\a \immersion \A_k^n\]
mit Bild $V(\a)$.

Betrachte die Homogenisierung von $\a$ in $k[T_0,\ldots,T_n]$:
Sei $\a = (f_1,\ldots,f_1)$. Definiere
\[ f_i^\text{homo}(T_0,\ldots,T_n) := T_0^{\deg f_i} f_i(\tfrac{T_1}{T_0},
    \ldots, \tfrac{T_n}{T_0}) \in k[T_0,\ldots,T_n].\]
Damit können wir nun folgenden Satz formulieren.

\begin{satz}
    Ist $\iota: X\immersion \A_k^n$ eine abgeschlossene Immersion, 
    $X = \Spec k[Y_1,\ldots,Y_n]\big/\a$ und $\a = (f_1,\ldots,f_r)$,
    so nennen wir
    \[ \bar X := \Proj k[T_0,\ldots,T_n] \big/ 
        \a^\text{homo} \immersion \P^n_k\]
    mit $\a^\text{homo} := (f_1^\text{homo}, \ldots, f_r^\text{homo})$
    den \emph{projektiven Abschluss von $X$ in $\P^n_k$}. Es gilt 
    \[ \begin{tikzcd}
        \p^\text{homo} \symb{\ni} & 
            D_+(T_0) \cap \bar X  \rar[offene immersion]{\tikzmark[1]{}} & \bar X 
            \rar[immersion] & \P^n_k \\
        \p \uar[mapsto] \symb{\ni} & 
            X \uar{\cong}  \ar[immersion]{rr} &&
            \Spec k[Y_1,\ldots,Y_n] = \A^n_k 
            \tikzmark[2]{\ \cong\ } D_+(T_0)  \uar[hook]
    \end{tikzcd}\]
    wobei die Isomorphie an \tikzarrow[2]{mark above}{dieser} Stelle
    durch die Definition der homogenen Polynome herrührt.
\end{satz}
\tikzmargin[1]{north}{\color{red} Bei mir steht "`offene Inklusion"',
    soll wohl aber offene Immersion gemeint sein !?}
\begin{proof}
    klar.
\end{proof}


\begin{beispiel}
    Sei $E = \Spec k[X,Y] \big/(Y^2-X^3-aX-b) \subseteq \A_k^2$, so ist
    \[
        \bar E = \Proj k[X,Y,Z]\big/ (Y^2Z - X^3 - aXZ^2 -bZ^3) 
            \subseteq \P_k^2.
    \]
    Als Übung überlege man sich was $\bar E \cap (\P_k^2 \setminus D_+(T_0))$
    ist.
\end{beispiel}



\pagebreak
% vim: set ft=tex :
