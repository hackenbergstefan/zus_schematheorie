%% Basierend auf einer TeXnicCenter-Vorlage von Mark M�ller
%%%%%%%%%%%%%%%%%%%%%%%%%%%%%%%%%%%%%%%%%%%%%%%%%%%%%%%%%%%%%%%%%%%%%%%

% W�hlen Sie die Optionen aus, indem Sie % vor der Option entfernen  
% Dokumentation des KOMA-Script-Packets: scrguide

%%%%%%%%%%%%%%%%%%%%%%%%%%%%%%%%%%%%%%%%%%%%%%%%%%%%%%%%%%%%%%%%%%%%%%%
%% Optionen zum Layout des Artikels                                  %%
%%%%%%%%%%%%%%%%%%%%%%%%%%%%%%%%%%%%%%%%%%%%%%%%%%%%%%%%%%%%%%%%%%%%%%%
\documentclass[%
%a5paper,							% alle weiteren Papierformat einstellbar
%landscape,						% Querformat
%10pt,								% Schriftgr��e (12pt, 11pt (Standard))
%BCOR1cm,							% Bindekorrektur, bspw. 1 cm
%DIVcalc,							% f�hrt die Satzspiegelberechnung neu aus
%											  s. scrguide 2.4
%twoside,							% Doppelseiten
%twocolumn,						% zweispaltiger Satz
halfparskip*,				% Absatzformatierung s. scrguide 3.1
%headsepline,					% Trennline zum Seitenkopf	
%footsepline,					% Trennline zum Seitenfu�
%titlepage,						% Titelei auf eigener Seite
%normalheadings,			% �berschriften etwas kleiner (smallheadings)
%idxtotoc,						% Index im Inhaltsverzeichnis
%liststotoc,					% Abb.- und Tab.verzeichnis im Inhalt
%bibtotoc,						% Literaturverzeichnis im Inhalt
%abstracton,					% �berschrift �ber der Zusammenfassung an	
%leqno,   						% Nummerierung von Gleichungen links
%fleqn,								% Ausgabe von Gleichungen linksb�ndig
%draft								% �berlangen Zeilen in Ausgabe gekennzeichnet
DIV = 14,
%monochrome						% schwarz wei� output
]
{scrartcl}

%\pagestyle{empty}		% keine Kopf und Fu�zeile (k. Seitenzahl)
%\pagestyle{headings}	% lebender Kolumnentitel  


%% Deutsche Anpassungen %%%%%%%%%%%%%%%%%%%%%%%%%%%%%%%%%%%%%
\usepackage{amsmath}
\usepackage[ngerman]{babel}
\usepackage[T1]{fontenc}
\usepackage[ansinew]{inputenc}

\usepackage{lmodern} %Type1-Schriftart f�r nicht-englische Texte

\usepackage{xcolor}
\usepackage{tikz}
\usetikzlibrary{matrix,arrows}
\usepackage{tikz-cd}


\usepackage[automark]{scrpage2} % Headline styles


%% Packages tikztitle and tikztheorem %%%%%%%%%%%%%%%%%%%%%%%
\colorlet{mycolor}{blue!80!black}
\usepackage[color=mycolor, style=elegant]{tikztitle}
\usepackage{tikztheorems}
\newtikztheorem[
	style=elegantbreak,
	color=mycolor,
	font header=\normalfont\sffamily\bfseries,
	counter zero=section
	]{satz}{Satz}
	
\newtikztheorem[
	style=elegantbreak,
	color=mycolor,
	font header=\normalfont\sffamily\bfseries,
	font body=\normalfont,
	counter zero=section
	]{definition}{Definition}
 
\newtikztheorem[
	style=elegantinline,
	color=mycolor,
	font header=\normalfont\sffamily\bfseries,
	counter parent=satz
	]{lemma}{Lemma}
	
\newtikztheorem[
	style=plain,
	color=mycolor,
	font header=\normalfont\sffamily\bfseries,
	font body=\normalfont,
	counter parent=satz
	]{beispiel}{Beispiel}
	
\newtikztheorem[
	style=plain,
	color=mycolor,
	font header=\normalfont\sffamily\bfseries,
	font body=\normalfont,
	counter parent=satz
	]{bemerkung}{Bemerkung}
%%%%%%%%%%%%%%%%%%%%%%%%%%%%%%%%%%%%%%%%%%%%%%%%%%%%%%%%%%%%%


\usepackage{tikzmargin}

\usepackage{array}

%% Packages f�r Grafiken & Abbildungen %%%%%%%%%%%%%%%%%%%%%%
\usepackage{graphicx} %%Zum Laden von Grafiken
%\usepackage{subfig} %%Teilabbildungen in einer Abbildung
\usepackage{hyperref}


%% Math abbreviations %%%%%%%%%%%%%%%%%%%%%%%%%%%%%%%%%%%%%%%
\let\bb\mathbb
\let\cal\mathcal
\newcommand{\F}{\cal{F}}
\newcommand{\G}{\cal{G}}
\renewcommand{\O}{\cal{O}}
\newcommand{\R}{\bb{R}}
\newcommand{\C}{\bb{C}}
\newcommand{\rest}[1]{\big|_{#1}}
\newcommand{\kat}[1]{\mathbf{#1}}
\newcommand{\op}{^\mathrm{op}}
\DeclareMathOperator{\Obj}{Obj}
\DeclareMathOperator{\Morph}{Morph}
\DeclareMathOperator{\Hom}{Hom}
\DeclareMathOperator{\im}{im}
\newcommand{\funcdef}[1]{%
	\begin{array}[t]{>{\displaystyle}r>{\displaystyle}c>{\displaystyle}l}
	#1\end{array}}
\let\xto\xrightarrow
%%%%%%%%%%%%%%%%%%%%%%%%%%%%%%%%%%%%%%%%%%%%%%%%%%%%%%%%%%%%%


%% Bibliographiestil %%%%%%%%%%%%%%%%%%%%%%%%%%%%%%%%%%%%%%%%%%%%%%%%%%
%\usepackage{natbib}

\begin{document}
\pagestyle{empty}
%%%%%%%%%%%%%%%%%%%%%%%%%%%%%%%%%%%%%%%%%%%%%%%%%%%%%%%%%%%%%%%%%%%%%%%
%% Ihr Artikel                                                       %%
%%%%%%%%%%%%%%%%%%%%%%%%%%%%%%%%%%%%%%%%%%%%%%%%%%%%%%%%%%%%%%%%%%%%%%%

%% eigene Titelseitengestaltung %%%%%%%%%%%%%%%%%%%%%%%%%%%%%%%%%%%%%%%    
%\begin{titlepage}
%Einsetzen der TXC Vorlage "Deckblatt" m�glich
%\end{titlepage}

%% Angaben zur Standardformatierung des Titels %%%%%%%%%%%%%%%%%%%%%%%%
%\titlehead{Titelkopf }
\subject{Vorlesungszusammenfassung}
\title{Schematheorie}
\author[erstellt von]{Stefan Hackenberg \and Maximilian Huber}
%\thanks{Fu�note}			% entspr. \footnote im Flie�text
\date{\today}				% falls anderes, als das aktuelle gew�nscht
\publishers[gelesen im WS 2012/2013 und SS 2013 von]{Prof. Dr. Marco Hien}

%% Widmungsseite %%%%%%%%%%%%%%%%%%%%%%%%%%%%%%%%%%%%%%%%%%%%%%%%%%%%%%
%\dedication{Widmung}

\maketitle 						% Titelei wird erzeugt

%% Zusammenfassung nach Titel, vor Inhaltsverzeichnis %%%%%%%%%%%%%%%%%
%\begin{abstract}
% F�r eine kurze Zusammenfassung des folgenden Artikels.
% F�r die �berschrift s. \documentclass[abstracton].
%\end{abstract}

\KOMAoptions{twoside}
\cleardoublepage

%% Erzeugung von Verzeichnissen %%%%%%%%%%%%%%%%%%%%%%%%%%%%%%%%%%%%%%%
\tableofcontents			% Inhaltsverzeichnis
%\listoftables				% Tabellenverzeichnis
%\listoffigures				% Abbildungsverzeichnis


%% Der Text %%%%%%%%%%%%%%%%%%%%%%%%%%%%%%%%%%%%%%%%%%%%%%%%%%%%%%%%%%%

\pagestyle{scrheadings}

\section{Lokal geringte R�ume}

\subsection{Garben}

\begin{definition}[Pr�garbe]
	Sei $X$ ein topologischer Raum. Eine \emph{Pr�garbe} $\F$ auf $X$
	ist eine Zuordnung
	$$\F: U\mapsto \F(U) \,,$$
	die jedem offenen $U\subset X$ eine abelsche Gruppe
	$\F(U)$ zuordnet, zusammen mit Homomorphismen
	$$\rho_{UV}: \F(U) \to \F(V)$$
	f�r jedes Paar $V\subset U$, so dass
	\[
	\begin{tikzcd}
		\F(U) \arrow{r}{\rho_{UV}}
			\arrow[bend right]{rr}{\rho_{UW}}& \F(V) \arrow{r}{\rho_{VW}}& \F(W)
	\end{tikzcd}
	\]
	kommutiert.
	
	Wir nennen $\rho_{UV}$ \emph{Restriktion}, schreiben
	meist $\tikzmark{s\rest V} := \rho_{UV}(s)$.
	
	Man nennt $s\in \F(U)$ auch \emph{Schnitt �ber $U$}.
\end{definition}
	
\tikzmargin{north, above=1cm}{\color{red}
Bei mir steht hier im Skript $s\rest U$. Offenbar ein Fehler!?}

\begin{beispiel}
	$$\cal C_X^\circ: U \mapsto \cal C_X^\circ (U) := 
		\{f: U\to \R \mid \text{ $f$ stetig}\} $$
	mit $\rho_{VU}: \cal C_X^\circ(V) \mapsto \cal C_X^\circ(U)$,
	$f \mapsto f\rest U$.
\end{beispiel}

\begin{bemerkung}
	Ist $\kat{Ab}$ die Kategorie der abelschen Gruppen und
	\[
		\kat{Top}_X := 
		\begin{cases}
		\Obj: U\subset X \text{ offen}\\
		\Morph: \Hom(U,V) = 
			\begin{cases}
				\emptyset & U\not\subset V,\\
				U\to V & U\subset V,
			\end{cases}
		\end{cases}
	\]
	dann ist eine Pr�garbe gerade ein kontravarianter Funktor
	\[
		\F: \funcdef{\kat{Top}_X & \to & \kat{Ab}\\
			U & \mapsto & \F(U)\\
			(U\to V) & \mapsto & (\F(V)\to \F(U)).}
	\]
	Oder anders ausgedr�ckt: Es ist
	\[
		\F: \funcdef{\kat{Top}_X\op & \to & \kat{Ab}\\
			U & \mapsto & \F(U)\\
			(V\to U) & \mapsto & (\F(V)\to \F(U)).}
	\]
	ein kovarianter Funktor.
\end{bemerkung}

\begin{definition}[Morphismus von Pr�garben]
	Ein \emph{Morphismus von Pr�garben} $\F \xto{\phi} \G$ auf $X$ ist
	eine nat�rliche Transformation der Funktoren $\F$ und $\G$, d.h.
	f�r alle $U\subset X$ offen gibt es einen Morphismus
	$\F(U) \xto{\phi_U} \G(U)$, so dass f�r $U\subset V$
	\[
		\begin{tikzcd}
			\F(U) \arrow{r}{\phi_U} & \G(U)\\
			\F(V) \arrow{r}{\phi_U} \arrow{u} & \G(V) \arrow{u}
		\end{tikzcd}
	\] 
	kommutiert.
\end{definition}


\begin{definition}[Garbe]
	Eine Pr�garbe $\F$ auf $X$ hei�t \emph{Garbe}, falls gilt:
	Ist $U\subset X$ offen und $U=\bigcup_{i\in I} U_i$ f�r 
	offene $U_i\subset X$, so gilt
	\begin{enumerate}
	  \item Ist $s\in \F(U)$ und $s\rest{U_i} = 0$ f�r alle $i\in I$,
	  	so ist $s=0\in \F(U)$.
	  \item Sind $s_i \in \F(U_i)$ gegeben, mit
	  	$$s_i \rest{U_i\cap U_j} = s_j \rest{U_i\cap U_j}\qquad \forall i,j,$$
	  	so existiert ein $s\in \F(U)$ mit
	  	$$s_i = s \rest{U_i}\qquad\forall i.$$
	\end{enumerate}
\end{definition} 

\begin{bemerkung}
	$\F$ ist eine Garbe, genau dann, wenn die folgende Sequenz abelscher
	Gruppen exakt ist:
	\[	\everymath{\displaystyle}
		\begin{tikzcd}[row sep=tiny, column sep=small]
		0 \rar & \tikzmark[1]{\F(U)} \rar & 
			\tikzmark[2]{\prod_{i\in I} \F(U_i)} \rar
			& \prod_{(i,j)\in I^2} \F(U_i\cap U_j)\\
		& s \rar[mapsto] & \left(s\rest{U_i}\right)_{i\in I}\\
		&& (s_i)_{i\in I} \rar[mapsto] &
			\left(s_i\rest{U_i\cap U_j} - 
			s_j\rest{U_i\cap U_j}\right)_{(i,j)\in I^2}  
		\end{tikzcd}
	\]
	
	Exaktheit an \tikzarrow[1]{south, mark above}{dieser} Stelle ist �quivalent 
	zu Eigenschaft 1.
	Exaktheit \tikzarrow[2]{south, mark above}{hier} zu Eigenschaft 2.
\end{bemerkung}

\begin{beispiel}
	Sei $M$ eine $\mathrm C^\infty$ Mannigfaltigkeit, so ist
  	\[ \cal C^\infty_M: U \mapsto
  		\cal C^\infty_M(U) := \{f:U\to \R \mid f\in \mathrm C^\infty(U)\}
  	\]
  	eine Garbe.
\end{beispiel}

\begin{beispiel}
	Sei $M$ eine $\C$ Mannigfaltigkeit, so ist
  	\[ \cal O_M: U \mapsto
  		\cal O_M(U) := \{f:U\to \C \mid f \text{ holomorph}\}
  	\]
  	eine Garbe. F�r $M = \C$ haben wir zus�tzlich die Garbe
  	\[ \cal O_\C^\times: U \mapsto
  		\cal O_\C^\times(U) := \{f:U\to \C^\times \mid f \text{ holomorph}\},
  	\]
  	(wobei die Gruppenverkn�pfung multiplikativ zu lesen ist).
  	Dies liefert uns einen Morphismus von (Pr�)garben
  	\[ \O \to \O_C^\times,\ f \mapsto \exp(f).\]
  	Betrachte nun die Pr�garbe
  	\[\cal H := \tikzmark{\im^\text{naiv}}(\exp): U \mapsto \im(\exp_U) = 
  		\{\exp \circ f: U\to \C \mid f:U\to \C \text{ holomorph}\}.\]
  	Dies ist \emph{keine} Garbe:
  	Betrachte die Scheibe 
  	\[U = \{z\in \C \mid \tfrac{1}{2} < |z| < \tfrac{3}{2}\}\]
  	zerlegt in die beiden offenen Teilmengen
  	\begin{align*}
  		U_1 &= \{z \in U \mid \Re z > -\varepsilon\}\\
  		U_2 &= \{z \in U \mid \Re z < \varepsilon\}
  	\end{align*}
  	mit $U = U_1 \cup U_2$ f�r ein $\varepsilon > 0$ beliebig. F�r $i=1,2$
  	ist 
  	$(z: U_i \to \C, z\mapsto z) \in \cal H(U_i)$,
  	da sich der komplexe Logarithmus auf beiden $U_i$ problemlos definieren
  	l�sst.
  	Ferner ist auch
 	\[ (z: U_1 \to \C) \rest{U_1\cap U_2} = 
 		(z: U_2 \to \C) \rest{U_1 \cap U_2},\]
 	erf�llt, jedoch kommen diese nicht von einem gemeinsamen Schnitt
 	da
 	\[ (z: U\to \C) \notin \cal H(U). \]
\end{beispiel}

\tikzmargin{north}{Warum steht hier naiv??}


\begin{definition}
	F�r einen topologischen Raum $X$ bezeichne
	\begin{align*}
		\PSh_X & := \text{die Kategorie der Pr�garben auf $X$},\\
		\Sh_X & := \text{die Kategorie der Garben auf $X$, wobei
			} \Hom_{\Sh_X}(\F,\G) := \Hom_{\PSh_X}(\F,\G)
	\end{align*} 
\end{definition}

\begin{bemerkung}
	Man hat den Inklusionsfunktor
	\[ \iota: \Sh_X \to \PSh_X,\ \F \mapsto \F\]
\end{bemerkung}

\begin{definition}[Halm]
	Ist $\F$ eine (Pr�)Garbe auf $X$ und $x_0 \in X$, so hei�t
	\[ \F_{x_0} := \varinjlim_{x_0 \in U \subset X\text{ offen}} \F(U)
		 = \coprod_{U\subset X\text{ offen}} \F(U) \Big/ \sim\] 
	mit 
	\[ s \sim t \  :\Leftrightarrow \  
		\exists W \subset X \text{ offen}:\ x_0 \in W \subset U \cap U'
		\text{ und } s\rest W = t \rest W
	\]
	f�r $s \in \F(U)$, $t \in \F(U')$ der \emph{Halm von $\F$ bei $x_0$}.
	
	Die Elemente $[s] \in \F_{x_0}$ hei�en \emph{Keime von Schnitten bei $x_0$}.  
\end{definition}

\begin{beispiel}
	$(\cal C^\infty_M)_{x_0} = \{ [f: U \xto{C^\infty} \R]\mid
  	f\sim g \Leftrightarrow \exists W\subset M\text{ offen}, x_0 \in W
  	\text{ mit } f\rest W = g\rest W\}$
\end{beispiel}
\begin{beispiel}
	\begin{align*}
	  	O_{\C,x_0} &= \{[f:U \xto{\text{hol}} \C] \mid x_0 \in U\}\\
	  	&= \{\sum_{n=0}^\infty a_n(x-x_0)^n \mid \text{Reihe hat positiven 
	  	Konvergenzradius}\}\\
	  	&:= \C\{x-x_0\}
	\end{align*}
\end{beispiel}

\begin{definition}[push-forward]
	Ist $f:X \to Y$ stetig und $\F$ eine Garbe auf $X$, so ist durch
	\[ f_\ast \F: V \mapsto \F(f^{-1}(V))\]
	f�r $V\subset Y \text{ offen}$ eine Garbe definiert, 
	der \emph{push-forward von $\F$}.
\end{definition}



\subsection{Lokal geringte R�ume}
Betrachte nun 
\[\Ring := \text{ Kategorie der kommuativen Ringe mit $1$}\]
und entsprechend Garben
\[\F:\Top_X\op \to \Ring.\]

\begin{definition}[lokaler Ring]
	Sei  $R$ ein Ring. Dann hei�t $R$ \emph{lokal}, wenn $R$ genau ein
	maximales Ideal besitzt.
\end{definition}

\begin{beispiel}
	$\Z_{(p)} := \left\{\frac{a}{b} \in \Q \mid p \nmid b\right\}$
\end{beispiel}

\begin{bemerkung}
	Ist $R$ lokaler Ring und $\m \ideal R$ das maximale Ideal,
	so ist $R \setminus \m = R^\times$.
\end{bemerkung}

\begin{beispiel}
	Sei $M$ eine $C^\infty$ Mannigfaltigkeit und $x_0 \in M$.
	Dann ist $\cal C^\infty_{M,x_0}$ ein lokaler Ring, denn
	\[
		\cal C^\infty_{M,x_0} \setminus \big(\cal C^\infty_{M,x_0}\big)^\times
		= \{[f:U\xto{C^\infty} \R] \mid x_0 \in U\text{ mit } f(x_0) = 0\}
		=: \m,
	\]
	da $[f]$ eine Einheit ist, genau dann, wenn $f(x_0) \neq 0$: 
	Ist $f: U\xto{C^\infty} \R$ mit $f(x_0) \neq 0$, so existiert
	$W\subset U$ offen, $x_0\in W$ mit $f(x) \neq 0$ f�r alle $x\in W$.
	Damit folgt
	\[
		\left[\frac{1}{f}: W \to \R,\ x\mapsto \frac{1}{f(x)}\right]
		\in \cal C^\infty_{M,x_0}
	\]
	ist Inverses zu $[f]$.
	Zudem ist $\m$ ein Ideal.
\end{beispiel}

\begin{definition}[lokal geringter Raum]
	Ein \emph{lokal geringter Raum} ist ein Paar $(X, \O_X)$ bestehend aus:
	\begin{itemize}
	  \item einem topologischen Raum $X$ und
	  \item einer Garbe $\O_X$ auf $X$ von Ringen,
	\end{itemize}
	so dass $\O_{X,x_0}$ f�r alle $x_0\in X$ ein lokaler Ring ist.
	
	Man nennt $\O_X$ die \emph{Strukturgarbe von $(X,\O_X)$}. Ist
	$x_0\in X$, so hat man das maximale Ideal
	$\m_{x_0} \ideal \O_{X,x_0}$.
	
	Der K�rper 
	\[\kappa(x_0) := \O_{X,x_0} \big/ \m_{x_0} \]
	hei�t \emph{Restklassenk�rper von $x_0$ in $(X,\O_X)$}.
\end{definition}

\begin{beispiel}
	Sei $M$ eine $C^\infty$-Mannigfaltigkeit und $x_0 \in M$,
	so ist $\kappa(x_0) = \R$.
\end{beispiel}

\begin{definition}[lokale Ringhomomorphismen]
	Sind $R,S$ lokale Ringe mit den maximalen Idealen
	$\m_R \ideal R$, $\m_S \ideal S$, so hei�t der Ringhomomorphismus
	$\varphi: R\to S$ \emph{lokal},
	falls
	\[\varphi\inv (\m_S) = \m_R .\]
	�quivalent l�sst sich fordern, dass
	\[\varphi(\m_R) \subset \m_S.\]
\end{definition}

\begin{definition}[Morphismus lokal geringter R�ume]
	\label{def:morphismus lokal geringter raume}
	Ein \emph{Morphismus $f:(X,\O_X) \to (Y,\O_Y)$ lokal geringter R�ume}
	ist ein Paar $(f,f\fis)$ bestehend aus
	\begin{align*}
		f:  X &\to Y \text{ stetig},\\
		f\fis:  \O_Y &\to f_\ast\O_X \text{ Morphismus von Garben auf $Y$},
	\end{align*}
	so dass der von $f\fis$ induzierte Ringhomomorphismus f�r
	$x_0 \in X$, $y_0:= f(x_0) \in Y$
	\[f_{x_0}\fis:
		\funcdef{ \O_{Y,y_0} & \to & \O_{X,x_0}\\~ 
			[s] & \mapsto & [f_U\fis(s)]
		}
	\]
	f�r $s\in \O_Y(U)$ und $y_0\in U$  ein lokaler Ringhomomorphismus ist.
\end{definition}

\begin{bemerkung}
	In \cref{def:morphismus lokal geringter raume} ist $f_{x_0}\fis$
	wohldefiniert:
	
	Sei $[s] = [t] \in \O_{Y,y_0}$, d.h. es existiert $W\subset Y$ offen mit
	$y_0\in W$ und $s\rest W = t\rest W \in \O_Y(W)$.
	Betrachte nun $f_U\fis (s) \in \O_X(f\inv(U))$ f�r 
	$s\in \O_Y(U)$, $U\subset Y$, $y_0\in U$ und analog 
	$f_V\fis (t) \in \O_X(f\inv(V))$ f�r 
	$t\in \O_Y(V)$, $V\subset Y$, $y_0\in V$.
	Da $f\fis$ ein Garbenmorphismus ist, kommutiert damit folgendes Diagramm:
	\[
		\begin{tikzcd}[row sep=large, column sep=large]
		s \dar[mapsto] \symb{\in}
			&[-1cm] \O_Y(U) \rar{f_U\fis} \dar[swap]{\rest W} 
			& \O_X(f\inv(U)) \dar{\rest{f\inv(W)}} \symb{\ni}
			& \dar[mapsto] f_U\fis(s)\\
		s\rest W = t\rest W \symb{\in}
			& \O_Y(W) \rar{f_W\fis}				
			& \O_X(f\inv(W)) \symb{\ni}
			& f_U\fis(s)\rest{f\inv(W)} = f_V\fis(t)\rest{f\inv(W)}\\
		t \uar[mapsto] \symb{\in}
			& \O_Y(V) \rar{f_W\fis} \uar{\rest W} 
			& \O_X(f\inv(V)) \uar[swap]{\rest{f\inv(W)}} \symb{\ni}
			& \uar[mapsto] f_V\fis(t)\\
		\end{tikzcd}
	\]
\end{bemerkung}

\section{Affine Schemata}

\subsection{$\Spec A$ als topologischer Raum}

Sei im Folgenden $A$ ein kommuativer Ring mit $1$ und 
$\Spec A := \{\p \ideal A \mid \p \text{ Primideal}\}$.

\begin{definition}[Zariski Topologie]
	Ist $\a \ideal A$, ein Ideal, setze
	\[
		V(\a) := \{\p \in \Spec A \mid \a \subseteq \p \} \subseteq \Spec A\,.
	\]
	Dann ist durch
	\[
		\cal T := \{ U \subseteq \Spec A \mid
			\exists\ \a \ideal A:\ U = \Spec A \setminus V(\a)\}
	\]
	eine Topologie auf $\Spec A$ definiert. Sie hei�t \emph{Zariski-Topologie}.
\end{definition}

\begin{bemerkung}
	Die abgeschlossenen Teilmengen $M \subset \Spec A$ sind genau die 
	$M = V(\a)$ f�r ein $\a \ideal A$.
\end{bemerkung}



\pagebreak


\end{document}