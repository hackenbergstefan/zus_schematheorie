%% Basierend auf einer TeXnicCenter-Vorlage von Mark M�ller
%%%%%%%%%%%%%%%%%%%%%%%%%%%%%%%%%%%%%%%%%%%%%%%%%%%%%%%%%%%%%%%%%%%%%%%

% W�hlen Sie die Optionen aus, indem Sie % vor der Option entfernen  
% Dokumentation des KOMA-Script-Packets: scrguide

%%%%%%%%%%%%%%%%%%%%%%%%%%%%%%%%%%%%%%%%%%%%%%%%%%%%%%%%%%%%%%%%%%%%%%%
%% Optionen zum Layout des Artikels                                  %%
%%%%%%%%%%%%%%%%%%%%%%%%%%%%%%%%%%%%%%%%%%%%%%%%%%%%%%%%%%%%%%%%%%%%%%%
\documentclass[%
%a5paper,							% alle weiteren Papierformat einstellbar
%landscape,						% Querformat
10pt,								% Schriftgr��e (12pt, 11pt (Standard))
%BCOR1cm,							% Bindekorrektur, bspw. 1 cm
%DIVcalc,							% f�hrt die Satzspiegelberechnung neu aus
%											  s. scrguide 2.4
%twoside,							% Doppelseiten
%twocolumn,						% zweispaltiger Satz
halfparskip*,				% Absatzformatierung s. scrguide 3.1
%headsepline,					% Trennline zum Seitenkopf	
%footsepline,					% Trennline zum Seitenfu�
%titlepage,						% Titelei auf eigener Seite
%normalheadings,			% �berschriften etwas kleiner (smallheadings)
%idxtotoc,						% Index im Inhaltsverzeichnis
%liststotoc,					% Abb.- und Tab.verzeichnis im Inhalt
%bibtotoc,						% Literaturverzeichnis im Inhalt
%abstracton,					% �berschrift �ber der Zusammenfassung an	
%leqno,   						% Nummerierung von Gleichungen links
%fleqn,								% Ausgabe von Gleichungen linksb�ndig
%draft								% �berlangen Zeilen in Ausgabe gekennzeichnet
DIV = 14,
%monochrome						% schwarz wei� output
]
{scrartcl}

%\pagestyle{empty}		% keine Kopf und Fu�zeile (k. Seitenzahl)
%\pagestyle{headings}	% lebender Kolumnentitel  


%% Deutsche Anpassungen %%%%%%%%%%%%%%%%%%%%%%%%%%%%%%%%%%%%%
\usepackage{amsmath,amssymb}
\usepackage[ngerman]{babel}
\usepackage[T1]{fontenc}
\usepackage[ansinew]{inputenc}

\usepackage{lmodern} %Type1-Schriftart f�r nicht-englische Texte

\usepackage{hyperref}

\usepackage{xspace}


\usepackage{xcolor}
\usepackage{tikz}
\usetikzlibrary{matrix,arrows,fadings,decorations.markings}
\usepackage{tikz-cd2}

%% define colors %%%%%%%%%%%%%%%%%%%%%%%%%%%%%%%%%%%%%%%%%%%%
\colorlet{mycolor}{blue!80!black}
\colorlet{col1}{mycolor}
\colorlet{col1shade1}{mycolor!5}
\colorlet{col1shade2}{mycolor!50}
\colorlet{col2}{purple!80}
\colorlet{col2shade1}{col2!5}
\colorlet{col2shade2}{col2!50}

%% tikz setup %%%%%%%%%%%%%%%%%%%%%%%%%%%%%%%%%%%%%%%%%%%%%%%
\usetikzlibrary{decorations.shapes, shapes.geometric}
\tikzset{generic point/.style=
	{star, star points=20, minimum size=#1, inner sep=0pt, outer sep=0pt}
}


%% Packages tikztitle and tikztheorem %%%%%%%%%%%%%%%%%%%%%%%
\usepackage[color=mycolor, style=elegant, withheadings=true,
	backgroundcmd=titlepagebackground]{tikztitle}


\usepackage{tikztheorems}
\newtikztheorem[
	style=elegantbreak,
	color=mycolor,
	font header=\normalfont\sffamily\bfseries,
	counter zero=section
	]{satz}{Satz}
	
\newtikztheorem[
	style=elegantbreak,
	color=mycolor,
	font header=\normalfont\sffamily\bfseries,
	font body=\normalfont,
	counter parent=satz
	]{definition}{Definition}
 
\newtikztheorem[
	style=elegantinline,
	color=mycolor,
	font header=\normalfont\sffamily\bfseries,
	counter parent=satz
	]{lemma}{Lemma}
	
\newtikztheorem[
	style=elegantinline,
	color=mycolor,
	font header=\normalfont\sffamily\bfseries,
	counter parent=satz
	]{korollar}{Korollar}
	
\newtikztheorem[
	style=plain,
	color=mycolor,
	font header=\normalfont\sffamily\bfseries,
	font body=\normalfont,
	counter parent=satz
	]{beispiel}{Beispiel}
	
\newtikztheorem[
	style=plain,
	color=mycolor,
	font header=\normalfont\sffamily\bfseries,
	font body=\normalfont,
	counter parent=satz
	]{bemerkung}{Bemerkung}
	
	
\newtikztheorem[
	style=plain,
	color=mycolor,
	font header=\normalfont\sffamily\bfseries,
	font body=\normalfont,
	counter parent=satz
	]{hilfslemma}{Hilfslemma}
	
\newtikztheorem[
	style=elegantbreak,
	color=col2,
	font header=\normalfont\sffamily\bfseries,
	font body=\normalfont,
	nocounter=true
	]{uebung}{�bung}
%%%%%%%%%%%%%%%%%%%%%%%%%%%%%%%%%%%%%%%%%%%%%%%%%%%%%%%%%%%%%


\usepackage{tikzmargin}

\usepackage{array, enumitem}

%% Packages f�r Grafiken & Abbildungen %%%%%%%%%%%%%%%%%%%%%%
\usepackage{graphicx} %%Zum Laden von Grafiken
%\usepackage{subfig} %%Teilabbildungen in einer Abbildung
% \usepackage[ngerman, nameinlink]{cleveref}

%% Math abbreviations %%%%%%%%%%%%%%%%%%%%%%%%%%%%%%%%%%%%%%%
\let\bb\mathbb
\let\cal\mathcal
\let\fr\mathfrak
\newcommand{\F}{\cal{F}}
\newcommand{\G}{\cal{G}}
\renewcommand{\O}{\cal{O}}
\newcommand{\R}{\bb{R}}
\newcommand{\N}{\bb{N}}
\newcommand{\C}{\bb{C}}
\newcommand{\Q}{\bb{Q}}
\newcommand{\Z}{\bb{Z}}
\newcommand{\A}{\bb{A}}
\newcommand{\rest}[1]{\big|_{#1}}
\newcommand{\inv}{^{-1}}
\newcommand{\fis}{^{\#}}
\newcommand{\kat}[1]{\mathbf{#1}}
\newcommand{\PSh}{\kat{PSh}}
\newcommand{\Sh}{\kat{Sh}}
\newcommand{\Ring}{\kat{Ring}}
\newcommand{\Top}{\kat{Top}}
\newcommand{\Sch}{\kat{Sch}}
\newcommand{\AffSch}{\kat{Sch}^\kat{aff}}
\newcommand{\op}{^\mathrm{op}}
\newcommand{\m}{\fr{m}}
\newcommand{\p}{\fr{p}}
\newcommand{\q}{\fr{q}}
\renewcommand{\a}{\fr{a}}
\renewcommand{\b}{\fr{b}}
\newcommand{\ideal}{\vartriangleleft}
\DeclareMathOperator{\Obj}{Obj}
\DeclareMathOperator{\Morph}{Morph}
\DeclareMathOperator{\Hom}{Hom}
\DeclareMathOperator{\im}{im}
\DeclareMathOperator{\Spec}{Spec}
\DeclareMathOperator{\Nil}{Nil}
\DeclareMathOperator{\Quot}{Quot}
\newcommand{\funcdef}[1]{%
	\begin{array}[t]{>{\displaystyle}r>{\displaystyle}c>{\displaystyle}l}
	#1\end{array}}
\let\xto\xrightarrow

\newcommand{\obda}{{\small oBdA}\xspace}
\newcommand{\Obda}{{\small OBdA}\xspace}

\newenvironment{description sf}{%
	\begin{description}[font=\normalfont\sffamily]}
	{\end{description}}
\newenvironment{description mathquote}{%
	\renewcommand{\descriptionlabel}[1]{\glqq$##1$\grqq}
	\begin{description}[font=\normalfont]}
	{\end{description}}
%%%%%%%%%%%%%%%%%%%%%%%%%%%%%%%%%%%%%%%%%%%%%%%%%%%%%%%%%%%%%


%% Bibliographiestil %%%%%%%%%%%%%%%%%%%%%%%%%%%%%%%%%%%%%%%%%%%%%%%%%%
%\usepackage{natbib}

\begin{document}

%% Titlepage Background cmd (lrbox must be after begin{document}) %%%%%
\tikzfading[name=fade down,
	top color=transparent!90,
	bottom color=transparent!0]
\def\titlepagebackground{
   	\begin{scope}[transform shape, rotate=40, scale=5, opacity=0.5,
   		]
 		\node[scope fading=fade down] at (current page.center)
		{\usebox{\diagbox}};
   	\end{scope}
}
\newsavebox{\diagbox}
\begin{lrbox}{\diagbox}
\begin{minipage}{\textwidth}
\[
	\everymath{\displaystyle}
	\begin{tikzcd}[row sep=large, column sep=large]
		s \dar[mapsto] \symb{\in}
			&[-1cm] \O_Y(U) \rar{f_U\fis} \dar[swap]{\rest W} 
			& \O_X(f\inv(U)) \dar{\rest{f\inv(W)}} \symb{\ni}
			& \dar[mapsto] f_U\fis(s)\\
		s\rest W = t\rest W \symb{\in}
			& \O_Y(W) \rar{f_W\fis}				
			& \O_X(f\inv(W)) \symb{\ni}
			& f_U\fis(s)\rest{f\inv(W)} = f_V\fis(t)\rest{f\inv(W)}\\
		t \uar[mapsto] \symb{\in}
			& \O_Y(V) \rar{f_W\fis} \uar{\rest W} 
			& \O_X(f\inv(V)) \uar[swap]{\rest{f\inv(W)}} \symb{\ni}
			& \uar[mapsto] f_V\fis(t)\\
	\end{tikzcd}
\]
\end{minipage}
\end{lrbox}
%%%%%%%%%%%%%%%%%%%%%%%%%%%%%%%%%%%%%%%%%%%%%%%%%%%%%%%%%%%%%%%%%%%%%%%

\pagestyle{empty}
%%%%%%%%%%%%%%%%%%%%%%%%%%%%%%%%%%%%%%%%%%%%%%%%%%%%%%%%%%%%%%%%%%%%%%%
%% Ihr Artikel                                                       %%
%%%%%%%%%%%%%%%%%%%%%%%%%%%%%%%%%%%%%%%%%%%%%%%%%%%%%%%%%%%%%%%%%%%%%%%

%% eigene Titelseitengestaltung %%%%%%%%%%%%%%%%%%%%%%%%%%%%%%%%%%%%%%%    
%\begin{titlepage}
%Einsetzen der TXC Vorlage "Deckblatt" m�glich
%\end{titlepage}

%% Angaben zur Standardformatierung des Titels %%%%%%%%%%%%%%%%%%%%%%%%
%\titlehead{Titelkopf }
\subject{Vorlesungszusammenfassung}
\title{Schematheorie}
\author[erstellt von]{Stefan Hackenberg \and Maximilian Huber}
%\thanks{Fu�note}			% entspr. \footnote im Flie�text
\date{\today}				% falls anderes, als das aktuelle gew�nscht
\publishers[gelesen im WS 2012/2013 und SS 2013 von]{Prof. Dr. Marco Hien}

%% Widmungsseite %%%%%%%%%%%%%%%%%%%%%%%%%%%%%%%%%%%%%%%%%%%%%%%%%%%%%%
%\dedication{Widmung}

\maketitle 						% Titelei wird erzeugt

%% Zusammenfassung nach Titel, vor Inhaltsverzeichnis %%%%%%%%%%%%%%%%%
%\begin{abstract}
% F�r eine kurze Zusammenfassung des folgenden Artikels.
% F�r die �berschrift s. \documentclass[abstracton].
%\end{abstract}

\KOMAoptions{twoside}
\cleardoublepage

%% Erzeugung von Verzeichnissen %%%%%%%%%%%%%%%%%%%%%%%%%%%%%%%%%%%%%%%
\thispagestyle{plain}
\tableofcontents			% Inhaltsverzeichnis
%\listoftables				% Tabellenverzeichnis
%\listoffigures				% Abbildungsverzeichnis


%% Der Text %%%%%%%%%%%%%%%%%%%%%%%%%%%%%%%%%%%%%%%%%%%%%%%%%%%%%%%%%%%

\pagestyle{scrheadings}

\section{Lokal geringte R�ume}

\subsection{Garben}

\begin{definition}[Pr�garbe]
	Sei $X$ ein topologischer Raum. Eine \emph{Pr�garbe} $\F$ auf $X$
	ist eine Zuordnung
	$$\F: U\mapsto \F(U) \,,$$
	die jedem offenen $U\subset X$ eine abelsche Gruppe
	$\F(U)$ zuordnet, zusammen mit Homomorphismen
	$$\rho_{UV}: \F(U) \to \F(V)$$
	f�r jedes Paar $V\subset U$, so dass
	\[
	\begin{tikzcd}
		\F(U) \arrow{r}{\rho_{UV}}
			\arrow[bend right]{rr}{\rho_{UW}}& \F(V) \arrow{r}{\rho_{VW}}& \F(W)
	\end{tikzcd}
	\]
	kommutiert.
	
	Wir nennen $\rho_{UV}$ \emph{Restriktion}, schreiben
	meist $\tikzmark{s\rest V} := \rho_{UV}(s)$.
	
	Man nennt $s\in \F(U)$ auch \emph{Schnitt �ber $U$}.
\end{definition}
	
\tikzmargin{north, above=1cm}{\color{red}
Bei mir steht hier im Skript $s\rest U$. Offenbar ein Fehler!?}

\begin{beispiel}
	$$\cal C_X^\circ: U \mapsto \cal C_X^\circ (U) := 
		\{f: U\to \R \mid \text{ $f$ stetig}\} $$
	mit $\rho_{VU}: \cal C_X^\circ(V) \mapsto \cal C_X^\circ(U)$,
	$f \mapsto f\rest U$.
\end{beispiel}

\begin{bemerkung}
	Ist $\kat{Ab}$ die Kategorie der abelschen Gruppen und
	\[
		\kat{Top}_X := 
		\begin{cases}
		\Obj: U\subset X \text{ offen}\\
		\Morph: \Hom(U,V) = 
			\begin{cases}
				\emptyset & U\not\subset V,\\
				U\to V & U\subset V,
			\end{cases}
		\end{cases}
	\]
	dann ist eine Pr�garbe gerade ein kontravarianter Funktor
	\[
		\F: \funcdef{\kat{Top}_X & \to & \kat{Ab}\\
			U & \mapsto & \F(U)\\
			(U\to V) & \mapsto & (\F(V)\to \F(U)).}
	\]
	Oder anders ausgedr�ckt: Es ist
	\[
		\F: \funcdef{\kat{Top}_X\op & \to & \kat{Ab}\\
			U & \mapsto & \F(U)\\
			(V\to U) & \mapsto & (\F(V)\to \F(U)).}
	\]
	ein kovarianter Funktor.
\end{bemerkung}

\begin{definition}[Morphismus von Pr�garben]
	Ein \emph{Morphismus von Pr�garben} $\F \xto{\phi} \G$ auf $X$ ist
	eine nat�rliche Transformation der Funktoren $\F$ und $\G$, d.h.
	f�r alle $U\subset X$ offen gibt es einen Morphismus
	$\F(U) \xto{\phi_U} \G(U)$, so dass f�r $U\subset V$
	\[
		\begin{tikzcd}
			\F(U) \arrow{r}{\phi_U} & \G(U)\\
			\F(V) \arrow{r}{\phi_U} \arrow{u} & \G(V) \arrow{u}
		\end{tikzcd}
	\] 
	kommutiert.
\end{definition}


\begin{definition}[Garbe]
	Eine Pr�garbe $\F$ auf $X$ hei�t \emph{Garbe}, falls gilt:
	Ist $U\subset X$ offen und $U=\bigcup_{i\in I} U_i$ f�r 
	offene $U_i\subset X$, so gilt
	\begin{enumerate}
	  \item Ist $s\in \F(U)$ und $s\rest{U_i} = 0$ f�r alle $i\in I$,
	  	so ist $s=0\in \F(U)$.
	  \item Sind $s_i \in \F(U_i)$ gegeben, mit
	  	$$s_i \rest{U_i\cap U_j} = s_j \rest{U_i\cap U_j}\qquad \forall i,j,$$
	  	so existiert ein $s\in \F(U)$ mit
	  	$$s_i = s \rest{U_i}\qquad\forall i.$$
	\end{enumerate}
\end{definition} 

\begin{bemerkung}
	$\F$ ist eine Garbe, genau dann, wenn die folgende Sequenz abelscher
	Gruppen exakt ist:
	\[	\everymath{\displaystyle}
		\begin{tikzcd}[row sep=tiny, column sep=small]
		0 \rar & \tikzmark[1]{\F(U)} \rar & 
			\tikzmark[2]{\prod_{i\in I} \F(U_i)} \rar
			& \prod_{(i,j)\in I^2} \F(U_i\cap U_j)\\
		& s \rar[mapsto] & \left(s\rest{U_i}\right)_{i\in I}\\
		&& (s_i)_{i\in I} \rar[mapsto] &
			\left(s_i\rest{U_i\cap U_j} - 
			s_j\rest{U_i\cap U_j}\right)_{(i,j)\in I^2}  
		\end{tikzcd}
	\]
	
	Exaktheit an \tikzarrow[1]{south, mark above}{dieser} Stelle ist �quivalent 
	zu Eigenschaft 1.
	Exaktheit \tikzarrow[2]{south, mark above}{hier} zu Eigenschaft 2.
\end{bemerkung}

\begin{beispiel}
	Sei $M$ eine $\mathrm C^\infty$ Mannigfaltigkeit, so ist
  	\[ \cal C^\infty_M: U \mapsto
  		\cal C^\infty_M(U) := \{f:U\to \R \mid f\in \mathrm C^\infty(U)\}
  	\]
  	eine Garbe.
\end{beispiel}

\begin{beispiel}
	Sei $M$ eine $\C$ Mannigfaltigkeit, so ist
  	\[ \cal O_M: U \mapsto
  		\cal O_M(U) := \{f:U\to \C \mid f \text{ holomorph}\}
  	\]
  	eine Garbe. F�r $M = \C$ haben wir zus�tzlich die Garbe
  	\[ \cal O_\C^\times: U \mapsto
  		\cal O_\C^\times(U) := \{f:U\to \C^\times \mid f \text{ holomorph}\},
  	\]
  	(wobei die Gruppenverkn�pfung multiplikativ zu lesen ist).
  	Dies liefert uns einen Morphismus von (Pr�)garben
  	\[ \O \to \O_C^\times,\ f \mapsto \exp(f).\]
  	Betrachte nun die Pr�garbe
  	\[\cal H := \tikzmark{\im^\text{naiv}}(\exp): U \mapsto \im(\exp_U) = 
  		\{\exp \circ f: U\to \C \mid f:U\to \C \text{ holomorph}\}.\]
  	Dies ist \emph{keine} Garbe:
  	Betrachte die Scheibe 
  	\[U = \{z\in \C \mid \tfrac{1}{2} < |z| < \tfrac{3}{2}\}\]
  	zerlegt in die beiden offenen Teilmengen
  	\begin{align*}
  		U_1 &= \{z \in U \mid \Re z > -\varepsilon\}\\
  		U_2 &= \{z \in U \mid \Re z < \varepsilon\}
  	\end{align*}
  	mit $U = U_1 \cup U_2$ f�r ein $\varepsilon > 0$ beliebig. F�r $i=1,2$
  	ist 
  	$(z: U_i \to \C, z\mapsto z) \in \cal H(U_i)$,
  	da sich der komplexe Logarithmus auf beiden $U_i$ problemlos definieren
  	l�sst.
  	Ferner ist auch
 	\[ (z: U_1 \to \C) \rest{U_1\cap U_2} = 
 		(z: U_2 \to \C) \rest{U_1 \cap U_2},\]
 	erf�llt, jedoch kommen diese nicht von einem gemeinsamen Schnitt
 	da
 	\[ (z: U\to \C) \notin \cal H(U). \]
\end{beispiel}

\tikzmargin{north}{Warum steht hier naiv??}


\begin{definition}
	F�r einen topologischen Raum $X$ bezeichne
	\begin{align*}
		\PSh_X & := \text{die Kategorie der Pr�garben auf $X$},\\
		\Sh_X & := \text{die Kategorie der Garben auf $X$, wobei
			} \Hom_{\Sh_X}(\F,\G) := \Hom_{\PSh_X}(\F,\G)
	\end{align*} 
\end{definition}

\begin{bemerkung}
	Man hat den Inklusionsfunktor
	\[ \iota: \Sh_X \to \PSh_X,\ \F \mapsto \F\]
\end{bemerkung}

\begin{definition}[Halm]
	Ist $\F$ eine (Pr�)Garbe auf $X$ und $x_0 \in X$, so hei�t
	\[ \F_{x_0} := \varinjlim_{x_0 \in U \subset X\text{ offen}} \F(U)
		 = \coprod_{U\subset X\text{ offen}} \F(U) \Big/ \sim\] 
	mit 
	\[ s \sim t \  :\Leftrightarrow \  
		\exists W \subset X \text{ offen}:\ x_0 \in W \subset U \cap U'
		\text{ und } s\rest W = t \rest W
	\]
	f�r $s \in \F(U)$, $t \in \F(U')$ der \emph{Halm von $\F$ bei $x_0$}.
	
	Die Elemente $[s] \in \F_{x_0}$ hei�en \emph{Keime von Schnitten bei $x_0$}.  
\end{definition}

\begin{beispiel}
	$(\cal C^\infty_M)_{x_0} = \{ [f: U \xto{C^\infty} \R]\mid
  	f\sim g \Leftrightarrow \exists W\subset M\text{ offen}, x_0 \in W
  	\text{ mit } f\rest W = g\rest W\}$
\end{beispiel}
\begin{beispiel}
	\begin{align*}
	  	O_{\C,x_0} &= \{[f:U \xto{\text{hol}} \C] \mid x_0 \in U\}\\
	  	&= \{\sum_{n=0}^\infty a_n(x-x_0)^n \mid \text{Reihe hat positiven 
	  	Konvergenzradius}\}\\
	  	&:= \C\{x-x_0\}
	\end{align*}
\end{beispiel}

\begin{definition}[push-forward]
	Ist $f:X \to Y$ stetig und $\F$ eine Garbe auf $X$, so ist durch
	\[ f_\ast \F: V \subset Y \text{ offen} \mapsto \F(f^{-1}(V))\]
	eine Garbe definiert, der \emph{push-forward von $\F$}.
\end{definition}



\subsection{Lokal geringte R�ume}
Betrachte nun 
\[\Ring := \text{ Kategorie der kommuativen Ringe mit $1$}\]
und entsprechend Garben
\[\F:\Top_X\op \to \Ring.\]

\begin{definition}[lokaler Ring]
	Sei  $R$ ein Ring. Dann hei�t $R$ \emph{lokal}, wenn $R$ genau ein
	maximales Ideal besitzt.
\end{definition}

\begin{beispiel}
	$\Z_{(p)} := \left\{\frac{a}{b} \in \Q \mid p \nmid b\right\}$
\end{beispiel}

\begin{bemerkung}
	Ist $R$ lokaler Ring und $\m \ideal R$ das maximale Ideal,
	so ist $R \setminus \m = R^\times$.
\end{bemerkung}

\begin{beispiel}
	Sei $M$ eine $C^\infty$ Mannigfaltigkeit und $x_0 \in M$.
	Dann ist $\cal C^\infty_{M,x_0}$ ein lokaler Ring, denn
	\[
		\cal C^\infty_{M,x_0} \setminus \big(\cal C^\infty_{M,x_0}\big)^\times
		= \{[f:U\xto{C^\infty} \R] \mid x_0 \in U\text{ mit } f(x_0) = 0\}
		=: \m,
	\]
	da $[f]$ eine Einheit ist, genau dann, wenn $f(x_0) \neq 0$: 
	Ist $f: U\xto{C^\infty} \R$ mit $f(x_0) \neq 0$, so existiert
	$W\subset U$ offen, $x_0\in W$ mit $f(x) \neq 0$ f�r alle $x\in W$.
	Damit folgt
	\[
		\left[\frac{1}{f}: W \to \R,\ x\mapsto \frac{1}{f(x)}\right]
		\in \cal C^\infty_{M,x_0}
	\]
	ist Inverses zu $[f]$.
	Zudem ist $\m$ ein Ideal.
\end{beispiel}

\begin{definition}[lokal geringter Raum]
	Ein \emph{lokal geringter Raum} ist ein Paar $(X, \O_X)$ bestehend aus:
	\begin{itemize}
	  \item einem topologischen Raum $X$ und
	  \item einer Garbe $\O_X$ auf $X$ von Ringen,
	\end{itemize}
	so dass $\O_{X,x_0}$ f�r alle $x_0\in X$ ein lokaler Ring ist.
	
	Man nennt $\O_X$ die \emph{Strukturgarbe von $(X,\O_X)$}. Ist
	$x_0\in X$, so hat man das maximale Ideal
	$\m_{x_0} \ideal \O_{X,x_0}$.
	
	Der K�rper 
	\[\kappa(x_0) := \O_{X,x_0} \big/ \m_{x_0} \]
	hei�t \emph{Restklassenk�rper von $x_0$ in $(X,\O_X)$}.
\end{definition}

\begin{beispiel}
	Sei $M$ eine $C^\infty$-Mannigfaltigkeit und $x_0 \in M$,
	so ist $\kappa(x_0) = \R$.
\end{beispiel}
\section{Affine Schemata}

\subsection{$\Spec A$ als topologischer Raum}

Sei im Folgenden $A$ ein kommuativer Ring mit $1$ und 
$\Spec A := \{\p \ideal A \mid \p \text{ Primideal}\}$.

\begin{definition}[Zariski Topologie]
	Ist $\a \ideal A$, ein Ideal, setze
	\[
		V(\a) := \{\p \in \Spec A \mid \a \subseteq \p \} \subseteq \Spec A\,.
	\]
	Dann ist durch
	\[
		\cal T := \{ U \subseteq \Spec A \mid
			\exists\ \a \ideal A:\ U = \Spec A \setminus V(\a)\}
	\]
	eine Topologie auf $\Spec A$ definiert. Sie hei�t \emph{Zariski-Topologie}.
\end{definition}

\begin{proof}[der Topologie-Eigenschaften]
	\begin{enumerate}
	  \item Zeige: $\emptyset$, $\Spec A$ offen $\Longleftrightarrow$ 
	  	$\Spec A$, $\emptyset$ abgeschlossen.\\
	  	Dazu: $V(A) = \emptyset$, $V((0)) = \Spec A$
	  \item Zeige: $U_1, U_2$ offen $\Rightarrow$ $U_1 \cap U_2$ offen
	  	$\Longleftrightarrow$ $M_1,M_2$ abgeschlossen $\Rightarrow$
	  	$M_1 \cup M_2$ abgeschlossen.\\
	  	Dazu:
	  	$V(\a) \cup V(\fr b) = V(\a \cap \fr b)$
	  \item $(U_i)_{i\in I}$ offen $\Rightarrow$ $\cup_{i\in I} U_i$ offen
	  	$\Longleftrightarrow$ $(M_i)_{i\in I}$ abgeschlossen
	  	$\Rightarrow$ $\cap_{i\in I} M_i$ abgeschlossen.\\
	  	Dazu:
	  	$\cap_{i\in I} V(\a_i) = V(\sum_{i\in I} \a_i)$
	\end{enumerate}
\end{proof}

\begin{bemerkung}
	Die abgeschlossenen Teilmengen $M \subset \Spec A$ sind genau die 
	$M = V(\a)$ f�r ein $\a \ideal A$.
\end{bemerkung}

\begin{beispiel}[$\Spec \Z$]
	F�r $\a \ideal \Z$ ist $\a = (a)$. Falls $a \neq 0,1,-1$ sei
	$a = \pm p_1^{\nu_1} \cdot \dots \cdot p_r^{\nu_r}$ die 
	Primfaktorzerlegung. F�r $p$ Primzahl ist
	\[
		(p) \in V((a)) \Leftrightarrow
		(a) \subseteq (p) \Leftrightarrow
		p \mid a \Leftrightarrow
		p \in \{p_1,\ldots, p_r\}
	\]
	Das bedeutet, die abgeschlossenen Mengen in $\Spec \Z$ sind genau die 
	Mengen $\emptyset$, $\Spec \Z$ und
	$\{(p_1), \ldots, (p_r)\}$ f�r eine endliche Anzahl an Primzahlen.
	
	Insbesondere gilt
	\begin{itemize}
	  \item $\Spec\Z$ ist nicht hausdorffsch.
	  \item $(0) =: \eta \in \Spec\Z$ liegt in \emph{jeder} nichtleeren 
	  	offenen Teilmenge.
	\end{itemize}
\end{beispiel}

\begin{lemma}
	Sei $x \in \Spec A$, so ist der Abschluss $\overline{\{x\}}$ der
	Menge $\{x\}$ in $\Spec A$ gleich
	\[\overline{\{x\}} = V(x).\]
\end{lemma}
\begin{proof}
	\[
		\overline{\{x\}} = 
		\bigcap_{B\subseteq \Spec A \text{ abg.}\atop x\in B} B
		= \bigcap_{\a\ideal A\atop \a \subseteq x}
		= V(x)
	\]
\end{proof}

\begin{bemerkung}
	Beachte, dass
	\[
		\a \subseteq \fr b \quad \Rightarrow\quad
		V(\fr b) \subseteq V(\a)
	\]
\end{bemerkung}

\begin{definition}[abgeschlossener Punkt, generischer Punkt]
	Sei $X$ ein topologischer Raum.
	Ein $x\in X$ hei�t \emph{abgeschlossener Punkt}, wenn
	$\overline{\{x\}} = \{x\}$.
	
	Er hei�t \emph{generischer Punkt}, wenn $\overline{\{x\}} = X$ gilt.
	
	Die Menge der abgeschlossenen Punkte bezeichnen wir mit
	$|X|$.
\end{definition}

\begin{beispiel}
	Sei $A = \C[X,Y]$. 
	\begin{itemize}
	  \item $x = (0) \in \Spec A$ ist generisch.
	  \item $x = (X-\alpha, Y-\beta) \ideal A$ ist abgeschlossen,
	  	da aus $x \ideal A$ maximal $V(x) = \{x\}$ und somit $x$ abgeschlossen
	  	folgt.
	  \item $x = (X) \ideal A$ ist weder abgeschlossen noch generisch.
	\end{itemize}
	Wir k�nnen die bisherigen Ergebnisse in
	\cref{fig:spec c xy} zusammenfassen. 
\end{beispiel}

\begin{figure}
	\caption{$\Spec \C[X,Y]$}
	\label{fig:spec c xy}
	\centering
	\begin{tikzpicture}
		\fill[col1shade1] (-3,-2) rectangle (3,2);
		\node[right, text=col1] 
			at (-2.8,-1.5)
			{$|\Spec\C[X,Y]|$};
		\draw[very thick]
			(-3,0) -- (3,0) node[near end, auto]{$\alpha$}
			(0,-2) -- (0,2) node[near end, auto]{$\beta$};
		\fill[col1]
			(1,1) circle[radius=2pt]
			node[above right] {$(X-\alpha, Y-\beta)$};
		\node[generic point=10pt, fill=black!60,
			label={above right:$(0)$}]
			at (4,0)
			{};
		\draw[line width=4pt, col2shade2, opacity=0.5]
			(0,-2) -- (0,2);
		\node[generic point=5pt, fill=col2shade2,
			label={[text=col2]below:$(X)$}]
			at (0,-2.1)
			{};
	\end{tikzpicture}
\end{figure}


\begin{definition}[basisoffene Menge]
	F�r $f\in A$ nennt man
	\[ D(f) := \Spec A \setminus V((f)) = \{ \p \in \Spec A \mid f \notin \p\}
	\]
	die \emph{zu $f$ geh�rige basisoffene Menge}.
\end{definition}

\begin{lemma}
	\label{lemma:basisoffene mengen sind basis}
	Die Menge $\fr B := \{D(f) \mid f \in A\}$ ist eine Basis der
	Topologie, d.h. jedes offene $U\subseteq \Spec A$ ist eine Vereinigung
	von $D(f) \in \fr B$ und $\fr B$ ist unter endlichen Schnitten 
	abgeschlossen.  
\end{lemma}
\begin{proof}
	Sei $U = \Spec A \setminus V(\a)$ offen und $\p \in U$, so ist
	$\p \notin V(\a)$, also $\a \not\subseteq \p$. Damit existiert
	$f \in \a \setminus \p$ mit $f \notin \p$, also $\p \in D(f)$
	und $f \in \a$. Also $(f) \subseteq \a$ und
	$V(\a) \subseteq V((f))$. Damit folgt $D(f) \subseteq U$.
	
	Zusammenfassend gilt f�r $U\subseteq \Spec A$ offen: $\forall \p \in U$
	$\exists f\p \in A$: $\p \in D(f\p) \subseteq U$.
	Also
	\[ U = \bigcup_{\p \in U} D(f\p)\]
	Ferner folgt mit \cref{lemma:vereinigungen von v sind produkt}
	$D(f) \cap D(g) = D(fg)$.
\end{proof}

\begin{lemma}
	\label{lemma:vereinigungen von v sind produkt}
	F�r $\a, \fr b\ideal A$ gilt
	\[
		V(\a) \cup V(\fr b) = V(\a \cap \fr b) = V(\a \cdot \fr b).
	\]
\end{lemma}
\begin{proof}
	Es ist 
	$\a\fr b \subseteq \a \cap \fr b \subseteq \a, \fr b$.
	Also 
	\[V(\a) \cup V(\fr b) \subseteq V(\a \cap \fr b) 
	\subseteq V(\a\fr b).\]
	Angenommen $V(\a) \cup V(\fr b) \subsetneq V(\a\fr b)$, 
	d.h. $\exists \p \in V(\a \fr b) \setminus \big(V(\a) \cup V(\fr b)\big)$,
	also $\a\fr b \subseteq \p$ aber nicht
	$\a,\fr b \not \subseteq \p$.
	Also existiert $s \in \a \setminus \p$ und $t\in\fr b\setminus \p$.
	Damit ist $st \in \a\fr b \setminus \p$.
	Dies ist ein Widerspruch, da $\p$ ein Primideal ist.
	Folglich herrscht Gleichheit in obiger Inklusionskette.
\end{proof}

\begin{definition}[Radikal]
	F�r $\a \ideal A$ hei�t
	\[
		\sqrt \a := \{ f\in A \mid \exists n \in \N:\ f^n\in \a\}
	\]
	\emph{Radikal} von $\a$.
\end{definition}

\begin{lemma}
	\label{lemma:radikal ist ideal}
	$\sqrt a \ideal A$.
\end{lemma}
\begin{proof}
	\begin{itemize}
	  \item $0\in \sqrt{\a}$ \checkmark
	  \item Sei $f \in \sqrt \a$, $r\in A$. Dann
	  	$f^n \in \a$, $r\in A$. Also 
	  	$(rf)^n \in \a$ und damit $rf\in \sqrt\a$.
	  \item $f,g\in \sqrt\a$ mit $f^n \in \a$, $g^m \in \a$.
	  	\begin{align*}
	  		(f+g)^{n+m-1} &= \sum_{i=0}^{n-1} \binom{n+m-1}{i} f^i g^{n+m-1-i}
	  			+ \sum_{i=n}^{n+m-1} \binom{n+m-1}{i}
	  				f^i g^{n+m-1-i}\\
  				&= \left( \sum_{i=0}^{n-1} \binom{n+m-1}{i} 
  					f^i g^{n-1-i}\right) g^m
  					+  \left(\sum_{i=n}^{n+m-1} \binom{n+m-1}{i}
  						f^i g^{m-1-i}\right) f^n
	  	\end{align*}
	  	Da $g^m$ und $f^n$ jeweils in $\a$ liegen, ist auch die Summe dort.
	\end{itemize}
\end{proof} 

\begin{definition}[Radikalideal (radiziell)]
	Ein Ideal $\fr b \ideal A$ hei�t \emph{Radikalideal (radiziell)},
	falls
	\[\sqrt \fr b = \fr b.\]
\end{definition}

\begin{bemerkung}
	Es gilt $\sqrt{\sqrt \a} = \sqrt\a$.
\end{bemerkung}

\begin{lemma}
	\label{lemma:radikal ist schnitt}
	F�r $\a \ideal A$ gilt
	\[
		\sqrt\a = \bigcap_{\p\in V(\a)} \p
	\]
\end{lemma}
\begin{proof}
	  \newcommand{\bmax}{\b_\text{max}}
	\begin{itemize}
	  \item["`$\subseteq$"']
	  	Sei $f \in \sqrt\a$, $f^n \in \a$. Ist $\p \in V(\a)$, d.h.
	  	$\a \subseteq\p$. Also
	  	$f^n \in \p$ und da $\p$ prim, folgt $f\in \p$.
	  \item["`$\supseteq$"']
	  	Ist $f\notin \sqrt\a$, so zu zeigen, dass 
	  	$f \notin \cap_{\p\in V(\a)} \p$.
	  	Sei also 
	  	$f^n \notin \a$ f�r alle $n\in \N$.
	  	
	  	Betrachte
	  	\[ M := \{\b \ideal A\mid a\subseteq \b,
	  		f^n \notin \b \forall n\in \N\},
	  	\]
	  	so gilt
	  	\begin{itemize}
	  	  \item $\a \in M$,
	  	  \item $M$ ist angeordnet durch "`$\subseteq$"',
	  	  \item ist $(\b_i)_{i\in I}$ eine total geordnete Teilmenge,
	  	  	so ist $\b:= \cup_{i\in I} \b_i \ideal A$ mit $\b \in M$.
	  	\end{itemize}
	  	Damit hat $M$ mit dem Lemma von Zorn ein maximales Element
	  	$\bmax \in M$.
	\end{itemize}
	Nun sei behauptet, dass $\bmax \ideal A$ ein Primideal ist.
	Dazu sei $xy \in \bmax$, wobei wir annehmen, dass 
	$x,y \notin \bmax$.
	Betrachte
	$\bmax \subsetneq (x) + \bmax$, was ein Ideal in $A$ ist, aber nicht 
	in $M$ liegt. Analog k�nnen wir dies von $(y) + \bmax$ sagen. Damit
	existieren $n,m \in \N$ mit
	\[
		f^n \in (x) + \bmax
		\qquad
		f^m \in (y) + \bmax.
	\]
	Ergo ist
	\[
		f^{n+m} \in
			(x)\bmax + (y)\bmax + \bmax\bmax + (xy),
	\]
	wobei jeder Summand Teilmenge von $\bmax$ ist und wir folgern
	$f^{n+m} \in \bmax \in M$, wodurch man den Widerspruch erh�lt.
	
	Damit ist $\bmax \in V(\a)$ und $f\notin \bmax$. 
\end{proof}

\begin{satz}
	\label{satz:v und radikal}
	F�r $\a, \b \ideal A$ gilt
	\[ V(\a) \subseteq V(\b) \quad\Leftrightarrow\quad
		\b \subseteq \sqrt\a.
	\]
	Insbesondere gilt
	\[ V(\a) = V(\b) \quad\Leftrightarrow\quad
		\b = \sqrt\a.
	\]
\end{satz}
\begin{proof}
	\begin{itemize}
	  \item["`$\Leftarrow$"']
	  	Aus $V(\a) \subseteq V(\b)$ folgt
	  	\[
	  		\bigcap_{\p \in V(\a)} \p 
	  		\supseteq \bigcap_{\p\in V(\b)} \p
	  	\]
	  	und mit \cref{lemma:radikal ist schnitt}
	  	folgt $\sqrt \a \supseteq \sqrt\b \supseteq \b$.
	  \item["`$\Rightarrow$"']
	  	Aus $\b \subseteq \sqrt\a$, d.h.
	  	$\b \subseteq \cap_{\p\in V(\a)} \p$, folgt
	  	$\b \subseteq \p$ f�r alle $\p\in V(\a)$.
	  	Also $\p \in V(\a)$.
	\end{itemize}
\end{proof}


\pagebreak
\section{Beispiele}

\subsection{$\Spec \Z$}

Jeder Ring $A$ hat einen eindeutigen Homomorphismus
\[
	\funcdef{
		\Z & \to & A\\
		 1 & \mapsto & 1\\
		 z & \mapsto & \begin{cases} 1 + 1 + \ldots + 1 & z > 0\\
		 	0 & z = 0\\
		 	-1 -1 - \ldots- 1 & z < 0
		 \end{cases}.}
\]
$\Z$ ist daher ein \emph{initiales Objekt} in der Kategorie $\Ring$.

Wir haben daher einen eindeutigen Morphismus $\Spec A \to \Spec \Z$ von
affinen Schemata. $\Spec \Z$ ist ein \emph{finales Objekt} in
der Kategorie $\affSch$.

Ferner können wir zusammenfassen
\paragraph{Offene Mengen}
	$\emptyset \neq U\subseteq \Spec \Z$ offen 
	$\Leftrightarrow$ $U = \Spec \Z \setminus \{(p_1),\ldots,(p_r)\}$
	
\paragraph{Basisoffene Mengen}
	$D(f) = \{\p \in \Spec\Z \mid f \notin \p\} = 
	\Spec\Z \setminus \{(p_1),\ldots,(p_r)\}$ für 
	$f = p_1^{\nu_1}\ldots p_r^{\nu_r}$.
	
\paragraph{Strukturgarbe}
	\begin{align*}
		\O_{\Spec\Z} (D(f)) &= \Z_f  = 
			\left\{ \frac{a}{f^n} \mid n\in \N_0, a\in \Z\right\} \\
		\O_{\Spec\Z, (p)} &= \Z_{(p)} = 
			\left\{ \frac{a}{b} \mid p\nmid b, a\in \Z\right\} 
 	\end{align*}

\subsection{$\Spec k$ für einen Körper $k$}
\paragraph{Als topologischer Raum}
	$\Spec k = \{(0)\}$.

\paragraph{Strukturgarbe}
	$\O_{\Spec k}(\{(0)\}) = k$.

\begin{bemerkung}
   Sei $A$ ein Ring. Angenommen wir haben 
  	$\Spec A \xto{(f,f\fis)} \Spec k$ für einen Körper $k$, so haben wir
  	\[
  		f\fis_{\Spec k}: k = \O_{\Spec k} \to f_\ast\O_{\Spec A}(\Spec k)
  			\tikzmark{=} A,
  	\]
  	wobei \tikzarrow{mark above}{} aus 
  	$\O_{\Spec A}(f\inv(\{(0)\})) = \O_{\Spec A}(\Spec A)$ resultiert.
  	Insgesamt ist $A$ also eine $k$-Algebra (d.h. ein Ring zusammen mit
  	$k\to A$).
  	
  	Bemerke hierbei "`Grothendiecks Gesamtphilosophie"':
  	\begin{quote}\itshape
  		Alles relativ lesen!
  	\end{quote}
\end{bemerkung}

\begin{definition}[$S$-Schema]
	Sei $S$ ein Schema. Dann ist ein \emph{$S$-Schema} ein Schema $X$
	zusammen mit einem Strukturmorphismus $X \xto{\varphi} S$.
	Dies ergibt die Kategorie $\Sch_S$, wenn man
	\[
		\Hom( X\xto{\varphi}S, Y\xto{\varphi} S) := 
		\left\{ 
		\begin{tikzcd}
		X \arrow{rr}{f} \drar{\varphi} & & Y \dlar{\psi} \\ & S &
		\end{tikzcd}
		\right\}
	\]
	setzt.
\end{definition}

\begin{beispiel}
	$\Sch_k := \Sch_{\Spec k}$ sind die sog. \emph{$k$-Schemata}.
	Ein Beispiel hierfür ist
	$\Spec k[X_1,\ldots,X_n] \to \Spec k$ via 
	$k \hookrightarrow k[X_1,\ldots,X_n]$.
\end{beispiel}


\begin{bemerkung}
	Sei $X$ ein Schema und $x\in X$ und weiter $\m_x \ideal \O_{X,x}$ das
	maximale Ideal.
	Dann ist 
	\[
		\kappa(x) := k(x) := \O_{X,x} \big/ \m_x
	\]
	der \emph{Restklassenkörper von $x$}.
	
	Betrachte nun $(f,f\fis): \Spec k \to X$ mit
	\[
		f: \funcdef{\Spec k(x) & \to & X \\
			\eta_x & \mapsto & x,}
	\]
	wobei topologisch gesehen $\eta_x \in \Spec k(x)$ der einzige Punkt 
	dieses Schemas ist.
	Für $U\subseteq X$ offen haben wir:
	\[
		f\fis_U : \O_X \to 
			f_\ast \O_{\Spec k(x)}(U) = 
			\begin{cases} 0 & x\notin U \\ k(x) & x \in U. \end{cases}
	\]
	Im Fall $x \in U$ geht dies via
	\[
		\O_X(U) \to \O_{X,x} = \varinjlim_{x\in V} \O_X(V)
			\overset\pi\twoheadrightarrow  \O_{X,x}\big/ \m_x = k(x). 
	\]
	
	Ist umgekehrt $(f,f\fis):\Spec k \to X$ ein Schemamorphismus, so
	setze $x := f((0)) \in X$ und
	$f\fis: \O_X \to f_\ast \O_{\Spec k}$ liefert einen Ringhomomorphismus der
	Halme:
	\[
		f_x\fis: \O_{X,x} \to \O_{\Spec k, (0)} = k.
	\]
	Dieser ist lokal (also $f\fis_x (\m_x) = (0)$). Damit ist
	\[
		\begin{tikzcd}
		k(x) = \O_{X,x} \big/ \m_x \rar[hookrightarrow]{f_x\fis \mod \m_x} 
		&[7ex]  {f_x\fis \mod \m_x} k
		\end{tikzcd}
	\]
	wohldefiniert und somit ist $k \mid k(x)$ eine Körpererweiterung.
	
	Zusammengefasst haben wir:
	\[	\fbox{\parbox{5cm}{
			Einen Punkt $x\in X$ wählen mit Restklassenkörper
			$k(x)$ und eine Körpererweiterung $k\mid k(x)$.}}
		\Longleftrightarrow
		\fbox{\parbox{5cm}{
			Einen Schemamorphismus $\Spec k \to X$ wählen
			für eine Körpererweiterung $k\mid k(x)$.}}
	\]
\end{bemerkung}

\subsection{Der Affine $n$-dimensionale Raum über $k$}
Sei $k$ wieder ein Körper. Der affine $n$-dimensionale Raum über $k$ ist
$\A_k^n := \Spec k[X_1,\ldots, X_n]$.

Wir erinnern an den Hilbertschen Nullstellensatz:
\begin{satz}[Hilbertscher Nullstellensatz]
	\label{satz:hilbertscher nullstellensatz}
	Sei $k$ algebraisch abgeschlossen. Dann ist jedes maximale Ideal
	in $k[X_1,\ldots, X_n]$ von der Form
	$(X_1-a_1, \ldots, X_n - a_n)$.
\end{satz}
\begin{proof}
	ohne Beweis.
\end{proof}

Wir haben bereits gezeigt:
\[
	|\A_k^n| = k^n, \qquad\text{via } 
		(X_1-a_1,\ldots,X_n-a_n) \mapsto (a_1,\ldots,a_n).
\]
Sei $\p = (f_1, \ldots, f_r)$ ein nicht maximales Ideal in $k[X_1,\ldots,X_n]$
(die Darstellung ist nach \thref{satz:hilbertscher nullstellensatz}) möglich,
so gilt
\[
	\p \subseteq (X_1 - a_1, \ldots, X_n - a_n)
	\quad\Leftrightarrow\quad
	f_1(a_1,\ldots,a_n) = 0, \ldots,
	f_r(a_1,\ldots,a_n) = 0
\]
Wir können dies in \autoref{fig:spec k xy 2} "`sehen"'.

\begin{figure}\centering
	\caption{$\Spec k[X_1,\ldots,X_n]$}
	\label{fig:spec k xy 2}
	\begin{tikzpicture}
		\draw[very thick] 
			(-3,0) -- (3,0) node[near end, above] {$X_1$}
			(0,-2) -- (0,2) node[near end, right] {$X_2$};
		
% 		\draw[col1,thick] 
% 			(-3,1) to[out=-5, in=135]  (0.2,-0.2) 
% 			to[out=-45, in=225, looseness=2] (-1,0) 
% 			to[out=45, in=180, looseness=0.5] (3,1.5)
% 			node[pos=0.9] {$f(X,Y) = 0$};
		\draw[col1, thick]
			(-3,1) 
			.. controls (5,-2) and (-8,-2) .. 
			(3,1.5)
			node[pos=1, right, text width=2.9cm, font=\scriptsize] 
				{$\{(a_1,\ldots,a_n) \mid f_j(a_1,\ldots,a_n) = 0,$\\ 
					$j=1\ldots r\}$}
			coordinate[pos=0.05] (a);
		
		\fill[col1shade2] (a) circle[radius=2pt]
			node[above right, col1] {$(\alpha,\beta)$};
		
		\path (-3,1)
			node[generic point=10pt, fill=col1shade2] {}
			node[above left, col1] {$\p$};
	\end{tikzpicture}
\end{figure}

\subsection{Ohne Titel}
Betrachte $k\ldbrack X_1, \ldots, X_n\rdbrack = 
	k\ldbrack X_1,\ldots,X_{n-1}\rdbrack\ldbrack X_n\rdbrack$
mit $R\ldbrack X\rdbrack = \{\sum_{i=0}^\infty a_i X^i \mid a_i \in R\}$.

\begin{bemerkung}
	$g \in k\ldbrack X_1, \ldots, X_n\rdbrack \setminus (X_1,\ldots,X_n)$
	ist eine Einheit.
\end{bemerkung}
\begin{proof}
	Idee: Ansatz für eine Variable:
	$g(X) = a_0 + a_1X + a_2X^2+ \ldots$. Dann
	\[
		1 = g(X)h(X) = 
		\underbrace{a_0 b_0}{= 1} + 
		(\underbrace{a_0b_1+a_1b_0}{= 0})X + \ldots
	\] 
\end{proof}

\paragraph{Funktor $\Spec$} Wir haben den Funktor $\Spec$:
Die Ringhomomorphismen
\[\everymath{\displaystyle} \begin{tikzcd}[row sep=tiny, outer sep=5pt]
	k[X_1,\ldots,X_n] \rar & k[X_1,\ldots,X_N]_{(X_1,\ldots,X_n)} \rar &
		k\ldbrack X_1,\ldots,X_n \rdbrack \rar & k \\
	f \rar[mapsto] & \frac{f}{1} \\
	& \frac f g \rar[mapsto] & f g\inv \\
	&& h \rar[mapsto] & h(0)
\end{tikzcd}\]
induzieren 
\[\everymath{\displaystyle} \begin{tikzcd}[row sep=tiny, outer sep=5pt]
	&\Spec k \rar & k\ldbrack X_1,\ldots,X_n \rdbrack \rar & 
	k[X_1,\ldots,X_N]_{(X_1,\ldots,X_n)} \rar &
	\Spec k[X_1,\ldots,X_n] \\
	\text{topologisch:} &  
	(0) \rar[mapsto] & (X_1,\ldots,X_N)  \rar[mapsto]& 
	(X_1,\ldots,X_n) \rar[mapsto] & (X_1 , \ldots,X_n).\\
	&&\makebox[0pt]{\parbox{3cm}{\centering\small 
		einziger abgeschlossener Punkt}} 
	&\makebox[0pt]{\parbox{3cm}{\centering\small 
		einziger abgeschlossener Punkt}}
	&\makebox[0pt]{\parbox{3cm}{\centering\small 
		entspricht dem abgeschlossenen Punkt $(0,\ldots,0) \in k^n$}}
\end{tikzcd}\]
Dies ist ein Homöomorphismus auf $\{\p \in \A_k^n \mid 
\p \subseteq (X_1,\ldots,X_n) = V(\p) = \overline{\{\p\}} \subseteq \A_k^n$.

Was passiert aber auf Schemaniveau?
\begin{center}\begin{tikzcd}[column sep=large]
	\node{\tikz{
		\fill[col1shade2] circle[radius=2pt];
	}};
	\rar &
	\node{\tikz{
		\draw[->]
			(-1,0) -- (1,0) node[very near end, above] {$X_1$};
		\draw[->]
			(0,-1) -- (0,1) node[very near end, right] {$X_n$};
		\fill[col1shade2] circle[radius=2pt];
		\node[text width=2cm, font=\scriptsize, text=col1shade2, right]
			 at (0.2,-0.5)
			 (text)
			 {einziger abgeschlossener Punkt};
	}};
	\rar &
	\node{\tikz{
		\draw[->]
			(-1,0) -- (1,0) node[very near end, above] {$X_1$};
		\draw[->]
			(0,-1) -- (0,1) node[very near end, right] {$X_n$};
		\fill[col1shade2] circle[radius=2pt];
		\node[text width=2cm, font=\scriptsize, text=col1shade2, right]
			 at (0.2,-0.5)
			 (text)
			 {einziger abgeschlossener Punkt};
		\draw[col1, thick, dashed]
			(-1,1) 
			.. controls (1.5,-0.8) and (-1.5,-0.8) .. 
			(1,1);
		\node[generic point=5pt, fill=col1] at (1,1) {};
		\node[right] at (1,1) {$\p$};
	}};
	\rar &
	\node{\tikz{
		\draw[->]
			(-1,0) -- (1,0) node[very near end, above] {$X_1$};
		\draw[->]
			(0,-1) -- (0,1) node[very near end, right] {$X_n$};
		\fill[col1shade2] circle[radius=2pt];
		\draw[col1, thick]
			(-1,1) 
			.. controls (1.5,-0.8) and (-1.5,-0.8) .. 
			(1,1);
		\node[generic point=5pt, fill=col1] at (1,1) {};
		\node[right] at (1,1) {$\p$};
	}};
\end{tikzcd}\end{center}
Betrachte dazu
\[\everymath{\displaystyle} \begin{tikzcd}[row sep=tiny, outer sep=5pt]
	\Spec k \rar & k\ldbrack X_1,\ldots,X_n \rdbrack \big/ \p \rar & 
	k[X_1,\ldots,X_N]_{(X_1,\ldots,X_n)} \big/ \p \rar &
	\Spec k[X_1,\ldots,X_n]\big/\p \quad \approx\quad V(\p)
	\end{tikzcd}
\]
Nehmen wir das explizite Beispiel $\p = (Y^2 - X^2(X+1))$. Es ist $\p$ ein
Primideal und $V(\p)$ irreduzibel.

Beachte: $1+X \in k\ldbrack X \rdbrack$ hat eine Wurzel, wie man durch
folgenden Ansatz mit $h(X) = a_0 + a_1 X + \ldots$ sieht:
\[
	1+ X = (h(X))^2 = a_0^2 + 2a_0a_1 X + \ldots
\]
Setze $a_0 := 1$ oder $-1$ und löse sukzessizve auf. Demnach ist
$Y^2 - X^2(X+1) = (Y-Xh(X))(Y + X h(X))$ nicht mehr prim, also
$V(\p) \subseteq k\ldbrack X,Y \rdbrack$ nicht mehr irreduziebel!

Betrachte genauer
\[\begin{tikzcd}[row sep=tiny]
	k \ldbrack u,v\rdbrack \big/(uv) \rar{\cong} & 
	k \ldbrack z,w\rdbrack \big/(z^2-w^2) \rar{\cong} &
	k \ldbrack X,Y\rdbrack \big/(Y^2 - X^2(h(X))^2)\\
	u \rar[mapsto] & z+w & z \rar[mapsto] & Y\\
	v \rar[mapsto] & z-w & wz \rar[mapsto] & Xh(X)\\
\end{tikzcd}\]
In Bildern:
\[\begin{tikzcd}[row sep=-15pt]
	\Spec k\ldbrack u,v\rdbrack \big/(uv) \rar & \Spec k\ldbrack X,Y\rdbrack
		\big/ (Y^2 - X^2(X+1)) \\
	\node{\tikz{
		\draw[->]
			(-1,0) -- (1,0) node[very near end, above] {$X$};
		\draw[->]
			(0,-1) -- (0,1) node[very near end, right] {$Y$};
		\draw[col1, opacity=0.4, line width=3pt]
			(-.8,0) -- (.8,0)
			(0,-.8) -- (0,.8);
	}}; \rar &
	\node{\tikz{
		\draw[->]
			(-1,0) -- (1,0) node[very near end, above] {$X$};
		\draw[->]
			(0,-1) -- (0,1) node[very near end, right] {$Y$};
		\draw[dashed]
			(1,1) 
			.. controls (-1.4,-2) and (-1.4,2) .. 
			(1,-1);
		\clip (-0.3,-0.3) rectangle (0.3,0.3);
		\draw[line width=3pt, col1, opacity=0.4]
			(1,1) 
			.. controls (-1.4,-2) and (-1.4,2) .. 
			(1,-1);
	}};
\end{tikzcd}\]

\subsection{Spezielles Beispiel $\A_\Z^1 = \Spec \Z[X]$}
Wir haben $\pi: \A_\Z^1 \to \Spec \Z$. Topologisch ist
\[
	\A_\Z^1 = \bigcup_{p \text{ prim}} \pi\inv((p)) \cup \pi\inv((0)).
\]
\autoref{fig:A 1 Z to Spec Z} verdeutlicht dies.

\begin{figure}
	\caption{Veranschaulichung von $\A_\Z^1 \to \Spec\Z$}
	\label{fig:A 1 Z to Spec Z}
	\centering
	\begin{tikzpicture}
		\draw[fill=col1shade1, draw=col1]
			(-0.5,0) rectangle (6,4);
		\node[right, text=col1shade2] at (6,2) {$\A_\Z^1$};
		
		\draw[col1, thick]
			(0,0) -- (0,4)
			node[near end, below, sloped] {$\pi\inv((0))$};
		\draw[col1, thick]
			(2,0) -- (2,4)
			node[near end, below, sloped] {$\pi\inv((2))$};
		\draw[col1, thick]
			(3,0) -- (3,4)
			node[near end, below, sloped] {$\pi\inv((p))$};
		
		\draw[->,thick] 
			(3,-0.5) -- (3,-1.5);
		
		\draw[col1shade2, thick]
			(-0.5,-2) -- (6,-2)
			node[right, text=col1shade2]{$\Spec \Z$};
		\draw[col1, thick]
			(0,-1.8) -- +(0,-0.4)
			node[below] {$(0)$};
		\draw[col1, thick]
			(2,-1.8) -- +(0,-0.4)
			node[below] {$(2)$};
		\draw[col1, thick]
			(3,-1.8) -- +(0,-0.4)
			node[below] {$(p)$};
	\end{tikzpicture}
\end{figure}

\paragraph{Zu $\pi\inv((0))$}
Betrachte nun $\p \in \Spec \Z[X]$, so gilt
$\p \in \pi\inv((0))$ $\Leftrightarrow$ $\p \cap \Z = (0)$.

Betrachte $S:= \Z \setminus \{0\} \subseteq \Z[X]$ und die Lokalisierung
$g: \Z[X] \hookrightarrow \Z[X]_S$. Es ist klar: $\Z[X]_S = \Q[X]$

Ferner gilt $\Spec \Q[X] \to \Spec \Z[X]$ ist ein Homöomorphismus auf sein 
Bild:
\[
	\{\p \in \Spec\Z[X] \mid \p \cap S = \emptyset \} = 
	\{\p \in \A_\Z^1 \mid \p \cap \Z = (0) \} = \pi\inv(0),
\]

\paragraph{Zu $\pi\inv((p))$}
Es ist $\p \in \pi\inv((p))$ $\Leftrightarrow$ $p \in \p$.
Dann betrachte
$\rho: \Z[X] \twoheadrightarrow \bb F_p[X]$ und
$\rho^\ast: \Spec \bb F_p[X] \to \A_\Z^1$.
Wegen $\bb F_p[X] \cong \Z[X] \big/ \ker\rho$ ist $\rho^\ast$ ein Homöomorphismus
auf 
\[
	V(\ker \rho) = \{\p \in \Spec\Z[X] \mid \ker \rho \subseteq \p\} = 
	\pi\inv((p)) \subseteq \A_\Z^1.
\] 

Zusammengefasst ist:
\begin{align*}
	\pi\inv((0)) &= \A_\Q^1\\
	\pi\inv((p)) &= \A_{\bb F_p}^1,
\end{align*}
wobei die Gleichheiten topologisch zu lesen sind.

\paragraph{Betrachte $\p\in \Spec\Z[X]$}
\begin{description}
\item[1. Fall.]
	$\p\in \pi\inv((0))\ \Leftrightarrow\ \p\cap \Z = (0)$, also
	\[
		\p = (\mu(X))
	\]
	mit $\mu(X) \in \Z[X]$ einem primitiven, irreduziblen Polynom.
\item[2. Fall.]
	$\p\in\pi\inv((p))$, so ist $\p = \rho\inv(\q)$ für ein 
	$\q\in \Spec\bb F_p[X]$, also
	$\p = \rho\inv((q(X)))$ für ein irreduzibles $q(X)\in \bb F_p[X]$
	oder $(0)$. Dann ist
	\[
		\p = (r(X), p)
	\]
	mit $r(X) \in \Z[X]$ und $r(X) \equiv q(X) \bmod p$.
\end{description}
Es stellt sich die Frage, wie für $f\in \Z[X]$ die $D(f) \subseteq \A_\Z^1$
aussehen. Dazu
\begin{description}
\item[1. Fall $\p\in \pi\inv((0))$.] Sei $f(X) \in \Q[X]$. Dann
	$f(X) = \xi q_1(X)^{\nu_1} \ldots q_r(X)^{\nu_r}$ und es gilt
	\[
		f\notin \p \ \Leftrightarrow\ \p = (q(X))
	\]
	mit $q \neq q_1, \ldots, q_r$.
\item[2. Fall $\p \in \pi\inv((p))$.] $f(X) \notin (r(X), p)$
	mit $r(X) \mod p \in \bb F_p[X]$ irreduzibel. Für eine Primzahl $p$,
	betrachte $\bar f(X) \in \bb F_p[X]$.
	Ist $\bar f(X) = 0$, so ist $f(X) \in (r(X),p)$ für alle $r(X)$.
	Für $\bar f(X) = \bar q_1(X)^{\nu_1} \ldots \bar q_s(X)^{\nu_s}$, ist
	$f(X) \in (q_i(X), p)$ für diese $i$.
\end{description}
Dargestellt ist dies wieder in \autoref{fig:A 1 Z to Spec Z 2}.

\begin{figure}
	\caption{Veranschaulichung von $D(f) \subseteq \A_\Z^1$}
	\label{fig:A 1 Z to Spec Z 2}
	\centering
	\begin{tikzpicture}
		\draw[fill=col1shade1, draw=col1]
			(-0.5,0) rectangle (6,4);
		\node[right, text=col1shade2, font=\scriptsize] at (6,2) {$\A_\Z^1$};
		
		\draw[col1, thick]
			(0,0) -- (0,4)
			(2,0) -- (2,4)
			(3,0) -- (3,4)
			(3.5,0) -- (3.5,4)
			(4,0) -- (4,4)
			(4.5,0) -- (4.5,4);
		
		\draw[->,thick] 
			(3,-0.5) -- (3,-1.5);
		
		\draw[col1shade2, thick]
			(-0.5,-2) -- (6,-2)
			node[right, text=col1shade2]{$\Spec \Z$};
		\draw[col1, thick]
			(0,-1.8) -- +(0,-0.4)
			(3,-1.8) -- +(0,-0.4)
			(3.5,-1.8) -- +(0,-0.4)
			(4,-1.8) -- +(0,-0.4)
			(4.5,-1.8) -- +(0,-0.4);
		\draw[thick, col2]
			(2,-1.8) -- +(0,-0.4);
			
		\foreach \x in {0, 3}{	
			\foreach \y in {0.5,1,...,3}{
				\fill[col2] (\x,\y) circle[radius=2pt];
			}
		}
		\fill[col2] (4,3) circle[radius=2pt];
		\fill[col2] (4.5,2) circle[radius=2pt];
		\draw[col2] (2,0) -- (2,4);
		
		\node[right, col2, text width=3cm, font=\scriptsize] at (3.5,-0.5) 
			{irreduzible Teiler von $\bar f\in \bb F_p[X]$};
			
		\node[below, col2, text width=2cm, font=\scriptsize] at (0,0) 
			{irreduzible Faktoren von $f$};
			
		\node[below, col2, text width=4cm, font=\scriptsize] at (2,-2.5)
			{Primteiler aller Koeffizienten von $f$};
			
		\fill[col2] (7.5,3) circle[radius=3pt] 
			node[right] {\ $\notin D(f)$};
		\fill[col1] (7.5,2) circle[radius=3pt] 
			node[right] {\ $\in D(f)$}; 
	\end{tikzpicture}
\end{figure}
\pagebreak

% vim: set ft=tex :
\section{Projektive Schemata}
% vim: set ft=tex :

\section{Eigenschaften von Schemata} %Seite 79
% vim: set ft=tex :

\section{Tensorprodukt} %Seite 105
% vim: set ft=tex :

\section{Glatt, regulär & normal} %Seite 125
% vim: set ft=tex :

\section{$k$-Varietät} %Seite 154

\begin{beispiel}
    Ein einführendes Beispiel einer $k$-Varietät ist gegeben durch
    abgeschlossene Unterschemata, wie beispielsweise
    \[ \Spec k[X_1,\ldots,X_n] \big/\a \to \Spec k[X_1,\ldots,X_n] = \A^n_k.\]
\end{beispiel}


\begin{definition}[endlich]
    \label{def:ringhom endlich}
    \index[def]{Ringhomormophismus!endlich}
    Ein Ringhomomorphismus $\varphi: B \to A$ heißt \emph{endlich},
    wenn $A$ dadurch zu einem endlich erzeugten $B$-Modul wird.
\end{definition}

\begin{satz}[Noether-Normalisierung]
    Sei $A$ eine endlich erzeugte $k$-Algebr. Dann existiert $d\geq 0$ und
    ein endlicher injektiver Ringhomomorphismus
    \[k[T_1,\ldots,T_n] \hookrightarrow A.\]
\end{satz}
\begin{proof}
\TODO
\end{proof}

\begin{korollar}
    Ist $A$ eine endlich erzeugte $k$-Algebra und $\m \ideal A$ maximal,
    dann ist $A\big/\m$ eine endliche Körpererweiterung von $k$.
\end{korollar}
\begin{proof}
\TODO
\end{proof}

\begin{korollar}
    Sei $X$ eine $k$-Varietät und $x\in X$ ein abgeschlossener Punkt, so ist
    $k(x)\mid k$ endlich.
\end{korollar}
\begin{proof}
klar.
\end{proof}

\begin{satz}[(schwacher) Hilbertscher Nullstellensatz]
    Sei $k$ algebraisch abgeschlossen, $\m\ideal k[X_1,\ldots,X_n]$ ein
    maximales Ideal, so gilt
    \[\m = (X_1-a_1,\ldots,X_n-a_n)\]
    für geeignete $a_1,\ldots,a_n \in k$.
\end{satz}
\begin{proof}
\TODO
\end{proof}


\begin{lemma}
    Sei $X$ eine irreduzible algebraische $k$-Varietät, dann gilt für
    $x\in |X|$:
    \[\dim \O_{X,x} = \dim X.\]
\end{lemma}
\begin{proof}
\TODO
\end{proof}

\begin{lemma}
    Ist $\m \ideal k[X_1,\ldots,X_n]$ ein maximales Ideal, so existieren
    Polynome
    \[f_1(X_1), f_2(X_1,X_2), \ldots, f_n(X_1,\ldots,X_n)\]
    mit
    \[\m = (f_1,\ldots,f_r).\]
\end{lemma}
\begin{proof}
Induktion mit Noethernormalisierung.
\end{proof}

\begin{folgerung}
    $\A^n_k$ ist regulär bei allen $x \in |\A^n_k|$.
\end{folgerung}


% vim: set ft=tex :

\section{Der Punktefunktor} %Seite 159
% vim: set ft=tex :


\end{document}
% vim: set ft=tex :
