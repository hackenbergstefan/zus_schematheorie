%% Basierend auf einer TeXnicCenter-Vorlage von Mark Müller
%%%%%%%%%%%%%%%%%%%%%%%%%%%%%%%%%%%%%%%%%%%%%%%%%%%%%%%%%%%%%%%%%%%%%%%
% Wählen Sie die Optionen aus, indem Sie % vor der Option entfernen  
% Dokumentation des KOMA-Script-Packets: scrguide

%%%%%%%%%%%%%%%%%%%%%%%%%%%%%%%%%%%%%%%%%%%%%%%%%%%%%%%%%%%%%%%%%%%%%%%
%% Optionen zum Layout des Artikels                                  %%
%%%%%%%%%%%%%%%%%%%%%%%%%%%%%%%%%%%%%%%%%%%%%%%%%%%%%%%%%%%%%%%%%%%%%%%
\documentclass[%
%a5paper,							% alle weiteren Papierformat einstellbar
%landscape,						% Querformat
10pt,								% Schriftgröße (12pt, 11pt (Standard))
%BCOR1cm,							% Bindekorrektur, bspw. 1 cm
%DIVcalc,							% führt die Satzspiegelberechnung neu aus
%											  s. scrguide 2.4
%twoside,							% Doppelseiten
%twocolumn,						% zweispaltiger Satz
halfparskip*,				% Absatzformatierung s. scrguide 3.1
%headsepline,					% Trennline zum Seitenkopf	
%footsepline,					% Trennline zum Seitenfuß
%titlepage,						% Titelei auf eigener Seite
%normalheadings,			% Überschriften etwas kleiner (smallheadings)
%idxtotoc,						% Index im Inhaltsverzeichnis
%liststotoc,					% Abb.- und Tab.verzeichnis im Inhalt
%bibtotoc,						% Literaturverzeichnis im Inhalt
%abstracton,					% Überschrift über der Zusammenfassung an	
%leqno,   						% Nummerierung von Gleichungen links
%fleqn,								% Ausgabe von Gleichungen linksbündig
%draft								% Überlangen Zeilen in Ausgabe gekennzeichnet
DIV = 14,
%monochrome,						% schwarz weiß output
]
{scrartcl}

\makeatletter
\@ifundefined{myDevelopVariable}{%
  \xdef\myDevelopVariable{0}}{}
\makeatother

%\pagestyle{empty}		% keine Kopf und Fußzeile (k. Seitenzahl)
%\pagestyle{headings}	% lebender Kolumnentitel  


%% Deutsche Anpassungen %%%%%%%%%%%%%%%%%%%%%%%%%%%%%%%%%%%%%
\usepackage{amsmath,amssymb,mathtools,stmaryrd}
\usepackage[ngerman]{babel}
\usepackage[T1]{fontenc}
%\usepackage[ansinew]{inputenc}
\usepackage[utf8]{inputenc}

\usepackage{lmodern} %Type1-Schriftart für nicht-englische Texte

%% Indexing %%%%%%%%%%%%%%%%%%%%%%%%%%%%%%%%%%%%%%%%%%%%%%%%%
\usepackage{imakeidx}
\makeindex[name=def, title=Definitionen, columns=2, columnsep=0pt, intoc]
%%%%%%%%%%%%%%%%%%%%%%%%%%%%%%%%%%%%%%%%%%%%%%%%%%%%%%%%%%%%%

\usepackage{hyperref}

\usepackage{xspace}


\usepackage{xcolor}
\usepackage{tikz}
\usetikzlibrary{matrix,arrows,fadings,decorations.markings}
\usepackage{tikz-cd2}

%% define colors %%%%%%%%%%%%%%%%%%%%%%%%%%%%%%%%%%%%%%%%%%%%
\colorlet{mycolor}{blue!80!black}
\colorlet{col1}{mycolor}
\colorlet{col1shade1}{mycolor!5}
\colorlet{col1shade2}{mycolor!50}
\colorlet{col2}{purple!80}
\colorlet{col2shade1}{col2!5}
\colorlet{col2shade2}{col2!50}

%% tikz setup %%%%%%%%%%%%%%%%%%%%%%%%%%%%%%%%%%%%%%%%%%%%%%%
\usetikzlibrary{decorations.shapes, shapes.geometric}
\tikzset{generic point/.style=
	{star, star points=20, minimum size=#1, inner sep=0pt, outer sep=0pt}
}


%% Packages tikztitle and tikztheorem %%%%%%%%%%%%%%%%%%%%%%%
\usepackage[color=mycolor, style=elegant, withheadings=true,
  backgroundcmd=titlepagebackground]{tikztitle}

\ifnum\myDevelopVariable=1 
  \RequirePackage[thref, hyperref, thmmarks]{ntheorem}
  \usepackage{thmbox}
  \theoremstyle{plain}% default
  \newtheorem{thm}{Satz}[section]
  \newtheorem{satz}[thm]{Satz}
  \newtheorem{lemma}[thm]{Lemma}
  \newtheorem{hilfslemma}[thm]{Hilfslemma}
  \newtheorem{lem}[thm]{Lemma}
  \newtheorem{kor}[thm]{Korollar}
  \newtheorem{cor}[thm]{Korollar}
  \newtheorem{korollar}[thm]{Korollar}
  %\newtheorem{prop}[thm]{Proposition}

  \theoremstyle{definition}
  \newtheorem{defn}[thm]{Definition}
  \newtheorem{definition}[thm]{Definition}
  \newtheorem{bsp}[thm]{Beispiel}
  \newtheorem{beispiel}[thm]{Beispiel}
  \newtheorem{exmp}[thm]{Beispiel}
  %\newtheorem{conj}[thm]{Conjecture}

  \theoremstyle{remark}
  \newtheorem{bem}[thm]{Bemerkung}
  \newtheorem{bemerkung}[thm]{Bemerkung}
  \newtheorem{rem}[thm]{Bemerkung}
  \newtheorem{uebung}[thm]{Übung}
  %\newtheorem{note}[thm]{Notiz}
  %\newtheorem{case}{Fall}
\else
  \usepackage{tikztheorems}
  \newtikztheorem[
    style=elegantbreak,
    color=mycolor,
    font header=\normalfont\sffamily\bfseries,
    counter zero=section
    ]{satz}{Satz}
    
  \newtikztheorem[
    style=elegantbreak,
    color=mycolor,
    font header=\normalfont\sffamily\bfseries,
    font body=\normalfont,
    counter parent=satz
    ]{definition}{Definition}
   
  \newtikztheorem[
    style=elegantinline,
    color=mycolor,
    font header=\normalfont\sffamily\bfseries,
    counter parent=satz
    ]{lemma}{Lemma}
    
  \newtikztheorem[
    style=elegantinline,
    color=mycolor,
    font header=\normalfont\sffamily\bfseries,
    counter parent=satz
    ]{korollar}{Korollar}
    
  \newtikztheorem[
    style=plain,
    color=mycolor,
    font header=\normalfont\sffamily\bfseries,
    font body=\normalfont,
    counter parent=satz
    ]{beispiel}{Beispiel}
    
  \newtikztheorem[
    style=plain,
    color=mycolor,
    font header=\normalfont\sffamily\bfseries,
    font body=\normalfont,
    counter parent=satz
    ]{bemerkung}{Bemerkung}
    
    
  \newtikztheorem[
    style=plain,
    color=mycolor,
    font header=\normalfont\sffamily\bfseries,
    font body=\normalfont,
    counter parent=satz
    ]{hilfslemma}{Hilfslemma}
    
  \newtikztheorem[
    style=elegantbreak,
    color=col2,
    font header=\normalfont\sffamily\bfseries,
    font body=\normalfont,
    nocounter=true
    ]{uebung}{Übung}
\fi
%%%%%%%%%%%%%%%%%%%%%%%%%%%%%%%%%%%%%%%%%%%%%%%%%%%%%%%%%%%%%

\usepackage{tikzmargin}

\usepackage{array, enumitem}

%Buchstaben durchstreichen
\usepackage{cancel}

%% Packages für Grafiken & Abbildungen %%%%%%%%%%%%%%%%%%%%%%
\usepackage{graphicx} %%Zum Laden von Grafiken
%\usepackage{subfig} %%Teilabbildungen in einer Abbildung
% \usepackage[ngerman, nameinlink]{cleveref}


%% Bibliography %%%%%%%%%%%%%%%%%%%%%%%%%%%%%%%%%%%%%%%%%%%%%
\usepackage[style=numeric, backend=biber]{biblatex}
\usepackage[babel,german=guillemets]{csquotes}
\addbibresource{bib.bib}
\setlength{\bibitemsep}{2ex}


%% Math abbreviations %%%%%%%%%%%%%%%%%%%%%%%%%%%%%%%%%%%%%%%
\let\bb\mathbb
\let\cal\mathcal
\let\fr\mathfrak
\newcommand{\F}{\cal{F}}
\newcommand{\G}{\cal{G}}
\renewcommand{\O}{\cal{O}}
\renewcommand{\L}{\cal{L}}
\newcommand{\U}{\cal{U}}
\newcommand{\M}{\cal{M}}
\newcommand{\R}{\bb{R}}
\newcommand{\N}{\bb{N}}
\newcommand{\C}{\bb{C}}
\newcommand{\Q}{\bb{Q}}
\newcommand{\Z}{\bb{Z}}
\newcommand{\A}{\bb{A}}
\renewcommand{\P}{\bb{P}}
\newcommand{\rest}[1]{\big|_{#1}}
\newcommand{\inv}{^{-1}}
\newcommand{\fis}{^{\#}}
\newcommand{\kat}[1]{\mathbf{#1}}
\newcommand{\PSh}{\kat{PSh}}
\newcommand{\Sh}{\kat{Sh}}
\newcommand{\Ring}{\kat{Ring}}
\newcommand{\Ab}{\kat{Ab}}
\newcommand{\Top}{\kat{Top}}
\newcommand{\Sch}{\kat{Sch}}
\newcommand{\Moduln}[1]{#1\text{-}\kat{Mod}}
\newcommand{\affSch}{\kat{Sch}^\kat{aff}}
\newcommand{\op}{^\mathrm{op}}
\newcommand{\m}{\fr{m}}
\newcommand{\p}{\fr{p}}
\newcommand{\q}{\fr{q}}
\renewcommand{\a}{\fr{a}}
\renewcommand{\b}{\fr{b}}
\newcommand{\ideal}{\vartriangleleft}
\newcommand{\immersion}{\ \tikz[baseline=-0.6ex]{
    \draw[/tikz/commutative diagrams/immersion,right hook->]
        (0,0) -- +(1.3em,0);}\ }
\newcommand{\oimmersion}{\ \tikz[baseline=-0.6ex]{
    \draw[/tikz/commutative diagrams/offene immersion,right hook->]
        (0,0) -- +(1.3em,0);}\ }
\DeclareMathOperator{\Obj}{Obj}
\DeclareMathOperator{\Morph}{Morph}
\DeclareMathOperator{\Hom}{Hom}
\DeclareMathOperator{\Aut}{Aut}
\DeclareMathOperator{\im}{im}
\DeclareMathOperator{\Spec}{Spec}
\DeclareMathOperator{\Proj}{Proj}
\DeclareMathOperator{\Nil}{Nil}
\DeclareMathOperator{\Quot}{Quot}
\DeclareMathOperator{\charak}{char}
\DeclareMathOperator{\Span}{span}
\DeclareMathOperator{\res}{res}
\DeclareMathOperator{\pr}{pr}
\DeclareMathOperator{\id}{id}
\newcommand{\funcdef}[1]{%
	\begin{array}[t]{>{\displaystyle}r>{\displaystyle}c>{\displaystyle}l}
	#1\end{array}}
\let\xto\xrightarrow

\newcommand{\osubset}{\subseteq^\circ}

%Cech 
\newcommand{\Cech}{\v Cech}
\DeclareMathOperator{\Hv}{\check H}
\DeclareMathOperator{\Cv}{\check C}
\makeatletter
\let\H\@undefined
\makeatother
\DeclareMathOperator{\H}{H}

\newcommand{\obda}{{\small oBdA}\xspace}
\newcommand{\Obda}{{\small OBdA}\xspace}

%für doppelklammer eckig
\newcommand{\dk}[1]{\ensuremath\llbracket #1\rrbracket}
% für doppelklammer rund
\newcommand{\dr}[1]{\ensuremath(\!( #1 )\!)}

\newenvironment{description sf}{%
	\begin{description}[font=\normalfont\sffamily]}
	{\end{description}}
\newenvironment{description mathquote}{%
	\renewcommand{\descriptionlabel}[1]{\glqq$##1$\grqq.}
	\begin{description}[font=\normalfont]}
	{\end{description}}
	
\newcommand{\TODO}{\textcolor{red}{\sffamily\bfseries 
    Leider noch nicht fertig :-(}}    
    

%%  comment boxen  %%%%%%%%%%%%%%%%%%%%%%%%%%%%%%%%%%%%%%%%%%
\ifnum\myDevelopVariable=1 
  \usepackage{xcolor}
  \usepackage{color}
  \usepackage{framed}

  \newenvironment{fshaded}{%
  \def\FrameCommand{\fcolorbox{framecolor}{shadecolor}}%
  \MakeFramed{\FrameRestore}}%
  {\endMakeFramed}

  \newenvironment{comment}{\color{gray}%
  \definecolor{shadecolor}{rgb}{.95,.95,.95}%
  \definecolor{framecolor}{rgb}{.8,.8,.8}%
  \fshaded%
  \textbf{\tiny Kommentar: } }{\endfshaded}
\else
  \usepackage{verbatim}
\fi

%% Bibliographiestil %%%%%%%%%%%%%%%%%%%%%%%%%%%%%%%%%%%%%%%%%%%%%%%%%%
%\usepackage{natbib}

\begin{document}

\ifnum\myDevelopVariable=0 
  %% Titlepage Background cmd (lrbox must be after begin{document}) %%%%%
  \tikzfading[name=fade down,
    top color=transparent!90,
    bottom color=transparent!0]
  \def\titlepagebackground{
      \begin{scope}[transform shape, rotate=40, scale=5, opacity=0.5,
        ]
      \node[scope fading=fade down] at (current page.center)
      {\usebox{\diagbox}};
      \end{scope}
  }
  \newsavebox{\diagbox}
  \begin{lrbox}{\diagbox}
  \begin{minipage}{\textwidth}
  \[
    \everymath{\displaystyle}
    \begin{tikzcd}[row sep=large, column sep=large]
      s \dar[mapsto] \symb{\in}
        &[-1cm] \O_Y(U) \rar{f_U\fis} \dar[swap]{\rest W} 
        & \O_X(f\inv(U)) \dar{\rest{f\inv(W)}} \symb{\ni}
        & \dar[mapsto] f_U\fis(s)\\
      s\rest W = t\rest W \symb{\in}
        & \O_Y(W) \rar{f_W\fis}				
        & \O_X(f\inv(W)) \symb{\ni}
        & f_U\fis(s)\rest{f\inv(W)} = f_V\fis(t)\rest{f\inv(W)}\\
      t \uar[mapsto] \symb{\in}
        & \O_Y(V) \rar{f_W\fis} \uar{\rest W} 
        & \O_X(f\inv(V)) \uar[swap]{\rest{f\inv(W)}} \symb{\ni}
        & \uar[mapsto] f_V\fis(t)\\
    \end{tikzcd}
  \]
  \end{minipage}
  \end{lrbox}
  %%%%%%%%%%%%%%%%%%%%%%%%%%%%%%%%%%%%%%%%%%%%%%%%%%%%%%%%%%%%%%%%%%%%%%%
\fi

\pagestyle{empty}
%%%%%%%%%%%%%%%%%%%%%%%%%%%%%%%%%%%%%%%%%%%%%%%%%%%%%%%%%%%%%%%%%%%%%%%
%% Ihr Artikel                                                       %%
%%%%%%%%%%%%%%%%%%%%%%%%%%%%%%%%%%%%%%%%%%%%%%%%%%%%%%%%%%%%%%%%%%%%%%%

%% eigene Titelseitengestaltung %%%%%%%%%%%%%%%%%%%%%%%%%%%%%%%%%%%%%%%    
%\begin{titlepage}
%Einsetzen der TXC Vorlage "Deckblatt" möglich
%\end{titlepage}

%% Angaben zur Standardformatierung des Titels %%%%%%%%%%%%%%%%%%%%%%%%
%\titlehead{Titelkopf }
\subject{Vorlesungszusammenfassung}
\title{Schematheorie}
\author[erstellt von]{Stefan Hackenberg \and Maximilian Huber}
%\thanks{Fuünote}			% entspr. \footnote im Flieütext
\date{\today}				% falls anderes, als das aktuelle gewünscht
\publishers[gelesen im WS 2012/2013 und SS 2013 von]{Prof. Dr. Marco Hien}

%% Widmungsseite %%%%%%%%%%%%%%%%%%%%%%%%%%%%%%%%%%%%%%%%%%%%%%%%%%%%%%
%\dedication{Widmung}

\maketitle 						% Titelei wird erzeugt


%% Zusammenfassung nach Titel, vor Inhaltsverzeichnis %%%%%%%%%%%%%%%%%
%\begin{abstract}
% Für eine kurze Zusammenfassung des folgenden Artikels.
% Für die überschrift s. \documentclass[abstracton].
%\end{abstract}

\KOMAoptions{twoside}
\cleardoublepage

%% Erzeugung von Verzeichnissen %%%%%%%%%%%%%%%%%%%%%%%%%%%%%%%%%%%%%%%
\thispagestyle{plain}
\tableofcontents			% Inhaltsverzeichnis
%\listoftables				% Tabellenverzeichnis
%\listoffigures				% Abbildungsverzeichnis


%% Der Text %%%%%%%%%%%%%%%%%%%%%%%%%%%%%%%%%%%%%%%%%%%%%%%%%%%%%%%%%%%

\pagestyle{scrheadings}

%\part{Erstes Semester}
\section{Lokal geringte R�ume}

\subsection{Garben}

\begin{definition}[Pr�garbe]
	Sei $X$ ein topologischer Raum. Eine \emph{Pr�garbe} $\F$ auf $X$
	ist eine Zuordnung
	$$\F: U\mapsto \F(U) \,,$$
	die jedem offenen $U\subset X$ eine abelsche Gruppe
	$\F(U)$ zuordnet, zusammen mit Homomorphismen
	$$\rho_{UV}: \F(U) \to \F(V)$$
	f�r jedes Paar $V\subset U$, so dass
	\[
	\begin{tikzcd}
		\F(U) \arrow{r}{\rho_{UV}}
			\arrow[bend right]{rr}{\rho_{UW}}& \F(V) \arrow{r}{\rho_{VW}}& \F(W)
	\end{tikzcd}
	\]
	kommutiert.
	
	Wir nennen $\rho_{UV}$ \emph{Restriktion}, schreiben
	meist $\tikzmark{s\rest V} := \rho_{UV}(s)$.
	
	Man nennt $s\in \F(U)$ auch \emph{Schnitt �ber $U$}.
\end{definition}
	
\tikzmargin{north, above=1cm}{\color{red}
Bei mir steht hier im Skript $s\rest U$. Offenbar ein Fehler!?}

\begin{beispiel}
	$$\cal C_X^\circ: U \mapsto \cal C_X^\circ (U) := 
		\{f: U\to \R \mid \text{ $f$ stetig}\} $$
	mit $\rho_{VU}: \cal C_X^\circ(V) \mapsto \cal C_X^\circ(U)$,
	$f \mapsto f\rest U$.
\end{beispiel}

\begin{bemerkung}
	Ist $\kat{Ab}$ die Kategorie der abelschen Gruppen und
	\[
		\kat{Top}_X := 
		\begin{cases}
		\Obj: U\subset X \text{ offen}\\
		\Morph: \Hom(U,V) = 
			\begin{cases}
				\emptyset & U\not\subset V,\\
				U\to V & U\subset V,
			\end{cases}
		\end{cases}
	\]
	dann ist eine Pr�garbe gerade ein kontravarianter Funktor
	\[
		\F: \funcdef{\kat{Top}_X & \to & \kat{Ab}\\
			U & \mapsto & \F(U)\\
			(U\to V) & \mapsto & (\F(V)\to \F(U)).}
	\]
	Oder anders ausgedr�ckt: Es ist
	\[
		\F: \funcdef{\kat{Top}_X\op & \to & \kat{Ab}\\
			U & \mapsto & \F(U)\\
			(V\to U) & \mapsto & (\F(V)\to \F(U)).}
	\]
	ein kovarianter Funktor.
\end{bemerkung}

\begin{definition}[Morphismus von Pr�garben]
	Ein \emph{Morphismus von Pr�garben} $\F \xto{\phi} \G$ auf $X$ ist
	eine nat�rliche Transformation der Funktoren $\F$ und $\G$, d.h.
	f�r alle $U\subset X$ offen gibt es einen Morphismus
	$\F(U) \xto{\phi_U} \G(U)$, so dass f�r $U\subset V$
	\[
		\begin{tikzcd}
			\F(U) \arrow{r}{\phi_U} & \G(U)\\
			\F(V) \arrow{r}{\phi_U} \arrow{u} & \G(V) \arrow{u}
		\end{tikzcd}
	\] 
	kommutiert.
\end{definition}


\begin{definition}[Garbe]
	Eine Pr�garbe $\F$ auf $X$ hei�t \emph{Garbe}, falls gilt:
	Ist $U\subset X$ offen und $U=\bigcup_{i\in I} U_i$ f�r 
	offene $U_i\subset X$, so gilt
	\begin{enumerate}
	  \item Ist $s\in \F(U)$ und $s\rest{U_i} = 0$ f�r alle $i\in I$,
	  	so ist $s=0\in \F(U)$.
	  \item Sind $s_i \in \F(U_i)$ gegeben, mit
	  	$$s_i \rest{U_i\cap U_j} = s_j \rest{U_i\cap U_j}\qquad \forall i,j,$$
	  	so existiert ein $s\in \F(U)$ mit
	  	$$s_i = s \rest{U_i}\qquad\forall i.$$
	\end{enumerate}
\end{definition} 

\begin{bemerkung}
	$\F$ ist eine Garbe, genau dann, wenn die folgende Sequenz abelscher
	Gruppen exakt ist:
	\[	\everymath{\displaystyle}
		\begin{tikzcd}[row sep=tiny, column sep=small]
		0 \rar & \tikzmark[1]{\F(U)} \rar & 
			\tikzmark[2]{\prod_{i\in I} \F(U_i)} \rar
			& \prod_{(i,j)\in I^2} \F(U_i\cap U_j)\\
		& s \rar[mapsto] & \left(s\rest{U_i}\right)_{i\in I}\\
		&& (s_i)_{i\in I} \rar[mapsto] &
			\left(s_i\rest{U_i\cap U_j} - 
			s_j\rest{U_i\cap U_j}\right)_{(i,j)\in I^2}  
		\end{tikzcd}
	\]
	
	Exaktheit an \tikzarrow[1]{south, mark above}{dieser} Stelle ist �quivalent 
	zu Eigenschaft 1.
	Exaktheit \tikzarrow[2]{south, mark above}{hier} zu Eigenschaft 2.
\end{bemerkung}

\begin{beispiel}
	Sei $M$ eine $\mathrm C^\infty$ Mannigfaltigkeit, so ist
  	\[ \cal C^\infty_M: U \mapsto
  		\cal C^\infty_M(U) := \{f:U\to \R \mid f\in \mathrm C^\infty(U)\}
  	\]
  	eine Garbe.
\end{beispiel}

\begin{beispiel}
	Sei $M$ eine $\C$ Mannigfaltigkeit, so ist
  	\[ \cal O_M: U \mapsto
  		\cal O_M(U) := \{f:U\to \C \mid f \text{ holomorph}\}
  	\]
  	eine Garbe. F�r $M = \C$ haben wir zus�tzlich die Garbe
  	\[ \cal O_\C^\times: U \mapsto
  		\cal O_\C^\times(U) := \{f:U\to \C^\times \mid f \text{ holomorph}\},
  	\]
  	(wobei die Gruppenverkn�pfung multiplikativ zu lesen ist).
  	Dies liefert uns einen Morphismus von (Pr�)garben
  	\[ \O \to \O_C^\times,\ f \mapsto \exp(f).\]
  	Betrachte nun die Pr�garbe
  	\[\cal H := \tikzmark{\im^\text{naiv}}(\exp): U \mapsto \im(\exp_U) = 
  		\{\exp \circ f: U\to \C \mid f:U\to \C \text{ holomorph}\}.\]
  	Dies ist \emph{keine} Garbe:
  	Betrachte die Scheibe 
  	\[U = \{z\in \C \mid \tfrac{1}{2} < |z| < \tfrac{3}{2}\}\]
  	zerlegt in die beiden offenen Teilmengen
  	\begin{align*}
  		U_1 &= \{z \in U \mid \Re z > -\varepsilon\}\\
  		U_2 &= \{z \in U \mid \Re z < \varepsilon\}
  	\end{align*}
  	mit $U = U_1 \cup U_2$ f�r ein $\varepsilon > 0$ beliebig. F�r $i=1,2$
  	ist 
  	$(z: U_i \to \C, z\mapsto z) \in \cal H(U_i)$,
  	da sich der komplexe Logarithmus auf beiden $U_i$ problemlos definieren
  	l�sst.
  	Ferner ist auch
 	\[ (z: U_1 \to \C) \rest{U_1\cap U_2} = 
 		(z: U_2 \to \C) \rest{U_1 \cap U_2},\]
 	erf�llt, jedoch kommen diese nicht von einem gemeinsamen Schnitt
 	da
 	\[ (z: U\to \C) \notin \cal H(U). \]
\end{beispiel}

\tikzmargin{north}{Warum steht hier naiv??}


\begin{definition}
	F�r einen topologischen Raum $X$ bezeichne
	\begin{align*}
		\PSh_X & := \text{die Kategorie der Pr�garben auf $X$},\\
		\Sh_X & := \text{die Kategorie der Garben auf $X$, wobei
			} \Hom_{\Sh_X}(\F,\G) := \Hom_{\PSh_X}(\F,\G)
	\end{align*} 
\end{definition}

\begin{bemerkung}
	Man hat den Inklusionsfunktor
	\[ \iota: \Sh_X \to \PSh_X,\ \F \mapsto \F\]
\end{bemerkung}

\begin{definition}[Halm]
	Ist $\F$ eine (Pr�)Garbe auf $X$ und $x_0 \in X$, so hei�t
	\[ \F_{x_0} := \varinjlim_{x_0 \in U \subset X\text{ offen}} \F(U)
		 = \coprod_{U\subset X\text{ offen}} \F(U) \Big/ \sim\] 
	mit 
	\[ s \sim t \  :\Leftrightarrow \  
		\exists W \subset X \text{ offen}:\ x_0 \in W \subset U \cap U'
		\text{ und } s\rest W = t \rest W
	\]
	f�r $s \in \F(U)$, $t \in \F(U')$ der \emph{Halm von $\F$ bei $x_0$}.
	
	Die Elemente $[s] \in \F_{x_0}$ hei�en \emph{Keime von Schnitten bei $x_0$}.  
\end{definition}

\begin{beispiel}
	$(\cal C^\infty_M)_{x_0} = \{ [f: U \xto{C^\infty} \R]\mid
  	f\sim g \Leftrightarrow \exists W\subset M\text{ offen}, x_0 \in W
  	\text{ mit } f\rest W = g\rest W\}$
\end{beispiel}
\begin{beispiel}
	\begin{align*}
	  	O_{\C,x_0} &= \{[f:U \xto{\text{hol}} \C] \mid x_0 \in U\}\\
	  	&= \{\sum_{n=0}^\infty a_n(x-x_0)^n \mid \text{Reihe hat positiven 
	  	Konvergenzradius}\}\\
	  	&:= \C\{x-x_0\}
	\end{align*}
\end{beispiel}

\begin{definition}[push-forward]
	Ist $f:X \to Y$ stetig und $\F$ eine Garbe auf $X$, so ist durch
	\[ f_\ast \F: V \subset Y \text{ offen} \mapsto \F(f^{-1}(V))\]
	eine Garbe definiert, der \emph{push-forward von $\F$}.
\end{definition}



\subsection{Lokal geringte R�ume}
Betrachte nun 
\[\Ring := \text{ Kategorie der kommuativen Ringe mit $1$}\]
und entsprechend Garben
\[\F:\Top_X\op \to \Ring.\]

\begin{definition}[lokaler Ring]
	Sei  $R$ ein Ring. Dann hei�t $R$ \emph{lokal}, wenn $R$ genau ein
	maximales Ideal besitzt.
\end{definition}

\begin{beispiel}
	$\Z_{(p)} := \left\{\frac{a}{b} \in \Q \mid p \nmid b\right\}$
\end{beispiel}

\begin{bemerkung}
	Ist $R$ lokaler Ring und $\m \ideal R$ das maximale Ideal,
	so ist $R \setminus \m = R^\times$.
\end{bemerkung}

\begin{beispiel}
	Sei $M$ eine $C^\infty$ Mannigfaltigkeit und $x_0 \in M$.
	Dann ist $\cal C^\infty_{M,x_0}$ ein lokaler Ring, denn
	\[
		\cal C^\infty_{M,x_0} \setminus \big(\cal C^\infty_{M,x_0}\big)^\times
		= \{[f:U\xto{C^\infty} \R] \mid x_0 \in U\text{ mit } f(x_0) = 0\}
		=: \m,
	\]
	da $[f]$ eine Einheit ist, genau dann, wenn $f(x_0) \neq 0$: 
	Ist $f: U\xto{C^\infty} \R$ mit $f(x_0) \neq 0$, so existiert
	$W\subset U$ offen, $x_0\in W$ mit $f(x) \neq 0$ f�r alle $x\in W$.
	Damit folgt
	\[
		\left[\frac{1}{f}: W \to \R,\ x\mapsto \frac{1}{f(x)}\right]
		\in \cal C^\infty_{M,x_0}
	\]
	ist Inverses zu $[f]$.
	Zudem ist $\m$ ein Ideal.
\end{beispiel}

\begin{definition}[lokal geringter Raum]
	Ein \emph{lokal geringter Raum} ist ein Paar $(X, \O_X)$ bestehend aus:
	\begin{itemize}
	  \item einem topologischen Raum $X$ und
	  \item einer Garbe $\O_X$ auf $X$ von Ringen,
	\end{itemize}
	so dass $\O_{X,x_0}$ f�r alle $x_0\in X$ ein lokaler Ring ist.
	
	Man nennt $\O_X$ die \emph{Strukturgarbe von $(X,\O_X)$}. Ist
	$x_0\in X$, so hat man das maximale Ideal
	$\m_{x_0} \ideal \O_{X,x_0}$.
	
	Der K�rper 
	\[\kappa(x_0) := \O_{X,x_0} \big/ \m_{x_0} \]
	hei�t \emph{Restklassenk�rper von $x_0$ in $(X,\O_X)$}.
\end{definition}

\begin{beispiel}
	Sei $M$ eine $C^\infty$-Mannigfaltigkeit und $x_0 \in M$,
	so ist $\kappa(x_0) = \R$.
\end{beispiel}
\section{Affine Schemata}

\subsection{$\Spec A$ als topologischer Raum}

Sei im Folgenden $A$ ein kommuativer Ring mit $1$ und 
$\Spec A := \{\p \ideal A \mid \p \text{ Primideal}\}$.

\begin{definition}[Zariski Topologie]
	Ist $\a \ideal A$, ein Ideal, setze
	\[
		V(\a) := \{\p \in \Spec A \mid \a \subseteq \p \} \subseteq \Spec A\,.
	\]
	Dann ist durch
	\[
		\cal T := \{ U \subseteq \Spec A \mid
			\exists\ \a \ideal A:\ U = \Spec A \setminus V(\a)\}
	\]
	eine Topologie auf $\Spec A$ definiert. Sie hei�t \emph{Zariski-Topologie}.
\end{definition}

\begin{proof}[der Topologie-Eigenschaften]
	\begin{enumerate}
	  \item Zeige: $\emptyset$, $\Spec A$ offen $\Longleftrightarrow$ 
	  	$\Spec A$, $\emptyset$ abgeschlossen.\\
	  	Dazu: $V(A) = \emptyset$, $V((0)) = \Spec A$
	  \item Zeige: $U_1, U_2$ offen $\Rightarrow$ $U_1 \cap U_2$ offen
	  	$\Longleftrightarrow$ $M_1,M_2$ abgeschlossen $\Rightarrow$
	  	$M_1 \cup M_2$ abgeschlossen.\\
	  	Dazu:
	  	$V(\a) \cup V(\fr b) = V(\a \cap \fr b)$
	  \item $(U_i)_{i\in I}$ offen $\Rightarrow$ $\cup_{i\in I} U_i$ offen
	  	$\Longleftrightarrow$ $(M_i)_{i\in I}$ abgeschlossen
	  	$\Rightarrow$ $\cap_{i\in I} M_i$ abgeschlossen.\\
	  	Dazu:
	  	$\cap_{i\in I} V(\a_i) = V(\sum_{i\in I} \a_i)$
	\end{enumerate}
\end{proof}

\begin{bemerkung}
	Die abgeschlossenen Teilmengen $M \subset \Spec A$ sind genau die 
	$M = V(\a)$ f�r ein $\a \ideal A$.
\end{bemerkung}

\begin{beispiel}[$\Spec \Z$]
	F�r $\a \ideal \Z$ ist $\a = (a)$. Falls $a \neq 0,1,-1$ sei
	$a = \pm p_1^{\nu_1} \cdot \dots \cdot p_r^{\nu_r}$ die 
	Primfaktorzerlegung. F�r $p$ Primzahl ist
	\[
		(p) \in V((a)) \Leftrightarrow
		(a) \subseteq (p) \Leftrightarrow
		p \mid a \Leftrightarrow
		p \in \{p_1,\ldots, p_r\}
	\]
	Das bedeutet, die abgeschlossenen Mengen in $\Spec \Z$ sind genau die 
	Mengen $\emptyset$, $\Spec \Z$ und
	$\{(p_1), \ldots, (p_r)\}$ f�r eine endliche Anzahl an Primzahlen.
	
	Insbesondere gilt
	\begin{itemize}
	  \item $\Spec\Z$ ist nicht hausdorffsch.
	  \item $(0) =: \eta \in \Spec\Z$ liegt in \emph{jeder} nichtleeren 
	  	offenen Teilmenge.
	\end{itemize}
\end{beispiel}

\begin{lemma}
	Sei $x \in \Spec A$, so ist der Abschluss $\overline{\{x\}}$ der
	Menge $\{x\}$ in $\Spec A$ gleich
	\[\overline{\{x\}} = V(x).\]
\end{lemma}
\begin{proof}
	\[
		\overline{\{x\}} = 
		\bigcap_{B\subseteq \Spec A \text{ abg.}\atop x\in B} B
		= \bigcap_{\a\ideal A\atop \a \subseteq x}
		= V(x)
	\]
\end{proof}

\begin{bemerkung}
	Beachte, dass
	\[
		\a \subseteq \fr b \quad \Rightarrow\quad
		V(\fr b) \subseteq V(\a)
	\]
\end{bemerkung}

\begin{definition}[abgeschlossener Punkt, generischer Punkt]
	Sei $X$ ein topologischer Raum.
	Ein $x\in X$ hei�t \emph{abgeschlossener Punkt}, wenn
	$\overline{\{x\}} = \{x\}$.
	
	Er hei�t \emph{generischer Punkt}, wenn $\overline{\{x\}} = X$ gilt.
	
	Die Menge der abgeschlossenen Punkte bezeichnen wir mit
	$|X|$.
\end{definition}

\begin{beispiel}
	Sei $A = \C[X,Y]$. 
	\begin{itemize}
	  \item $x = (0) \in \Spec A$ ist generisch.
	  \item $x = (X-\alpha, Y-\beta) \ideal A$ ist abgeschlossen,
	  	da aus $x \ideal A$ maximal $V(x) = \{x\}$ und somit $x$ abgeschlossen
	  	folgt.
	  \item $x = (X) \ideal A$ ist weder abgeschlossen noch generisch.
	\end{itemize}
	Wir k�nnen die bisherigen Ergebnisse in
	\cref{fig:spec c xy} zusammenfassen. 
\end{beispiel}

\begin{figure}
	\caption{$\Spec \C[X,Y]$}
	\label{fig:spec c xy}
	\centering
	\begin{tikzpicture}
		\fill[col1shade1] (-3,-2) rectangle (3,2);
		\node[right, text=col1] 
			at (-2.8,-1.5)
			{$|\Spec\C[X,Y]|$};
		\draw[very thick]
			(-3,0) -- (3,0) node[near end, auto]{$\alpha$}
			(0,-2) -- (0,2) node[near end, auto]{$\beta$};
		\fill[col1]
			(1,1) circle[radius=2pt]
			node[above right] {$(X-\alpha, Y-\beta)$};
		\node[generic point=10pt, fill=black!60,
			label={above right:$(0)$}]
			at (4,0)
			{};
		\draw[line width=4pt, col2shade2, opacity=0.5]
			(0,-2) -- (0,2);
		\node[generic point=5pt, fill=col2shade2,
			label={[text=col2]below:$(X)$}]
			at (0,-2.1)
			{};
	\end{tikzpicture}
\end{figure}


\begin{definition}[basisoffene Menge]
	F�r $f\in A$ nennt man
	\[ D(f) := \Spec A \setminus V((f)) = \{ \p \in \Spec A \mid f \notin \p\}
	\]
	die \emph{zu $f$ geh�rige basisoffene Menge}.
\end{definition}

\begin{lemma}
	\label{lemma:basisoffene mengen sind basis}
	Die Menge $\fr B := \{D(f) \mid f \in A\}$ ist eine Basis der
	Topologie, d.h. jedes offene $U\subseteq \Spec A$ ist eine Vereinigung
	von $D(f) \in \fr B$ und $\fr B$ ist unter endlichen Schnitten 
	abgeschlossen.  
\end{lemma}
\begin{proof}
	Sei $U = \Spec A \setminus V(\a)$ offen und $\p \in U$, so ist
	$\p \notin V(\a)$, also $\a \not\subseteq \p$. Damit existiert
	$f \in \a \setminus \p$ mit $f \notin \p$, also $\p \in D(f)$
	und $f \in \a$. Also $(f) \subseteq \a$ und
	$V(\a) \subseteq V((f))$. Damit folgt $D(f) \subseteq U$.
	
	Zusammenfassend gilt f�r $U\subseteq \Spec A$ offen: $\forall \p \in U$
	$\exists f\p \in A$: $\p \in D(f\p) \subseteq U$.
	Also
	\[ U = \bigcup_{\p \in U} D(f\p)\]
	Ferner folgt mit \cref{lemma:vereinigungen von v sind produkt}
	$D(f) \cap D(g) = D(fg)$.
\end{proof}

\begin{lemma}
	\label{lemma:vereinigungen von v sind produkt}
	F�r $\a, \fr b\ideal A$ gilt
	\[
		V(\a) \cup V(\fr b) = V(\a \cap \fr b) = V(\a \cdot \fr b).
	\]
\end{lemma}
\begin{proof}
	Es ist 
	$\a\fr b \subseteq \a \cap \fr b \subseteq \a, \fr b$.
	Also 
	\[V(\a) \cup V(\fr b) \subseteq V(\a \cap \fr b) 
	\subseteq V(\a\fr b).\]
	Angenommen $V(\a) \cup V(\fr b) \subsetneq V(\a\fr b)$, 
	d.h. $\exists \p \in V(\a \fr b) \setminus \big(V(\a) \cup V(\fr b)\big)$,
	also $\a\fr b \subseteq \p$ aber nicht
	$\a,\fr b \not \subseteq \p$.
	Also existiert $s \in \a \setminus \p$ und $t\in\fr b\setminus \p$.
	Damit ist $st \in \a\fr b \setminus \p$.
	Dies ist ein Widerspruch, da $\p$ ein Primideal ist.
	Folglich herrscht Gleichheit in obiger Inklusionskette.
\end{proof}

\begin{definition}[Radikal]
	F�r $\a \ideal A$ hei�t
	\[
		\sqrt \a := \{ f\in A \mid \exists n \in \N:\ f^n\in \a\}
	\]
	\emph{Radikal} von $\a$.
\end{definition}

\begin{lemma}
	\label{lemma:radikal ist ideal}
	$\sqrt a \ideal A$.
\end{lemma}
\begin{proof}
	\begin{itemize}
	  \item $0\in \sqrt{\a}$ \checkmark
	  \item Sei $f \in \sqrt \a$, $r\in A$. Dann
	  	$f^n \in \a$, $r\in A$. Also 
	  	$(rf)^n \in \a$ und damit $rf\in \sqrt\a$.
	  \item $f,g\in \sqrt\a$ mit $f^n \in \a$, $g^m \in \a$.
	  	\begin{align*}
	  		(f+g)^{n+m-1} &= \sum_{i=0}^{n-1} \binom{n+m-1}{i} f^i g^{n+m-1-i}
	  			+ \sum_{i=n}^{n+m-1} \binom{n+m-1}{i}
	  				f^i g^{n+m-1-i}\\
  				&= \left( \sum_{i=0}^{n-1} \binom{n+m-1}{i} 
  					f^i g^{n-1-i}\right) g^m
  					+  \left(\sum_{i=n}^{n+m-1} \binom{n+m-1}{i}
  						f^i g^{m-1-i}\right) f^n
	  	\end{align*}
	  	Da $g^m$ und $f^n$ jeweils in $\a$ liegen, ist auch die Summe dort.
	\end{itemize}
\end{proof} 

\begin{definition}[Radikalideal (radiziell)]
	Ein Ideal $\fr b \ideal A$ hei�t \emph{Radikalideal (radiziell)},
	falls
	\[\sqrt \fr b = \fr b.\]
\end{definition}

\begin{bemerkung}
	Es gilt $\sqrt{\sqrt \a} = \sqrt\a$.
\end{bemerkung}

\begin{lemma}
	\label{lemma:radikal ist schnitt}
	F�r $\a \ideal A$ gilt
	\[
		\sqrt\a = \bigcap_{\p\in V(\a)} \p
	\]
\end{lemma}
\begin{proof}
	  \newcommand{\bmax}{\b_\text{max}}
	\begin{itemize}
	  \item["`$\subseteq$"']
	  	Sei $f \in \sqrt\a$, $f^n \in \a$. Ist $\p \in V(\a)$, d.h.
	  	$\a \subseteq\p$. Also
	  	$f^n \in \p$ und da $\p$ prim, folgt $f\in \p$.
	  \item["`$\supseteq$"']
	  	Ist $f\notin \sqrt\a$, so zu zeigen, dass 
	  	$f \notin \cap_{\p\in V(\a)} \p$.
	  	Sei also 
	  	$f^n \notin \a$ f�r alle $n\in \N$.
	  	
	  	Betrachte
	  	\[ M := \{\b \ideal A\mid a\subseteq \b,
	  		f^n \notin \b \forall n\in \N\},
	  	\]
	  	so gilt
	  	\begin{itemize}
	  	  \item $\a \in M$,
	  	  \item $M$ ist angeordnet durch "`$\subseteq$"',
	  	  \item ist $(\b_i)_{i\in I}$ eine total geordnete Teilmenge,
	  	  	so ist $\b:= \cup_{i\in I} \b_i \ideal A$ mit $\b \in M$.
	  	\end{itemize}
	  	Damit hat $M$ mit dem Lemma von Zorn ein maximales Element
	  	$\bmax \in M$.
	\end{itemize}
	Nun sei behauptet, dass $\bmax \ideal A$ ein Primideal ist.
	Dazu sei $xy \in \bmax$, wobei wir annehmen, dass 
	$x,y \notin \bmax$.
	Betrachte
	$\bmax \subsetneq (x) + \bmax$, was ein Ideal in $A$ ist, aber nicht 
	in $M$ liegt. Analog k�nnen wir dies von $(y) + \bmax$ sagen. Damit
	existieren $n,m \in \N$ mit
	\[
		f^n \in (x) + \bmax
		\qquad
		f^m \in (y) + \bmax.
	\]
	Ergo ist
	\[
		f^{n+m} \in
			(x)\bmax + (y)\bmax + \bmax\bmax + (xy),
	\]
	wobei jeder Summand Teilmenge von $\bmax$ ist und wir folgern
	$f^{n+m} \in \bmax \in M$, wodurch man den Widerspruch erh�lt.
	
	Damit ist $\bmax \in V(\a)$ und $f\notin \bmax$. 
\end{proof}

\begin{satz}
	\label{satz:v und radikal}
	F�r $\a, \b \ideal A$ gilt
	\[ V(\a) \subseteq V(\b) \quad\Leftrightarrow\quad
		\b \subseteq \sqrt\a.
	\]
	Insbesondere gilt
	\[ V(\a) = V(\b) \quad\Leftrightarrow\quad
		\b = \sqrt\a.
	\]
\end{satz}
\begin{proof}
	\begin{itemize}
	  \item["`$\Leftarrow$"']
	  	Aus $V(\a) \subseteq V(\b)$ folgt
	  	\[
	  		\bigcap_{\p \in V(\a)} \p 
	  		\supseteq \bigcap_{\p\in V(\b)} \p
	  	\]
	  	und mit \cref{lemma:radikal ist schnitt}
	  	folgt $\sqrt \a \supseteq \sqrt\b \supseteq \b$.
	  \item["`$\Rightarrow$"']
	  	Aus $\b \subseteq \sqrt\a$, d.h.
	  	$\b \subseteq \cap_{\p\in V(\a)} \p$, folgt
	  	$\b \subseteq \p$ f�r alle $\p\in V(\a)$.
	  	Also $\p \in V(\a)$.
	\end{itemize}
\end{proof}


\pagebreak
\section{Beispiele}

\subsection{$\Spec \Z$}

Jeder Ring $A$ hat einen eindeutigen Homomorphismus
\[
	\funcdef{
		\Z & \to & A\\
		 1 & \mapsto & 1\\
		 z & \mapsto & \begin{cases} 1 + 1 + \ldots + 1 & z > 0\\
		 	0 & z = 0\\
		 	-1 -1 - \ldots- 1 & z < 0
		 \end{cases}.}
\]
$\Z$ ist daher ein \emph{initiales Objekt} in der Kategorie $\Ring$.

Wir haben daher einen eindeutigen Morphismus $\Spec A \to \Spec \Z$ von
affinen Schemata. $\Spec \Z$ ist ein \emph{finales Objekt} in
der Kategorie $\affSch$.

Ferner können wir zusammenfassen
\paragraph{Offene Mengen}
	$\emptyset \neq U\subseteq \Spec \Z$ offen 
	$\Leftrightarrow$ $U = \Spec \Z \setminus \{(p_1),\ldots,(p_r)\}$
	
\paragraph{Basisoffene Mengen}
	$D(f) = \{\p \in \Spec\Z \mid f \notin \p\} = 
	\Spec\Z \setminus \{(p_1),\ldots,(p_r)\}$ für 
	$f = p_1^{\nu_1}\ldots p_r^{\nu_r}$.
	
\paragraph{Strukturgarbe}
	\begin{align*}
		\O_{\Spec\Z} (D(f)) &= \Z_f  = 
			\left\{ \frac{a}{f^n} \mid n\in \N_0, a\in \Z\right\} \\
		\O_{\Spec\Z, (p)} &= \Z_{(p)} = 
			\left\{ \frac{a}{b} \mid p\nmid b, a\in \Z\right\} 
 	\end{align*}

\subsection{$\Spec k$ für einen Körper $k$}
\paragraph{Als topologischer Raum}
	$\Spec k = \{(0)\}$.

\paragraph{Strukturgarbe}
	$\O_{\Spec k}(\{(0)\}) = k$.

\begin{bemerkung}
   Sei $A$ ein Ring. Angenommen wir haben 
  	$\Spec A \xto{(f,f\fis)} \Spec k$ für einen Körper $k$, so haben wir
  	\[
  		f\fis_{\Spec k}: k = \O_{\Spec k} \to f_\ast\O_{\Spec A}(\Spec k)
  			\tikzmark{=} A,
  	\]
  	wobei \tikzarrow{mark above}{} aus 
  	$\O_{\Spec A}(f\inv(\{(0)\})) = \O_{\Spec A}(\Spec A)$ resultiert.
  	Insgesamt ist $A$ also eine $k$-Algebra (d.h. ein Ring zusammen mit
  	$k\to A$).
  	
  	Bemerke hierbei "`Grothendiecks Gesamtphilosophie"':
  	\begin{quote}\itshape
  		Alles relativ lesen!
  	\end{quote}
\end{bemerkung}

\begin{definition}[$S$-Schema]
	Sei $S$ ein Schema. Dann ist ein \emph{$S$-Schema} ein Schema $X$
	zusammen mit einem Strukturmorphismus $X \xto{\varphi} S$.
	Dies ergibt die Kategorie $\Sch_S$, wenn man
	\[
		\Hom( X\xto{\varphi}S, Y\xto{\varphi} S) := 
		\left\{ 
		\begin{tikzcd}
		X \arrow{rr}{f} \drar{\varphi} & & Y \dlar{\psi} \\ & S &
		\end{tikzcd}
		\right\}
	\]
	setzt.
\end{definition}

\begin{beispiel}
	$\Sch_k := \Sch_{\Spec k}$ sind die sog. \emph{$k$-Schemata}.
	Ein Beispiel hierfür ist
	$\Spec k[X_1,\ldots,X_n] \to \Spec k$ via 
	$k \hookrightarrow k[X_1,\ldots,X_n]$.
\end{beispiel}


\begin{bemerkung}
	Sei $X$ ein Schema und $x\in X$ und weiter $\m_x \ideal \O_{X,x}$ das
	maximale Ideal.
	Dann ist 
	\[
		\kappa(x) := k(x) := \O_{X,x} \big/ \m_x
	\]
	der \emph{Restklassenkörper von $x$}.
	
	Betrachte nun $(f,f\fis): \Spec k \to X$ mit
	\[
		f: \funcdef{\Spec k(x) & \to & X \\
			\eta_x & \mapsto & x,}
	\]
	wobei topologisch gesehen $\eta_x \in \Spec k(x)$ der einzige Punkt 
	dieses Schemas ist.
	Für $U\subseteq X$ offen haben wir:
	\[
		f\fis_U : \O_X \to 
			f_\ast \O_{\Spec k(x)}(U) = 
			\begin{cases} 0 & x\notin U \\ k(x) & x \in U. \end{cases}
	\]
	Im Fall $x \in U$ geht dies via
	\[
		\O_X(U) \to \O_{X,x} = \varinjlim_{x\in V} \O_X(V)
			\overset\pi\twoheadrightarrow  \O_{X,x}\big/ \m_x = k(x). 
	\]
	
	Ist umgekehrt $(f,f\fis):\Spec k \to X$ ein Schemamorphismus, so
	setze $x := f((0)) \in X$ und
	$f\fis: \O_X \to f_\ast \O_{\Spec k}$ liefert einen Ringhomomorphismus der
	Halme:
	\[
		f_x\fis: \O_{X,x} \to \O_{\Spec k, (0)} = k.
	\]
	Dieser ist lokal (also $f\fis_x (\m_x) = (0)$). Damit ist
	\[
		\begin{tikzcd}
		k(x) = \O_{X,x} \big/ \m_x \rar[hookrightarrow]{f_x\fis \mod \m_x} 
		&[7ex]  {f_x\fis \mod \m_x} k
		\end{tikzcd}
	\]
	wohldefiniert und somit ist $k \mid k(x)$ eine Körpererweiterung.
	
	Zusammengefasst haben wir:
	\[	\fbox{\parbox{5cm}{
			Einen Punkt $x\in X$ wählen mit Restklassenkörper
			$k(x)$ und eine Körpererweiterung $k\mid k(x)$.}}
		\Longleftrightarrow
		\fbox{\parbox{5cm}{
			Einen Schemamorphismus $\Spec k \to X$ wählen
			für eine Körpererweiterung $k\mid k(x)$.}}
	\]
\end{bemerkung}

\subsection{Der Affine $n$-dimensionale Raum über $k$}
Sei $k$ wieder ein Körper. Der affine $n$-dimensionale Raum über $k$ ist
$\A_k^n := \Spec k[X_1,\ldots, X_n]$.

Wir erinnern an den Hilbertschen Nullstellensatz:
\begin{satz}[Hilbertscher Nullstellensatz]
	\label{satz:hilbertscher nullstellensatz}
	Sei $k$ algebraisch abgeschlossen. Dann ist jedes maximale Ideal
	in $k[X_1,\ldots, X_n]$ von der Form
	$(X_1-a_1, \ldots, X_n - a_n)$.
\end{satz}
\begin{proof}
	ohne Beweis.
\end{proof}

Wir haben bereits gezeigt:
\[
	|\A_k^n| = k^n, \qquad\text{via } 
		(X_1-a_1,\ldots,X_n-a_n) \mapsto (a_1,\ldots,a_n).
\]
Sei $\p = (f_1, \ldots, f_r)$ ein nicht maximales Ideal in $k[X_1,\ldots,X_n]$
(die Darstellung ist nach \thref{satz:hilbertscher nullstellensatz}) möglich,
so gilt
\[
	\p \subseteq (X_1 - a_1, \ldots, X_n - a_n)
	\quad\Leftrightarrow\quad
	f_1(a_1,\ldots,a_n) = 0, \ldots,
	f_r(a_1,\ldots,a_n) = 0
\]
Wir können dies in \autoref{fig:spec k xy 2} "`sehen"'.

\begin{figure}\centering
	\caption{$\Spec k[X_1,\ldots,X_n]$}
	\label{fig:spec k xy 2}
	\begin{tikzpicture}
		\draw[very thick] 
			(-3,0) -- (3,0) node[near end, above] {$X_1$}
			(0,-2) -- (0,2) node[near end, right] {$X_2$};
		
% 		\draw[col1,thick] 
% 			(-3,1) to[out=-5, in=135]  (0.2,-0.2) 
% 			to[out=-45, in=225, looseness=2] (-1,0) 
% 			to[out=45, in=180, looseness=0.5] (3,1.5)
% 			node[pos=0.9] {$f(X,Y) = 0$};
		\draw[col1, thick]
			(-3,1) 
			.. controls (5,-2) and (-8,-2) .. 
			(3,1.5)
			node[pos=1, right, text width=2.9cm, font=\scriptsize] 
				{$\{(a_1,\ldots,a_n) \mid f_j(a_1,\ldots,a_n) = 0,$\\ 
					$j=1\ldots r\}$}
			coordinate[pos=0.05] (a);
		
		\fill[col1shade2] (a) circle[radius=2pt]
			node[above right, col1] {$(\alpha,\beta)$};
		
		\path (-3,1)
			node[generic point=10pt, fill=col1shade2] {}
			node[above left, col1] {$\p$};
	\end{tikzpicture}
\end{figure}

\subsection{Ohne Titel}
Betrachte $k\ldbrack X_1, \ldots, X_n\rdbrack = 
	k\ldbrack X_1,\ldots,X_{n-1}\rdbrack\ldbrack X_n\rdbrack$
mit $R\ldbrack X\rdbrack = \{\sum_{i=0}^\infty a_i X^i \mid a_i \in R\}$.

\begin{bemerkung}
	$g \in k\ldbrack X_1, \ldots, X_n\rdbrack \setminus (X_1,\ldots,X_n)$
	ist eine Einheit.
\end{bemerkung}
\begin{proof}
	Idee: Ansatz für eine Variable:
	$g(X) = a_0 + a_1X + a_2X^2+ \ldots$. Dann
	\[
		1 = g(X)h(X) = 
		\underbrace{a_0 b_0}{= 1} + 
		(\underbrace{a_0b_1+a_1b_0}{= 0})X + \ldots
	\] 
\end{proof}

\paragraph{Funktor $\Spec$} Wir haben den Funktor $\Spec$:
Die Ringhomomorphismen
\[\everymath{\displaystyle} \begin{tikzcd}[row sep=tiny, outer sep=5pt]
	k[X_1,\ldots,X_n] \rar & k[X_1,\ldots,X_N]_{(X_1,\ldots,X_n)} \rar &
		k\ldbrack X_1,\ldots,X_n \rdbrack \rar & k \\
	f \rar[mapsto] & \frac{f}{1} \\
	& \frac f g \rar[mapsto] & f g\inv \\
	&& h \rar[mapsto] & h(0)
\end{tikzcd}\]
induzieren 
\[\everymath{\displaystyle} \begin{tikzcd}[row sep=tiny, outer sep=5pt]
	&\Spec k \rar & k\ldbrack X_1,\ldots,X_n \rdbrack \rar & 
	k[X_1,\ldots,X_N]_{(X_1,\ldots,X_n)} \rar &
	\Spec k[X_1,\ldots,X_n] \\
	\text{topologisch:} &  
	(0) \rar[mapsto] & (X_1,\ldots,X_N)  \rar[mapsto]& 
	(X_1,\ldots,X_n) \rar[mapsto] & (X_1 , \ldots,X_n).\\
	&&\makebox[0pt]{\parbox{3cm}{\centering\small 
		einziger abgeschlossener Punkt}} 
	&\makebox[0pt]{\parbox{3cm}{\centering\small 
		einziger abgeschlossener Punkt}}
	&\makebox[0pt]{\parbox{3cm}{\centering\small 
		entspricht dem abgeschlossenen Punkt $(0,\ldots,0) \in k^n$}}
\end{tikzcd}\]
Dies ist ein Homöomorphismus auf $\{\p \in \A_k^n \mid 
\p \subseteq (X_1,\ldots,X_n) = V(\p) = \overline{\{\p\}} \subseteq \A_k^n$.

Was passiert aber auf Schemaniveau?
\begin{center}\begin{tikzcd}[column sep=large]
	\node{\tikz{
		\fill[col1shade2] circle[radius=2pt];
	}};
	\rar &
	\node{\tikz{
		\draw[->]
			(-1,0) -- (1,0) node[very near end, above] {$X_1$};
		\draw[->]
			(0,-1) -- (0,1) node[very near end, right] {$X_n$};
		\fill[col1shade2] circle[radius=2pt];
		\node[text width=2cm, font=\scriptsize, text=col1shade2, right]
			 at (0.2,-0.5)
			 (text)
			 {einziger abgeschlossener Punkt};
	}};
	\rar &
	\node{\tikz{
		\draw[->]
			(-1,0) -- (1,0) node[very near end, above] {$X_1$};
		\draw[->]
			(0,-1) -- (0,1) node[very near end, right] {$X_n$};
		\fill[col1shade2] circle[radius=2pt];
		\node[text width=2cm, font=\scriptsize, text=col1shade2, right]
			 at (0.2,-0.5)
			 (text)
			 {einziger abgeschlossener Punkt};
		\draw[col1, thick, dashed]
			(-1,1) 
			.. controls (1.5,-0.8) and (-1.5,-0.8) .. 
			(1,1);
		\node[generic point=5pt, fill=col1] at (1,1) {};
		\node[right] at (1,1) {$\p$};
	}};
	\rar &
	\node{\tikz{
		\draw[->]
			(-1,0) -- (1,0) node[very near end, above] {$X_1$};
		\draw[->]
			(0,-1) -- (0,1) node[very near end, right] {$X_n$};
		\fill[col1shade2] circle[radius=2pt];
		\draw[col1, thick]
			(-1,1) 
			.. controls (1.5,-0.8) and (-1.5,-0.8) .. 
			(1,1);
		\node[generic point=5pt, fill=col1] at (1,1) {};
		\node[right] at (1,1) {$\p$};
	}};
\end{tikzcd}\end{center}
Betrachte dazu
\[\everymath{\displaystyle} \begin{tikzcd}[row sep=tiny, outer sep=5pt]
	\Spec k \rar & k\ldbrack X_1,\ldots,X_n \rdbrack \big/ \p \rar & 
	k[X_1,\ldots,X_N]_{(X_1,\ldots,X_n)} \big/ \p \rar &
	\Spec k[X_1,\ldots,X_n]\big/\p \quad \approx\quad V(\p)
	\end{tikzcd}
\]
Nehmen wir das explizite Beispiel $\p = (Y^2 - X^2(X+1))$. Es ist $\p$ ein
Primideal und $V(\p)$ irreduzibel.

Beachte: $1+X \in k\ldbrack X \rdbrack$ hat eine Wurzel, wie man durch
folgenden Ansatz mit $h(X) = a_0 + a_1 X + \ldots$ sieht:
\[
	1+ X = (h(X))^2 = a_0^2 + 2a_0a_1 X + \ldots
\]
Setze $a_0 := 1$ oder $-1$ und löse sukzessizve auf. Demnach ist
$Y^2 - X^2(X+1) = (Y-Xh(X))(Y + X h(X))$ nicht mehr prim, also
$V(\p) \subseteq k\ldbrack X,Y \rdbrack$ nicht mehr irreduziebel!

Betrachte genauer
\[\begin{tikzcd}[row sep=tiny]
	k \ldbrack u,v\rdbrack \big/(uv) \rar{\cong} & 
	k \ldbrack z,w\rdbrack \big/(z^2-w^2) \rar{\cong} &
	k \ldbrack X,Y\rdbrack \big/(Y^2 - X^2(h(X))^2)\\
	u \rar[mapsto] & z+w & z \rar[mapsto] & Y\\
	v \rar[mapsto] & z-w & wz \rar[mapsto] & Xh(X)\\
\end{tikzcd}\]
In Bildern:
\[\begin{tikzcd}[row sep=-15pt]
	\Spec k\ldbrack u,v\rdbrack \big/(uv) \rar & \Spec k\ldbrack X,Y\rdbrack
		\big/ (Y^2 - X^2(X+1)) \\
	\node{\tikz{
		\draw[->]
			(-1,0) -- (1,0) node[very near end, above] {$X$};
		\draw[->]
			(0,-1) -- (0,1) node[very near end, right] {$Y$};
		\draw[col1, opacity=0.4, line width=3pt]
			(-.8,0) -- (.8,0)
			(0,-.8) -- (0,.8);
	}}; \rar &
	\node{\tikz{
		\draw[->]
			(-1,0) -- (1,0) node[very near end, above] {$X$};
		\draw[->]
			(0,-1) -- (0,1) node[very near end, right] {$Y$};
		\draw[dashed]
			(1,1) 
			.. controls (-1.4,-2) and (-1.4,2) .. 
			(1,-1);
		\clip (-0.3,-0.3) rectangle (0.3,0.3);
		\draw[line width=3pt, col1, opacity=0.4]
			(1,1) 
			.. controls (-1.4,-2) and (-1.4,2) .. 
			(1,-1);
	}};
\end{tikzcd}\]

\subsection{Spezielles Beispiel $\A_\Z^1 = \Spec \Z[X]$}
Wir haben $\pi: \A_\Z^1 \to \Spec \Z$. Topologisch ist
\[
	\A_\Z^1 = \bigcup_{p \text{ prim}} \pi\inv((p)) \cup \pi\inv((0)).
\]
\autoref{fig:A 1 Z to Spec Z} verdeutlicht dies.

\begin{figure}
	\caption{Veranschaulichung von $\A_\Z^1 \to \Spec\Z$}
	\label{fig:A 1 Z to Spec Z}
	\centering
	\begin{tikzpicture}
		\draw[fill=col1shade1, draw=col1]
			(-0.5,0) rectangle (6,4);
		\node[right, text=col1shade2] at (6,2) {$\A_\Z^1$};
		
		\draw[col1, thick]
			(0,0) -- (0,4)
			node[near end, below, sloped] {$\pi\inv((0))$};
		\draw[col1, thick]
			(2,0) -- (2,4)
			node[near end, below, sloped] {$\pi\inv((2))$};
		\draw[col1, thick]
			(3,0) -- (3,4)
			node[near end, below, sloped] {$\pi\inv((p))$};
		
		\draw[->,thick] 
			(3,-0.5) -- (3,-1.5);
		
		\draw[col1shade2, thick]
			(-0.5,-2) -- (6,-2)
			node[right, text=col1shade2]{$\Spec \Z$};
		\draw[col1, thick]
			(0,-1.8) -- +(0,-0.4)
			node[below] {$(0)$};
		\draw[col1, thick]
			(2,-1.8) -- +(0,-0.4)
			node[below] {$(2)$};
		\draw[col1, thick]
			(3,-1.8) -- +(0,-0.4)
			node[below] {$(p)$};
	\end{tikzpicture}
\end{figure}

\paragraph{Zu $\pi\inv((0))$}
Betrachte nun $\p \in \Spec \Z[X]$, so gilt
$\p \in \pi\inv((0))$ $\Leftrightarrow$ $\p \cap \Z = (0)$.

Betrachte $S:= \Z \setminus \{0\} \subseteq \Z[X]$ und die Lokalisierung
$g: \Z[X] \hookrightarrow \Z[X]_S$. Es ist klar: $\Z[X]_S = \Q[X]$

Ferner gilt $\Spec \Q[X] \to \Spec \Z[X]$ ist ein Homöomorphismus auf sein 
Bild:
\[
	\{\p \in \Spec\Z[X] \mid \p \cap S = \emptyset \} = 
	\{\p \in \A_\Z^1 \mid \p \cap \Z = (0) \} = \pi\inv(0),
\]

\paragraph{Zu $\pi\inv((p))$}
Es ist $\p \in \pi\inv((p))$ $\Leftrightarrow$ $p \in \p$.
Dann betrachte
$\rho: \Z[X] \twoheadrightarrow \bb F_p[X]$ und
$\rho^\ast: \Spec \bb F_p[X] \to \A_\Z^1$.
Wegen $\bb F_p[X] \cong \Z[X] \big/ \ker\rho$ ist $\rho^\ast$ ein Homöomorphismus
auf 
\[
	V(\ker \rho) = \{\p \in \Spec\Z[X] \mid \ker \rho \subseteq \p\} = 
	\pi\inv((p)) \subseteq \A_\Z^1.
\] 

Zusammengefasst ist:
\begin{align*}
	\pi\inv((0)) &= \A_\Q^1\\
	\pi\inv((p)) &= \A_{\bb F_p}^1,
\end{align*}
wobei die Gleichheiten topologisch zu lesen sind.

\paragraph{Betrachte $\p\in \Spec\Z[X]$}
\begin{description}
\item[1. Fall.]
	$\p\in \pi\inv((0))\ \Leftrightarrow\ \p\cap \Z = (0)$, also
	\[
		\p = (\mu(X))
	\]
	mit $\mu(X) \in \Z[X]$ einem primitiven, irreduziblen Polynom.
\item[2. Fall.]
	$\p\in\pi\inv((p))$, so ist $\p = \rho\inv(\q)$ für ein 
	$\q\in \Spec\bb F_p[X]$, also
	$\p = \rho\inv((q(X)))$ für ein irreduzibles $q(X)\in \bb F_p[X]$
	oder $(0)$. Dann ist
	\[
		\p = (r(X), p)
	\]
	mit $r(X) \in \Z[X]$ und $r(X) \equiv q(X) \bmod p$.
\end{description}
Es stellt sich die Frage, wie für $f\in \Z[X]$ die $D(f) \subseteq \A_\Z^1$
aussehen. Dazu
\begin{description}
\item[1. Fall $\p\in \pi\inv((0))$.] Sei $f(X) \in \Q[X]$. Dann
	$f(X) = \xi q_1(X)^{\nu_1} \ldots q_r(X)^{\nu_r}$ und es gilt
	\[
		f\notin \p \ \Leftrightarrow\ \p = (q(X))
	\]
	mit $q \neq q_1, \ldots, q_r$.
\item[2. Fall $\p \in \pi\inv((p))$.] $f(X) \notin (r(X), p)$
	mit $r(X) \mod p \in \bb F_p[X]$ irreduzibel. Für eine Primzahl $p$,
	betrachte $\bar f(X) \in \bb F_p[X]$.
	Ist $\bar f(X) = 0$, so ist $f(X) \in (r(X),p)$ für alle $r(X)$.
	Für $\bar f(X) = \bar q_1(X)^{\nu_1} \ldots \bar q_s(X)^{\nu_s}$, ist
	$f(X) \in (q_i(X), p)$ für diese $i$.
\end{description}
Dargestellt ist dies wieder in \autoref{fig:A 1 Z to Spec Z 2}.

\begin{figure}
	\caption{Veranschaulichung von $D(f) \subseteq \A_\Z^1$}
	\label{fig:A 1 Z to Spec Z 2}
	\centering
	\begin{tikzpicture}
		\draw[fill=col1shade1, draw=col1]
			(-0.5,0) rectangle (6,4);
		\node[right, text=col1shade2, font=\scriptsize] at (6,2) {$\A_\Z^1$};
		
		\draw[col1, thick]
			(0,0) -- (0,4)
			(2,0) -- (2,4)
			(3,0) -- (3,4)
			(3.5,0) -- (3.5,4)
			(4,0) -- (4,4)
			(4.5,0) -- (4.5,4);
		
		\draw[->,thick] 
			(3,-0.5) -- (3,-1.5);
		
		\draw[col1shade2, thick]
			(-0.5,-2) -- (6,-2)
			node[right, text=col1shade2]{$\Spec \Z$};
		\draw[col1, thick]
			(0,-1.8) -- +(0,-0.4)
			(3,-1.8) -- +(0,-0.4)
			(3.5,-1.8) -- +(0,-0.4)
			(4,-1.8) -- +(0,-0.4)
			(4.5,-1.8) -- +(0,-0.4);
		\draw[thick, col2]
			(2,-1.8) -- +(0,-0.4);
			
		\foreach \x in {0, 3}{	
			\foreach \y in {0.5,1,...,3}{
				\fill[col2] (\x,\y) circle[radius=2pt];
			}
		}
		\fill[col2] (4,3) circle[radius=2pt];
		\fill[col2] (4.5,2) circle[radius=2pt];
		\draw[col2] (2,0) -- (2,4);
		
		\node[right, col2, text width=3cm, font=\scriptsize] at (3.5,-0.5) 
			{irreduzible Teiler von $\bar f\in \bb F_p[X]$};
			
		\node[below, col2, text width=2cm, font=\scriptsize] at (0,0) 
			{irreduzible Faktoren von $f$};
			
		\node[below, col2, text width=4cm, font=\scriptsize] at (2,-2.5)
			{Primteiler aller Koeffizienten von $f$};
			
		\fill[col2] (7.5,3) circle[radius=3pt] 
			node[right] {\ $\notin D(f)$};
		\fill[col1] (7.5,2) circle[radius=3pt] 
			node[right] {\ $\in D(f)$}; 
	\end{tikzpicture}
\end{figure}
\pagebreak

% vim: set ft=tex :
\section{Projektive Schemata}
% vim: set ft=tex :

\section{Eigenschaften von Schemata} %Seite 79
% vim: set ft=tex :

\section{Faserprodukt} %Seite 105
% vim: set ft=tex :

\section{Glatt, regulär & normal} %Seite 125
% vim: set ft=tex :

\section{$k$-Varietät} %Seite 154

\begin{beispiel}
    Ein einführendes Beispiel einer $k$-Varietät ist gegeben durch
    abgeschlossene Unterschemata, wie beispielsweise
    \[ \Spec k[X_1,\ldots,X_n] \big/\a \to \Spec k[X_1,\ldots,X_n] = \A^n_k.\]
\end{beispiel}


\begin{definition}[endlich]
    \label{def:ringhom endlich}
    \index[def]{Ringhomormophismus!endlich}
    Ein Ringhomomorphismus $\varphi: B \to A$ heißt \emph{endlich},
    wenn $A$ dadurch zu einem endlich erzeugten $B$-Modul wird.
\end{definition}

\begin{satz}[Noether-Normalisierung]
    Sei $A$ eine endlich erzeugte $k$-Algebr. Dann existiert $d\geq 0$ und
    ein endlicher injektiver Ringhomomorphismus
    \[k[T_1,\ldots,T_n] \hookrightarrow A.\]
\end{satz}
\begin{proof}
\TODO
\end{proof}

\begin{korollar}
    Ist $A$ eine endlich erzeugte $k$-Algebra und $\m \ideal A$ maximal,
    dann ist $A\big/\m$ eine endliche Körpererweiterung von $k$.
\end{korollar}
\begin{proof}
\TODO
\end{proof}

\begin{korollar}
    Sei $X$ eine $k$-Varietät und $x\in X$ ein abgeschlossener Punkt, so ist
    $k(x)\mid k$ endlich.
\end{korollar}
\begin{proof}
klar.
\end{proof}

\begin{satz}[(schwacher) Hilbertscher Nullstellensatz]
    Sei $k$ algebraisch abgeschlossen, $\m\ideal k[X_1,\ldots,X_n]$ ein
    maximales Ideal, so gilt
    \[\m = (X_1-a_1,\ldots,X_n-a_n)\]
    für geeignete $a_1,\ldots,a_n \in k$.
\end{satz}
\begin{proof}
\TODO
\end{proof}


\begin{lemma}
    Sei $X$ eine irreduzible algebraische $k$-Varietät, dann gilt für
    $x\in |X|$:
    \[\dim \O_{X,x} = \dim X.\]
\end{lemma}
\begin{proof}
\TODO
\end{proof}

\begin{lemma}
    Ist $\m \ideal k[X_1,\ldots,X_n]$ ein maximales Ideal, so existieren
    Polynome
    \[f_1(X_1), f_2(X_1,X_2), \ldots, f_n(X_1,\ldots,X_n)\]
    mit
    \[\m = (f_1,\ldots,f_r).\]
\end{lemma}
\begin{proof}
Induktion mit Noethernormalisierung.
\end{proof}

\begin{folgerung}
    $\A^n_k$ ist regulär bei allen $x \in |\A^n_k|$.
\end{folgerung}


% vim: set ft=tex :

\section{Der Punktefunktor} %Seite 159
% vim: set ft=tex :

%\part{Zweites Semester}
\section{$\O_X$-Moduln}
\newcommand{\OX}{$\O_X$-}

\subsection{$\O_X$-Moduln}
\begin{definition}[$\O_X$-Modul]
    \index[def]{O-Modul@\OX-Modul@\OX-Modul|hypertarget{def:ox modul}{}}
    Ein \emph{$\O_X$-Modul} (oder eine \emph{$\O_X$-Modulgarbe}) ist eine 
    Garbe $\M$ zusammen mit einer $\O_X(U)$-Modulstruktur auf $\M(U)$
    für jedes offene $U\osubset X$, so dass für 
    $V \osubset U \osubset X$ folgendes Diagramm kommutiert:
    \[\begin{tikzcd}
        \O_X(U) \times \M(U) \rar \dar{\cdot\rest V \times \cdot \rest V}
        & \M(U) \dar{\cdot \rest V}\\
        \O_X(V) \times \M(V) \rar & \M(V)
    \end{tikzcd}\]
    
    Ein \emph{Morphismus} $\M \to \M'$ von solchen ist ein Garbenmorphismus
    $\alpha: \M\to \M'$, so dass für jedes $U\osubset X$ 
    $\alpha(U):\M(U) \to \M'(U)$ $\O_X(U)$-linear ist.
\end{definition}


\begin{bemerkung}
    Man hat einige Konstruktionen aus der kommutativen Algebra auch für
    \OX Moduln, wie z.B.
    \begin{itemize}
      \item $\M \otimes_{\O_X} \M': U \mapsto \M(U) \otimes_{\O_X(U)} \M'(U)$.
      \item $\oplus_{i\in I} \M_i$ von $\O_X$-Moduln $\M_i$.
      \item Für $\alpha:\M\to \M'$ $\O_X$-Modul-Morphismus haben wir
        $\ker \alpha$ und $\im \alpha$,
        wobei Kern und Bild in $\Sh_X$ zu lesen sind.
    \end{itemize}
\end{bemerkung}


\begin{definition}[frei, lokal frei]
    \index[def]{O-Modul@\OX-Modul!frei}
    \index[def]{O-Modul@\OX-Modul!lokal frei}
    Ein \hyperlink{def:ox modul}{\OX Modul} $\M$ heißt
    \begin{itemize}
      \item \emph{frei}, wenn es eine Menge $I$ und einen 
        \OX Modul-Isomorphismus 
        \[ \O_X^{(I)} := \bigoplus_{i\in I} \O_X \xto{\cong} \M\]
        gibt,
      \item \emph{lokal frei} oder \emph{Vektorbündel von Rang $r$}, 
        wenn es zu jedem $x\in X$ ein $x \in U\osubset X$ und einen
        $\O_U$-Modul-Isomorphismus
        \[ \O_U^r \xto{\cong} \M\rest U\]
        gibt.
    \end{itemize}
\end{definition}

\subsection{Exkurs: Vektorbündel in der Topologie}
Sei $X$ ein topologischer Raum. Dann ist ein $\R$-Vektorbündel vom Rang $r$
eine stetige Abbildung $\pi: E \to X$ mit einer $\R$-Vektorraumstruktur
auf $E_x := \pi\inv(\{x\})$ zusammen mit einem sog Bündelatlas,
bestehend aus Karten
\[ \psi_U: E\rest U := \pi\inv(U) \to U\times \R^r\]
mit $\pr_U \circ \psi_U = \pi\rest{\pi\inv(U)}$, d.h.
\[\begin{tikzcd}
    E\rest U = \pi\inv(U) \ar{rr}{\approx} 
    \drar{\pi}& &  U\times \R \dlar{\pr_U}\\
    & U &
\end{tikzcd}\]
kommutiert und die Karten sind
\begin{itemize}
  \item Homöomorphismen und so, dass
  \item $\psi_x: E_x \to \{x\} \times \R^r$ ein linearer Isomorphismus ist.
\end{itemize}

\paragraph{Wie verstehen wir das als Garbe von Moduln?}

Setze $\O_X := U \mapsto \O_X(U) := \{ f: U \to \R \mid f\text{ stetig}\}$,
also die Garbe der stetigen Funktionen. Dann ist $(X, \O_X)$ ein lokal
geringter Raum. Weiter haben wir $E \xto{\pi} X$ stetig.
Setze 
\[ \cal E : U \mapsto \cal E(U) := \{\sigma: U \to \pi\inv(U) \subseteq E \mid
    \sigma\text{ stetig, } \pi \circ \sigma = \id_U\}.\]
Dies ist eine Garbe. $\cal E$ ist sogar eine \OX-Modulgarbe:
Für $U\osubset X$ gilt
\[ \O_X(U) \times \cal E(U) \to \cal E(U),\ (f,\sigma) \mapsto f\cdot \sigma.\]
wobei
\[f\cdot \sigma : \funcdef{ U & \to & \pi\inv(U) \\
    x & \mapsto & \underbrace{f(x)}_{\in \R} \cdot 
    \underbrace{\sigma(x)}_{\in E_x}}
\]
und $E_x$ ein $\R$-Vektorraum ist.
 
Bleibt nur noch zu klären, wie die Bündelkarten 
$\psi_U: E\rest U = \pi\inv(U) \xto{\cong} U \times \R^r$ eingehen:
\[\begin{tikzcd}
    \pi\inv(U) \rar \drar[swap]{\pi} & U \times \R^r \dar[swap]{\pr_U} 
        & \lar[empty][description]{\ni} 
        (x,\alpha(x)) := (x, \pi_{\R^r} \circ \psi_U \circ \sigma(x)) \\
    & U \ular[bend left, mapsto]{\cal E(U) \ni \sigma}
        \uar[bend right,swap]{\psi_U \circ \sigma} & 
        x \lar[empty][description]{\ni} \uar[mapsto]
\end{tikzcd}\]

$\alpha: U \to \R^r$ ist eine stetige Abbildung, also
$\alpha \in \O_X(U)^r$. Weiter liefert $\psi_U$ einen 
$\O_X(U)$-Modul-Isomorphismus
\[
    \funcdef{ \cal E(U) & \xto{\cong} & \O_X(U)^r \\ 
        \sigma & \mapsto & \pr_{\R^r} \circ \psi_U \circ \sigma \\
        \psi_U\inv \circ (\id_U \times \alpha)  & \mapsfrom & \alpha.}
\]
Schränkt man auf $V \osubset U$ ein, ist dies verträglich. Also
\[ \cal E\rest U \cong \O_X(U)\]
als $\O_X\rest U$-Modulgarben.

\subsection{Quasi-Kohärenz}

\begin{definition}[quasi-kohärent]
    \index[def]{O-Modul@\OX-Modul!quasi-kohärent}
    Eine $\O_X$-Modulgarbe $\M$ heißt \emph{quasi-kohärent}, wenn es
    zu jedem $x\in X$ ein $x \in U \osubset X$ und Mengen $I,J$ und
    eine exakte Sequenz von $\O_U$-Modulgarben
    \[\begin{tikzcd}
        \O_X\rest U^{(J)} \rar &\O_X\rest U^{(J)} \rar &
        \M\rest U \rar & 0
    \end{tikzcd}\]
    gibt.
\end{definition}

\begin{definition}[von seinen globalen Schnitten erzeugt]
    \index[def]{O-Modul@\OX-Modul!von seinen globalen Schnitten erzeugt}
    Ein \OX Modul $\M$ wird \emph{von seinen globalen Schnitten erzeugt},
    wenn für jedes $x\in X$ der Morphismus von $\O_{X,x}$-Moduln
    \[ \M(X) \otimes_{\O_X(X)} \O_{X,x} \to \M_x\]
    surjektiv ist.
\end{definition}

Mit anderen Worten: Jeder Keim $m_x \in \M_x$ lässt sich schreiben als
\[ m_x = \sum_{\text{endl. viele }i} \lambda_i [\sigma_i]_x\]
für $\lambda_i \in \O_{X,x}$ und $\sigma_i \in \M(X)$.

Dies gilt nicht für $\O_X$ selbst; betrachte beispielsweise
$X = \C\P^1$ und $\O_X$ die Garbe der holomorphen Funktionen.

\begin{bemerkung}
    Es existiert ein surjektives 
    $\O_X\rest U^{(I)} \twoheadrightarrow \M\rest U$ genau dann, wenn
    $\M\rest U$ durch seine auf $U$ globalen Schnitte erzeugt wird.
    
    $\M$ ist quasi-kohärent genau dann, wenn
    $\M\rest U$ durch seine globalen Schnitte erzeugt wird und die Relationen
    (also $\ker(\O_X\rest U^{(I)}) \to \M)$) auch.  
\end{bemerkung}

\subsection{Quasikohärente Garben auf $\Spec A$}

\paragraph{Beachte folgende Konstruktion}
Ist $M$ ein $A$-Modul, so betrachte
\begin{itemize}
  \item für $f \in A$: $M_f = M \otimes_A A_f$ 
    als $A_f = \O_{\Spec A}(D(f))$-Modul.
  \item für $\p \in \Spec A$: $M_\p = M\otimes_A A_\p$
    als $A_\p = \O_{\Spec A,\p}$-Modul.
\end{itemize}
Dies ist eine $\fr B$-Garbe für $\fr B = \{D(f)\mid f\in A\}$ der Basis 
der Topologie auf $\Spec A$. Dann folgt analog zu 
\thref{satz:spec a hat eindeutige ringgarbe} folgender Satz.

\begin{satz}
    \label{satz:a modul hat ein o spec a modulgarbe}
    Zu gegebenem $A$-Modul $M$ existiert (bis auf Isomorphie) genau eine
    $\O_{\Spec A}$-Modulgarbe $M^\sim$ auf $X = \Spec A$ mit
    \begin{align*}
        M^\sim (D(f)) &\cong M_f\\
        (M^\sim)_\p &\cong M_\p
    \end{align*} 
    Insbesondere ist $M^\sim(\Spec A) = M$.
\end{satz}


% 18.04.2013

\begin{satz}
    \label{satz:sim exakt}
    Der Funktor
    \[
        ^\sim: \funcdef{ \Moduln{A} & \to & \Moduln{\O_{\Spec A}} \\
            M & \mapsto & M^\sim\\
            (M \xto{\varphi} N) & \mapsto &  (M^\sim \xto{\varphi^\sim}N^\sim)}
    \]
    ist exakt.
\end{satz}
\begin{proof}
    Es ist zu zeigen: Ist
    \[ M' \xto\alpha M \xto\beta M''\]
    eine exakte Sequenz in $\Moduln{A}$, so ist
    \[ (M')^\sim \xto{\alpha^\sim} M^\sim \xto{\beta^\sim} (M'')^\sim\]
    eine exakte Sequenz in $\Moduln{\O_{\Spec A}}$. Letzteres ist aber
    äquivalent dazu, dass
    \[ (M')_\p^\sim \xto{\alpha_\p^\sim} M_\p^\sim \xto{\beta_\p^\sim}
         (M'')_\p^\sim\]
    eine exakte Halmsequenz für alle $\p\in\Spec A$ ist.
    Dies ist aber klar, weil $\A_\p$ flach über $A$ ist 
    (\autocite[Example 9.1.1]{hartshorne1977algebraic} oder
    \autocite[Abschnitt 7 Satz 8]{bosch2009algebra}) und
    $M_\p^\sim = M_\p \cong M\otimes_A A_\p$. 
\end{proof}

\begin{korollar}
    \label{kor:m a modul dann m sim quaiskohaerent}
    Für einen $A$-Modul $M$ ist $M^\sim$ quasi-kohärent.
\end{korollar}
\begin{proof}
    Für $M$ hat man
    \[ A^{(J)} \to A^{(I)} \xto\varphi M \to 0.\]
    Nun wähle beispielsweise $I := M$ und $J := \ker\varphi$.
    Ferner ist 
    \[ (A^{(J)})^\sim = (\oplus_{j\in J}A)^\sim = \oplus_{j\in J}
        A^\sim = \oplus_{j  \in J} \O_X = \O_X^{(J)}\]
    und da $^\sim$ exakt ist, folgt die Exaktheit von
    \[ \O_X^{(J)} \to \O_X^{(I)} \to M^\sim \to 0.\] 
\end{proof}

\begin{bemerkung}
    Sind $M$ und $N$ $A$-Moduln, so ist
    \[ (M\otimes_A N)^\sim = M^\sim \otimes_{\O_{\Spec A}} N^\sim.\]
\end{bemerkung}

\begin{satz}
    \label{satz:modulgarbe quasikohaerent <=> iso auf einschraenkung}
    Sei $(X,\O_X)$ ein Schema. Dann ist eine $\O_X$-Modulgarbe $\M$
    genau dann quasi-kohärent, wenn für jede affin offene Teilmenge
    $U$ ein Isomorphismus
    \[ \M\rest U \cong (\M(U))^\sim\]
    existiert.
\end{satz}
\begin{proof}
    \begin{description mathquote}
    \item[\Leftarrow] Folgt aus \thref{kor:m a modul dann m sim quaiskohaerent}.
    \item[\Rightarrow]
        Aus nachstehenden Hilfslemmas haben wir die Behauptung, da
        $\M(U)^\sim$ durch die Eigenschaft auf den $D(f)$s festgelegt ist.
    \end{description mathquote}
\end{proof}

\begin{hilfslemma}
    In der Situation von 
    \thref{satz:modulgarbe quasikohaerent <=> iso auf einschraenkung} gilt:
    Für jedes $x \in X$ existiert ein affin offenes $ x \in U\osubset X$
    mit $\M\rest U \cong (\M(U))^\sim$.
\end{hilfslemma}
\begin{proof}
    Man hat den kanonischen Garbenmorphismus
    \[ (\M(U))^\sim \to \M\rest U.\]
    Dieser rührt her von
    \[(\M(U))^\sim (D(f)) = \M(U)_f \xto\varrho \M(D(f)),\]
    welcher induziert wird von den beiden Restriktionen
    $\res^\M: \M(U) \to \M(D(f))$ und
    $\res^\O: \O_X(U) \to \O_X(D(f))$, da 
    wird $\M(U)$ als einen $\O_X(U)$-Modul und $\M(D(f))$ als einen
    $\O_X(D(f))$-Modul auffassen wollen. Demnach haben wir für
    $\lambda \in \O_X(U)$ und $m\in \M(U)$
    \[\res^\M(\lambda m) = \res^\O(\lambda) \res(m).\]
    Weiter ist $f \in A_f^\times = (\O_U(D(f)))^\times$, also 
    dort invertierbar und wir können setzen
    \[\rho(\tfrac{m}{f^n}) := \res(m) f^{-n}.\]
    
    Da $\M$ quasi-kohärent existiert für alle $x \in X$ ein affin offenes
    $x\in U\osubset X$, so dass
    \[\O_X\rest U^{(J)} \xto\beta \O_X\rest U^{(I)} \xto\alpha 
    \M\rest U \xto{} 0 \] 
    exakt ist. Insbesondere haben wir
    \[\O_X(U)^{(J)} \to \O_X(U)^{(I)} \to 
    \M(U).\]
    Setze nun $N:= \im(\alpha(U))\subseteq \M(U)$. $N$ ist ein 
    $\O_X(U)$-Untermodul. Damit ist
    \[\O_X(U)^{(J)} \to \O_X(U)^{(I)} \to 
        N \to 0\]
    eine exakte Sequenz von $\O_X(U)$-Moduln. Wir wenden $^\sim$ an 
    und da $^\sim$ exakt (\thref{satz:sim exakt}) erhalten wir
    \[\O_X\rest U^{(J)} \to \O_X\rest U^{(I)} \to
        N^\sim \to 0.\]
    Mit dem Homomorphiesatz folgt dann $\M\rest U \cong N^\sim$. 
\end{proof}

\begin{hilfslemma}
    In der Situation von 
    \thref{satz:modulgarbe quasikohaerent <=> iso auf einschraenkung} gilt:
    Für beliebiges $U = \Spec A \osubset X$ und $f\in A = \O_X(U)$ gilt
    \[ \M(U)_f \cong \M(D(f)).\]
\end{hilfslemma}
\begin{proof}
    Wir überdecken $U = \bigcup_{i=1}^r U_i$ durch endlich viele affin 
    offene $U_i$ (es reichen endlich viele, da $\Spec A$ quasi-kompakt!).
    Die $U_i$ wählen wir dabei so, dass sie die Eigenschaften 
    im ersten Hilfslemma genügen
    und setzen $V_i = U_i \cap D(f) = D(f\rest{U_i})$.
    Dann haben wir
    \[\begin{tikzcd}
        0 \rar & \M(U)_f \rar \dar{\alpha}[swap]{\text{kanonisch}} & 
        \oplus_i \M(U_i)_f \rar \dar{\cong}[swap]{\beta} & 
        \oplus_{(i,j)} \M(U_i \cap U_j)_f \dar{\cong}[swap]{\gamma}\\
        0 \rar & \M(D(f)) \rar & \oplus_i \M(V_i) \rar & 
        \oplus_{(i,j)} \M(V_i \cap V_j),
    \end{tikzcd}\]
    wobei die Zeilen jeweils exakt sind und die Isomorphismen sich aus
    dem ersten Hilfslemma ergeben. Man erjagt sich aus $\beta$ ein 
    Isomorphismus, dass $\alpha$ injektiv ist und zusammen mit $\gamma$
    einem Isomorphismus, kann man erneut auf Jagd gehen und
    die Surjektivität von $\alpha$ erlegen.    
\end{proof}


\begin{satz}
    \label{satz:kurze exakte sequenz mit quasikohaerent bleibt exakt}
    Ist $X = \Spec A$ affin und 
    \[\begin{tikzcd}
        0 \rar & \M' \rar{\alpha} & \M \rar{\beta} & \M'' \rar & 0
    \end{tikzcd}\]
    eine kurze exakte Sequenz von $\O_X$-Modulgarben und ist
    $\M'$ quasikohärent, so ist
    \[\begin{tikzcd}
        0 \rar & \M'(X) \rar{\alpha(X)} & \M(X) \rar{\beta(X)} & 
        \M''(X) \rar & 0
    \end{tikzcd}\]
    eine kurze exakte Sequenz von $A$-Moduln.
\end{satz}

Bevor wir den Beweis des Satzes angeben, wollen wir in folgendem Lemma und 
anschließendem Beispiel sehen, dass die Bedingung der Quasikohärenz wirklich
notwendig ist, um Rechtsexaktheit zu garantieren.

\begin{lemma}
    \label{lemma:raum in garbe einsetzen ist linksexakt}
    Für jeden topologischen Raum $X$ ist
    \[\Gamma(X,\_): \Sh_X \to \Ab,\ \F \mapsto \F(X) =: \Gamma(X,\F)\] 
    linksexakt,
    d.h. ist
    \[\begin{tikzcd}
        0 \rar & \F \rar{\alpha} & \G \rar{\beta} & \cal H \rar & 0
    \end{tikzcd}\]
    eine kurze exakte Sequenz in $\Sh_X$, so ist
    \[\begin{tikzcd}
        0 \rar & \F(X) \rar{\alpha(X)} & \G(X) \rar{\beta(X)} & 
        \cal H(X)
    \end{tikzcd}\]
    eine exakte Sequenz in $\Ab$.
\end{lemma}
\begin{proof}
    Zeigen wir zunächst die Injektivität von $\alpha(X)$: Sei
    $\sigma \in \F(X)$ mit $\alpha(X)\sigma = 0\in \G(X)$, so ist
    $[\alpha(X)\sigma]_x  = 0 \in \G_x$ für alle $x \in X$, also ist
    $\alpha_x([\sigma]_x) = 0$ mit $[\sigma]_x \in \F_x$
    und da $\alpha_x$ injektiv nach Voraussetzung, folgt
    $[\sigma]_x = 0$, ergo $\sigma = 0$.
    
    Als zweites folgern wir $\ker \beta(X) = \im \alpha(X)$:
    Da $\beta \circ \alpha = 0$, folgt $\beta(X) \circ \alpha(X) = 0$,
    also $\im\alpha(X) \subseteq \ker \beta (X)$.
    Sei nun $\sigma \in \ker\beta (X)$. Insbesondere gilt für jedes 
    $U \osubset X$, dass $\beta(U)\sigma\rest U = 0 \in \cal H(U)$.
    Da $\ker \beta = \im\alpha$ nach Voraussetzung, existiert eine
    offene Überdeckung $X = \cup_{i\in I} U_i$ mit 
    $\ker\beta(U_i) = \im\alpha(U_i)$. Also finden wir zu jedem
    $i \in I$ ein $\tau_i \in \F(U_i)$ mit 
    $\alpha(U_i)\tau_i = \sigma\rest{U_i}$. Wir müssen nur noch sehen, 
    dass diese geeignet verkleben:
    Es gilt
    \[\alpha(U_i \cap U_j)\tau_i\rest{U_i \cap U_j} = \sigma\rest{U_j\cap U_j}
         = \alpha(U_i \cap U_j) \tau_j\rest{U_j \cap U_j}\]
    und mit der Injektivität von $\alpha(U_i \cap U_j)$ folgt
    \[\tau_i\rest{U_i \cap U_j} = \tau_j\rest{U_i \cap U_j}.\]
    Also verkleben die $(\tau_i)_{i\in I}$ zu $\tau \in \F(X)$ mit
    $\alpha(X)\tau = \sigma$.
\end{proof}

\begin{beispiel}
    In \thref{lemma:raum in garbe einsetzen ist linksexakt} ist
    die Rechtsexaktheit im Allgemeinen nicht gegeben, wie man am Beispiel
    $X = \C\setminus\{0\}$ sieht: Setze $\G := \O_{\C^\times}$ die 
    Garbe der holomorphen Funktionen und 
    $\cal H := \O_{\C^\times}^\times$ die Garbe der nirgends verschwindenden
    holomorphen Funktionen, so ist
    \[\begin{tikzcd}
        0 \rar & 2\pi i \Z \rar & \G \rar{\exp} & 
        \cal H \rar & 0
    \end{tikzcd}\]
    eine kurze exakte Sequenz, aber
    \[\begin{tikzcd}
        0 \rar & 2\pi i \Z \rar & \G(X) = \O_{\C^\times}(\C\setminus\{0\}) 
        \rar{\exp} & 
        \cal H(X) =  \O_{\C^\times}^\times(\C\setminus\{0\})
    \end{tikzcd}\]
    ist alles, da die letzte Abbildung nicht surjektiv ist (es gibt keinen
    komplexen Logarithmus auf $\C\setminus\{0\}$).
\end{beispiel}


\begin{proof}[von 
    \thref{satz:kurze exakte sequenz mit quasikohaerent bleibt exakt}]
    Nach \thref{lemma:raum in garbe einsetzen ist linksexakt} bleibt noch
    zu zeigen, dass $\M(X) \to \M''(X)$ surjektiv ist. Wir wählen 
    eine Überdeckung $X = \cup_i U_i$, von offenen $\cal U = (U_i)_i$,
    so dass auf den $U_i$ die 
    Sequenz \tikzmark{exakt} ist. \tikzmargin{south}{\color{red}
    Mir ist noch nicht ganz klar, warum das geht.}
    Nun können wir \obda annehmen, dass 
    \begin{enumerate}
      \item $U_i$ basisoffen sind, also $U_i = D(f_i)$ für geeignete 
        $f_i \in A$
      \item und $\# I < \infty$, da $X = \Spec A$ quasikompakt ist.
    \end{enumerate}
    \newcommand{\Uij}{U_{ij}}
    Sei $\sigma \in M''(X)$ beliebige. Zu jedem $i\in I$ 
    wähle $\tau_i \in \M(U_i)$, so dass
    $\beta(U_i)(\tau_i) = \sigma\rest{U_i}$. Wir führen die
    Schreibweise $U_{ij} := U_i \cap U_j$ ein und damit ist
    $\beta(U_{ij})(\tau_i\rest{U_{ij}}) = \sigma\rest{U_{ij}} = 
    \beta(U_{ij})(\tau_j\rest{U_{ij}})$, also
    \[\tau_i\rest{U_{ij}} - \tau_j\rest{U_{ij}} \in \ker\beta(U_{ij}) 
        = \im\alpha(U_{ij}),\]
     wobei wir die letzte Gleichheit aus der Linksexaktheit haben. Damit
     können wir \obda $\M'\subseteq \M$ als untergarbe ansehen, also
     $\M'(U) \subseteq \M(U)$ als Untermodul.
     Setze nun $\eta_{ij} := \tau_i\rest{\Uij} - \tau_j\rest{\Uij} \in 
     \M'(\Uij)$ für jedes Paar $(i,j)$.
     
     Diese $(\eta_{ij})_{i,j}$ sind also das "`Hindernis"', dass die $(\tau_i)_i$
     verkleben zu einem $\tau \in \M(X)$!
     Es ist
     \newcommand{\ijk}{_{ijk}}
     \renewcommand{\ij}{_{ij}}
     \newcommand{\ik}{_{ik}}
     \newcommand{\jk}{_{jk}}
     \[0 = d(\eta_{ij})_{i,j} = \left(
        \eta\ij\rest{U\ijk} - \eta\ik\rest{U\ijk} + \eta\jk\rest{U\ijk}
        \right)_{i,j,k}.\]
     Das $d$ werden wir später erklären!
     Nach Wahl der $U_i = D(f_i)$, $U\ij = D(f_if_j)$ ist
     \[\eta\ij = \frac{a\ij}{(f_if_j)^r} \in \M'(D(f_if_j))
        \tikzmark{=} M'(X)^\sim(D(f_if_j)) = \M'(X)_{f_if_j}\]
     mit $a\ij \in \M'(X)$, wobei die Gleichheit \tikzarrow{mark above}{hier}
     durch die Quasikohärenz von $\M'$ mit 
     \thref{satz:modulgarbe quasikohaerent <=> iso auf einschraenkung}
     gegeben ist. Ferner ist zu bemerken, dass $r$ nicht von $i,j$ abhängt.
     Dies können wir \obda erreichen, da $\# I < \infty$.
     Damit haben wir:
     \[0 = \frac{a\ij}{(f_if_j)^r}\rest{U\ijk} - 
        \frac{a\ik}{(f_if_k)^r}\rest{U\ijk} + 
        \frac{a\jk}{(f_jf_k)^r}\rest{U\ijk} \in \M'(X)_{f_if_jf_k}.\]
    Die Restriktionen sind aber gerade gegeben durch    
    \[ \begin{tikzcd} \M'(D(f_if_j)) \dar[empty]{\rotatebox{90}{=}}
        \rar{\_\rest{U_{ijk}}} 
        & M'(D(f_if_jf_k)) \dar[empty]{\rotatebox{90}{=}} \\
        M'(X)_{f_if_j} \rar & \M'(X)_{f_if_jf_k} \\
        \frac{a}{(f_if_j)^r} \rar[mapsto] & 
        \frac{a f_k^r}{(f_if_jf_k)^r},
    \end{tikzcd}\]
    also haben wir
    \[0 = \frac{a\ij f_k^r}{(f_if_jf_k)^r} - 
        \frac{a\ik f_j^r}{(f_if_jf_k)^r} + 
        \frac{a\jk f_i^r}{(f_if_jf_k)^r} \in \M'(X)_{f_if_jf_k}.\]
    Da aber die Lokalisierung an $f_i f_j f_k$ gerade die Lokalisierung an
    $f_k$ von der Lokalisierung an $f_if_j$ ist, existiert $l\in \N$, so dass
    \begin{equation}\label{eq:1} 0 = f_k^{l+r} \frac{a\ij}{(f_if_j)^r} - 
        f_k^l f_j^r\frac{a\ik}{(f_if_k)^r} + 
        f_K^l f_i^r\frac{a\jk}{(f_jf_k)^r} \in \M'(X)_{f_if_j}
    \end{equation}
    Da es nur endlich viele Indizes gibt, haben wir diese Gleichheit
    für alle $k\in I$ und für alle $(i,j)\in I^2$.
    
    Nun ist $D(f_k) = D(f_k^{r+l})$ und $\Spec A = \cup_{k\in I} D(f_k^{r+l})$,
    also
    \[ \bigcap_{k\in I} V((f_k^{r+l})) = V\left(\sum_{k\in I} f_k^{r+l}\right)
    = \emptyset = V(A) \quad\Leftrightarrow\quad 
    1 \in \sum_{k\in I} (f_k^{r+l}).\]
    Damit ist $1 = \sum_{k\in I} h_k f_k^{r+l}$ für geeignete $h_k \in A$.
    Setzen wir nun 
    \[g_i := \sum_{k\in I} h_k f_k^l \frac{a\ik}{f_i^r} \in 
        M'(X)_{f_i} = (M'(X))^\sim(D(f_i)) = \M'(U_i),\]
    wobei sich letzte Gleichheit wieder aus der Quasikohärenz ergibt,
    so haben wir
    \[g_i\rest{U\ij} - g_j\rest{U\ij} = \sum_{k\in I} h_k f_k^l
        \left(f_j^r\frac{a\ik}{(f_if_j)^r} - 
        f_i^r\frac{a\jk}{(f_if_j)^r}\right)
        \tikzmark[1]{=} \underbrace{\sum_{k\in I} h_k f_k^{r+l}}{=1} 
        \frac{a\ij}{(f_if_j)^r} = \eta\ij \in \M'(U\ij),\]
    wobei wir \tikzarrow{mark above}{diesen} Schritt durch Umformung von 
    \autoref{eq:1} erhalten haben. Definieren wir nun
    $\mu_i := \tau_i - g_i \in \M(U_i)$, so haben wir für alle $i,j \in I$
    \[ \mu_i \rest{U\ij} - \mu_j \rest{U\ij} = \eta\ij - \eta\ij = 0\]
    Also existiert ein eindeutiger globaler Schnitt $\mu \in \M(X)$ mit
    $\mu\rest{U\ij} = \mu_i$ für alle $i\in I$.
    
    Benutzen wir nun alles bisherige, so erhalten wir für alle $i\in I$
    \[ \beta(X)(\mu)\rest{U_i} = \beta(U_i)(\mu\rest{U_j}) = 
        \beta(U_i)(\mu_i) = \beta(U_i)(\tau_i) = \sigma\rest{U_i}.\]
    Damit stimmen $\beta(X)(\mu)$ und $\sigma \in \M''(X)$ auf 
    $U_i$ überein. Daher sind sie gleich und wir haben die Surjektivität 
    von $\beta$ gezeigt.
\end{proof}

\subsection{Der \Cech-Komplex}
Wir gehen hier genauer auf die Verwendung des $d$ in vorherigem Beweis ein.
Der Beweis liefert nämlich gerade, dass $\Hv^1(\cal U, \M'') = 0$,
wie wir mit nachstehender Definition sehen.

\begin{definition}[\Cech-Komplex, \Cech-Kohomologie]
    \index[def]{\Cech-Kohomologie!\Cech-Komplex}
    \index[def]{\Cech-Kohomologie}
     \newcommand{\ijk}{_{ijk}}
     \renewcommand{\ij}{_{ij}}
     \newcommand{\ik}{_{ik}}
     \newcommand{\jk}{_{jk}}
    Sie $X$ ein topologischer Raum. $\cal U = (U_i)_{i\in I}$ eine offene
    Überdeckung, $\F \in \Sh_X$. Betrachte den folgenden Kettenkomplex
    \[\everymath{\displaystyle}\begin{tikzcd}
        \Cv^0 \dar[empty]{\rotatebox{90}{=}}\rar &
             \Cv^1 \dar[empty]{\rotatebox{90}{=}}\rar &
             \Cv^2 \dar[empty]{\rotatebox{90}{=}}\rar & \ldots\\
        \prod_{i\in I} \F(U_i) \rar{d} & 
            \prod_{(i,j)\in I^2} \F(U_{ij}) \rar{d} &
            \prod_{(i,j,k)\in I^3} \F(U_{ijk}) \rar{d} & \ldots\\
        (\eta_i)_i \rar[mapsto] & 
            (\eta_i \rest{U_{ij}} - \eta_j \rest{U_{ij}})_{i,j} \\
        &(y_{ij})_{i,j} \rar[mapsto] & 
            (y\ij \rest{U\ijk} - y\ik  \rest{U\ijk} + y\jk\rest{U\ijk})_{i,j,k},
    \end{tikzcd}\]
    so heißt
    \[\Hv^k(\cal U,\F) := \H^k(\text{\Cech-Komplex}) := 
        \ker(d: \Cv^k \to \Cv^{k+1}) \big/ \im(d: \Cv^{k-1} \to \Cv^k)
    \]
    die $k$-te \emph{\Cech-Kohomologie von $\F$ bzgl. $\cal U$}.
\end{definition}

\begin{bemerkung}
    \newcommand{\ijk}{_{ijk}}
    \renewcommand{\ij}{_{ij}}
    \newcommand{\ik}{_{ik}}
    \newcommand{\jk}{_{jk}}
    Da $\F$ eine Garbe ist, haben wir $\Hv^0(\cal U,\F) = \F(X)$!
    Ferner gilt $d \circ d = 0$ und
    für $[(y_{ij})_{i,j}] \in \Hv^1$ haben wir
    $d(y_{ij})_{i,j} = 0$, d.h.
    \[(y\ij \rest{U\ijk} - y\ik  \rest{U\ijk} + y\jk\rest{U\ijk})_{i,j,k}\]
    Diese Bedingung nennen wir \emph{Ko-Zykel-Bedingnung}.
    Ferner ist $[(y\ij)_{i,j}] \in \Hv^1$ per definitionem, falls
    ein $(\eta_i)_i \in \Cv^0$ existiert, so dass
    $(y\ij)_{i,j} = d(\eta_i)_i$, also
    \[y\ij = \eta_i \rest{U\ij} - \eta_j \rest{U\ij}.\]
    Daher nennen wir in dieser Situation $(y\ij)_{i,j}$ einen \emph{Ko-Rand}.
\end{bemerkung}



\subsection{Kohärenz}
\begin{definition}[endlich erzeugt, kohärent]
    \index[def]{O-Modul@\OX-Modul!endlich erzeugt}
    \index[def]{O-Modul@\OX-Modul!kohärent}
    
    \begin{itemize}
      \item Eine \OX Modulgarbe $\M$ heißt \emph{endlich erzeugt}, 
        falls es zu jedem $x\in X$ ein offenes $x\in U\osubset X$ gibt 
        und eine exakte Sequenz
        \[\O_X\rest U^n \to \M\rest U \to 0\]
        für ein $n\in \N$ gibt.
      \item $\M$ heißt \emph{kohärent}, falls $\M$ endlich erzeugt ist und
        wenn für jedes $\alpha$ in
        \[\O_X\rest U^n \xto\alpha \M\rest U \to 0\]
        der $\ker \alpha$ als $\O_U$-Modulgarbe endlich erzeugt ist.
    \end{itemize}
\end{definition}

\begin{bemerkung}
    Sei $A$ ein Ring, so ist ein endlich erzeugter $A$-Modul $M$ nicht anderes, 
    als dass analog zu oben eine exakte Sequenz
    $ A^n \xto\alpha M \to 0$ für ein $n\in \N$ gibt. $M$ ist
    kohärent (oder endlich präsentierter), falls $M$ endlich erzeugt ist
    und $\ker \alpha$ endlich erzeugt ist. Letzteres ist bei immer der Fall,
    falls $A$ noethersch ist.
    
    Der Unterschied zu Ringmoduln wird in nachstehendem Satz deutlich,
    wo wir die Quasikohärenz fordern müssen, um garantieren zu können,
    dass $\M(U)$ überhaupt erzeugbar ist.  
\end{bemerkung}

\begin{satz}
    Sei $(X,\O_X)$ ein lokal noethersches Schema und $\F$ eine
    quasikohärente $\O_X$-Modulgarbe. Dann ist äquivalent:
    \begin{enumerate}[label=(\roman*)]
      \item $\F$ ist kohärent.
      \item $\F$ ist endlich erzeugt.
      \item $\forall U \osubset X$ affin und offen ist
        $\F(U)$ ein endlich erzeugter $\O_X(U)$-Modul. 
    \end{enumerate}
\end{satz}
\begin{proof}
    \begin{description mathquote}
    \item[\text{(ii)}\Rightarrow\text{(iii)}]
        Da $\F$ endlich erzeugt ist, existiert eine offene Überdeckung
        $U = \cup_{i\in I} U_i$ für \obda $U_i = D(f_i)$ mit 
        $f_i \in \O_U(U)$, so dass
        \[\begin{tikzcd}[row sep=small]
            0 \rar & \ker\alpha \rar & 
            \O_U\rest{U_i}^n \rar{\alpha} & \F\rest{U_i} \rar & 0
        \end{tikzcd}\]
        eine kurze exakte Sequenz ist.
        Aus nachstehendem Hilfslemma wissen wir, dass $\ker\alpha$ ebenfalls
        quasikohärent ist und diese bleibt mit 
        \thref{satz:kurze exakte sequenz mit quasikohaerent bleibt exakt}
        beim Einsetzen von $U_i$ exakt, also
        \[\begin{tikzcd}[row sep=small] 
            \O_X(U_i)^{n_i} \rar{\alpha(U_i)} & \F(U_i) \rar & 0.
        \end{tikzcd}\]
        Damit ist $\F(U_i)$ ein endlich erzeugter $\O_X(U_i)$-Modul.
        
        Ferner gilt
        \[\F(U_i) = \F(U) \otimes_{\O_X(U)} \O_X(U)_{f_i} = 
            \F(U) \otimes_{\O_X(U)} \O_X(U_i)\]
        und andererseits aufgrund der Quasikohärenz
        \[\F(U_i) = \F(U)^\sim(U_i) = \F(U)^\sim(D(f_i)) = 
            \F(U)_{f_i}.\]
        Also existiert ein endlich erzeugter $\O_X(U)$-Untermodul
        $M_i \subseteq \F(U)$ mit
        \[\F(U_i) = M_i \otimes_{\O_X(U)} \O_X(U_i),\]
        denn: Seien 
        $\alpha_1, \ldots, \alpha_r \in \F(U)\otimes_{\O_X(U)}\O_X(U_i)$
        ein Erzeugendensystem über $\O_X(U_i)$ mit
        \[\alpha_k = \sum_{k \text{ endlich}} m_{kj} \otimes\lambda_j\]
        mit $m_{kj} \in \F(U)$ und $\lambda_j \in \O_X(U_i)$.
        Damit erzeugen $\{m_{kj}\}_{k,j}$ ein solches $M_i$.
        
        Da die anfangs gewählte Überdeckung \obda endlich ist 
        ($U$ affin, also quasikompakt), existiert en endlich erzeugtes
        $\O_X(U)$-Modul $M$ mit
        \[ M\otimes_{\O_X(U)} \O_X(U_i) \to \F(U_i) \to 0\]
        exakt als Sequenz von $\O_X(U_i)$-Moduln. Betrachte nun 
        Dies ist aber gerade
        $M^\sim(U_i) \to \F(U)^\sim(U_i) \to 0$,
        was die Exaktheit von
        $M^\sim \to \F(U)^\sim \to 0$ als Sequenz von $\O_U$-Modulgarben
        zur Folge hat. Wir setzen wieder $U$ ein und erhalten mit
        \thref{satz:kurze exakte sequenz mit quasikohaerent bleibt exakt}
        $M(U) \to \F(U) \to 0$ exakt.
        Damit ist $\F(U)$ endlich erzeugt.
      \item[\text{(iii)}\Rightarrow\text{(i)}]
        Für jedes offene affine $U$ ist $\F(U)$ endlich erzeugt,
        es existiert also eine exakte Sequenz der Form 
        $\O_X(U)^n \to \F(U) \to 0$. Da $\F$ quasikohärent, ist 
        $\F\rest U = \F(U)^\sim$, wobei  $^\sim$ auf $U = \Spec A$ zu lesen
        ist. Ergo ist auch $\F\rest U$ endlich erzeugt als $\O_U$-Modul.
        Zu zeigen bleibt: Ist $\O_X\rest U ^n \xto\alpha \F\rest U \to 0$ 
        exakt, so ist $\ker \alpha$ endlich erzeugt. Dazu sei
        \obda $U = \Spec A$ affin und $A$ noethersch. Dann
        ist
        \[0\to \ker\alpha\to \O_X\rest U^n \to \F\rest U \to 0\]
        exakt. Mit der Linksexaktheit von $\Gamma(U,\_)$ ist
        \[0\to \ker\alpha(U) \to \O_X(U)^n \to \F(U)\]
        und wieder mit dex Exaktheit von $^\sim$ erhalten wir
        \[0 \to (\ker\alpha(U))^\sim \to \O_X\rest U^n \to \F\rest U.\]
        Also ist $\ker\alpha = (\ker\alpha(U))^\sim$. Weiter
        ist $\ker(\alpha(U)) \subseteq \O_X(U)^n$ ein endlich erzeugter
        $\O_X(U)$-Modul, da nach Voraussetzung $\O_X(U)$ noethersch ist.
        Folglich ist $(\ker\alpha(U))^\sim$ eine endlich erzeugte
        $\O_X\rest U$-Modulgarbe.
    \end{description mathquote}
\end{proof}

\begin{hilfslemma}
    Ist 
    \[0 \to \cal K \to \cal N \xto\alpha \M \to 0\]
    eine kurze exakte Sequenz $\O_X$-Moduln und sind $\M$ und $\cal N$ 
    quasikohärent, so ist $\cal K$ quasikohärent.
\end{hilfslemma}
\begin{proof}

\end{proof}

\pagebreak


\pagebreak
\nocite{*}
\printbibliography

\printindex[def]
\end{document}
% vim: set ft=tex :
