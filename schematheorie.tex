%% Basierend auf einer TeXnicCenter-Vorlage von Mark Müller
%%%%%%%%%%%%%%%%%%%%%%%%%%%%%%%%%%%%%%%%%%%%%%%%%%%%%%%%%%%%%%%%%%%%%%%
% Wählen Sie die Optionen aus, indem Sie % vor der Option entfernen  
% Dokumentation des KOMA-Script-Packets: scrguide

%%%%%%%%%%%%%%%%%%%%%%%%%%%%%%%%%%%%%%%%%%%%%%%%%%%%%%%%%%%%%%%%%%%%%%%
%% Optionen zum Layout des Artikels                                  %%
%%%%%%%%%%%%%%%%%%%%%%%%%%%%%%%%%%%%%%%%%%%%%%%%%%%%%%%%%%%%%%%%%%%%%%%
\documentclass[%
%a5paper,							% alle weiteren Papierformat einstellbar
%landscape,						% Querformat
10pt,								% Schriftgröße (12pt, 11pt (Standard))
%BCOR1cm,							% Bindekorrektur, bspw. 1 cm
%DIVcalc,							% führt die Satzspiegelberechnung neu aus
%											  s. scrguide 2.4
%twoside,							% Doppelseiten
%twocolumn,						% zweispaltiger Satz
halfparskip*,				% Absatzformatierung s. scrguide 3.1
%headsepline,					% Trennline zum Seitenkopf	
%footsepline,					% Trennline zum Seitenfuß
%titlepage,						% Titelei auf eigener Seite
%normalheadings,			% Überschriften etwas kleiner (smallheadings)
%idxtotoc,						% Index im Inhaltsverzeichnis
%liststotoc,					% Abb.- und Tab.verzeichnis im Inhalt
%bibtotoc,						% Literaturverzeichnis im Inhalt
%abstracton,					% Überschrift über der Zusammenfassung an	
%leqno,   						% Nummerierung von Gleichungen links
%fleqn,								% Ausgabe von Gleichungen linksbündig
%draft								% Überlangen Zeilen in Ausgabe gekennzeichnet
DIV = 14,
%monochrome						% schwarz weiß output
]
{scrartcl}

\xdef\myDevelopVariable{1}

%\pagestyle{empty}		% keine Kopf und Fußzeile (k. Seitenzahl)
%\pagestyle{headings}	% lebender Kolumnentitel  


%% Deutsche Anpassungen %%%%%%%%%%%%%%%%%%%%%%%%%%%%%%%%%%%%%
\usepackage{amsmath,amssymb,mathabx}
\usepackage[ngerman]{babel}
\usepackage[T1]{fontenc}
%\usepackage[ansinew]{inputenc}
\usepackage[utf8]{inputenc}

\usepackage{lmodern} %Type1-Schriftart für nicht-englische Texte

\usepackage{hyperref}

\usepackage{xspace}


\usepackage{xcolor}
\usepackage{tikz}
\usetikzlibrary{matrix,arrows,fadings,decorations.markings}
\usepackage{tikz-cd2}

%% define colors %%%%%%%%%%%%%%%%%%%%%%%%%%%%%%%%%%%%%%%%%%%%
\colorlet{mycolor}{blue!80!black}
\colorlet{col1}{mycolor}
\colorlet{col1shade1}{mycolor!5}
\colorlet{col1shade2}{mycolor!50}
\colorlet{col2}{purple!80}
\colorlet{col2shade1}{col2!5}
\colorlet{col2shade2}{col2!50}

%% tikz setup %%%%%%%%%%%%%%%%%%%%%%%%%%%%%%%%%%%%%%%%%%%%%%%
\usetikzlibrary{decorations.shapes, shapes.geometric}
\tikzset{generic point/.style=
	{star, star points=20, minimum size=#1, inner sep=0pt, outer sep=0pt}
}


%% Packages tikztitle and tikztheorem %%%%%%%%%%%%%%%%%%%%%%%
\usepackage[color=mycolor, style=elegant, withheadings=true,
  backgroundcmd=titlepagebackground]{tikztitle}

\ifnum\myDevelopVariable=1 
  \usepackage{amsthm}
  \usepackage{thmbox}
  \theoremstyle{plain}% default
  \newtheorem{thm}{Satz}[section]
  \newtheorem{satz}[thm]{Satz}
  \newtheorem{lemma}[thm]{Lemma}
  \newtheorem{hilfslemma}[thm]{Hilfslemma}
  \newtheorem{lem}[thm]{Lemma}
  \newtheorem{kor}[thm]{Korollar}
  \newtheorem{cor}[thm]{Korollar}
  \newtheorem{korollar}[thm]{Korollar}
  %\newtheorem{prop}[thm]{Proposition}

  \theoremstyle{definition}
  \newtheorem{defn}[thm]{Definition}
  \newtheorem{definition}[thm]{Definition}
  \newtheorem{bsp}[thm]{Beispiel}
  \newtheorem{beispiel}[thm]{Beispiel}
  \newtheorem{exmp}[thm]{Beispiel}
  %\newtheorem{conj}[thm]{Conjecture}

  \theoremstyle{remark}
  \newtheorem{bem}[thm]{Bemerkung}
  \newtheorem{bemerkung}[thm]{Bemerkung}
  \newtheorem{rem}[thm]{Bemerkung}
  \newtheorem{uebung}[thm]{Übung}
  %\newtheorem{note}[thm]{Notiz}
  %\newtheorem{case}{Fall}
  \def\thref{\ref}
\else
  \usepackage{tikztheorems}
  \newtikztheorem[
    style=elegantbreak,
    color=mycolor,
    font header=\normalfont\sffamily\bfseries,
    counter zero=section
    ]{satz}{Satz}
    
  \newtikztheorem[
    style=elegantbreak,
    color=mycolor,
    font header=\normalfont\sffamily\bfseries,
    font body=\normalfont,
    counter parent=satz
    ]{definition}{Definition}
   
  \newtikztheorem[
    style=elegantinline,
    color=mycolor,
    font header=\normalfont\sffamily\bfseries,
    counter parent=satz
    ]{lemma}{Lemma}
    
  \newtikztheorem[
    style=elegantinline,
    color=mycolor,
    font header=\normalfont\sffamily\bfseries,
    counter parent=satz
    ]{korollar}{Korollar}
    
  \newtikztheorem[
    style=plain,
    color=mycolor,
    font header=\normalfont\sffamily\bfseries,
    font body=\normalfont,
    counter parent=satz
    ]{beispiel}{Beispiel}
    
  \newtikztheorem[
    style=plain,
    color=mycolor,
    font header=\normalfont\sffamily\bfseries,
    font body=\normalfont,
    counter parent=satz
    ]{bemerkung}{Bemerkung}
    
    
  \newtikztheorem[
    style=plain,
    color=mycolor,
    font header=\normalfont\sffamily\bfseries,
    font body=\normalfont,
    counter parent=satz
    ]{hilfslemma}{Hilfslemma}
    
  \newtikztheorem[
    style=elegantbreak,
    color=col2,
    font header=\normalfont\sffamily\bfseries,
    font body=\normalfont,
    nocounter=true
    ]{uebung}{Übung}
\fi
%%%%%%%%%%%%%%%%%%%%%%%%%%%%%%%%%%%%%%%%%%%%%%%%%%%%%%%%%%%%%

\usepackage{tikzmargin}

\usepackage{array, enumitem}

%Buchstaben durchstreichen
\usepackage{cancel}

%% Packages für Grafiken & Abbildungen %%%%%%%%%%%%%%%%%%%%%%
\usepackage{graphicx} %%Zum Laden von Grafiken
%\usepackage{subfig} %%Teilabbildungen in einer Abbildung
% \usepackage[ngerman, nameinlink]{cleveref}

%% Math abbreviations %%%%%%%%%%%%%%%%%%%%%%%%%%%%%%%%%%%%%%%
\let\bb\mathbb
\let\cal\mathcal
\let\fr\mathfrak
\newcommand{\F}{\cal{F}}
\newcommand{\G}{\cal{G}}
\renewcommand{\O}{\cal{O}}
\newcommand{\R}{\bb{R}}
\newcommand{\N}{\bb{N}}
\newcommand{\C}{\bb{C}}
\newcommand{\Q}{\bb{Q}}
\newcommand{\Z}{\bb{Z}}
\newcommand{\A}{\bb{A}}
\renewcommand{\P}{\bb{P}}
\newcommand{\rest}[1]{\big|_{#1}}
\newcommand{\inv}{^{-1}}
\newcommand{\fis}{^{\#}}
\newcommand{\kat}[1]{\mathbf{#1}}
\newcommand{\PSh}{\kat{PSh}}
\newcommand{\Sh}{\kat{Sh}}
\newcommand{\Ring}{\kat{Ring}}
\newcommand{\Top}{\kat{Top}}
\newcommand{\Sch}{\kat{Sch}}
\newcommand{\affSch}{\kat{Sch}^\kat{aff}}
\newcommand{\op}{^\mathrm{op}}
\newcommand{\m}{\fr{m}}
\newcommand{\p}{\fr{p}}
\newcommand{\q}{\fr{q}}
\renewcommand{\a}{\fr{a}}
\renewcommand{\b}{\fr{b}}
\newcommand{\ideal}{\vartriangleleft}
\DeclareMathOperator{\Obj}{Obj}
\DeclareMathOperator{\Morph}{Morph}
\DeclareMathOperator{\Hom}{Hom}
\DeclareMathOperator{\im}{im}
\DeclareMathOperator{\Spec}{Spec}
\DeclareMathOperator{\Nil}{Nil}
\DeclareMathOperator{\Quot}{Quot}
\DeclareMathOperator{\charak}{char}
\newcommand{\funcdef}[1]{%
	\begin{array}[t]{>{\displaystyle}r>{\displaystyle}c>{\displaystyle}l}
	#1\end{array}}
\let\xto\xrightarrow

\newcommand{\obda}{{\small oBdA}\xspace}
\newcommand{\Obda}{{\small OBdA}\xspace}

%für doppelklammer
\newcommand{\dk}[1]{\ensuremath\ldbrack #1\rdbrack}
% für doppelrund
\newcommand{\dr}[1]{\ensuremath(\!( #1 )\!)}

\newenvironment{description sf}{%
	\begin{description}[font=\normalfont\sffamily]}
	{\end{description}}
\newenvironment{description mathquote}{%
	\renewcommand{\descriptionlabel}[1]{\glqq$##1$\grqq}
	\begin{description}[font=\normalfont]}
	{\end{description}}
%%%%%%%%%%%%%%%%%%%%%%%%%%%%%%%%%%%%%%%%%%%%%%%%%%%%%%%%%%%%%


%% Bibliographiestil %%%%%%%%%%%%%%%%%%%%%%%%%%%%%%%%%%%%%%%%%%%%%%%%%%
%\usepackage{natbib}

\begin{document}

\ifnum\myDevelopVariable=0 
  %% Titlepage Background cmd (lrbox must be after begin{document}) %%%%%
  \tikzfading[name=fade down,
    top color=transparent!90,
    bottom color=transparent!0]
  \def\titlepagebackground{
      \begin{scope}[transform shape, rotate=40, scale=5, opacity=0.5,
        ]
      \node[scope fading=fade down] at (current page.center)
      {\usebox{\diagbox}};
      \end{scope}
  }
  \newsavebox{\diagbox}
  \begin{lrbox}{\diagbox}
  \begin{minipage}{\textwidth}
  \[
    \everymath{\displaystyle}
    \begin{tikzcd}[row sep=large, column sep=large]
      s \dar[mapsto] \symb{\in}
        &[-1cm] \O_Y(U) \rar{f_U\fis} \dar[swap]{\rest W} 
        & \O_X(f\inv(U)) \dar{\rest{f\inv(W)}} \symb{\ni}
        & \dar[mapsto] f_U\fis(s)\\
      s\rest W = t\rest W \symb{\in}
        & \O_Y(W) \rar{f_W\fis}				
        & \O_X(f\inv(W)) \symb{\ni}
        & f_U\fis(s)\rest{f\inv(W)} = f_V\fis(t)\rest{f\inv(W)}\\
      t \uar[mapsto] \symb{\in}
        & \O_Y(V) \rar{f_W\fis} \uar{\rest W} 
        & \O_X(f\inv(V)) \uar[swap]{\rest{f\inv(W)}} \symb{\ni}
        & \uar[mapsto] f_V\fis(t)\\
    \end{tikzcd}
  \]
  \end{minipage}
  \end{lrbox}
  %%%%%%%%%%%%%%%%%%%%%%%%%%%%%%%%%%%%%%%%%%%%%%%%%%%%%%%%%%%%%%%%%%%%%%%
\fi

\pagestyle{empty}
%%%%%%%%%%%%%%%%%%%%%%%%%%%%%%%%%%%%%%%%%%%%%%%%%%%%%%%%%%%%%%%%%%%%%%%
%% Ihr Artikel                                                       %%
%%%%%%%%%%%%%%%%%%%%%%%%%%%%%%%%%%%%%%%%%%%%%%%%%%%%%%%%%%%%%%%%%%%%%%%

%% eigene Titelseitengestaltung %%%%%%%%%%%%%%%%%%%%%%%%%%%%%%%%%%%%%%%    
%\begin{titlepage}
%Einsetzen der TXC Vorlage "Deckblatt" möglich
%\end{titlepage}

%% Angaben zur Standardformatierung des Titels %%%%%%%%%%%%%%%%%%%%%%%%
%\titlehead{Titelkopf }
\subject{Vorlesungszusammenfassung}
\title{Schematheorie}
\author[erstellt von]{Stefan Hackenberg \and Maximilian Huber}
%\thanks{Fuünote}			% entspr. \footnote im Flieütext
\date{\today}				% falls anderes, als das aktuelle gewünscht
\publishers[gelesen im WS 2012/2013 und SS 2013 von]{Prof. Dr. Marco Hien}

%% Widmungsseite %%%%%%%%%%%%%%%%%%%%%%%%%%%%%%%%%%%%%%%%%%%%%%%%%%%%%%
%\dedication{Widmung}

\maketitle 						% Titelei wird erzeugt


%% Zusammenfassung nach Titel, vor Inhaltsverzeichnis %%%%%%%%%%%%%%%%%
%\begin{abstract}
% Für eine kurze Zusammenfassung des folgenden Artikels.
% Für die überschrift s. \documentclass[abstracton].
%\end{abstract}

\KOMAoptions{twoside}
\cleardoublepage

%% Erzeugung von Verzeichnissen %%%%%%%%%%%%%%%%%%%%%%%%%%%%%%%%%%%%%%%
\thispagestyle{plain}
\tableofcontents			% Inhaltsverzeichnis
%\listoftables				% Tabellenverzeichnis
%\listoffigures				% Abbildungsverzeichnis


%% Der Text %%%%%%%%%%%%%%%%%%%%%%%%%%%%%%%%%%%%%%%%%%%%%%%%%%%%%%%%%%%

\pagestyle{scrheadings}
 
\section{Lokal geringte R�ume}

\subsection{Garben}

\begin{definition}[Pr�garbe]
	Sei $X$ ein topologischer Raum. Eine \emph{Pr�garbe} $\F$ auf $X$
	ist eine Zuordnung
	$$\F: U\mapsto \F(U) \,,$$
	die jedem offenen $U\subset X$ eine abelsche Gruppe
	$\F(U)$ zuordnet, zusammen mit Homomorphismen
	$$\rho_{UV}: \F(U) \to \F(V)$$
	f�r jedes Paar $V\subset U$, so dass
	\[
	\begin{tikzcd}
		\F(U) \arrow{r}{\rho_{UV}}
			\arrow[bend right]{rr}{\rho_{UW}}& \F(V) \arrow{r}{\rho_{VW}}& \F(W)
	\end{tikzcd}
	\]
	kommutiert.
	
	Wir nennen $\rho_{UV}$ \emph{Restriktion}, schreiben
	meist $\tikzmark{s\rest V} := \rho_{UV}(s)$.
	
	Man nennt $s\in \F(U)$ auch \emph{Schnitt �ber $U$}.
\end{definition}
	
\tikzmargin{north, above=1cm}{\color{red}
Bei mir steht hier im Skript $s\rest U$. Offenbar ein Fehler!?}

\begin{beispiel}
	$$\cal C_X^\circ: U \mapsto \cal C_X^\circ (U) := 
		\{f: U\to \R \mid \text{ $f$ stetig}\} $$
	mit $\rho_{VU}: \cal C_X^\circ(V) \mapsto \cal C_X^\circ(U)$,
	$f \mapsto f\rest U$.
\end{beispiel}

\begin{bemerkung}
	Ist $\kat{Ab}$ die Kategorie der abelschen Gruppen und
	\[
		\kat{Top}_X := 
		\begin{cases}
		\Obj: U\subset X \text{ offen}\\
		\Morph: \Hom(U,V) = 
			\begin{cases}
				\emptyset & U\not\subset V,\\
				U\to V & U\subset V,
			\end{cases}
		\end{cases}
	\]
	dann ist eine Pr�garbe gerade ein kontravarianter Funktor
	\[
		\F: \funcdef{\kat{Top}_X & \to & \kat{Ab}\\
			U & \mapsto & \F(U)\\
			(U\to V) & \mapsto & (\F(V)\to \F(U)).}
	\]
	Oder anders ausgedr�ckt: Es ist
	\[
		\F: \funcdef{\kat{Top}_X\op & \to & \kat{Ab}\\
			U & \mapsto & \F(U)\\
			(V\to U) & \mapsto & (\F(V)\to \F(U)).}
	\]
	ein kovarianter Funktor.
\end{bemerkung}

\begin{definition}[Morphismus von Pr�garben]
	Ein \emph{Morphismus von Pr�garben} $\F \xto{\phi} \G$ auf $X$ ist
	eine nat�rliche Transformation der Funktoren $\F$ und $\G$, d.h.
	f�r alle $U\subset X$ offen gibt es einen Morphismus
	$\F(U) \xto{\phi_U} \G(U)$, so dass f�r $U\subset V$
	\[
		\begin{tikzcd}
			\F(U) \arrow{r}{\phi_U} & \G(U)\\
			\F(V) \arrow{r}{\phi_U} \arrow{u} & \G(V) \arrow{u}
		\end{tikzcd}
	\] 
	kommutiert.
\end{definition}


\begin{definition}[Garbe]
	Eine Pr�garbe $\F$ auf $X$ hei�t \emph{Garbe}, falls gilt:
	Ist $U\subset X$ offen und $U=\bigcup_{i\in I} U_i$ f�r 
	offene $U_i\subset X$, so gilt
	\begin{enumerate}
	  \item Ist $s\in \F(U)$ und $s\rest{U_i} = 0$ f�r alle $i\in I$,
	  	so ist $s=0\in \F(U)$.
	  \item Sind $s_i \in \F(U_i)$ gegeben, mit
	  	$$s_i \rest{U_i\cap U_j} = s_j \rest{U_i\cap U_j}\qquad \forall i,j,$$
	  	so existiert ein $s\in \F(U)$ mit
	  	$$s_i = s \rest{U_i}\qquad\forall i.$$
	\end{enumerate}
\end{definition} 

\begin{bemerkung}
	$\F$ ist eine Garbe, genau dann, wenn die folgende Sequenz abelscher
	Gruppen exakt ist:
	\[	\everymath{\displaystyle}
		\begin{tikzcd}[row sep=tiny, column sep=small]
		0 \rar & \tikzmark[1]{\F(U)} \rar & 
			\tikzmark[2]{\prod_{i\in I} \F(U_i)} \rar
			& \prod_{(i,j)\in I^2} \F(U_i\cap U_j)\\
		& s \rar[mapsto] & \left(s\rest{U_i}\right)_{i\in I}\\
		&& (s_i)_{i\in I} \rar[mapsto] &
			\left(s_i\rest{U_i\cap U_j} - 
			s_j\rest{U_i\cap U_j}\right)_{(i,j)\in I^2}  
		\end{tikzcd}
	\]
	
	Exaktheit an \tikzarrow[1]{south, mark above}{dieser} Stelle ist �quivalent 
	zu Eigenschaft 1.
	Exaktheit \tikzarrow[2]{south, mark above}{hier} zu Eigenschaft 2.
\end{bemerkung}

\begin{beispiel}
	Sei $M$ eine $\mathrm C^\infty$ Mannigfaltigkeit, so ist
  	\[ \cal C^\infty_M: U \mapsto
  		\cal C^\infty_M(U) := \{f:U\to \R \mid f\in \mathrm C^\infty(U)\}
  	\]
  	eine Garbe.
\end{beispiel}

\begin{beispiel}
	Sei $M$ eine $\C$ Mannigfaltigkeit, so ist
  	\[ \cal O_M: U \mapsto
  		\cal O_M(U) := \{f:U\to \C \mid f \text{ holomorph}\}
  	\]
  	eine Garbe. F�r $M = \C$ haben wir zus�tzlich die Garbe
  	\[ \cal O_\C^\times: U \mapsto
  		\cal O_\C^\times(U) := \{f:U\to \C^\times \mid f \text{ holomorph}\},
  	\]
  	(wobei die Gruppenverkn�pfung multiplikativ zu lesen ist).
  	Dies liefert uns einen Morphismus von (Pr�)garben
  	\[ \O \to \O_C^\times,\ f \mapsto \exp(f).\]
  	Betrachte nun die Pr�garbe
  	\[\cal H := \tikzmark{\im^\text{naiv}}(\exp): U \mapsto \im(\exp_U) = 
  		\{\exp \circ f: U\to \C \mid f:U\to \C \text{ holomorph}\}.\]
  	Dies ist \emph{keine} Garbe:
  	Betrachte die Scheibe 
  	\[U = \{z\in \C \mid \tfrac{1}{2} < |z| < \tfrac{3}{2}\}\]
  	zerlegt in die beiden offenen Teilmengen
  	\begin{align*}
  		U_1 &= \{z \in U \mid \Re z > -\varepsilon\}\\
  		U_2 &= \{z \in U \mid \Re z < \varepsilon\}
  	\end{align*}
  	mit $U = U_1 \cup U_2$ f�r ein $\varepsilon > 0$ beliebig. F�r $i=1,2$
  	ist 
  	$(z: U_i \to \C, z\mapsto z) \in \cal H(U_i)$,
  	da sich der komplexe Logarithmus auf beiden $U_i$ problemlos definieren
  	l�sst.
  	Ferner ist auch
 	\[ (z: U_1 \to \C) \rest{U_1\cap U_2} = 
 		(z: U_2 \to \C) \rest{U_1 \cap U_2},\]
 	erf�llt, jedoch kommen diese nicht von einem gemeinsamen Schnitt
 	da
 	\[ (z: U\to \C) \notin \cal H(U). \]
\end{beispiel}

\tikzmargin{north}{Warum steht hier naiv??}


\begin{definition}
	F�r einen topologischen Raum $X$ bezeichne
	\begin{align*}
		\PSh_X & := \text{die Kategorie der Pr�garben auf $X$},\\
		\Sh_X & := \text{die Kategorie der Garben auf $X$, wobei
			} \Hom_{\Sh_X}(\F,\G) := \Hom_{\PSh_X}(\F,\G)
	\end{align*} 
\end{definition}

\begin{bemerkung}
	Man hat den Inklusionsfunktor
	\[ \iota: \Sh_X \to \PSh_X,\ \F \mapsto \F\]
\end{bemerkung}

\begin{definition}[Halm]
	Ist $\F$ eine (Pr�)Garbe auf $X$ und $x_0 \in X$, so hei�t
	\[ \F_{x_0} := \varinjlim_{x_0 \in U \subset X\text{ offen}} \F(U)
		 = \coprod_{U\subset X\text{ offen}} \F(U) \Big/ \sim\] 
	mit 
	\[ s \sim t \  :\Leftrightarrow \  
		\exists W \subset X \text{ offen}:\ x_0 \in W \subset U \cap U'
		\text{ und } s\rest W = t \rest W
	\]
	f�r $s \in \F(U)$, $t \in \F(U')$ der \emph{Halm von $\F$ bei $x_0$}.
	
	Die Elemente $[s] \in \F_{x_0}$ hei�en \emph{Keime von Schnitten bei $x_0$}.  
\end{definition}

\begin{beispiel}
	$(\cal C^\infty_M)_{x_0} = \{ [f: U \xto{C^\infty} \R]\mid
  	f\sim g \Leftrightarrow \exists W\subset M\text{ offen}, x_0 \in W
  	\text{ mit } f\rest W = g\rest W\}$
\end{beispiel}
\begin{beispiel}
	\begin{align*}
	  	O_{\C,x_0} &= \{[f:U \xto{\text{hol}} \C] \mid x_0 \in U\}\\
	  	&= \{\sum_{n=0}^\infty a_n(x-x_0)^n \mid \text{Reihe hat positiven 
	  	Konvergenzradius}\}\\
	  	&:= \C\{x-x_0\}
	\end{align*}
\end{beispiel}

\begin{definition}[push-forward]
	Ist $f:X \to Y$ stetig und $\F$ eine Garbe auf $X$, so ist durch
	\[ f_\ast \F: V \mapsto \F(f^{-1}(V))\]
	f�r $V\subset Y \text{ offen}$ eine Garbe definiert, 
	der \emph{push-forward von $\F$}.
\end{definition}



\subsection{Lokal geringte R�ume}
Betrachte nun 
\[\Ring := \text{ Kategorie der kommuativen Ringe mit $1$}\]
und entsprechend Garben
\[\F:\Top_X\op \to \Ring.\]

\begin{definition}[lokaler Ring]
	Sei  $R$ ein Ring. Dann hei�t $R$ \emph{lokal}, wenn $R$ genau ein
	maximales Ideal besitzt.
\end{definition}

\begin{beispiel}
	$\Z_{(p)} := \left\{\frac{a}{b} \in \Q \mid p \nmid b\right\}$
\end{beispiel}

\begin{bemerkung}
	Ist $R$ lokaler Ring und $\m \ideal R$ das maximale Ideal,
	so ist $R \setminus \m = R^\times$.
\end{bemerkung}

\begin{beispiel}
	Sei $M$ eine $C^\infty$ Mannigfaltigkeit und $x_0 \in M$.
	Dann ist $\cal C^\infty_{M,x_0}$ ein lokaler Ring, denn
	\[
		\cal C^\infty_{M,x_0} \setminus \big(\cal C^\infty_{M,x_0}\big)^\times
		= \{[f:U\xto{C^\infty} \R] \mid x_0 \in U\text{ mit } f(x_0) = 0\}
		=: \m,
	\]
	da $[f]$ eine Einheit ist, genau dann, wenn $f(x_0) \neq 0$: 
	Ist $f: U\xto{C^\infty} \R$ mit $f(x_0) \neq 0$, so existiert
	$W\subset U$ offen, $x_0\in W$ mit $f(x) \neq 0$ f�r alle $x\in W$.
	Damit folgt
	\[
		\left[\frac{1}{f}: W \to \R,\ x\mapsto \frac{1}{f(x)}\right]
		\in \cal C^\infty_{M,x_0}
	\]
	ist Inverses zu $[f]$.
	Zudem ist $\m$ ein Ideal.
\end{beispiel}

\begin{definition}[lokal geringter Raum]
	Ein \emph{lokal geringter Raum} ist ein Paar $(X, \O_X)$ bestehend aus:
	\begin{itemize}
	  \item einem topologischen Raum $X$ und
	  \item einer Garbe $\O_X$ auf $X$ von Ringen,
	\end{itemize}
	so dass $\O_{X,x_0}$ f�r alle $x_0\in X$ ein lokaler Ring ist.
	
	Man nennt $\O_X$ die \emph{Strukturgarbe von $(X,\O_X)$}. Ist
	$x_0\in X$, so hat man das maximale Ideal
	$\m_{x_0} \ideal \O_{X,x_0}$.
	
	Der K�rper 
	\[\kappa(x_0) := \O_{X,x_0} \big/ \m_{x_0} \]
	hei�t \emph{Restklassenk�rper von $x_0$ in $(X,\O_X)$}.
\end{definition}

\begin{beispiel}
	Sei $M$ eine $C^\infty$-Mannigfaltigkeit und $x_0 \in M$,
	so ist $\kappa(x_0) = \R$.
\end{beispiel}

\begin{definition}[lokale Ringhomomorphismen]
	Sind $R,S$ lokale Ringe mit den maximalen Idealen
	$\m_R \ideal R$, $\m_S \ideal S$, so hei�t der Ringhomomorphismus
	$\varphi: R\to S$ \emph{lokal},
	falls
	\[\varphi\inv (\m_S) = \m_R .\]
	�quivalent l�sst sich fordern, dass
	\[\varphi(\m_R) \subset \m_S.\]
\end{definition}

\begin{definition}[Morphismus lokal geringter R�ume]
	\label{def:morphismus lokal geringter raume}
	Ein \emph{Morphismus $f:(X,\O_X) \to (Y,\O_Y)$ lokal geringter R�ume}
	ist ein Paar $(f,f\fis)$ bestehend aus
	\begin{align*}
		f:  X &\to Y \text{ stetig},\\
		f\fis:  \O_Y &\to f_\ast\O_X \text{ Morphismus von Garben auf $Y$},
	\end{align*}
	so dass der von $f\fis$ induzierte Ringhomomorphismus f�r
	$x_0 \in X$, $y_0:= f(x_0) \in Y$
	\[f_{x_0}\fis:
		\funcdef{ \O_{Y,y_0} & \to & \O_{X,x_0}\\~ 
			[s] & \mapsto & [f_U\fis(s)]
		}
	\]
	f�r $s\in \O_Y(U)$ und $y_0\in U$  ein lokaler Ringhomomorphismus ist.
\end{definition}

\begin{bemerkung}
	In \cref{def:morphismus lokal geringter raume} ist $f_{x_0}\fis$
	wohldefiniert:
	
	Sei $[s] = [t] \in \O_{Y,y_0}$, d.h. es existiert $W\subset Y$ offen mit
	$y_0\in W$ und $s\rest W = t\rest W \in \O_Y(W)$.
	Betrachte nun $f_U\fis (s) \in \O_X(f\inv(U))$ f�r 
	$s\in \O_Y(U)$, $U\subset Y$, $y_0\in U$ und analog 
	$f_V\fis (t) \in \O_X(f\inv(V))$ f�r 
	$t\in \O_Y(V)$, $V\subset Y$, $y_0\in V$.
	Da $f\fis$ ein Garbenmorphismus ist, kommutiert damit folgendes Diagramm:
	\[
		\begin{tikzcd}[row sep=large, column sep=large]
		s \dar[mapsto] \symb{\in}
			&[-1cm] \O_Y(U) \rar{f_U\fis} \dar[swap]{\rest W} 
			& \O_X(f\inv(U)) \dar{\rest{f\inv(W)}} \symb{\ni}
			& \dar[mapsto] f_U\fis(s)\\
		s\rest W = t\rest W \symb{\in}
			& \O_Y(W) \rar{f_W\fis}				
			& \O_X(f\inv(W)) \symb{\ni}
			& f_U\fis(s)\rest{f\inv(W)} = f_V\fis(t)\rest{f\inv(W)}\\
		t \uar[mapsto] \symb{\in}
			& \O_Y(V) \rar{f_W\fis} \uar{\rest W} 
			& \O_X(f\inv(V)) \uar[swap]{\rest{f\inv(W)}} \symb{\ni}
			& \uar[mapsto] f_V\fis(t)\\
		\end{tikzcd}
	\]
\end{bemerkung}

\section{Affine Schemata}

\subsection{$\Spec A$ als topologischer Raum}

Sei im Folgenden $A$ ein kommuativer Ring mit $1$ und 
$\Spec A := \{\p \ideal A \mid \p \text{ Primideal}\}$.

\begin{definition}[Zariski Topologie]
	Ist $\a \ideal A$, ein Ideal, setze
	\[
		V(\a) := \{\p \in \Spec A \mid \a \subseteq \p \} \subseteq \Spec A\,.
	\]
	Dann ist durch
	\[
		\cal T := \{ U \subseteq \Spec A \mid
			\exists\ \a \ideal A:\ U = \Spec A \setminus V(\a)\}
	\]
	eine Topologie auf $\Spec A$ definiert. Sie hei�t \emph{Zariski-Topologie}.
\end{definition}

\begin{bemerkung}
	Die abgeschlossenen Teilmengen $M \subset \Spec A$ sind genau die 
	$M = V(\a)$ f�r ein $\a \ideal A$.
\end{bemerkung}



\pagebreak
\section{Affine Schemata}

\subsection{$\Spec A$ als topologischer Raum}

Sei im Folgenden $A$ ein kommuativer Ring mit $1$ und 
$\Spec A := \{\p \ideal A \mid \p \text{ Primideal}\}$.

\begin{definition}[Zariski Topologie]
    \index[def]{Zariski Topologie}
	Ist $\a \ideal A$, ein Ideal, setze
	\[
		V(\a) := \{\p \in \Spec A \mid \a \subseteq \p \} \subseteq \Spec A\,.
	\]
	Dann ist durch
	\[
		\cal T := \{ U \subseteq \Spec A \mid
			\exists\ \a \ideal A:\ U = \Spec A \setminus V(\a)\}
	\]
	eine Topologie auf $\Spec A$ definiert. Sie heißt \emph{Zariski-Topologie}.
\end{definition}

\begin{proof}[der Topologie-Eigenschaften]
	\begin{enumerate}
	  \item Zeige: $\emptyset$, $\Spec A$ offen $\Longleftrightarrow$ 
	  	$\Spec A$, $\emptyset$ abgeschlossen.\\
	  	Dazu: $V(A) = \emptyset$, $V((0)) = \Spec A$
	  \item Zeige: $U_1, U_2$ offen $\Rightarrow$ $U_1 \cap U_2$ offen
	  	$\Longleftrightarrow$ $M_1,M_2$ abgeschlossen $\Rightarrow$
	  	$M_1 \cup M_2$ abgeschlossen.\\
	  	Dazu:
	  	$V(\a) \cup V(\fr b) = V(\a \cap \fr b)$
	  \item $(U_i)_{i\in I}$ offen $\Rightarrow$ $\cup_{i\in I} U_i$ offen
	  	$\Longleftrightarrow$ $(M_i)_{i\in I}$ abgeschlossen
	  	$\Rightarrow$ $\cap_{i\in I} M_i$ abgeschlossen.\\
	  	Dazu:
	  	$\cap_{i\in I} V(\a_i) = V(\sum_{i\in I} \a_i)$
	\end{enumerate}
\end{proof}

\begin{bemerkung}
	Die abgeschlossenen Teilmengen $M \subset \Spec A$ sind genau die 
	$M = V(\a)$ für ein $\a \ideal A$.
\end{bemerkung}

\begin{beispiel}[$\Spec \Z$]
	Für $\a \ideal \Z$ ist $\a = (a)$. Falls $a \neq 0,1,-1$ sei
	$a = \pm p_1^{\nu_1} \cdot \dots \cdot p_r^{\nu_r}$ die 
	Primfaktorzerlegung. Für $p$ Primzahl ist
	\[
		(p) \in V((a)) \Leftrightarrow
		(a) \subseteq (p) \Leftrightarrow
		p \mid a \Leftrightarrow
		p \in \{p_1,\ldots, p_r\}
	\]
	Das bedeutet, die abgeschlossenen Mengen in $\Spec \Z$ sind genau die 
	Mengen $\emptyset$, $\Spec \Z$ und
	$\{(p_1), \ldots, (p_r)\}$ für eine endliche Anzahl an Primzahlen.
	
	Insbesondere gilt
	\begin{itemize}
	  \item $\Spec\Z$ ist nicht hausdorffsch.
	  \item $(0) =: \eta \in \Spec\Z$ liegt in \emph{jeder} nichtleeren 
	  	offenen Teilmenge.
	\end{itemize}
\end{beispiel}

\begin{lemma}
	Sei $x \in \Spec A$, so ist der Abschluss $\overline{\{x\}}$ der
	Menge $\{x\}$ in $\Spec A$ gleich
	\[\overline{\{x\}} = V(x).\]
\end{lemma}
\begin{proof}
	\[
		\overline{\{x\}} = 
		\bigcap_{B\subseteq \Spec A \text{ abg.}\atop x\in B} B
		= \bigcap_{\a\ideal A\atop \a \subseteq x}
		= V(x)
	\]
\end{proof}

\begin{bemerkung}
	Beachte, dass
	\[
		\textcolor{purple}{\a} \subseteq \textcolor{blue}{\fr b} \quad 
    \Rightarrow\quad
		V(\textcolor{blue}{\fr b}) \subseteq V(\textcolor{purple}{\a})
	\]
\end{bemerkung}

\begin{definition}[abgeschlossener Punkt, generischer Punkt]
    \index[def]{Abgeschlossener Punkt}
    \index[def]{Generischer Punkt}
	Sei $X$ ein topologischer Raum.
	Ein $x\in X$ heißt \emph{abgeschlossener Punkt}, wenn
	$\overline{\{x\}} = \{x\}$.
	
	Er heißt \emph{generischer Punkt}, wenn $\overline{\{x\}} = X$ gilt.
	
	Die Menge der abgeschlossenen Punkte bezeichnen wir mit
	$|X|$.
\end{definition}

\begin{beispiel}
	Sei $A = \C[X,Y]$. 
	\begin{itemize}
	  \item $x = (0) \in \Spec A$ ist generisch.
	  \item $x = (X-\alpha, Y-\beta) \ideal A$ ist abgeschlossen,
	  	da aus $x \ideal A$ maximal $V(x) = \{x\}$ und somit $x$ abgeschlossen
	  	folgt.
	  \item $x = (X) \ideal A$ ist weder abgeschlossen noch generisch.
    \item $x = (XY-1) \ideal A$ ist ebenfalls weder abgeschlossen noch
      generisch.
	\end{itemize}
  Wir können die bisherigen Ergebnisse in \thref{fig:spec c xy} zusammenfassen. 
\end{beispiel}

\begin{figure}
	\caption{$\Spec \C[X,Y]$}
	\label{fig:spec c xy}
	\centering
	\begin{tikzpicture}
		\fill[col1shade1] (-4,-2) rectangle (4,2);
		\node[right, text=col1] 
			at (-3.8,-1.5)
			{$|\Spec\C[X,Y]|$};
		\draw[very thick]
			(-4,0) -- (4,0) node[near end, auto]{$\alpha$}
			(0,-2) -- (0,2) node[near end, auto]{$\beta$};
		\fill[col1]
			(-2.8,1) circle[radius=2pt]
			node[above right] {$(X-\alpha, Y-\beta)$};
		\node[generic point=10pt, fill=black!60,
			label={above right:$(0)$}]
			at (5,0)
			{};
		\draw[line width=4pt, col2shade2, opacity=0.5]
			(0,-2) -- (0,2);
		\node[generic point=5pt, fill=col2shade2,
			label={[text=col2]below:$(X)$}]
			at (0,-2.1)
			{};
    \draw[scale=1,domain=.25:2,smooth,variable=\t,green] 
      plot ({1/\t},{\t});
    \draw[scale=1,domain=.25:2,smooth,variable=\t,green] 
      plot ({-1/\t},{-\t});
		\fill[green]
			(1,1) circle[radius=2pt]
			node[above right] {$(XY-1)$};
	\end{tikzpicture}
\end{figure}


\begin{definition}[basisoffene Menge]
    \index[def]{Basisoffene Menge}
	Für $f\in A$ nennt man
	\[ D(f) := \Spec A \setminus V((f)) = \{ \p \in \Spec A \mid f \notin \p\}
	\]
	die \emph{zu $f$ gehörige basisoffene Menge}.
\end{definition}

\begin{lemma}
	\label{lemma:basisoffene mengen sind basis}
	Die Menge $\fr B := \{D(f) \mid f \in A\}$ ist eine Basis der
	Topologie, d.h. jedes offene $U\subseteq \Spec A$ ist eine Vereinigung
	von $D(f) \in \fr B$ und $\fr B$ ist unter endlichen Schnitten 
	abgeschlossen.  
\end{lemma}
\begin{proof}
	Sei $U = \Spec A \setminus V(\a)$ offen und $\p \in U$, so ist
	$\p \notin V(\a)$, also $\a \not\subseteq \p$. Damit existiert
	$f \in \a \setminus \p$ mit $f \notin \p$, also $\p \in D(f)$
	und $f \in \a$. Also $(f) \subseteq \a$ und
	$V(\a) \subseteq V((f))$. Damit folgt $D(f) \subseteq U$.
	
	Zusammenfassend gilt für $U\subseteq \Spec A$ offen: $\forall \p \in U$
	$\exists f\p \in A$: $\p \in D(f\p) \subseteq U$.
	Also
	\[ U = \bigcup_{\p \in U} D(f\p)\]
	Ferner folgt mit \thref{lemma:vereinigungen von v sind produkt}
	$D(f) \cap D(g) = D(fg)$.
\end{proof}

\begin{lemma}
	\label{lemma:vereinigungen von v sind produkt}
	Für $\a, \fr b\ideal A$ gilt
	\[
		V(\a) \cup V(\fr b) = V(\a \cap \fr b) = V(\a \cdot \fr b).
	\]
\end{lemma}
\begin{proof}
	Es ist 
	$\a\fr b \subseteq \a \cap \fr b \subseteq \a, \fr b$.
	Also 
	\[V(\a) \cup V(\fr b) \subseteq V(\a \cap \fr b) 
	\subseteq V(\a\fr b).\]
	Angenommen $V(\a) \cup V(\fr b) \subsetneq V(\a\fr b)$, 
	d.h. $\exists \p \in V(\a \fr b) \setminus \big(V(\a) \cup V(\fr b)\big)$,
	also $\a\fr b \subseteq \p$ aber nicht
	$\a,\fr b \not \subseteq \p$.
	Also existiert $s \in \a \setminus \p$ und $t\in\fr b\setminus \p$.
	Damit ist $st \in \a\fr b \setminus \p$.
	Dies ist ein Widerspruch, da $\p$ ein Primideal ist.
	Folglich herrscht Gleichheit in obiger Inklusionskette.
\end{proof}

\begin{definition}[Radikal]
    \index[def]{Ring!Radikal}
	Für $\a \ideal A$ heißt
	\[
		\sqrt \a := \{ f\in A \mid \exists n \in \N:\ f^n\in \a\}
	\]
	\emph{Radikal} von $\a$.
\end{definition}

\begin{lemma}
	\label{lemma:radikal ist ideal}
	$\sqrt a \ideal A$.
\end{lemma}
\begin{proof}
	\begin{itemize}
	  \item $0\in \sqrt{\a}$ \checkmark
	  \item Sei $f \in \sqrt \a$, $r\in A$. Dann
	  	$f^n \in \a$, $r\in A$. Also 
	  	$(rf)^n \in \a$ und damit $rf\in \sqrt\a$.
	  \item $f,g\in \sqrt\a$ mit $f^n \in \a$, $g^m \in \a$.
	  	\begin{align*}
	  		(f+g)^{n+m-1} &= \sum_{i=0}^{n-1} \binom{n+m-1}{i} f^i g^{n+m-1-i}
	  			+ \sum_{i=n}^{n+m-1} \binom{n+m-1}{i}
	  				f^i g^{n+m-1-i}\\
  				&= \left( \sum_{i=0}^{n-1} \binom{n+m-1}{i} 
  					f^i g^{n-1-i}\right) g^m
  					+  \left(\sum_{i=n}^{n+m-1} \binom{n+m-1}{i}
  						f^i g^{m-1-i}\right) f^n
	  	\end{align*}
	  	Da $g^m$ und $f^n$ jeweils in $\a$ liegen, ist auch die Summe dort.
	\end{itemize}
\end{proof} 

\begin{definition}[Radikalideal (radiziell)]
    \index[def]{Ring!radiziell}
	Ein Ideal $\fr b \ideal A$ heißt \emph{Radikalideal (radiziell)},
	falls
	\[\sqrt \fr b = \fr b.\]
\end{definition}

\begin{bemerkung}
	Es gilt $\sqrt{\sqrt \a} = \sqrt\a$.
\end{bemerkung}

\begin{lemma}
	\label{lemma:radikal ist schnitt}
	Für $\a \ideal A$ gilt
	\[
		\sqrt\a = \bigcap_{\p\in V(\a)} \p
	\]
\end{lemma}
\begin{proof}
	  \newcommand{\bmax}{\b_\text{max}}
	\begin{itemize}
	  \item["`$\subseteq$"']
	  	Sei $f \in \sqrt\a$, $f^n \in \a$. Ist $\p \in V(\a)$, d.h.
	  	$\a \subseteq\p$. Also
	  	$f^n \in \p$ und da $\p$ prim, folgt $f\in \p$.
	  \item["`$\supseteq$"']
	  	Ist $f\notin \sqrt\a$, so zu zeigen, dass 
	  	$f \notin \cap_{\p\in V(\a)} \p$.
	  	Sei also 
	  	$f^n \notin \a$ für alle $n\in \N$.
	  	
	  	Betrachte
	  	\[ M := \{\b \ideal A\mid a\subseteq \b,
	  		f^n \notin \b \forall n\in \N\},
	  	\]
	  	so gilt
	  	\begin{itemize}
	  	  \item $\a \in M$,
	  	  \item $M$ ist angeordnet durch "`$\subseteq$"',
	  	  \item ist $(\b_i)_{i\in I}$ eine total geordnete Teilmenge,
	  	  	so ist $\b:= \cup_{i\in I} \b_i \ideal A$ mit $\b \in M$.
	  	\end{itemize}
	  	Damit hat $M$ mit dem Lemma von Zorn ein maximales Element
	  	$\bmax \in M$.
	\end{itemize}
	Nun sei behauptet, dass $\bmax \ideal A$ ein Primideal ist.
	Dazu sei $xy \in \bmax$, wobei wir annehmen, dass 
	$x,y \notin \bmax$.
	Betrachte
	$\bmax \subsetneq (x) + \bmax$, was ein Ideal in $A$ ist, aber nicht 
	in $M$ liegt. Analog künnen wir dies von $(y) + \bmax$ sagen. Damit
	existieren $n,m \in \N$ mit
	\[
		f^n \in (x) + \bmax
		\qquad
		f^m \in (y) + \bmax.
	\]
	Ergo ist
	\[
		f^{n+m} \in
			(x)\bmax + (y)\bmax + \bmax\bmax + (xy),
	\]
	wobei jeder Summand Teilmenge von $\bmax$ ist und wir folgern
	$f^{n+m} \in \bmax \in M$, wodurch man den Widerspruch erhült.
	
	Damit ist $\bmax \in V(\a)$ und $f\notin \bmax$. 
\end{proof}

\begin{satz}
	\label{satz:v und radikal}
	Für $\a, \b \ideal A$ gilt
	\[ V(\a) \subseteq V(\b) \quad\Leftrightarrow\quad
		\b \subseteq \sqrt\a.
	\]
	Insbesondere gilt sogar
	\[ V(\a) = V(\b) \quad\Leftrightarrow\quad
		\b = \sqrt\a.
	\]
\end{satz}
\begin{proof}
	\begin{itemize}
	  \item["`$\Leftarrow$"']
	  	Aus $V(\a) \subseteq V(\b)$ folgt
	  	\[
	  		\bigcap_{\p \in V(\a)} \p 
	  		\supseteq \bigcap_{\p\in V(\b)} \p
	  	\]
	  	und mit \thref{lemma:radikal ist schnitt}
	  	folgt $\sqrt \a \supseteq \sqrt\b \supseteq \b$.
	  \item["`$\Rightarrow$"']
	  	Aus $\b \subseteq \sqrt\a$, d.h.
	  	$\b \subseteq \cap_{\p\in V(\a)} \p$, folgt
	  	$\b \subseteq \p$ für alle $\p\in V(\a)$.
	  	Also $\p \in V(\a)$.
	\end{itemize}
\end{proof}

\begin{definition}[irreduzibel]
    \index[def]{topologischer Raum!irreduzibel}
	Ein topologischer Raum $X$ heißt \emph{irreduzibel}, wenn gilt:
	Ist $X = A_1 \cup A_2$ mit $A_{1,2}\subseteq X$ abgeschlossen, so ist
	$X = A_1$ oder $X = A_2$.
	
	Eine Teilmenge $Z\subseteq X$ heißt \emph{irreduzibel}, wenn $Z$ mit der
	Teilraumtopologie irreduzibel ist.
\end{definition}

\begin{beispiel}
	$\Spec\Z$ ist irreduzibel. Ist nämlich $A_1 \subsetneq \Spec\Z$ 
	abgeschlossen, so ist $A_1 = \{(p_1), \ldots, (p_r)\}$ für
	irgendwelche Primzahlen $p_i$. 
\end{beispiel}

\begin{lemma}
	\label{lemma:v irreduzibel <=> radikal prim}
	In $\Spec A$ gilt:
	\[
		V(\a) \text{ irreduzibel} \quad\Leftrightarrow\quad
		\sqrt\a \text{ Primideal}.
	\]
\end{lemma}
\begin{proof}
	\begin{itemize}
	  \item["`$\Rightarrow$"']
	  	Sei $xy \in \sqrt\a$, so ist $(xy) \subseteq \sqrt\a$ und mit
	  	\thref{satz:v und radikal} $V(\a) \subseteq V((xy))$.
	  	
	  	Für $\p\in V(\a) \subseteq V((xy))$, gilt:
	  	Ist $xy \in \p$, so folgt $x \in \p$ oder $y\in \p$. Damit
	  	\[
	  		V(\a) \subseteq V((x)) \cup V((y))
	  		\ \Rightarrow\ 
	  		V(\a) = \big(V(\a) \cap V((x))\big) \cup 
	  				\big(V(\a) \cap V((y))\big).
	  	\]
	  	Da $V(\a)$ irreduzibel nach Voraussetzung, folgt
	  	\obda $V(\a) = V(\a) \cap V((x))$, also $V(\a) \subseteq V((x))$.
	  	Wieder mit \thref{satz:v und radikal} folgt
	  	$(x) \subseteq \sqrt\a$ und damit $x \in \sqrt\a$. 
	 \item["`$\Leftarrow$"']
	 	Schreibe $V(\a) = V(\b) \cup V(\fr c) = V(\b\cap \fr c)$.
	 	Dann folgt wiederum mit \thref{satz:v und radikal}
	 	$\sqrt\a = \sqrt{\b \cap \fr c}$.
	 	
	 	Ist $V(\a) \neq V(\b)$, also $V(\b) \subsetneq V(\a)$, also
	 	$\sqrt\a \subsetneq \sqrt\b$, so existiert 
	 	$x\in \sqrt\b \setminus \sqrt\a$. Für $y \in \fr c$, ist 
	 	\[
	 		xy \in \sqrt{\b\fr c} \subseteq \sqrt{\b \cap \fr c} = \sqrt\a.
	 	\]
	 	Nach Voraussetzung ist $\sqrt\a$ Primideal, also 
	 	nach Wahl von $x$ ist $y\in \sqrt\a$.
	 	Insgesamt ist $\fr c \subseteq \sqrt\a$, also
	 	$V(\a) \subseteq V(\fr c)$	und damit $V(\a) = V(\fr c)$.
	\end{itemize}
\end{proof}

\begin{definition}[Nilradikal]
    \index[def]{Ring!Nilradikal}
	\[
		\Nil(A) := \sqrt{(0)}
	\]
	heißt \emph{Nilradikal} von $A$.
\end{definition}

\begin{korollar}
	Es gilt
	\[
		\Spec A \text{ irreduzibel}
		\quad\Leftrightarrow\quad
		\Nil(A) \text{ Primideal}.
	\]
\end{korollar}
\begin{proof}
	\thref{lemma:v irreduzibel <=> radikal prim} mit $\a = (0)$.
\end{proof}

\begin{definition}[noethersch]
    \index[def]{topologischer Raum!noethersch}
	Ein topologischer Raum heißt \emph{noethersch}, wenn gilt:
	Ist 
	\[
		A_1 \supseteq A_2 \supseteq A_3 \supseteq\ldots
	\]
	eine Folge abgeschlosser Teilmengen, so existiert
	$n_0\in \N$ mit $A_i = A_{i+1}$ für alle $i\geq n_0$. 
\end{definition}

\begin{lemma}
	\label{lemma:A noethersch => Spec A noethersch}
	Ist $A$ noethersch, so ist auch $\Spec A$ noethersch.
\end{lemma}
\begin{proof}
	Sei 
	\[
		A_1 \supseteq A_2 \supseteq \ldots
	\]
	eine Folge abgeschlossener Teilmengen, also
	\[
		V(\a_1) \supseteq V(\a_2) \supseteq \ldots
	\]
	mit $A_i = V(\a_i)$ für geeignete $\a_i \in \Spec A$, so ist
	\[
		\sqrt{\a_1} \subseteq \sqrt{\a_2} \subseteq \ldots
	\]
	eine aufsteigende Idealkette in $A$.
\end{proof}

\begin{satz}
	Ist $X$ noetherschscher topologischer Raum und 
	$\emptyset \neq A \subseteq X$ abgeschlossen, so zerlegt sich
	\[
		A = A_1 \cup \ldots \cup A_r
	\]
	in abgeschlosse irreduzible Teilmengen $A_i \subseteq A$.
	Nimmt man $A_i \not\subseteq A_j$ für $i\neq j$, so ist die Zerlegung
	bis auf Reihenfolge eindeutig.
	
	Die $A_i$ heißen \emph{(irreduzible) Komponenten} von $A$.
\end{satz}
\begin{proof}
	\begin{description sf}
	\item[Existenz.]
		Sei 
		\[
			\cal V := \{ A\subseteq X \mid \emptyset \neq A 
				\text{ abgeschlossen, $A$ hat keine solche Zerlegung} \}. 
		\]
		Angenommen $\cal V \neq\emptyset$, so hütte man
		ein inklusionsminimales $A \in \cal V$, denn falls nicht gübe es
		\[
			A_1 \supsetneq A_2 \supsetneq \ldots
		\]
		mit $A_i \in \cal V$. Da $X$ noethersch, müsste diese Folge
		stationür werden, wodurch man einen Widerspruch erhült.
		
		Dieses $A \in \cal V$ hat keine solche Zerlegung, ist also
		insbesondere nicht irreduzibel. Damit gibt es
		\[
			A = A_1 \cup A_2\quad A_i \subseteq X \text{ abgeschlossen, }
			A_i \neq A
		\]
		Da $A \in \cal V$ minimal sind $A_1, A_2 \notin \cal V$.
		Aber damit ist
		$A = A_1 \cup A_2 \notin \cal V$. Ein Widerspruch, der wie gewünscht
		$\cal V = \emptyset$ liefert.
	\item[Eindeutigkeit.]
		Sind 
		\[
			A = A_1 \cup \ldots \cup A_r = 
				A_1' \cup \ldots \cup A_s'
		\]
		zwei solcher Zerlegungen, so ist
		$A_1 \subseteq A_1' \cup \ldots \cup A_s'$, also
		$A_1 = (A_1' \cap A_1) \cup \ldots \cup (A_s' \cap A_1)$.
		Da $A_1$ irreduzibel künnen wir \obda $A_1 = A_1 \cap A_1'$ 
		annehmen. Also ist $A_1 \subseteq A_1'$.
		
		Analog ist $A_1' \subseteq A_k$ für ein $k=1,\ldots,r$.
		Zusammenfassend gilt
		\[
			A_1 \subseteq A_1' \subseteq A_k,
		\]
		was nach Voraussetzung $k = 1$ impliziert. Also $A_1 = A_1'$.
		
		Nun sukzessive weiter. 
	\end{description sf}
\end{proof}

\begin{beispiel}
	In $\Spec k[X,Y]$ zerfüllt
	\[
		V((XY)) = V((X)) \cup V((Y)).
	\]
	Im Bild
	\tikz[baseline, scale=0.8]{
		\draw (-1,0) -- (1,0) node[right,auto]{$V((Y))$}
			(0,-1) -- (0,1) node[above,auto]{$V((X))$};
	}
\end{beispiel}

\begin{beispiel}
	Sei $k$ algebraisch abgeschlossen. Betrachte $\Spec k[X,Y]$.
	Die \tikzmark[1]{maximalen Ideale} sind gerade 
	$\m = (X-\alpha, Y-\beta)$ für $\alpha,\beta \in k$.
	Ein abgeschlosser Punkt $\m \in \Spec k[X,Y]$ wird eindeutig durch
	$(\alpha, \beta) \in k^2$ gegeben.
	\tikzmargin[1]{north}{Quelle suchen!}
	
	$\A_k^2 := \Spec k[X,Y]$ wird der 
	\emph{2 dimensionale affine Raum über $k$} genannt.
	Man hat die Bijektion
	\[
		|\A_k^2| \xto{\phi} k^2.
	\]
	Eine abgeschlossene Teilmenge $A = V(\a) \subseteq \A_k^2$ liefert
	\[
		A \cap |\A_k^2| \cong_\phi \{ (\alpha,\beta) \in k^2 \mid 
			f(\alpha,\beta) = 0\ \forall f\in \a\},
	\]
	denn
	\begin{align*}
		A \cap |\A_k^2| &= V(\a) \cap |\A_k^2| = |V(\a)| \\
		&= \{ \m \in \Spec k[X,Y] \mid \a \subseteq \m,\ \m
			\text{ maximal}\}
			= 
			\{(X-\alpha, Y-\beta) \ideal k[X,Y] \mid \a \subseteq
				(X-\alpha, Y-\beta) \} \\
		&= \{(X-\alpha, Y-\beta) \mid f(X,Y) \in \alpha\ \Rightarrow\ 
			f(X,Y) \in (X-\alpha, Y-\beta)\}\\
		&\ \tikzmark[2]{=}\  
			\{(X-\alpha, Y-\beta) \mid f(X,Y) \in \alpha\ \Rightarrow\ 
			f(X,Y) = (X-\alpha)g(X,Y) + (Y-\beta)h(X,Y)\} \\
%ich weiü, das ist unschün, aber nach dem align wird der rand der nüchsten
%seite benutzt :-(
\tikzmargin[2]{north, above=2cm}{
	"`$\Rightarrow$"' ist klar. Also zu "`$\Leftarrow$"'.\\
	Es ist $f(\alpha,\beta) = 0$, also
	$f(X,Y) = (X-\beta) h(X,Y)$ für gewisses $h$.
	Es ist 
	$f(X,Y) - f(\alpha,Y) = (X-\alpha) g(X,Y)$,
	da die linke Seite $X = \alpha$ als Nullstelle hat.
}
		&= \{(X-\alpha, Y-\beta) \mid f(\alpha,\beta) = 0
			\ \forall f\in \a\}\\
		&\xto{\phi}
			\{(\alpha,\beta) \in k^2 \mid f(\alpha, \beta) = 0\ 
			\forall f\in \a\}.
	\end{align*}
	
	In $\A_k^2$ hat man aber noch mehr Punkte:
	Sei $\p \ideal k[X,Y]$ Primideal, aber nicht maximal, so ist
	$\p \in \A_k^2$ kein abgeschlossener Punkt.
	Ist beispielsweise $\p = (f(X,Y))$ für $f\in k[X,Y]$ irreduzibel, 
	so liegen alle $(\alpha,\beta) \in k^2$ mit $f(\alpha,\beta) = 0$
	auf der entsprechenden Menge in $k^2$, d.h.
	\[
		\p = (f(X,Y)) \subseteq 
		\m_{\alpha,\beta} := (X-\alpha, Y-\beta)
		\quad \Rightarrow\quad
		\m_{\alpha,\beta} \in \overline{\{\p\}}.
	\]
	\thref{fig:spec k xy} verdeutlicht dies.
\end{beispiel}

\begin{figure}\centering
	\caption{$\Spec k[X,Y]$}
	\label{fig:spec k xy}
	\begin{tikzpicture}
		\draw[very thick] 
			(-3,0) -- (3,0) node[near end, above] {$X$}
			(0,-2) -- (0,2) node[near end, right] {$Y$};
		
% 		\draw[col1,thick] 
% 			(-3,1) to[out=-5, in=135]  (0.2,-0.2) 
% 			to[out=-45, in=225, looseness=2] (-1,0) 
% 			to[out=45, in=180, looseness=0.5] (3,1.5)
% 			node[pos=0.9] {$f(X,Y) = 0$};
		\draw[col1, thick]
			(-3,1) 
			.. controls (5,-2) and (-8,-2) .. 
			(3,1.5)
			node[pos=0.98, above, sloped] {$f(X,Y) = 0$}
			coordinate[pos=0.05] (a);
		
		\fill[col1shade2] (a) circle[radius=2pt]
			node[above right, col1] {$(\alpha,\beta)$};
		
		\path (-3,1)
			node[generic point=10pt, fill=col1shade2] {}
			node[above left, col1] {$(f(X,Y))$};
			
		\path (4,0)
			node[generic point=10pt, fill=col2shade2] {}
			node[above right, col2] {$(0)$};
	\end{tikzpicture}
\end{figure}

\begin{lemma}
	\label{lemma:pi inv ist homöo auf bild}
	Ist $A$ ein Ring, $\a \in \Spec A$ und 
	$\pi: A \twoheadrightarrow A\big/ \a$ die Projektion, so ist
	\[
		\varphi := \pi\inv: 
			\funcdef{\Spec A\big/\a & \to & \Spec A \\
				\overline{\p} & \mapsto & \pi\inv(\overline{\p})}
	\]
	ein Homöomorphismus auf sein Bild
	\[
		\Spec A\big/\a \xto[\approx]{\pi\inv} V(\a) \subseteq \Spec A.
	\]
\end{lemma}
\begin{proof}

\end{proof}


\begin{definition}[(quasi)-kompakt]
    \index[def]{topoloischer Raum!quasi-kompakt}
	Ein topologischer Raum $X$ heißt \emph{quasi-kompakt}, wenn gilt:
	Ist $X = \bigcap_{i\in I} U_i$ mit $U_i$ offen, so existiert eine endliche
	Teilmenge $F\subset I$ mit
	$X = \bigcap_{i\in F} U_i$.
	
	$X$ heißt \emph{kompakt}, wenn $X$ hausdorffsch und quasi-kompakt ist.
\end{definition}


\begin{satz}
	Ist $A$ ein Ring, so ist $\Spec A$ quasi-kompakt.
\end{satz}
\begin{proof}
	Wir zeigen: Ist $\emptyset = \bigcap_{i\in I} Z_i$ für abgeschlossene
	$Z_i$, so existiert $F\subset I$ endlich mit
	$\emptyset = \bigcap_{i\in F} Z_i$.
	
	Sei also $Z_i = V(\a_i)$, $\a_i \ideal A$ und
	\[
		V(A) = \emptyset = \bigcap_{i\in I} V(\a_i)
		= V\left(\sum_{i\in I} \a_i\right)
	\]
	Nach \thref{satz:v und radikal} ist damit
	\[
		A = \sqrt{\sum_{i\in I} \a_i},
	\]
	also insbesondere $1 \in \sqrt{\sum_{i\in I} \a_i}$ und
	$1 \in \sum_{i\in I} \a_i$. Ergo
	\[
		1 = a_{i_1} + \ldots + a_{i_r},
	\]
	für $F:= \{i_1, \ldots, i_r\} \subset I$.
	Nun ist
	$1 \in \a_{i_1} + \ldots + \a_{i_r}$,
	also 
	\[
		(1) = A \subseteq \a_{i_1} + \ldots + \a_{i_r}.
	\]
	Wiederum mit \thref{satz:v und radikal} ist
	\[
		\emptyset = V(A) \supseteq \bigcap_{k=1}^r V(\a_{i_k}).
	\] 
\end{proof}

\subsection{$\Spec A$ als lokal geringter Raum}

Wir wollen $\O_{\Spec A}$ als die "`guten Funktionen"' auf $\Spec A$ auffassen,
aber dazu müssen wir es besser verstehen. 

\begin{definition}[multiplikative Teilmenge, Lokalisierung]
    \index[def]{Ring!Lokalisierung}
    \index[def]{Ring!Multiplikative Teilmenge}
	\label{def:lokalisierung}
	Sei $A$ ein Ring, dann heißt $S\subseteq A$ \emph{multiplikative Teilmenge},
	wenn $1\in S$ ist und aus $a,b\in S$ auch $ab\in S$ folgt.
	
	Die \emph{Lokalisierung} $A_S$ oder $A[S\inv]$ von $A$ bezüglich $S$ ist
	der Ring
	\[
		A_S := \big(A \times S \big) \big/ \sim
	\]
	mit
	\[
		(a,s) \sim (b,t) \quad\Leftrightarrow\quad
		\exists u \in S:\ u(at - bs) = 0.
	\]
	Schreibe $\frac a s := [(a,s)]$ und definiere eine Ringstruktur auf
	$A_S$ durch Bruchrechnen.
\end{definition}

\begin{lemma}[Universelle Eigenschaft der Lokalisierung]
	\label{lemma:universelle eigenschaft lokalisierung}
	Wir haben die folgende universelle Eigenschaft: Ist
	$S\subseteq A$ wie in \thref{def:lokalisierung}, $\varphi: A \to R$
	ein Ringhomomorphismus, so dass $\varphi(S) \subseteq R^\times$, so
	existiert ein eindeutiger Ringhomomorphismus, der das
	Diagramm
	\[\begin{tikzcd}
		A \rar{\iota} \arrow{dr}{\varphi} & A_S \dar{\exists!} \\
		& R
	\end{tikzcd}\]
	kommutativ macht, wobei
	$\iota: A \to A_S,\ a \mapsto \frac a 1$.
\end{lemma}
\begin{proof}
	Klar, weil dieses $\psi: A_S \to R$ durch
	\[
		\psi\left(\frac a s\right) = 
		\psi\left(\frac a 1\right) \psi\left(\frac 1 s\right) =
		\varphi(a) \varphi(s)\inv
	\]
	eindeutig festgelegt ist.
\end{proof}

\begin{beispiel}
	\begin{itemize}
	  \item $S = \{f^n \mid n \in \N_0\}$, $f \in A$ fest.
	  	\[ A_S =: A_f := \left\{ \frac{a}{f^n} \mid n\in \N_0\right\}\]
	  \item $S = A \setminus \p$, $\p \in \Spec A$.
	  	\[ A_\p := \left\{ \frac a b \mid a \in A,\ b\notin \p \right\}\]
	  	ist ein lokaler Ring mit dem maximalen Ideal $\p A_\p$.
	\end{itemize}
\end{beispiel}


\begin{satz}
	\label{satz:spec a hat eindeutige ringgarbe}
	Sei $X = \Spec A$. Dann existiert auf $X$ eine bis auf Isomorphie 
	eindeutige Ringgarbe $\O_X$ mit:
	\begin{enumerate}[label=\roman{*})]
	  \item Es existiert ein Ringhomomorphismus
	  	$\varphi: A \xto{\cong} \O_X(X)$.
	  \item Für $f\in A$ betrachte 
	  	\[\funcdef{ \O_X(X) & \to & \O_X(D(f)) \\
	  		\varphi(f) & \mapsto & \varphi(f) \rest{D(f)}.}\]
	  	Dann ist $\varphi(f)\rest{D(f)} \in \O_X(D(f))^\times$ eine Einheit
	  	und der eindeutig durch 
	  	\[
	  	\begin{tikzcd}
	  		A \rar{\iota} \dar{\varphi}[swap]{\cong}
	  			\drar & A_f \dar{\exists!}[swap]{\varphi_f}\\
	  		\O_X(X) \rar{\cdot\rest{D(f)}} & \O_X(D(f))\\
	  	\end{tikzcd}
	  	\]
	  	gegebene Ringhomomorphismus $\varphi_f$ ist ein Isomorphismus.
	  \item Für $\p\in \Spec A$ hat man das koanonische Diagramm
	  	\[\begin{tikzcd}
	  		A \rar{\varphi}[swap]{\cong}  \dar{\iota} 
	  			& \O_X(X) \dar \\
	  		A_\p \rar{\varphi_\p} & \O_{X,\p}
	  	\end{tikzcd}\]
	  	und $\varphi_\p: A_\p \to \O_{X,\p}$ ist ein Isomorphismus.
	\end{enumerate}
\end{satz}

\subsubsection{Beweis von \autoref{satz:spec a hat eindeutige ringgarbe}}

Für den Beweis benötigen wir noch eine Definition.

\begin{definition}[$\fr B$-(Prä)Garbe]
    \index[def]{$\fr B$-(Prä-)Garbe}
	$\F:D(f) \mapsto A_f$ heißt \emph{$\fr B$-Prügarbe} auf
	$X = \Spec A$, wenn 
	$\F$ eine Prägarbe auf 
	\[
		\fr B := \{D(f) \subset X \mid f \in A\}
	\]
	ist.
	
	$\F$ heißt \emph{$\fr B$-Garbe}, wenn $\F$ eine $\fr B$-Prägarbe ist
	und die Garbenbedingungen für die $D(f)$ erfüllt sind.
\end{definition}

\begin{hilfslemma}
	\label{hilfslemma:1}
	Es gilt:
	\begin{enumerate}
	  \item $\O_X: D(f) \mapsto A_f$ ist eine $\fr B$-Garbe.
	  \item Ist $\F$ eine $\fr B$-Garbe, so existiert eine bis auf
	  	Isomorphie eindeutige Garbe $\bar \F$ auf $X$ mit
	  	$\bar\F(D(f)) = \F(D(f))$ für alle $D(f) \in \fr B$.
	\end{enumerate}
\end{hilfslemma}
\begin{proof}
	\begin{enumerate}
	  \item 
	\end{enumerate}
\end{proof}


 TODO
 
 
\begin{definition}[(affines) Schema]
    \index[def]{affines Schema}
    \index[def]{Schema}
	Ein \emph{affines Schema} ist ein lokal geringter Raum
	$(X,\O_X)$, der zu einem $(\Spec A, \O_{\Spec A})$ als lokal geringter
	Raum isomorph ist.
	
	Ein \emph{Schema} ist ein lokal geringter Raum $(X,\O_X)$, der eine
	offene Überdeckung durch affine Schemata besitzt, d.h.
	$X = \bigcup_{i\in I} U_i$ mit $U_i \subseteq X$ offen und 
	$(U_i, \O_X\rest{U_i})$ ist ein affines Schema.
\end{definition}

\begin{bemerkung}
	Beachte dabei: Ist $X$ ein topologischer Raum, $\F$ eine Garbe auf $X$,
	$U\subseteq X$ offen, so ist durch
	\[
		\F\rest U:\ V \mapsto \F\rest U (V) := \F(V)
	\]
	eine Garbe $\F\rest U$ auf $U$ definiert.
\end{bemerkung}

\begin{definition}[Morphismus von Schemata]
    \index[def]{Schema!Morphismus von Schemata}
	Ein \emph{Morphismus von Schemata} ist ein
	Morphismus von lokal geringten Räumen
	\[
		(f,f\fis): (X, \O_X)  \to  (Y, \O_Y).
	\]
  mit $f:X \to Y$ stetig und $f^{\fis} : \O_Y \to f_*\O_X$ Garbenmorphismus
  auf $Y$ so dass $\O_{Y,f(x)}\to\O_{X,x}$ lokaler Ringhomomorphismus
\end{definition}

\begin{bemerkung}
	Man hat einen kontravarianten Funktor
	\[
		\funcdef{ \Ring & \to & \affSch \\
			A & \mapsto & (\Spec A, \O_{\Spec A}) \\
			A \xto{\varphi} B & \mapsto & 
			(f,f\fis): (\Spec B, \O_{\Spec B}) \to (\Spec A, \O_{\Spec A})}
	\]
	durch 
	\[
		f: \funcdef{\Spec B & \to & \Spec A\\
			\q & \mapsto & \varphi\inv(\q)},
	\]
	wobei die Stetigkeit hier klar ist, und
	\[
		f\fis: \O_{\Spec A} \to f_\ast \O_{\Spec B}.
	\]
	Letzterer ist für $g\in A$ gegeben durch
	\[
		f\fis_{D(g)}: \funcdef{\O_{\Spec A}(D(g)) = A_g & \to &  
			\big(f_\ast\O_{\Spec B}\big)(D(g))
			 \tikzmark{=} B_{\varphi(g)}\\
			 \frac{a}{g^n} & \mapsto & \frac{\varphi(a)}{\varphi(g)^n}}
	\]
	wobei wir \tikzarrow{mark above}{$\textcolor{lightgray}{\bullet}$} durch
	\[
		f\inv(D(g)) = \{\q\in \Spec B \mid f(\q) \in D(g) \}
			= \{\q\in \Spec B \mid \varphi\inv(\q) \not\ni g\}\\
			= \{\q \in \Spec B \mid \q \not\ni \varphi(g) \}
	\]
	erhalten.
	Diese Abbildung ist funktoriell und lokal, da für $\p\in \Spec A$
	\[
		f\fis_\p: \funcdef{ A_\p & \to & \O_{\Spec B, \q} \\
			\frac a \gamma & \mapsto & \frac{\varphi(a)}{\varphi(\gamma)}}
	\]
	für $\p = \varphi\inv(\q)$, $\gamma \notin \p$ 
	(also $\varphi(\gamma) \notin \q$) ein lokaler Ringhomomorphismus ist. 	
\end{bemerkung}
\pagebreak

% vim: set ft=tex :

\section{Beispiele}

\subsection{$\Spec \Z$}

Jeder Ring $A$ hat einen eindeutigen Homomorphismus
\[
	\funcdef{
		\Z & \to & A\\
		 1 & \mapsto & 1\\
		 z & \mapsto & \begin{cases} 1 + 1 + \ldots + 1 & z > 0\\
		 	0 & z = 0\\
		 	-1 -1 - \ldots- 1 & z < 0
		 \end{cases}.}
\]
$\Z$ ist daher ein \emph{initiales Objekt} in der Kategorie $\Ring$.

Wir haben daher einen eindeutigen Morphismus $\Spec A \to \Spec \Z$ von
affinen Schemata. $\Spec \Z$ ist ein \emph{finales Objekt} in
der Kategorie $\affSch$.

Ferner können wir zusammenfassen
\paragraph{Offene Mengen}
	$\emptyset \neq U\subseteq \Spec \Z$ offen 
	$\Leftrightarrow$ $U = \Spec \Z \setminus \{(p_1),\ldots,(p_r)\}$
	
\paragraph{Basisoffene Mengen}
	$D(f) = \{\p \in \Spec\Z \mid f \notin \p\} = 
	\Spec\Z \setminus \{(p_1),\ldots,(p_r)\}$ für 
	$f = p_1^{\nu_1}\ldots p_r^{\nu_r}$.
	
\paragraph{Strukturgarbe}
	\begin{align*}
		\O_{\Spec\Z} (D(f)) &= \Z_f  = 
			\left\{ \frac{a}{f^n} \mid n\in \N_0, a\in \Z\right\} \\
		\O_{\Spec\Z, (p)} &= \Z_{(p)} = 
			\left\{ \frac{a}{b} \mid p\nmid b, a\in \Z\right\} 
 	\end{align*}

\subsection{$\Spec k$ für einen Körper $k$}
\paragraph{Als topologischer Raum}
	$\Spec k = \{(0)\}$.

\paragraph{Strukturgarbe}
	$\O_{\Spec k}(\{(0)\}) = k$.

\begin{bemerkung}
   Sei $A$ ein Ring. Angenommen wir haben 
  	$\Spec A \xto{(f,f\fis)} \Spec k$ für einen Körper $k$, so haben wir
  	\[
  		f\fis_{\Spec k}: k = \O_{\Spec k} \to f_\ast\O_{\Spec A}(\Spec k)
  			\tikzmark{=} A,
  	\]
  	wobei \tikzarrow{mark above}{} aus 
  	$\O_{\Spec A}(f\inv(\{(0)\})) = \O_{\Spec A}(\Spec A)$ resultiert.
  	Insgesamt ist $A$ also eine $k$-Algebra (d.h. ein Ring zusammen mit
  	$k\to A$).
  	
  	Bemerke hierbei "`Grothendiecks Gesamtphilosophie"':
  	\begin{quote}\itshape
  		Alles relativ lesen!
  	\end{quote}
\end{bemerkung}

\begin{definition}[$S$-Schema]
	Sei $S$ ein Schema. Dann ist ein \emph{$S$-Schema} ein Schema $X$
	zusammen mit einem Strukturmorphismus $X \xto{\varphi} S$.
	Dies ergibt die Kategorie $\Sch_S$, wenn man
	\[
		\Hom( X\xto{\varphi}S, Y\xto{\varphi} S) := 
		\left\{ 
		\begin{tikzcd}
		X \arrow{rr}{f} \drar{\varphi} & & Y \dlar{\psi} \\ & S &
		\end{tikzcd}
		\right\}
	\]
	setzt.
\end{definition}

\begin{beispiel}
	$\Sch_k := \Sch_{\Spec k}$ sind die sog. \emph{$k$-Schemata}.
	Ein Beispiel hierfür ist
	$\Spec k[X_1,\ldots,X_n] \to \Spec k$ via 
	$k \hookrightarrow k[X_1,\ldots,X_n]$.
\end{beispiel}


\begin{bemerkung}
	Sei $X$ ein Schema und $x\in X$ und weiter $\m_x \ideal \O_{X,x}$ das
	maximale Ideal.
	Dann ist 
	\[
		\kappa(x) := k(x) := \O_{X,x} \big/ \m_x
	\]
	der \emph{Restklassenkörper von $x$}.
	
	Betrachte nun $(f,f\fis): \Spec k \to X$ mit
	\[
		f: \funcdef{\Spec k(x) & \to & X \\
			\eta_x & \mapsto & x,}
	\]
	wobei topologisch gesehen $\eta_x \in \Spec k(x)$ der einzige Punkt 
	dieses Schemas ist.
	Für $U\subseteq X$ offen haben wir:
	\[
		f\fis_U : \O_X \to 
			f_\ast \O_{\Spec k(x)}(U) = 
			\begin{cases} 0 & x\notin U \\ k(x) & x \in U. \end{cases}
	\]
	Im Fall $x \in U$ geht dies via
	\[
		\O_X(U) \to \O_{X,x} = \varinjlim_{x\in V} \O_X(V)
			\overset\pi\twoheadrightarrow  \O_{X,x}\big/ \m_x = k(x). 
	\]
	
	Ist umgekehrt $(f,f\fis):\Spec k \to X$ ein Schemamorphismus, so
	setze $x := f((0)) \in X$ und
	$f\fis: \O_X \to f_\ast \O_{\Spec k}$ liefert einen Ringhomomorphismus der
	Halme:
	\[
		f_x\fis: \O_{X,x} \to \O_{\Spec k, (0)} = k.
	\]
	Dieser ist lokal (also $f\fis_x (\m_x) = (0)$). Damit ist
	\[
		\begin{tikzcd}
		k(x) = \O_{X,x} \big/ \m_x \rar[hookrightarrow]{f_x\fis \mod \m_x} 
		&[7ex]  {f_x\fis \mod \m_x} k
		\end{tikzcd}
	\]
	wohldefiniert und somit ist $k \mid k(x)$ eine Körpererweiterung.
	
	Zusammengefasst haben wir:
	\[	\fbox{\parbox{5cm}{
			Einen Punkt $x\in X$ wählen mit Restklassenkörper
			$k(x)$ und eine Körpererweiterung $k\mid k(x)$.}}
		\Longleftrightarrow
		\fbox{\parbox{5cm}{
			Einen Schemamorphismus $\Spec k \to X$ wählen
			für eine Körpererweiterung $k\mid k(x)$.}}
	\]
\end{bemerkung}

\subsection{Der Affine $n$-dimensionale Raum über $k$}
Sei $k$ wieder ein Körper. Der affine $n$-dimensionale Raum über $k$ ist
$\A_k^n := \Spec k[X_1,\ldots, X_n]$.

Wir erinnern an den Hilbertschen Nullstellensatz:
\begin{satz}[Hilbertscher Nullstellensatz]
	\label{satz:hilbertscher nullstellensatz}
	Sei $k$ algebraisch abgeschlossen. Dann ist jedes maximale Ideal
	in $k[X_1,\ldots, X_n]$ von der Form
	$(X_1-a_1, \ldots, X_n - a_n)$.
\end{satz}
\begin{proof}
	ohne Beweis.
\end{proof}

Wir haben bereits gezeigt:
\[
	|\A_k^n| = k^n, \qquad\text{via } 
		(X_1-a_1,\ldots,X_n-a_n) \mapsto (a_1,\ldots,a_n).
\]
Sei $\p = (f_1, \ldots, f_r)$ ein nicht maximales Ideal in $k[X_1,\ldots,X_n]$
(die Darstellung ist nach \thref{satz:hilbertscher nullstellensatz}) möglich,
so gilt
\[
	\p \subseteq (X_1 - a_1, \ldots, X_n - a_n)
	\quad\Leftrightarrow\quad
	f_1(a_1,\ldots,a_n) = 0, \ldots,
	f_r(a_1,\ldots,a_n) = 0
\]
Wir können dies in \autoref{fig:spec k xy 2} "`sehen"'.

\begin{figure}\centering
	\caption{$\Spec k[X_1,\ldots,X_n]$}
	\label{fig:spec k xy 2}
	\begin{tikzpicture}
		\draw[very thick] 
			(-3,0) -- (3,0) node[near end, above] {$X_1$}
			(0,-2) -- (0,2) node[near end, right] {$X_2$};
		
% 		\draw[col1,thick] 
% 			(-3,1) to[out=-5, in=135]  (0.2,-0.2) 
% 			to[out=-45, in=225, looseness=2] (-1,0) 
% 			to[out=45, in=180, looseness=0.5] (3,1.5)
% 			node[pos=0.9] {$f(X,Y) = 0$};
		\draw[col1, thick]
			(-3,1) 
			.. controls (5,-2) and (-8,-2) .. 
			(3,1.5)
			node[pos=1, right, text width=2.9cm, font=\scriptsize] 
				{$\{(a_1,\ldots,a_n) \mid f_j(a_1,\ldots,a_n) = 0,$\\ 
					$j=1\ldots r\}$}
			coordinate[pos=0.05] (a);
		
		\fill[col1shade2] (a) circle[radius=2pt]
			node[above right, col1] {$(\alpha,\beta)$};
		
		\path (-3,1)
			node[generic point=10pt, fill=col1shade2] {}
			node[above left, col1] {$\p$};
	\end{tikzpicture}
\end{figure}

\subsection{Ohne Titel}
Betrachte $k\ldbrack X_1, \ldots, X_n\rdbrack = 
	k\ldbrack X_1,\ldots,X_{n-1}\rdbrack\ldbrack X_n\rdbrack$
mit $R\ldbrack X\rdbrack = \{\sum_{i=0}^\infty a_i X^i \mid a_i \in R\}$.

\begin{bemerkung}
	$g \in k\ldbrack X_1, \ldots, X_n\rdbrack \setminus (X_1,\ldots,X_n)$
	ist eine Einheit.
\end{bemerkung}
\begin{proof}
	Idee: Ansatz für eine Variable:
	$g(X) = a_0 + a_1X + a_2X^2+ \ldots$. Dann
	\[
		1 = g(X)h(X) = 
		\underbrace{a_0 b_0}{= 1} + 
		(\underbrace{a_0b_1+a_1b_0}{= 0})X + \ldots
	\] 
\end{proof}

\paragraph{Funktor $\Spec$} Wir haben den Funktor $\Spec$:
Die Ringhomomorphismen
\[\everymath{\displaystyle} \begin{tikzcd}[row sep=tiny, outer sep=5pt]
	k[X_1,\ldots,X_n] \rar & k[X_1,\ldots,X_N]_{(X_1,\ldots,X_n)} \rar &
		k\ldbrack X_1,\ldots,X_n \rdbrack \rar & k \\
	f \rar[mapsto] & \frac{f}{1} \\
	& \frac f g \rar[mapsto] & f g\inv \\
	&& h \rar[mapsto] & h(0)
\end{tikzcd}\]
induzieren 
\[\everymath{\displaystyle} \begin{tikzcd}[row sep=tiny, outer sep=5pt]
	&\Spec k \rar & k\ldbrack X_1,\ldots,X_n \rdbrack \rar & 
	k[X_1,\ldots,X_N]_{(X_1,\ldots,X_n)} \rar &
	\Spec k[X_1,\ldots,X_n] \\
	\text{topologisch:} &  
	(0) \rar[mapsto] & (X_1,\ldots,X_N)  \rar[mapsto]& 
	(X_1,\ldots,X_n) \rar[mapsto] & (X_1 , \ldots,X_n).\\
	&&\makebox[0pt]{\parbox{3cm}{\centering\small 
		einziger abgeschlossener Punkt}} 
	&\makebox[0pt]{\parbox{3cm}{\centering\small 
		einziger abgeschlossener Punkt}}
	&\makebox[0pt]{\parbox{3cm}{\centering\small 
		entspricht dem abgeschlossenen Punkt $(0,\ldots,0) \in k^n$}}
\end{tikzcd}\]
Dies ist ein Homöomorphismus auf $\{\p \in \A_k^n \mid 
\p \subseteq (X_1,\ldots,X_n) = V(\p) = \overline{\{\p\}} \subseteq \A_k^n$.

Was passiert aber auf Schemaniveau?
\begin{center}\begin{tikzcd}[column sep=large]
	\node{\tikz{
		\fill[col1shade2] circle[radius=2pt];
	}};
	\rar &
	\node{\tikz{
		\draw[->]
			(-1,0) -- (1,0) node[very near end, above] {$X_1$};
		\draw[->]
			(0,-1) -- (0,1) node[very near end, right] {$X_n$};
		\fill[col1shade2] circle[radius=2pt];
		\node[text width=2cm, font=\scriptsize, text=col1shade2, right]
			 at (0.2,-0.5)
			 (text)
			 {einziger abgeschlossener Punkt};
	}};
	\rar &
	\node{\tikz{
		\draw[->]
			(-1,0) -- (1,0) node[very near end, above] {$X_1$};
		\draw[->]
			(0,-1) -- (0,1) node[very near end, right] {$X_n$};
		\fill[col1shade2] circle[radius=2pt];
		\node[text width=2cm, font=\scriptsize, text=col1shade2, right]
			 at (0.2,-0.5)
			 (text)
			 {einziger abgeschlossener Punkt};
		\draw[col1, thick, dashed]
			(-1,1) 
			.. controls (1.5,-0.8) and (-1.5,-0.8) .. 
			(1,1);
		\node[generic point=5pt, fill=col1] at (1,1) {};
		\node[right] at (1,1) {$\p$};
	}};
	\rar &
	\node{\tikz{
		\draw[->]
			(-1,0) -- (1,0) node[very near end, above] {$X_1$};
		\draw[->]
			(0,-1) -- (0,1) node[very near end, right] {$X_n$};
		\fill[col1shade2] circle[radius=2pt];
		\draw[col1, thick]
			(-1,1) 
			.. controls (1.5,-0.8) and (-1.5,-0.8) .. 
			(1,1);
		\node[generic point=5pt, fill=col1] at (1,1) {};
		\node[right] at (1,1) {$\p$};
	}};
\end{tikzcd}\end{center}
Betrachte dazu
\[\everymath{\displaystyle} \begin{tikzcd}[row sep=tiny, outer sep=5pt]
	\Spec k \rar & k\ldbrack X_1,\ldots,X_n \rdbrack \big/ \p \rar & 
	k[X_1,\ldots,X_N]_{(X_1,\ldots,X_n)} \big/ \p \rar &
	\Spec k[X_1,\ldots,X_n]\big/\p \quad \approx\quad V(\p)
	\end{tikzcd}
\]
Nehmen wir das explizite Beispiel $\p = (Y^2 - X^2(X+1))$. Es ist $\p$ ein
Primideal und $V(\p)$ irreduzibel.

Beachte: $1+X \in k\ldbrack X \rdbrack$ hat eine Wurzel, wie man durch
folgenden Ansatz mit $h(X) = a_0 + a_1 X + \ldots$ sieht:
\[
	1+ X = (h(X))^2 = a_0^2 + 2a_0a_1 X + \ldots
\]
Setze $a_0 := 1$ oder $-1$ und löse sukzessizve auf. Demnach ist
$Y^2 - X^2(X+1) = (Y-Xh(X))(Y + X h(X))$ nicht mehr prim, also
$V(\p) \subseteq k\ldbrack X,Y \rdbrack$ nicht mehr irreduziebel!

Betrachte genauer
\[\begin{tikzcd}[row sep=tiny]
	k \ldbrack u,v\rdbrack \big/(uv) \rar{\cong} & 
	k \ldbrack z,w\rdbrack \big/(z^2-w^2) \rar{\cong} &
	k \ldbrack X,Y\rdbrack \big/(Y^2 - X^2(h(X))^2)\\
	u \rar[mapsto] & z+w & z \rar[mapsto] & Y\\
	v \rar[mapsto] & z-w & wz \rar[mapsto] & Xh(X)\\
\end{tikzcd}\]
In Bildern:
\[\begin{tikzcd}[row sep=-15pt]
	\Spec k\ldbrack u,v\rdbrack \big/(uv) \rar & \Spec k\ldbrack X,Y\rdbrack
		\big/ (Y^2 - X^2(X+1)) \\
	\node{\tikz{
		\draw[->]
			(-1,0) -- (1,0) node[very near end, above] {$X$};
		\draw[->]
			(0,-1) -- (0,1) node[very near end, right] {$Y$};
		\draw[col1, opacity=0.4, line width=3pt]
			(-.8,0) -- (.8,0)
			(0,-.8) -- (0,.8);
	}}; \rar &
	\node{\tikz{
		\draw[->]
			(-1,0) -- (1,0) node[very near end, above] {$X$};
		\draw[->]
			(0,-1) -- (0,1) node[very near end, right] {$Y$};
		\draw[dashed]
			(1,1) 
			.. controls (-1.4,-2) and (-1.4,2) .. 
			(1,-1);
		\clip (-0.3,-0.3) rectangle (0.3,0.3);
		\draw[line width=3pt, col1, opacity=0.4]
			(1,1) 
			.. controls (-1.4,-2) and (-1.4,2) .. 
			(1,-1);
	}};
\end{tikzcd}\]

\subsection{Spezielles Beispiel $\A_\Z^1 = \Spec \Z[X]$}
Wir haben $\pi: \A_\Z^1 \to \Spec \Z$. Topologisch ist
\[
	\A_\Z^1 = \bigcup_{p \text{ prim}} \pi\inv((p)) \cup \pi\inv((0)).
\]
\autoref{fig:A 1 Z to Spec Z} verdeutlicht dies.

\begin{figure}
	\caption{Veranschaulichung von $\A_\Z^1 \to \Spec\Z$}
	\label{fig:A 1 Z to Spec Z}
	\centering
	\begin{tikzpicture}
		\draw[fill=col1shade1, draw=col1]
			(-0.5,0) rectangle (6,4);
		\node[right, text=col1shade2] at (6,2) {$\A_\Z^1$};
		
		\draw[col1, thick]
			(0,0) -- (0,4)
			node[near end, below, sloped] {$\pi\inv((0))$};
		\draw[col1, thick]
			(2,0) -- (2,4)
			node[near end, below, sloped] {$\pi\inv((2))$};
		\draw[col1, thick]
			(3,0) -- (3,4)
			node[near end, below, sloped] {$\pi\inv((p))$};
		
		\draw[->,thick] 
			(3,-0.5) -- (3,-1.5);
		
		\draw[col1shade2, thick]
			(-0.5,-2) -- (6,-2)
			node[right, text=col1shade2]{$\Spec \Z$};
		\draw[col1, thick]
			(0,-1.8) -- +(0,-0.4)
			node[below] {$(0)$};
		\draw[col1, thick]
			(2,-1.8) -- +(0,-0.4)
			node[below] {$(2)$};
		\draw[col1, thick]
			(3,-1.8) -- +(0,-0.4)
			node[below] {$(p)$};
	\end{tikzpicture}
\end{figure}

\paragraph{Zu $\pi\inv((0))$}
Betrachte nun $\p \in \Spec \Z[X]$, so gilt
$\p \in \pi\inv((0))$ $\Leftrightarrow$ $\p \cap \Z = (0)$.

Betrachte $S:= \Z \setminus \{0\} \subseteq \Z[X]$ und die Lokalisierung
$g: \Z[X] \hookrightarrow \Z[X]_S$. Es ist klar: $\Z[X]_S = \Q[X]$

Ferner gilt $\Spec \Q[X] \to \Spec \Z[X]$ ist ein Homöomorphismus auf sein 
Bild:
\[
	\{\p \in \Spec\Z[X] \mid \p \cap S = \emptyset \} = 
	\{\p \in \A_\Z^1 \mid \p \cap \Z = (0) \} = \pi\inv(0),
\]

\paragraph{Zu $\pi\inv((p))$}
Es ist $\p \in \pi\inv((p))$ $\Leftrightarrow$ $p \in \p$.
Dann betrachte
$\rho: \Z[X] \twoheadrightarrow \bb F_p[X]$ und
$\rho^\ast: \Spec \bb F_p[X] \to \A_\Z^1$.
Wegen $\bb F_p[X] \cong \Z[X] \big/ \ker\rho$ ist $\rho^\ast$ ein Homöomorphismus
auf 
\[
	V(\ker \rho) = \{\p \in \Spec\Z[X] \mid \ker \rho \subseteq \p\} = 
	\pi\inv((p)) \subseteq \A_\Z^1.
\] 

Zusammengefasst ist:
\begin{align*}
	\pi\inv((0)) &= \A_\Q^1\\
	\pi\inv((p)) &= \A_{\bb F_p}^1,
\end{align*}
wobei die Gleichheiten topologisch zu lesen sind.

\paragraph{Betrachte $\p\in \Spec\Z[X]$}
\begin{description}
\item[1. Fall.]
	$\p\in \pi\inv((0))\ \Leftrightarrow\ \p\cap \Z = (0)$, also
	\[
		\p = (\mu(X))
	\]
	mit $\mu(X) \in \Z[X]$ einem primitiven, irreduziblen Polynom.
\item[2. Fall.]
	$\p\in\pi\inv((p))$, so ist $\p = \rho\inv(\q)$ für ein 
	$\q\in \Spec\bb F_p[X]$, also
	$\p = \rho\inv((q(X)))$ für ein irreduzibles $q(X)\in \bb F_p[X]$
	oder $(0)$. Dann ist
	\[
		\p = (r(X), p)
	\]
	mit $r(X) \in \Z[X]$ und $r(X) \equiv q(X) \bmod p$.
\end{description}
Es stellt sich die Frage, wie für $f\in \Z[X]$ die $D(f) \subseteq \A_\Z^1$
aussehen. Dazu
\begin{description}
\item[1. Fall $\p\in \pi\inv((0))$.] Sei $f(X) \in \Q[X]$. Dann
	$f(X) = \xi q_1(X)^{\nu_1} \ldots q_r(X)^{\nu_r}$ und es gilt
	\[
		f\notin \p \ \Leftrightarrow\ \p = (q(X))
	\]
	mit $q \neq q_1, \ldots, q_r$.
\item[2. Fall $\p \in \pi\inv((p))$.] $f(X) \notin (r(X), p)$
	mit $r(X) \mod p \in \bb F_p[X]$ irreduzibel. Für eine Primzahl $p$,
	betrachte $\bar f(X) \in \bb F_p[X]$.
	Ist $\bar f(X) = 0$, so ist $f(X) \in (r(X),p)$ für alle $r(X)$.
	Für $\bar f(X) = \bar q_1(X)^{\nu_1} \ldots \bar q_s(X)^{\nu_s}$, ist
	$f(X) \in (q_i(X), p)$ für diese $i$.
\end{description}
Dargestellt ist dies wieder in \autoref{fig:A 1 Z to Spec Z 2}.

\begin{figure}
	\caption{Veranschaulichung von $D(f) \subseteq \A_\Z^1$}
	\label{fig:A 1 Z to Spec Z 2}
	\centering
	\begin{tikzpicture}
		\draw[fill=col1shade1, draw=col1]
			(-0.5,0) rectangle (6,4);
		\node[right, text=col1shade2, font=\scriptsize] at (6,2) {$\A_\Z^1$};
		
		\draw[col1, thick]
			(0,0) -- (0,4)
			(2,0) -- (2,4)
			(3,0) -- (3,4)
			(3.5,0) -- (3.5,4)
			(4,0) -- (4,4)
			(4.5,0) -- (4.5,4);
		
		\draw[->,thick] 
			(3,-0.5) -- (3,-1.5);
		
		\draw[col1shade2, thick]
			(-0.5,-2) -- (6,-2)
			node[right, text=col1shade2]{$\Spec \Z$};
		\draw[col1, thick]
			(0,-1.8) -- +(0,-0.4)
			(3,-1.8) -- +(0,-0.4)
			(3.5,-1.8) -- +(0,-0.4)
			(4,-1.8) -- +(0,-0.4)
			(4.5,-1.8) -- +(0,-0.4);
		\draw[thick, col2]
			(2,-1.8) -- +(0,-0.4);
			
		\foreach \x in {0, 3}{	
			\foreach \y in {0.5,1,...,3}{
				\fill[col2] (\x,\y) circle[radius=2pt];
			}
		}
		\fill[col2] (4,3) circle[radius=2pt];
		\fill[col2] (4.5,2) circle[radius=2pt];
		\draw[col2] (2,0) -- (2,4);
		
		\node[right, col2, text width=3cm, font=\scriptsize] at (3.5,-0.5) 
			{irreduzible Teiler von $\bar f\in \bb F_p[X]$};
			
		\node[below, col2, text width=2cm, font=\scriptsize] at (0,0) 
			{irreduzible Faktoren von $f$};
			
		\node[below, col2, text width=4cm, font=\scriptsize] at (2,-2.5)
			{Primteiler aller Koeffizienten von $f$};
			
		\fill[col2] (7.5,3) circle[radius=3pt] 
			node[right] {\ $\notin D(f)$};
		\fill[col1] (7.5,2) circle[radius=3pt] 
			node[right] {\ $\in D(f)$}; 
	\end{tikzpicture}
\end{figure}
\pagebreak

% vim: set ft=tex :
\section{Projektive Schemata}

\subsection{Eine kurze Einführung in klassische projektive Geometrie}
Sei $k$ ein Körper. So ist
\[
\P^n(k):=\P(k^{n+1}):=\{L\subset k^{n+1} \text{UVR}\mid \dim_k L=1\}
\]
der n-dimensionale projektive Raum.
\paragraph{Homogene Koordinaten} $[x_0:\dots:x_n]\in\P^n(k)$ mit
$0\neq(x_0,\dots,x_n)\in k^{n+1}$ definiert als
\[
[x_0:\dots:x_n]:=\Span_k \begin{pmatrix}x_{0}\\ \vdots\\ x_{n} \end{pmatrix} 
\]
mit $[x_0:\dots:x_n]=[y_0,\dots,y_n]$ $\Leftrightarrow$ $\exists \lambda \in
k^\times$ mit $x_i=\lambda y_i \forall i$.
Damit gilt dann, dass $\P^n(k)=k^{n+1}/\sim$, wobei $\sim$ die gerade eben
definierte Äquivalenzrelation bezeichnet.
\paragraph{Überdeckung}
$\P^n(k)=\bigcup_{i=0}^nU_i$ mit
\begin{center}
\begin{tikzpicture}
\matrix (m) [ matrix of math nodes , row sep=3em ] {
  U_i & = \{[x_0:\dots:x_n]\in\P^n(k)\mid x_i\neq 0\} & \ni &
    {[}x_0:\dots:x_0{]}\\
  k^n   & & \ni & \Big( \frac{x_0}{x_i},\dots,\frac{x_{i-1}}{x_i},
  \frac{x_{i+1}}{x_i},\dots,\frac{x_n}{x_i} \Big) \\
};
\path[->,font=\scriptsize,>=angle 90]
(m-1-1) edge node[left]{$h_i$} node[right]{$b_{ij}$} (m-2-1) ;
\path[>=stealth,|->] (m-1-4) edge (m-2-4) ;
\end{tikzpicture}
\end{center}
als "'Karten"'.
\paragraph{Beachte}
\begin{tikzcd}
\P^n(k)\backslash U_i=\{[x_0:\dots:0:\dots:x_n]\mid (x_0,\dots,\not
i,\dots,x_n)\neq0\}  \arrow{r}{1-1} & \P^{n-1}(k)
\end{tikzcd}
\begin{bemerkung}
\begin{itemize}
    \item $\R\P^n:=\P^n(\R)$
    \item $\C\P^n:=\P^n(\C)$
    \item $\C\P^1\approx S^2$
\end{itemize}
\end{bemerkung}

\subsection{$\P^n(k)$ als Schema}
Statt einem Körper $k$ können wir einen Ring $A$ betrachten.

\subsubsection{1. Variante}
Betrachte $U_i := \Spec A[x_0,\ldots,\cancel i,\ldots,x_n] = \A_A^n$.

In $\R\P^n$ würden wir diese mit dem Kartenwechsel verkleben:
\[\everymath{\displaystyle}\begin{tikzcd}
	& [] [y_0,\ldots,\underset{i\text{-te}}{1},\ldots,y_n] & U_i \cap U_j & \\
	(y_0,\ldots,\cancel i,\ldots,y_n) \urar[mapsto] & h_i(U_i \cap U_j)
		\ar{rr}{\text{Kartenwechsel}} \urar 
		&& h_j(U_i \cap U_j) \ular \\
	&\{(y_0,\ldots,\cancel i,\ldots,y_n) \mid y_j \neq 0\} \ar{rr}
	\uar[empty]{\rotatebox{90}{=}}
	&&
	\{(z_0,\ldots,\cancel j,\ldots,z_n) \mid z_i \neq 0\}
	\uar[empty]{\rotatebox{90}{=}}\\
	& (y_0,\ldots,\cancel i,\ldots,y_n) \ar[mapsto]{rr} &&
		\left(\frac{y_0}{y_j},\ldots,\underset{i\text{-te}}{\frac{1}{y_j}},
		\ldots,\cancel j,\ldots,\frac{y_n}{y_j}\right)
\end{tikzcd}\]
Betrachte also
\begin{align*}
	U_{ij} := \Spec A[x_0,\ldots,\cancel i,\ldots,x_n][x_j\inv]
		&\hookrightarrow \Spec A[x_0,\ldots,\cancel i,\ldots,x_n] = U_i\\
	U_{ji} := \Spec A[x_0,\ldots,\cancel j,\ldots,x_n][x_i\inv]
		&\hookrightarrow \Spec A[x_0,\ldots,\cancel j,\ldots,x_n] = U_j\\
\end{align*}
und wähle einen Isomorphismus
\[
	\phi_{ij}: \funcdef{U_{ij} & \to & U_{ji}\\
		x_k &\mapsto& \frac{x_k}{x_j} \quad \text{für $k\neq i$}\\
		x_i & \mapsto& \frac{1}{x_j}.}
\]
Es gilt nun
$\phi_{ij}(U_{ij} \cap U_{ik}) = U_{ji} \cap U_{jk}$, denn
\begin{align*}
	U_{ij} \cap U_{ik} &= D(x_j x_k) \subseteq U_i\\
	U_{ji} \cap U_{jk} &= D(x_i x_k) \subseteq U_j
\end{align*}
sowie
\[
	\phi_{ik} \rest{U_{ij} \cap U_{ik}} = 
	\phi_{jk} \circ \phi_{ij} \rest{U_{ij} \cap U_{ik}}
\]

\paragraph{Wir haben also}
eine Familie $(U_i)_{i=0,\ldots,n}$ von (affinen) Schemata. Für jedes Paar
$(i,j)$ eine offene Imersion $U_{ij} \hookrightarrow U_i$ mit
(affinen) Schemata
und Isomorphismen
$\phi_{ij}: U_{ij} \xto{\cong} U_{ji}$, so dass
$\phi_{ik} \rest{U_{ij} \cap U_{ik}} = 
	\phi_{jk} \circ \phi_{ij} \rest{U_{ij} \cap U_{ik}}$.

Bleibt zur Übung lediglich zu zeigen, dass ein (bist auf Isomorphie) 
eindeutiges Schema
$\P_A^n$ mit Überdeckung $\P_A^n = \bigcup_{i=0}^n V_i$ für 
$V_i \subseteq \P_A^n$ offen und Isomorphismen
$V_i \xto\cong U_i$ von (affinen) Schemata existiert.


\subsubsection{2. Variante (Die $\Proj$-Konstruktion)}

\begin{definition}[graduierte $A$-Algebra]
    \index[def]{Graduierte Algebra}
	Sei $A$ ein Ring, dann heißt
	\[ S:= \oplus_{n\in\N_0} S_n\]
	eine \emph{graduierte $A$-Algebra}, wenn
	\begin{itemize}
	  \item $S$ ein Ring,
	  \item $S_n \subset S$ ein $\Z$-Untermodul,
	  \item $S_n S_m \subseteq S_{n+m}$ ist,
	  \item wir einen Ringhomomorphismus $A \xto \varphi S$ haben und
	  \item die $S_n$ $A$-Untermoduln sind.
	\end{itemize}
	
	Ein $s \in S_n$ heißt \emph{homogen vom Grad $n$}.
\end{definition}

\begin{definition}[homogenes Ideal]
    \index[def]{Homogenes Ideal}
	\label{def:homogenes ideal}
	Ein Ideal $\a \ideal S$ heißt \emph{homogen}, wenn
	\[
		\a = \oplus_{n\in\N_0} \a \cap S_n.
	\]
\end{definition}

\begin{lemma}
	\label{lemma:ideal homogen <=> von homogenen elementen erzeugt}
	Es ist äquivalent
	\begin{itemize}
		\item $\a$ homogen,
		\item $\a$ wird von homogenen Elementen erzeugt
		\item Aus $a \in \a$ mit $a = \sum_{n\in \N_0} a_n$ für
			$a_n\in S_n$ folgt $a_n \in \a$.
	\end{itemize}
\end{lemma}
\begin{proof}
	leicht.
\end{proof}


\begin{beispiel}
	$S = A[x_0,\ldots,x_n] = \oplus_{m\geq 0} S_m$ mit
	\[
		S_m = \{f(x_0,\ldots,x_n) \mid f\text{ homogen von Grad $m$}\},
	\]
	d.h. 
	\[
		f\in S_m \quad\Leftrightarrow\quad
			f = \sum_{\nu \in \N_0^{n+1}} \alpha_\nu X_0^{\nu_0} 
				\ldots X_n^{\nu_n} \quad\text{mit }
				\nu_0 + \ldots+\nu_n = m.
	\]
\end{beispiel}


\begin{definition}[$\Proj(S)$]
	Setze $S_+ := \oplus_{n\geq 1} S_n$, dann ist das
	\emph{projektive Spektrum $\Proj S$ von $S$} definiert als
	\[
		\Proj(S) := \{ \p \in \Spec S\text{ homogen} \mid
			S_+ \subsetneq \p\}.
	\]
\end{definition}

\begin{definition}[Zariski Topologie auf $\Proj(S)$]
    \index[def]{Zariski Topologie!auf $\Proj$}
	Für ein homogenes Ideal $\a \ideal S$ setze
	\[
		V_+(\a) := \{ \p \in \Proj(S)\mid \a\subseteq \p\} \subseteq 
			\Proj(S).
	\]
	Dann bilden diese $V_+(\a)$ die abgeschlossenen Mengen einer Topologie,
	der \emph{Zariski-Topologie auf $\Proj(S)$}.
\end{definition}
\begin{proof}
	Wie im inhomogenen Fall.
\end{proof}

\begin{bemerkung}
	Ein homogenes $\a \ideal S$, $\a\neq S$, ist prim genau dann, wenn
	gilt:
	\[ xy \in \a \quad \Rightarrow\quad x\in\a \text{ oder } y\in\a\]
	für alle homogenen $x,y$.
\end{bemerkung}

\begin{definition}[basisoffenen Mengen auf $\Proj(S)$]
    \index[def]{Basisoffene Menge!auf $\Proj$}
	Analog zu $\Spec A$ bilden für $f\in S$ 
	die \emph{basisoffenen Mengen in $\Proj(S)$}
	\[
		D_+(f) := \{ \p \in \Proj(S) \mid f\notin \p\}\subseteq \Proj(S)
	\]
	eine Basis der Topologie auf $\Proj(S)$.
\end{definition}

\begin{definition}[homogene Lokalisierung]
    \index[def]{Lokalisierung!homogene}
	\begin{itemize}
	  \item Für $\p\in \Proj(S)$ heißt
		  \[
		  	S_{(\p)} := \left\{ \frac s t \mid s,t \in S,\ t\notin \p,\ 
		  		s,t \text{ homogen von gleichem Grad}\right\}
		  \]
		  \emph{homogene Lokalisierung von $\p$}.
	  \item Für $f \in S $ homogen von Grad $m$ heißt
	  	\[ 
	  		S_{(f)} := \left\{ \frac{s}{f^k} \mid s\in S,\ k\in \N_0,\ 
	  			s\text{ homogen von Grad } k\deg f\right\}
	  	\]
	  	\emph{homogene Lokalisierung bezüglich $f$}.	  	
	\end{itemize}
\end{definition}

\begin{lemma}
	Es gilt:
	$S_{(\p)}$ ist ein lokaler Ring mit maximalem Ideal 
	\[
		\p_{(\p)} := \left\{\frac s t \mid s\in \p\right\}.
	\]
\end{lemma}
\begin{proof}
    \TODO
\end{proof}

\begin{satz}
    \label{satz:proj s eindeutige ringgarbe}
	Auf $\Proj(S)$ gibt es eine (bis auf Isomorphie) eindeutige Ringgarbe
	$\O_{\Proj(S)}$ mit:
	\begin{enumerate}
	  \item Für alle homogenen $f\in S_+$ hat man den Isomorphismus
	  	\[
	  		(\varphi, \varphi\fis): 
	  			\left(D_+(f), \O_{\Proj(S)}\rest{D_+(f)}\right)
	  			\to 
	  			\Spec(S_{(f)}, \O_{S_{(f)}})
	  	\]
	  \item
	  	Diese induzieren Isomorphismen
	  	\[
	  		\O_{\Proj(S), \p} \xto\cong S_{(\p)}.
	  	\]
	\end{enumerate}
	Damit wird $(\Proj(S), \O_{\Proj(S)})$ zu einem Schema.
\end{satz}
\begin{proof}
"`analog"' zum Beweis für $\Spec$ mit nachfolgendem Lemma.
\end{proof}

\begin{lemma}
	Ist $f\in S_+$ homogen, so ist
	\[
		\phi: \funcdef{D_+(f) & \to & \Spec(S_{(f)}) \\
			\p & \mapsto & \p S_f \cap S_{(f)}}
	\]
	ein Homöomorphismus.
\end{lemma}
\begin{proof}
	\newcommand{\Sf}{S_{(f)}}
	Sei $S \xto \lambda S_f \xhookleftarrow\iota \Sf$, so haben wir
	\[\begin{tikzcd}
		\Spec S & \Spec S_f \lar{\lambda^\ast} \ar{dd}{\iota^\ast}
		%\arrow[shift left=2pt]{dl}{(\lambda^\ast)\inv}
		\\
		D(f) \arrow{ur}{\lambda^\ast}[swap]{\approx}& \\
		D_+(f) \uar[hook]{\text{stetig}} \rar{\phi} & \Spec(\Sf)
	\end{tikzcd}\qquad
	\begin{tikzcd}
		& \p S_f \ar{dd}\\
		\p \urar[mapsto] & \\
		\p \uar[mapsto] \rar[mapsto] 
		& \p S_f \cap \Sf
	\end{tikzcd}
	\]
	Die Stetigkeit im linken Diagramm folgt aus der Tatsache, dass
	$V_+(\a) = V(\a) \cap \Proj(S)$ und $\Proj(S)$ trägt die Teilraumtopologie
	von $\Spec S$.
	Damit ist $\phi$ stetig.
	
	Wir wollen die Umkehrabbildung von $\phi$ angeben:
	\[
		\funcdef{D_+(f) & \xto\phi & \Spec(\Sf) \\
			\lambda\inv(\sqrt{\q S_f}) & \mapsfrom & \q.}
	\]
	Den Rest zeigen nachstehende Hilfslemmata. 
\end{proof}

\begin{hilfslemma}
	$\p := \lambda\inv(\sqrt{\q S_f})$ ist homogenes Primideal in $S$.
\end{hilfslemma}
\begin{proof}
	\[
		\q S_f = \left\{ \frac{b}{f^l} \frac{c}{f^n} \in S_f \left|
			\begin{array}{l}
				b\text{ homogen, } \deg b = l\deg f\\
				\frac{b}{f^n} \in \q,\ c\in S,n\in \N_0
			\end{array}\right. \right\}
	\]
	Bemerke, dass $\p$ ein homogenes Ideal ist, weil $\q S_f$ es ist.
	Genauer:
	$S_f = \oplus_{n\geq 0} S_{f,n}$ mit
	\[
		S_{f,n} := \left\{\frac{c}{f^m} \mid c\text{ homogen, }
			\deg c - m\deg f = n\right\}.
	\]
	Es bleibt also zu zeigen: Sind $a,a' \in S$ homogen und 
	$aa' \in \p$, so folgt $a\in \p$ oder $a'\in \p$.
	
	Sei dazu  $r = \deg a$, $s = \deg a'$.
	Aus $aa'\in \p$ folgt $\lambda(aa') = \frac{aa'}{1} \in \sqrt{\q S_f}$.
	Also existiert ein $k\in \N$ mit
	$\left(\frac{aa'}{1}\right)^k \in \q S_f$, also
	$\left( \frac{aa'}{1}\right)^k = \frac{b}{f^l} \frac{c}{f^n}$
	wie oben. Potenzieren mit $\deg f$ ergibt
	\[
		\frac{a^{k\deg f} a'^{k\deg f}}{f^{kr} f^{ks}} = 
		\frac{b^{\deg g}}{f^{l\deg f}} \frac{c^{\deg f}}{f^{n\deg f}}
		\frac{1}{f^{kr} f^{ks}} \in S_f.
	\]
	\TODO
\end{proof}

wir definieren $\P_A^n := \Proj(A[X_0,\ldots,X_n])$ als Schema. Dabei stellen
sich aber die Fragen, was dabei $D_+(X_i)$ sein soll und ob die beiden 
Varianten übereinstimmen.

\begin{lemma}
    Die beiden Varianten der Definition von $\P_A^n$ stimmen überein
    und es gilt
    \[ D_+(X_i) \cong \Spec S_{(X_i)} \cong \A_A^n.\]
\end{lemma}
\begin{proof}
    \TODO
\end{proof}

\subsection{Immersionen und projektive $A$-Schemata}

\begin{definition}[offene und abgeschlossene Immersion]
    \index[def]{offene Immersion}
    \index[def]{abgeschlossene Immersion}
    Ein Morphismus $f: Y \to X$ von Schemata heißt
    \begin{enumerate}
      \item \emph{offene Immersion}, wenn es $U\osubset X$ gibt, so dass
        \[ f: (Y,\O_Y) \xto\cong (U,\O_X\rest U) 
            \xhookrightarrow{(\iota,\iota\fis)} (X,\O_X)\]
      \item \emph{abgeschlossene Immerson}, wenn gilt:
      \begin{itemize}
        \item $f$ ist topologisch ein Homöomorphismus auf $\im f:= Z\subset X$
            abgeschlossen,
        \item $f\fis: \O_X \to f_\ast \O_Y$ ist ein surjektiver 
            Garbenmorphismus, d.h. für alle $y \in Y$ ist
            \[f_{(f(y))}\fis: \O_{X,f(y)} \to \O_{Y,y}\]
            surjektiv.
      \end{itemize}
        Wir schreiben dann auch $Y \immersion X \rightarrow Y$. 
    \end{enumerate}
\end{definition}


\begin{beispiel}
    Ist $A$ ein Ring, $a\ideal A$, so induziert
    \[A\xto{\pi} A\big/\a\]
    eine abgeschlossene Immersion
    \[ f:\Spec A\big/\a \to \Spec A\] 
\end{beispiel}
\begin{proof}
    \TODO
\end{proof}

\begin{bemerkung}
    Es ist $V(\a) = V(\sqrt\a) = V(\b)$ genau dann, wenn $\sqrt a = \sqrt b$.
    Aber es folgt nicht notwendigerweise $A\big/\a \overset?\cong A\big/\b$!
    
    Dazu betrachte einen Ring $A$ mit nilpotenten Elementen, d.h.
    $\Nil A := \sqrt{(0)} \neq (0)$ und
    \[f:\Spec A\big/\Nil(A) \hookrightarrow \Spec A\]
    ist eine abgeschlossene Immersion mit
    \[\im f = V(\Nil(A)) = \{\p\in \Spec A \mid \Nil(A) \subseteq \p\}
        = \Spec A.\]
    Jedoch ist dies \emph{kein} Isomorphismus.  
\end{bemerkung}


\begin{definition}[abgeschlossenes Unterschema]
    \index[def]{Abgeschlossenes Unterschema}
    Ist $f:Y\to X$ eine abgeschlossene Immersion, so nennen wir $Y$ 
    ein \emph{(bzgl. $f$) abgeschlossenes Unterschema von $X$}.
\end{definition}

\begin{definition}[projektives Schema über $A$]
    \index[def]{Projektives Schema über $A$}
    Sei $A$ ein Ring. Ein \emph{projektives Schema über $A$} ist ein
    $A$-Schema $X$ mit einer abgeschlossenen Immersion, so dass
    \[\begin{tikzcd}
        \iota: \ X \ar[immersion]{rr} \drar && \P_A^n \dlar\\
        & \Spec A & 
    \end{tikzcd}\]
    für ein $n \in\N_0$ kommutiert.
\end{definition}

\begin{bemerkung}
    \TODO
\end{bemerkung}

\subsubsection{Beispiele}
Zunächst ein etwas abstrakteres Beispiel.
\begin{satz}
    Sei $S := A[X_0,\ldots, X_n]$. Ist $\b\ideal S$ ein homogenes Ideal, so ist
    $B := S \big/\b$ in natürlicher Weise eine graduierte $A$-Algebra
    und $\Proj(B)$ ein projektives $A$-Schema.
\end{satz}
\begin{proof}
    \TODO
\end{proof}

Und nun einige konkrete!

\newcommand{\Pnklass}{\P_{\text{klass}}^n}
\newcommand{\Pn}{\P^n}
\paragraph{1. $\Pnklass(k)$ und $\Pn_k$.} Sei $k$ ein Körper.
Wir haben $\Pnklass(k) := k^{n+1}\setminus\{0\} \big/ \sim$ und
dagegen $\Pn_k := \Proj k[T_0,\ldots, T_n]$.

Eine algebraische Menge in $\Pnklass(k)$ ist per definitionem
\[Z:= \{[x_0:\ldots:x_n] \in \Pnklass(k) \mid f_i(x_0,\ldots,x_n) = 0\}\]
für $f_1(T_0,\ldots,T_n),\ldots,f_r(T_0,\ldots,T_n) \in k[T_0,\ldots,T_n]$ 
homogen.

\begin{satz}
    Die Abbildung
    \[ \rho: \funcdef{ \Pnklass(k) & \to & \Pn_k \\ {}
        [x_0:\ldots:x_n] & \mapsto & 
        \langle x_i T_j - x_j T_i \mid i,j \rangle
        }
     \]
    ist eine Bijektion auf 
    \[ \Pn_k(k) = \{\p \in \Pn_k \mid \p \text{ ist $k$-rational}\} = 
        \Hom_{\Sch_k}(\Spec k, \Pn_k).\]
\end{satz} 
\begin{proof}
    \TODO
\end{proof}


\begin{bemerkung}
    Wir haben dies auch schon affin gesehen:
    \[ \funcdef{ k^n = \A_{\text{klass}}^n(k) & \to & 
        \A_k^n = \Spec k[X_1,\ldots,X_n]\\
        (\alpha_1,\ldots,\alpha_n) & \mapsto & 
        (X_1 - \alpha_1, \ldots, X_n - \alpha_n)}.\]  
\end{bemerkung}


\begin{bemerkung}
    Sei $X$ ein Schema. Wir erinnern daran, dass
    \[X(K) := \Hom_{\Sch}(\Spec k, X) = 
        \{ (\varphi,\varphi\fis): \Spec k \to X\}\]
    mit 
    \[\varphi_\eta: \O_{X,x} \to \O_{\Spec k, \eta} = k\]
    mit $x = \varphi(\eta)$, wobei topologisch $\Spec k = \{\eta\}$.
    Damit haben wir
    \[ \overline{\varphi_\eta^n}: \O_{X,x}\big/\m_x = k(x) 
        \tikzmark{\hookrightarrow}
        k \]
    \tikzmargin{south}{\color{red} 
        Wir folgern eine Seite später dass $k(x) \cong k$ 
        kanonisch. Das ist mir nicht klar :-(}
    (Körperhomomorphismen sind immer injektiv) und wir
    erhalten folgende 1-1 Beziehung:
    \[ X(k) \overset{\text{1-1}}{=} \{x\in X \text{ zusammen mit Inklusionen }
        \iota: k(x) \hookrightarrow k\}. \]
        
    Beachte dabei:
    \begin{align*}
        X \in \Obj(\Sch) \quad&\leadsto\quad X(k) := \Hom_{\Sch}(\Spec k, X)\\
        Y \in \Obj(\Sch\rest k) \quad&\leadsto\quad 
            Y(k) := \Hom_{\Sch\rest k}(\Spec k, X) = 
            \left\{\begin{tikzcd}[ampersand replacement=\&]
                \varphi: \Spec k \ar{rr} \drar{\id} \&\& Y \dlar \\
                \& \Spec k
            \end{tikzcd}\right\}\\
    \end{align*} 
    In diesem Sinne ist $\P_k^n$ als $k$-Schema zu lesen mit 
    $\P^n_k \to \Spec k$. 
\end{bemerkung}

\paragraph{2. Projektiver Abschluss}
Sei $\a \ideal k[Y_1,\ldots, Y_n]$, so hat man die abgeschlossene Immersion
\[ \Spec k[Y_1,\ldots,Y_n]\big/\a \immersion \A_k^n\]
mit Bild $V(\a)$.

Betrachte die Homogenisierung von $\a$ in $k[T_0,\ldots,T_n]$:
Sei $\a = (f_1,\ldots,f_1)$. Definiere
\[ f_i^\text{homo}(T_0,\ldots,T_n) := T_0^{\deg f_i} f_i(\tfrac{T_1}{T_0},
    \ldots, \tfrac{T_n}{T_0}) \in k[T_0,\ldots,T_n].\]
Damit können wir nun folgenden Satz formulieren.

\begin{satz}
    Ist $\iota: X\immersion \A_k^n$ eine abgeschlossene Immersion, 
    $X = \Spec k[Y_1,\ldots,Y_n]\big/\a$ und $\a = (f_1,\ldots,f_r)$,
    so nennen wir
    \[ \bar X := \Proj k[T_0,\ldots,T_n] \big/ 
        \a^\text{homo} \immersion \P^n_k\]
    mit $\a^\text{homo} := (f_1^\text{homo}, \ldots, f_r^\text{homo})$
    den \emph{projektiven Abschluss von $X$ in $\P^n_k$}. Es gilt 
    \[ \begin{tikzcd}
        \p^\text{homo} \symb{\ni} & 
            D_+(T_0) \cap \bar X  \rar[offene immersion]{\tikzmark[1]{}} & \bar X 
            \rar[immersion] & \P^n_k \\
        \p \uar[mapsto] \symb{\ni} & 
            X \uar{\cong}  \ar[immersion]{rr} &&
            \Spec k[Y_1,\ldots,Y_n] = \A^n_k 
            \tikzmark[2]{\ \cong\ } D_+(T_0)  \uar[hook]
    \end{tikzcd}\]
    wobei die Isomorphie an \tikzarrow[2]{mark above}{dieser} Stelle
    durch die Definition der homogenen Polynome herrührt.
\end{satz}
\tikzmargin[1]{north}{\color{red} Bei mir steht "`offene Inklusion"',
    soll wohl aber offene Immersion gemeint sein !?}
\begin{proof}
    klar.
\end{proof}


\begin{beispiel}
    Sei $E = \Spec k[X,Y] \big/(Y^2-X^3-aX-b) \subseteq \A_k^2$, so ist
    \[
        \bar E = \Proj k[X,Y,Z]\big/ (Y^2Z - X^3 - aXZ^2 -bZ^3) 
            \subseteq \P_k^2.
    \]
    Als Übung überlege man sich was $\bar E \cap (\P_k^2 \setminus D_+(T_0))$
    ist.
\end{beispiel}



\pagebreak
% vim: set ft=tex :

\section{Eigenschaften von Schemata} %Seite 79
% vim: set ft=tex :

\section{Tensorprodukt} %Seite 105
% vim: set ft=tex :

\section{Glatt, regulär & normal} %Seite 125
% vim: set ft=tex :

\section{k-Variet�t} %Seite 154
% vim: set ft=tex :

\section{Der Punktefunktor} %Seite 159

Ein wenig Kategorientheorie:
\begin{definition}[treu, volltreu]
    \label{def:treu}
    \index[def]{Funktor!treu}
    \index[def]{Funktor!volltreu}
    Ein Funktor $F:\cal C \to \cal D$ heißt
    \emph{treu}, falls für alle $X,Y \in \Obj(\cal C)$
    \[F : \Hom_{\cal C}(X,Y) \to \Hom_{\cal D}(F(X),F(Y))\]
    injektiv ist.
    
    Er heißt \emph{volltreu}, falls für alle $X,Y \in \Obj(\cal C)$
    \[F :\Hom_{\cal C}(X,Y) \to \Hom_{\cal D}(F(X),F(Y))\]
    eine Bijektion ist.
\end{definition}


\begin{notation}
    Sind $\cal A$, $\cal B$ Kategorien, so definieren wir
    \[\cal B^{\cal A} := 
        \begin{cases}
            \Obj: \text{ Funktoren }F:\cal A \to \cal B\\
            \Morph: \Hom_{\cal B^{\cal A}} = 
                \text{ natürliche Transformationen}
        \end{cases}\]
\end{notation}


\begin{definition}[Punktefunktor, darstellbar]
    \label{def:punktefunktor}
    \index[def]{Punktefunktor}
    \index[def]{Funktor!darstellbar}
    Zu $X \in \Obj \cC$ heißt
    \[h_X: \funcdef{ \cC\op &\to& \Set \\
        T & \mapsto & \Hom_\cC(T,X)}\]
    der \emph{Punktefunktor zu $X$}.
    Es ist $h_X \in \Obj(\Set^{\cC\op})$.
    
    Ein Funktor $F \in \Obj(\Set^{\cC\op})$ heißt \emph{darstellbar}, 
    wenn es ein $X \in \Obj(\cC)$ gibt, so dass
    \[F \cong h_X\]
    in $\Set^{\cC\op}$ gilt.
\end{definition}


\begin{lemma}[Yoneda Lemma]
    \begin{enumerate}[label=(\roman*)]
      \item Ist $F: \cC\op \to \Set$ beliebig, so ist für $X\in \Obj\cC$:
          \[ \funcdef{ \Hom_{\Set^{\cC\op}}(h_X,F) &\to& F(X) \\
            \tau &\mapsto & \tau_X(\id_X) }\]
          eine Bijektion, wobei 
          \[\tau_X: 
          h_X(X) = \Hom(X,X) \to F(X).\]
      \item Es ist
        \[ h: \funcdef{ \C & \to & \Set^{\cC\op} \\
            X &\mapsto & h_X}\]
        eine Äquivalenz von $\cC$ zur vollen Unterkategorie der darstellbaren
        Funktoren, d.h. $h$ ist volltreu und jeder darstellbare Funktor
        ist insomorph zu einem $h_X$. Insbesondere gilt:
        \[h_X \cong h_{\widetilde X} \text{ als Funktoren}
            \quad\Rightarrow\quad
            X \cong \widetilde X \text{ in }\cC\]
    \end{enumerate}
\end{lemma}
\begin{proof}
\TODO
\end{proof}

\begin{definition}[Gruppenschema]
    \label{def:Gruppenschema}
    \index[def]{Schema!Gruppenschema}
    Ein \emph{Gruppenschema} ist ein Schema $G$, so dass
    \[ \begin{tikzcd}
        h_G:  \Sch\op \rar \drar{\exists} 
        &\Set
        \\
        & \Gr \uar
    \end{tikzcd}\]
    über $\Gr$ faktorisiert.
\end{definition}

\begin{bemerkung}
    Das bedeutet: Für jedes $T\in \Sch$ ist $G(T) = \Hom(T,G)$ ein Gruppe.
\end{bemerkung}

\begin{beispiel}
    Sei $k$ ein algebraisch abgeschlossener Körper, $X \in \Sch|_k$ so 
    hat man den Funktor
    \[ \Hilb_{X|k,n}: \funcdef{ \Sch|_k\op &\to& \Set \\
        T &\mapsto& \left\{ Z \immersion X \times_{\Spec k} T \left|
            \parbox{5cm}{$Z$ ist abgeschlossenes Unterschema;
            $Z\to T$ flach; jede Faster $Z_t$ für einen
            $k$-rationalen Punkt $t$ ist $0$-dimensional von Länge $n$}
             \right.\right\}}\]
             
    Betrachte nun 
    \[\Hilb_{X|_k,n}(\Spec k) = \{ Z \immersion X \mid 
        Z \text{ abgeschlossenes Unterschema }, \dim Z = 0, \dim_k \O_Z(Z) = n\}
    \]
    so stellt sich die Frage: Gibt es ein Schema $H$ mit
    \[H(\Spec k) = \Hilb_{X|_k,n}(\Spec k).\]
    Die Antwort sei vorweg genommen: Ja, falls $X$ gewisse Voraussetzungen 
    erfüllt.
\end{beispiel}

% vim: set ft=tex :


\end{document}
% vim: set ft=tex :
