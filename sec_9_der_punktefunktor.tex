\section{Der Punktefunktor} %Seite 159

Ein wenig Kategorientheorie:
\begin{definition}[treu, volltreu]
    \label{def:treu}
    \index[def]{Funktor!treu}
    \index[def]{Funktor!volltreu}
    Ein Funktor $F:\cal C \to \cal D$ heißt
    \emph{treu}, falls für alle $X,Y \in \Obj(\cal C)$
    \[F : \Hom_{\cal C}(X,Y) \to \Hom_{\cal D}(F(X),F(Y))\]
    injektiv ist.
    
    Er heißt \emph{volltreu}, falls für alle $X,Y \in \Obj(\cal C)$
    \[F :\Hom_{\cal C}(X,Y) \to \Hom_{\cal D}(F(X),F(Y))\]
    eine Bijektion ist.
\end{definition}


\begin{notation}
    Sind $\cal A$, $\cal B$ Kategorien, so definieren wir
    \[\cal B^{\cal A} := 
        \begin{cases}
            \Obj: \text{ Funktoren }F:\cal A \to \cal B\\
            \Morph: \Hom_{\cal B^{\cal A}} = 
                \text{ natürliche Transformationen}
        \end{cases}\]
\end{notation}


\begin{definition}[Punktefunktor, darstellbar]
    \label{def:punktefunktor}
    \index[def]{Punktefunktor}
    \index[def]{Funktor!darstellbar}
    Zu $X \in \Obj \cC$ heißt
    \[h_X: \funcdef{ \cC\op &\to& \Set \\
        T & \mapsto & \Hom_\cC(T,X)}\]
    der \emph{Punktefunktor zu $X$}.
    Es ist $h_X \in \Obj(\Set^{\cC\op})$.
    
    Ein Funktor $F \in \Obj(\Set^{\cC\op})$ heißt \emph{darstellbar}, 
    wenn es ein $X \in \Obj(\cC)$ gibt, so dass
    \[F \cong h_X\]
    in $\Set^{\cC\op}$ gilt.
\end{definition}


\begin{lemma}[Yoneda Lemma]
    \begin{enumerate}[label=(\roman*)]
      \item Ist $F: \cC\op \to \Set$ beliebig, so ist für $X\in \Obj\cC$:
          \[ \funcdef{ \Hom_{\Set^{\cC\op}}(h_X,F) &\to& F(X) \\
            \tau &\mapsto & \tau_X(\id_X) }\]
          eine Bijektion, wobei 
          \[\tau_X: 
          h_X(X) = \Hom(X,X) \to F(X).\]
      \item Es ist
        \[ h: \funcdef{ \C & \to & \Set^{\cC\op} \\
            X &\mapsto & h_X}\]
        eine Äquivalenz von $\cC$ zur vollen Unterkategorie der darstellbaren
        Funktoren, d.h. $h$ ist volltreu und jeder darstellbare Funktor
        ist insomorph zu einem $h_X$. Insbesondere gilt:
        \[h_X \cong h_{\widetilde X} \text{ als Funktoren}
            \quad\Rightarrow\quad
            X \cong \widetilde X \text{ in }\cC\]
    \end{enumerate}
\end{lemma}
\begin{proof}
\TODO
\end{proof}

\begin{definition}[Gruppenschema]
    \label{def:Gruppenschema}
    \index[def]{Schema!Gruppenschema}
    Ein \emph{Gruppenschema} ist ein Schema $G$, so dass
    \[ \begin{tikzcd}
        h_G:  \Sch\op \rar \drar{\exists} 
        &\Set
        \\
        & \Gr \uar
    \end{tikzcd}\]
    über $\Gr$ faktorisiert.
\end{definition}

\begin{bemerkung}
    Das bedeutet: Für jedes $T\in \Sch$ ist $G(T) = \Hom(T,G)$ ein Gruppe.
\end{bemerkung}

\begin{beispiel}
    Sei $k$ ein algebraisch abgeschlossener Körper, $X \in \Sch|_k$ so 
    hat man den Funktor
    \[ \Hilb_{X|k,n}: \funcdef{ \Sch|_k\op &\to& \Set \\
        T &\mapsto& \left\{ Z \immersion X \times_{\Spec k} T \left|
            \parbox{5cm}{$Z$ ist abgeschlossenes Unterschema;
            $Z\to T$ flach; jede Faster $Z_t$ für einen
            $k$-rationalen Punkt $t$ ist $0$-dimensional von Länge $n$}
             \right.\right\}}\]
             
    Betrachte nun 
    \[\Hilb_{X|_k,n}(\Spec k) = \{ Z \immersion X \mid 
        Z \text{ abgeschlossenes Unterschema }, \dim Z = 0, \dim_k \O_Z(Z) = n\}
    \]
    so stellt sich die Frage: Gibt es ein Schema $H$ mit
    \[H(\Spec k) = \Hilb_{X|_k,n}(\Spec k).\]
    Die Antwort sei vorweg genommen: Ja, falls $X$ gewisse Voraussetzungen 
    erfüllt.
\end{beispiel}

% vim: set ft=tex :
