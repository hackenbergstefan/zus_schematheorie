\section{Affine Schemata}

\subsection{$\Spec A$ als topologischer Raum}

Sei im Folgenden $A$ ein kommuativer Ring mit $1$ und 
$\Spec A := \{\p \ideal A \mid \p \text{ Primideal}\}$.

\begin{definition}[Zariski Topologie]
	Ist $\a \ideal A$, ein Ideal, setze
	\[
		V(\a) := \{\p \in \Spec A \mid \a \subseteq \p \} \subseteq \Spec A\,.
	\]
	Dann ist durch
	\[
		\cal T := \{ U \subseteq \Spec A \mid
			\exists\ \a \ideal A:\ U = \Spec A \setminus V(\a)\}
	\]
	eine Topologie auf $\Spec A$ definiert. Sie heißt \emph{Zariski-Topologie}.
\end{definition}

\begin{proof}[der Topologie-Eigenschaften]
	\begin{enumerate}
	  \item Zeige: $\emptyset$, $\Spec A$ offen $\Longleftrightarrow$ 
	  	$\Spec A$, $\emptyset$ abgeschlossen.\\
	  	Dazu: $V(A) = \emptyset$, $V((0)) = \Spec A$
	  \item Zeige: $U_1, U_2$ offen $\Rightarrow$ $U_1 \cap U_2$ offen
	  	$\Longleftrightarrow$ $M_1,M_2$ abgeschlossen $\Rightarrow$
	  	$M_1 \cup M_2$ abgeschlossen.\\
	  	Dazu:
	  	$V(\a) \cup V(\fr b) = V(\a \cap \fr b)$
	  \item $(U_i)_{i\in I}$ offen $\Rightarrow$ $\cup_{i\in I} U_i$ offen
	  	$\Longleftrightarrow$ $(M_i)_{i\in I}$ abgeschlossen
	  	$\Rightarrow$ $\cap_{i\in I} M_i$ abgeschlossen.\\
	  	Dazu:
	  	$\cap_{i\in I} V(\a_i) = V(\sum_{i\in I} \a_i)$
	\end{enumerate}
\end{proof}

\begin{bemerkung}
	Die abgeschlossenen Teilmengen $M \subset \Spec A$ sind genau die 
	$M = V(\a)$ für ein $\a \ideal A$.
\end{bemerkung}

\begin{beispiel}[$\Spec \Z$]
	Für $\a \ideal \Z$ ist $\a = (a)$. Falls $a \neq 0,1,-1$ sei
	$a = \pm p_1^{\nu_1} \cdot \dots \cdot p_r^{\nu_r}$ die 
	Primfaktorzerlegung. Für $p$ Primzahl ist
	\[
		(p) \in V((a)) \Leftrightarrow
		(a) \subseteq (p) \Leftrightarrow
		p \mid a \Leftrightarrow
		p \in \{p_1,\ldots, p_r\}
	\]
	Das bedeutet, die abgeschlossenen Mengen in $\Spec \Z$ sind genau die 
	Mengen $\emptyset$, $\Spec \Z$ und
	$\{(p_1), \ldots, (p_r)\}$ für eine endliche Anzahl an Primzahlen.
	
	Insbesondere gilt
	\begin{itemize}
	  \item $\Spec\Z$ ist nicht hausdorffsch.
	  \item $(0) =: \eta \in \Spec\Z$ liegt in \emph{jeder} nichtleeren 
	  	offenen Teilmenge.
	\end{itemize}
\end{beispiel}

\begin{lemma}
	Sei $x \in \Spec A$, so ist der Abschluss $\overline{\{x\}}$ der
	Menge $\{x\}$ in $\Spec A$ gleich
	\[\overline{\{x\}} = V(x).\]
\end{lemma}
\begin{proof}
	\[
		\overline{\{x\}} = 
		\bigcap_{B\subseteq \Spec A \text{ abg.}\atop x\in B} B
		= \bigcap_{\a\ideal A\atop \a \subseteq x}
		= V(x)
	\]
\end{proof}

\begin{bemerkung}
	Beachte, dass
	\[
		\textcolor{purple}{\a} \subseteq \textcolor{blue}{\fr b} \quad 
    \Rightarrow\quad
		V(\textcolor{blue}{\fr b}) \subseteq V(\textcolor{purple}{\a})
	\]
\end{bemerkung}

\begin{definition}[abgeschlossener Punkt, generischer Punkt]
	Sei $X$ ein topologischer Raum.
	Ein $x\in X$ heißt \emph{abgeschlossener Punkt}, wenn
	$\overline{\{x\}} = \{x\}$.
	
	Er heißt \emph{generischer Punkt}, wenn $\overline{\{x\}} = X$ gilt.
	
	Die Menge der abgeschlossenen Punkte bezeichnen wir mit
	$|X|$.
\end{definition}

\begin{beispiel}
	Sei $A = \C[X,Y]$. 
	\begin{itemize}
	  \item $x = (0) \in \Spec A$ ist generisch.
	  \item $x = (X-\alpha, Y-\beta) \ideal A$ ist abgeschlossen,
	  	da aus $x \ideal A$ maximal $V(x) = \{x\}$ und somit $x$ abgeschlossen
	  	folgt.
	  \item $x = (X) \ideal A$ ist weder abgeschlossen noch generisch.
    \item $x = (XY-1) \ideal A$ ist ebenfalls weder abgeschlossen noch
      generisch.
	\end{itemize}
  Wir können die bisherigen Ergebnisse in \thref{fig:spec c xy} zusammenfassen. 
\end{beispiel}

\begin{figure}
	\caption{$\Spec \C[X,Y]$}
	\label{fig:spec c xy}
	\centering
	\begin{tikzpicture}
		\fill[col1shade1] (-4,-2) rectangle (4,2);
		\node[right, text=col1] 
			at (-3.8,-1.5)
			{$|\Spec\C[X,Y]|$};
		\draw[very thick]
			(-4,0) -- (4,0) node[near end, auto]{$\alpha$}
			(0,-2) -- (0,2) node[near end, auto]{$\beta$};
		\fill[col1]
			(-2.8,1) circle[radius=2pt]
			node[above right] {$(X-\alpha, Y-\beta)$};
		\node[generic point=10pt, fill=black!60,
			label={above right:$(0)$}]
			at (5,0)
			{};
		\draw[line width=4pt, col2shade2, opacity=0.5]
			(0,-2) -- (0,2);
		\node[generic point=5pt, fill=col2shade2,
			label={[text=col2]below:$(X)$}]
			at (0,-2.1)
			{};
    \draw[scale=1,domain=.25:2,smooth,variable=\t,green] 
      plot ({1/\t},{\t});
    \draw[scale=1,domain=.25:2,smooth,variable=\t,green] 
      plot ({-1/\t},{-\t});
		\fill[green]
			(1,1) circle[radius=2pt]
			node[above right] {$(XY-1)$};
	\end{tikzpicture}
\end{figure}


\begin{definition}[basisoffene Menge]
	Für $f\in A$ nennt man
	\[ D(f) := \Spec A \setminus V((f)) = \{ \p \in \Spec A \mid f \notin \p\}
	\]
	die \emph{zu $f$ gehörige basisoffene Menge}.
\end{definition}

\begin{lemma}
	\label{lemma:basisoffene mengen sind basis}
	Die Menge $\fr B := \{D(f) \mid f \in A\}$ ist eine Basis der
	Topologie, d.h. jedes offene $U\subseteq \Spec A$ ist eine Vereinigung
	von $D(f) \in \fr B$ und $\fr B$ ist unter endlichen Schnitten 
	abgeschlossen.  
\end{lemma}
\begin{proof}
	Sei $U = \Spec A \setminus V(\a)$ offen und $\p \in U$, so ist
	$\p \notin V(\a)$, also $\a \not\subseteq \p$. Damit existiert
	$f \in \a \setminus \p$ mit $f \notin \p$, also $\p \in D(f)$
	und $f \in \a$. Also $(f) \subseteq \a$ und
	$V(\a) \subseteq V((f))$. Damit folgt $D(f) \subseteq U$.
	
	Zusammenfassend gilt für $U\subseteq \Spec A$ offen: $\forall \p \in U$
	$\exists f\p \in A$: $\p \in D(f\p) \subseteq U$.
	Also
	\[ U = \bigcup_{\p \in U} D(f\p)\]
	Ferner folgt mit \thref{lemma:vereinigungen von v sind produkt}
	$D(f) \cap D(g) = D(fg)$.
\end{proof}

\begin{lemma}
	\label{lemma:vereinigungen von v sind produkt}
	Für $\a, \fr b\ideal A$ gilt
	\[
		V(\a) \cup V(\fr b) = V(\a \cap \fr b) = V(\a \cdot \fr b).
	\]
\end{lemma}
\begin{proof}
	Es ist 
	$\a\fr b \subseteq \a \cap \fr b \subseteq \a, \fr b$.
	Also 
	\[V(\a) \cup V(\fr b) \subseteq V(\a \cap \fr b) 
	\subseteq V(\a\fr b).\]
	Angenommen $V(\a) \cup V(\fr b) \subsetneq V(\a\fr b)$, 
	d.h. $\exists \p \in V(\a \fr b) \setminus \big(V(\a) \cup V(\fr b)\big)$,
	also $\a\fr b \subseteq \p$ aber nicht
	$\a,\fr b \not \subseteq \p$.
	Also existiert $s \in \a \setminus \p$ und $t\in\fr b\setminus \p$.
	Damit ist $st \in \a\fr b \setminus \p$.
	Dies ist ein Widerspruch, da $\p$ ein Primideal ist.
	Folglich herrscht Gleichheit in obiger Inklusionskette.
\end{proof}

\begin{definition}[Radikal]
	Für $\a \ideal A$ heißt
	\[
		\sqrt \a := \{ f\in A \mid \exists n \in \N:\ f^n\in \a\}
	\]
	\emph{Radikal} von $\a$.
\end{definition}

\begin{lemma}
	\label{lemma:radikal ist ideal}
	$\sqrt a \ideal A$.
\end{lemma}
\begin{proof}
	\begin{itemize}
	  \item $0\in \sqrt{\a}$ \checkmark
	  \item Sei $f \in \sqrt \a$, $r\in A$. Dann
	  	$f^n \in \a$, $r\in A$. Also 
	  	$(rf)^n \in \a$ und damit $rf\in \sqrt\a$.
	  \item $f,g\in \sqrt\a$ mit $f^n \in \a$, $g^m \in \a$.
	  	\begin{align*}
	  		(f+g)^{n+m-1} &= \sum_{i=0}^{n-1} \binom{n+m-1}{i} f^i g^{n+m-1-i}
	  			+ \sum_{i=n}^{n+m-1} \binom{n+m-1}{i}
	  				f^i g^{n+m-1-i}\\
  				&= \left( \sum_{i=0}^{n-1} \binom{n+m-1}{i} 
  					f^i g^{n-1-i}\right) g^m
  					+  \left(\sum_{i=n}^{n+m-1} \binom{n+m-1}{i}
  						f^i g^{m-1-i}\right) f^n
	  	\end{align*}
	  	Da $g^m$ und $f^n$ jeweils in $\a$ liegen, ist auch die Summe dort.
	\end{itemize}
\end{proof} 

\begin{definition}[Radikalideal (radiziell)]
	Ein Ideal $\fr b \ideal A$ heißt \emph{Radikalideal (radiziell)},
	falls
	\[\sqrt \fr b = \fr b.\]
\end{definition}

\begin{bemerkung}
	Es gilt $\sqrt{\sqrt \a} = \sqrt\a$.
\end{bemerkung}

\begin{lemma}
	\label{lemma:radikal ist schnitt}
	Für $\a \ideal A$ gilt
	\[
		\sqrt\a = \bigcap_{\p\in V(\a)} \p
	\]
\end{lemma}
\begin{proof}
	  \newcommand{\bmax}{\b_\text{max}}
	\begin{itemize}
	  \item["`$\subseteq$"']
	  	Sei $f \in \sqrt\a$, $f^n \in \a$. Ist $\p \in V(\a)$, d.h.
	  	$\a \subseteq\p$. Also
	  	$f^n \in \p$ und da $\p$ prim, folgt $f\in \p$.
	  \item["`$\supseteq$"']
	  	Ist $f\notin \sqrt\a$, so zu zeigen, dass 
	  	$f \notin \cap_{\p\in V(\a)} \p$.
	  	Sei also 
	  	$f^n \notin \a$ für alle $n\in \N$.
	  	
	  	Betrachte
	  	\[ M := \{\b \ideal A\mid a\subseteq \b,
	  		f^n \notin \b \forall n\in \N\},
	  	\]
	  	so gilt
	  	\begin{itemize}
	  	  \item $\a \in M$,
	  	  \item $M$ ist angeordnet durch "`$\subseteq$"',
	  	  \item ist $(\b_i)_{i\in I}$ eine total geordnete Teilmenge,
	  	  	so ist $\b:= \cup_{i\in I} \b_i \ideal A$ mit $\b \in M$.
	  	\end{itemize}
	  	Damit hat $M$ mit dem Lemma von Zorn ein maximales Element
	  	$\bmax \in M$.
	\end{itemize}
	Nun sei behauptet, dass $\bmax \ideal A$ ein Primideal ist.
	Dazu sei $xy \in \bmax$, wobei wir annehmen, dass 
	$x,y \notin \bmax$.
	Betrachte
	$\bmax \subsetneq (x) + \bmax$, was ein Ideal in $A$ ist, aber nicht 
	in $M$ liegt. Analog künnen wir dies von $(y) + \bmax$ sagen. Damit
	existieren $n,m \in \N$ mit
	\[
		f^n \in (x) + \bmax
		\qquad
		f^m \in (y) + \bmax.
	\]
	Ergo ist
	\[
		f^{n+m} \in
			(x)\bmax + (y)\bmax + \bmax\bmax + (xy),
	\]
	wobei jeder Summand Teilmenge von $\bmax$ ist und wir folgern
	$f^{n+m} \in \bmax \in M$, wodurch man den Widerspruch erhült.
	
	Damit ist $\bmax \in V(\a)$ und $f\notin \bmax$. 
\end{proof}

\begin{satz}
	\label{satz:v und radikal}
	Für $\a, \b \ideal A$ gilt
	\[ V(\a) \subseteq V(\b) \quad\Leftrightarrow\quad
		\b \subseteq \sqrt\a.
	\]
	Insbesondere gilt sogar
	\[ V(\a) = V(\b) \quad\Leftrightarrow\quad
		\b = \sqrt\a.
	\]
\end{satz}
\begin{proof}
	\begin{itemize}
	  \item["`$\Leftarrow$"']
	  	Aus $V(\a) \subseteq V(\b)$ folgt
	  	\[
	  		\bigcap_{\p \in V(\a)} \p 
	  		\supseteq \bigcap_{\p\in V(\b)} \p
	  	\]
	  	und mit \thref{lemma:radikal ist schnitt}
	  	folgt $\sqrt \a \supseteq \sqrt\b \supseteq \b$.
	  \item["`$\Rightarrow$"']
	  	Aus $\b \subseteq \sqrt\a$, d.h.
	  	$\b \subseteq \cap_{\p\in V(\a)} \p$, folgt
	  	$\b \subseteq \p$ für alle $\p\in V(\a)$.
	  	Also $\p \in V(\a)$.
	\end{itemize}
\end{proof}

\begin{definition}[irreduzibel]
	Ein topologischer Raum $X$ heißt \emph{irreduzibel}, wenn gilt:
	Ist $X = A_1 \cup A_2$ mit $A_{1,2}\subseteq X$ abgeschlossen, so ist
	$X = A_1$ oder $X = A_2$.
	
	Eine Teilmenge $Z\subseteq X$ heißt \emph{irreduzibel}, wenn $Z$ mit der
	Teilraumtopologie irreduzibel ist.
\end{definition}

\begin{beispiel}
	$\Spec\Z$ ist irreduzibel. Ist nämlich $A_1 \subsetneq \Spec\Z$ 
	abgeschlossen, so ist $A_1 = \{(p_1), \ldots, (p_r)\}$ für
	irgendwelche Primzahlen $p_i$. 
\end{beispiel}

\begin{lemma}
	\label{lemma:v irreduzibel <=> radikal prim}
	In $\Spec A$ gilt:
	\[
		V(\a) \text{ irreduzibel} \quad\Leftrightarrow\quad
		\sqrt\a \text{ Primideal}.
	\]
\end{lemma}
\begin{proof}
	\begin{itemize}
	  \item["`$\Rightarrow$"']
	  	Sei $xy \in \sqrt\a$, so ist $(xy) \subseteq \sqrt\a$ und mit
	  	\thref{satz:v und radikal} $V(\a) \subseteq V((xy))$.
	  	
	  	Für $\p\in V(\a) \subseteq V((xy))$, gilt:
	  	Ist $xy \in \p$, so folgt $x \in \p$ oder $y\in \p$. Damit
	  	\[
	  		V(\a) \subseteq V((x)) \cup V((y))
	  		\ \Rightarrow\ 
	  		V(\a) = \big(V(\a) \cap V((x))\big) \cup 
	  				\big(V(\a) \cap V((y))\big).
	  	\]
	  	Da $V(\a)$ irreduzibel nach Voraussetzung, folgt
	  	\obda $V(\a) = V(\a) \cap V((x))$, also $V(\a) \subseteq V((x))$.
	  	Wieder mit \thref{satz:v und radikal} folgt
	  	$(x) \subseteq \sqrt\a$ und damit $x \in \sqrt\a$. 
	 \item["`$\Leftarrow$"']
	 	Schreibe $V(\a) = V(\b) \cup V(\fr c) = V(\b\cap \fr c)$.
	 	Dann folgt wiederum mit \thref{satz:v und radikal}
	 	$\sqrt\a = \sqrt{\b \cap \fr c}$.
	 	
	 	Ist $V(\a) \neq V(\b)$, also $V(\b) \subsetneq V(\a)$, also
	 	$\sqrt\a \subsetneq \sqrt\b$, so existiert 
	 	$x\in \sqrt\b \setminus \sqrt\a$. Für $y \in \fr c$, ist 
	 	\[
	 		xy \in \sqrt{\b\fr c} \subseteq \sqrt{\b \cap \fr c} = \sqrt\a.
	 	\]
	 	Nach Voraussetzung ist $\sqrt\a$ Primideal, also 
	 	nach Wahl von $x$ ist $y\in \sqrt\a$.
	 	Insgesamt ist $\fr c \subseteq \sqrt\a$, also
	 	$V(\a) \subseteq V(\fr c)$	und damit $V(\a) = V(\fr c)$.
	\end{itemize}
\end{proof}

\begin{definition}[Nilradikal]
	\[
		\Nil(A) := \sqrt{(0)}
	\]
	heißt \emph{Nilradikal} von $A$.
\end{definition}

\begin{korollar}8[]
	Es gilt
	\[
		\Spec A \text{ irreduzibel}
		\quad\Leftrightarrow\quad
		\Nil(A) \text{ Primideal}.
	\]
\end{korollar}
\begin{proof}
	\thref{lemma:v irreduzibel <=> radikal prim} mit $\a = (0)$.
\end{proof}

\begin{definition}[noethersch]
	Ein topologischer Raum heißt \emph{noethersch}, wenn gilt:
	Ist 
	\[
		A_1 \supseteq A_2 \supseteq A_3 \supseteq\ldots
	\]
	eine Folge abgeschlosser Teilmengen, so existiert
	$n_0\in \N$ mit $A_i = A_{i+1}$ für alle $i\geq n_0$. 
\end{definition}

\begin{lemma}
	\label{lemma:A noethersch => Spec A noethersch}
	Ist $A$ noethersch, so ist auch $\Spec A$ noethersch.
\end{lemma}
\begin{proof}
	Sei 
	\[
		A_1 \supseteq A_2 \supseteq \ldots
	\]
	eine Folge abgeschlossener Teilmengen, also
	\[
		V(\a_1) \supseteq V(\a_2) \supseteq \ldots
	\]
	mit $A_i = V(\a_i)$ für geeignete $\a_i \in \Spec A$, so ist
	\[
		\sqrt{\a_1} \subseteq \sqrt{\a_2} \subseteq \ldots
	\]
	eine aufsteigende Idealkette in $A$.
\end{proof}

\begin{satz}
	Ist $X$ noetherschscher topologischer Raum und 
	$\emptyset \neq A \subseteq X$ abgeschlossen, so zerlegt sich
	\[
		A = A_1 \cup \ldots \cup A_r
	\]
	in abgeschlosse irreduzible Teilmengen $A_i \subseteq A$.
	Nimmt man $A_i \not\subseteq A_j$ für $i\neq j$, so ist die Zerlegung
	bis auf Reihenfolge eindeutig.
	
	Die $A_i$ heißen \emph{(irreduzible) Komponenten} von $A$.
\end{satz}
\begin{proof}
	\begin{description sf}
	\item[Existenz.]
		Sei 
		\[
			\cal V := \{ A\subseteq X \mid \emptyset \neq A 
				\text{ abgeschlossen, $A$ hat keine solche Zerlegung} \}. 
		\]
		Angenommen $\cal V \neq\emptyset$, so hütte man
		ein inklusionsminimales $A \in \cal V$, denn falls nicht gübe es
		\[
			A_1 \supsetneq A_2 \supsetneq \ldots
		\]
		mit $A_i \in \cal V$. Da $X$ noethersch, müsste diese Folge
		stationür werden, wodurch man einen Widerspruch erhült.
		
		Dieses $A \in \cal V$ hat keine solche Zerlegung, ist also
		insbesondere nicht irreduzibel. Damit gibt es
		\[
			A = A_1 \cup A_2\quad A_i \subseteq X \text{ abgeschlossen, }
			A_i \neq A
		\]
		Da $A \in \cal V$ minimal sind $A_1, A_2 \notin \cal V$.
		Aber damit ist
		$A = A_1 \cup A_2 \notin \cal V$. Ein Widerspruch, der wie gewünscht
		$\cal V = \emptyset$ liefert.
	\item[Eindeutigkeit.]
		Sind 
		\[
			A = A_1 \cup \ldots \cup A_r = 
				A_1' \cup \ldots \cup A_s'
		\]
		zwei solcher Zerlegungen, so ist
		$A_1 \subseteq A_1' \cup \ldots \cup A_s'$, also
		$A_1 = (A_1' \cap A_1) \cup \ldots \cup (A_s' \cap A_1)$.
		Da $A_1$ irreduzibel künnen wir \obda $A_1 = A_1 \cap A_1'$ 
		annehmen. Also ist $A_1 \subseteq A_1'$.
		
		Analog ist $A_1' \subseteq A_k$ für ein $k=1,\ldots,r$.
		Zusammenfassend gilt
		\[
			A_1 \subseteq A_1' \subseteq A_k,
		\]
		was nach Voraussetzung $k = 1$ impliziert. Also $A_1 = A_1'$.
		
		Nun sukzessive weiter. 
	\end{description sf}
\end{proof}

\begin{beispiel}
	In $\Spec k[X,Y]$ zerfüllt
	\[
		V((XY)) = V((X)) \cup V((Y)).
	\]
	Im Bild
	\tikz[baseline, scale=0.8]{
		\draw (-1,0) -- (1,0) node[right,auto]{$V((Y))$}
			(0,-1) -- (0,1) node[above,auto]{$V((X))$};
	}
\end{beispiel}

\begin{beispiel}
	Sei $k$ algebraisch abgeschlossen. Betrachte $\Spec k[X,Y]$.
	Die \tikzmark[1]{maximalen Ideale} sind gerade 
	$\m = (X-\alpha, Y-\beta)$ für $\alpha,\beta \in k$.
	Ein abgeschlosser Punkt $\m \in \Spec k[X,Y]$ wird eindeutig durch
	$(\alpha, \beta) \in k^2$ gegeben.
	\tikzmargin[1]{north}{Quelle suchen!}
	
	$\A_k^2 := \Spec k[X,Y]$ wird der 
	\emph{2 dimensionale affine Raum über $k$} genannt.
	Man hat die Bijektion
	\[
		|\A_k^2| \xto{\phi} k^2.
	\]
	Eine abgeschlossene Teilmenge $A = V(\a) \subseteq \A_k^2$ liefert
	\[
		A \cap |\A_k^2| \cong_\phi \{ (\alpha,\beta) \in k^2 \mid 
			f(\alpha,\beta) = 0\ \forall f\in \a\},
	\]
	denn
	\begin{align*}
		A \cap |\A_k^2| &= V(\a) \cap |\A_k^2| = |V(\a)| \\
		&= \{ \m \in \Spec k[X,Y] \mid \a \subseteq \m,\ \m
			\text{ maximal}\}
			= 
			\{(X-\alpha, Y-\beta) \ideal k[X,Y] \mid \a \subseteq
				(X-\alpha, Y-\beta) \} \\
		&= \{(X-\alpha, Y-\beta) \mid f(X,Y) \in \alpha\ \Rightarrow\ 
			f(X,Y) \in (X-\alpha, Y-\beta)\}\\
		&\ \tikzmark[2]{=}\  
			\{(X-\alpha, Y-\beta) \mid f(X,Y) \in \alpha\ \Rightarrow\ 
			f(X,Y) = (X-\alpha)g(X,Y) + (Y-\beta)h(X,Y)\} \\
%ich weiü, das ist unschün, aber nach dem align wird der rand der nüchsten
%seite benutzt :-(
\tikzmargin[2]{north, above=2cm}{
	"`$\Rightarrow$"' ist klar. Also zu "`$\Leftarrow$"'.\\
	Es ist $f(\alpha,\beta) = 0$, also
	$f(X,Y) = (X-\beta) h(X,Y)$ für gewisses $h$.
	Es ist 
	$f(X,Y) - f(\alpha,Y) = (X-\alpha) g(X,Y)$,
	da die linke Seite $X = \alpha$ als Nullstelle hat.
}
		&= \{(X-\alpha, Y-\beta) \mid f(\alpha,\beta) = 0
			\ \forall f\in \a\}\\
		&\xto{\phi}
			\{(\alpha,\beta) \in k^2 \mid f(\alpha, \beta) = 0\ 
			\forall f\in \a\}.
	\end{align*}
	
	In $\A_k^2$ hat man aber noch mehr Punkte:
	Sei $\p \ideal k[X,Y]$ Primideal, aber nicht maximal, so ist
	$\p \in \A_k^2$ kein abgeschlossener Punkt.
	Ist beispielsweise $\p = (f(X,Y))$ für $f\in k[X,Y]$ irreduzibel, 
	so liegen alle $(\alpha,\beta) \in k^2$ mit $f(\alpha,\beta) = 0$
	auf der entsprechenden Menge in $k^2$, d.h.
	\[
		\p = (f(X,Y)) \subseteq 
		\m_{\alpha,\beta} := (X-\alpha, Y-\beta)
		\quad \Rightarrow\quad
		\m_{\alpha,\beta} \in \overline{\{\p\}}.
	\]
	\thref{fig:spec k xy} verdeutlicht dies.
\end{beispiel}

\begin{figure}\centering
	\caption{$\Spec k[X,Y]$}
	\label{fig:spec k xy}
	\begin{tikzpicture}
		\draw[very thick] 
			(-3,0) -- (3,0) node[near end, above] {$X$}
			(0,-2) -- (0,2) node[near end, right] {$Y$};
		
% 		\draw[col1,thick] 
% 			(-3,1) to[out=-5, in=135]  (0.2,-0.2) 
% 			to[out=-45, in=225, looseness=2] (-1,0) 
% 			to[out=45, in=180, looseness=0.5] (3,1.5)
% 			node[pos=0.9] {$f(X,Y) = 0$};
		\draw[col1, thick]
			(-3,1) 
			.. controls (5,-2) and (-8,-2) .. 
			(3,1.5)
			node[pos=0.98, above, sloped] {$f(X,Y) = 0$}
			coordinate[pos=0.05] (a);
		
		\fill[col1shade2] (a) circle[radius=2pt]
			node[above right, col1] {$(\alpha,\beta)$};
		
		\path (-3,1)
			node[generic point=10pt, fill=col1shade2] {}
			node[above left, col1] {$(f(X,Y))$};
			
		\path (4,0)
			node[generic point=10pt, fill=col2shade2] {}
			node[above right, col2] {$(0)$};
	\end{tikzpicture}
\end{figure}

\begin{lemma}
	\label{lemma:pi inv ist homöo auf bild}
	Ist $A$ ein Ring, $\a \in \Spec A$ und 
	$\pi: A \twoheadrightarrow A\big/ \a$ die Projektion, so ist
	\[
		\varphi := \pi\inv: 
			\funcdef{\Spec A\big/\a & \to & \Spec A \\
				\overline{\p} & \mapsto & \pi\inv(\overline{\p})}
	\]
	ein Homöomorphismus auf sein Bild
	\[
		\Spec A\big/\a \xto[\approx]{\pi\inv} V(\a) \subseteq \Spec A.
	\]
\end{lemma}
\begin{proof}

\end{proof}


\begin{definition}[(quasi)-kompakt]
	Ein topologischer Raum $X$ heißt \emph{quasi-kompakt}, wenn gilt:
	Ist $X = \bigcap_{i\in I} U_i$ mit $U_i$ offen, so existiert eine endliche
	Teilmenge $F\subset I$ mit
	$X = \bigcap_{i\in F} U_i$.
	
	$X$ heißt \emph{kompakt}, wenn $X$ hausdorffsch und quasi-kompakt ist.
\end{definition}


\begin{satz}
	Ist $A$ ein Ring, so ist $\Spec A$ quasi-kompakt.
\end{satz}
\begin{proof}
	Wir zeigen: Ist $\emptyset = \bigcap_{i\in I} Z_i$ für abgeschlossene
	$Z_i$, so existiert $F\subset I$ endlich mit
	$\emptyset = \bigcap_{i\in F} Z_i$.
	
	Sei also $Z_i = V(\a_i)$, $\a_i \ideal A$ und
	\[
		V(A) = \emptyset = \bigcap_{i\in I} V(\a_i)
		= V\left(\sum_{i\in I} \a_i\right)
	\]
	Nach \thref{satz:v und radikal} ist damit
	\[
		A = \sqrt{\sum_{i\in I} \a_i},
	\]
	also insbesondere $1 \in \sqrt{\sum_{i\in I} \a_i}$ und
	$1 \in \sum_{i\in I} \a_i$. Ergo
	\[
		1 = a_{i_1} + \ldots + a_{i_r},
	\]
	für $F:= \{i_1, \ldots, i_r\} \subset I$.
	Nun ist
	$1 \in \a_{i_1} + \ldots + \a_{i_r}$,
	also 
	\[
		(1) = A \subseteq \a_{i_1} + \ldots + \a_{i_r}.
	\]
	Wiederum mit \thref{satz:v und radikal} ist
	\[
		\emptyset = V(A) \supseteq \bigcap_{k=1}^r V(\a_{i_k}).
	\] 
\end{proof}

\subsection{$\Spec A$ als lokal geringter Raum}

Wir wollen $\O_{\Spec A}$ als die "`guten Funktionen"' auf $\Spec A$ auffassen,
aber dazu müssen wir es besser verstehen. 

\begin{definition}[multiplikative Teilmenge, Lokalisierung]
	\label{def:lokalisierung}
	Sei $A$ ein Ring, dann heißt $S\subseteq A$ \emph{multiplikative Teilmenge},
	wenn $1\in S$ ist und aus $a,b\in S$ auch $ab\in S$ folgt.
	
	Die \emph{Lokalisierung} $A_S$ oder $A[S\inv]$ von $A$ bezüglich $S$ ist
	der Ring
	\[
		A_S := \big(A \times S \big) \big/ \sim
	\]
	mit
	\[
		(a,s) \sim (b,t) \quad\Leftrightarrow\quad
		\exists u \in S:\ u(at - bs) = 0.
	\]
	Schreibe $\frac a s := [(a,s)]$ und definiere eine Ringstruktur auf
	$A_S$ durch Bruchrechnen.
\end{definition}

\begin{lemma}[Universelle Eigenschaft der Lokalisierung]
	\label{lemma:universelle eigenschaft lokalisierung}
	Wir haben die folgende universelle Eigenschaft: Ist
	$S\subseteq A$ wie in \thref{def:lokalisierung}, $\varphi: A \to R$
	ein Ringhomomorphismus, so dass $\varphi(S) \subseteq R^\times$, so
	existiert ein eindeutiger Ringhomomorphismus, der das
	Diagramm
	\[\begin{tikzcd}
		A \rar{\iota} \arrow{dr}{\varphi} & A_S \dar{\exists!} \\
		& R
	\end{tikzcd}\]
	kommutativ macht, wobei
	$\iota: A \to A_S,\ a \mapsto \frac a 1$.
\end{lemma}
\begin{proof}
	Klar, weil dieses $\psi: A_S \to R$ durch
	\[
		\psi\left(\frac a s\right) = 
		\psi\left(\frac a 1\right) \psi\left(\frac 1 s\right) =
		\varphi(a) \varphi(s)\inv
	\]
	eindeutig festgelegt ist.
\end{proof}

\begin{beispiel}
	\begin{itemize}
	  \item $S = \{f^n \mid n \in \N_0\}$, $f \in A$ fest.
	  	\[ A_S =: A_f := \left\{ \frac{a}{f^n} \mid n\in \N_0\right\}\]
	  \item $S = A \setminus \p$, $\p \in \Spec A$.
	  	\[ A_\p := \left\{ \frac a b \mid a \in A,\ b\notin \p \right\}\]
	  	ist ein lokaler Ring mit dem maximalen Ideal $\p A_\p$.
	\end{itemize}
\end{beispiel}


\begin{satz}
	\label{satz:spec a hat eindeutige ringgarbe}
	Sei $X = \Spec A$. Dann existiert auf $X$ eine bis auf Isomorphie 
	eindeutige Ringgarbe $\O_X$ mit:
	\begin{enumerate}[label=\roman{*})]
	  \item Es existiert ein Ringhomomorphismus
	  	$\varphi: A \xto{\cong} \O_X(X)$.
	  \item Für $f\in A$ betrachte 
	  	\[\funcdef{ \O_X(X) & \to & \O_X(D(f)) \\
	  		\varphi(f) & \mapsto & \varphi(f) \rest{D(f)}.}\]
	  	Dann ist $\varphi(f)\rest{D(f)} \in \O_X(D(f))^\times$ eine Einheit
	  	und der eindeutig durch 
	  	\[
	  	\begin{tikzcd}
	  		A \rar{\iota} \dar{\varphi}[swap]{\cong}
	  			\drar & A_f \dar{\exists!}[swap]{\varphi_f}\\
	  		\O_X(X) \rar{\cdot\rest{D(f)}} & \O_X(D(f))\\
	  	\end{tikzcd}
	  	\]
	  	gegebene Ringhomomorphismus $\varphi_f$ ist ein Isomorphismus.
	  \item Für $\p\in \Spec A$ hat man das koanonische Diagramm
	  	\[\begin{tikzcd}
	  		A \rar{\varphi}[swap]{\cong}  \dar{\iota} 
	  			& \O_X(X) \dar \\
	  		A_\p \rar{\varphi_\p} & \O_{X,\p}
	  	\end{tikzcd}\]
	  	und $\varphi_\p: A_\p \to \O_{X,\p}$ ist ein Isomorphismus.
	\end{enumerate}
\end{satz}

\subsubsection{Beweis von \autoref{satz:spec a hat eindeutige ringgarbe}}

Für den Beweis benötigen wir noch eine Definition.

\begin{definition}[$\cal B$-(Prä)Garbe]
	$\F:D(f) \mapsto A_f$ heißt \emph{$\fr B$-Prügarbe} auf
	$X = \Spec A$, wenn 
	$\F$ eine Prägarbe auf 
	\[
		\fr B := \{D(f) \subset X \mid f \in A\}
	\]
	ist.
	
	$\F$ heißt \emph{$\fr B$-Garbe}, wenn $\F$ eine $\fr B$-Prägarbe ist
	und die Garbenbedingungen für die $D(f)$ erfüllt sind.
\end{definition}

\begin{hilfslemma}
	\label{hilfslemma:1}
	Es gilt:
	\begin{enumerate}
	  \item $\O_X: D(f) \mapsto A_f$ ist eine $\fr B$-Garbe.
	  \item Ist $\F$ eine $\fr B$-Garbe, so existiert eine bis auf
	  	Isomorphie eindeutige Garbe $\bar \F$ auf $X$ mit
	  	$\bar\F(D(f)) = \F(D(f))$ für alle $D(f) \in \fr B$.
	\end{enumerate}
\end{hilfslemma}
\begin{proof}
	\begin{enumerate}
	  \item 
	\end{enumerate}
\end{proof}


 TODO
 
 
\begin{definition}[(affines) Schema]
	Ein \emph{affines Schema} ist ein lokal geringter Raum
	$(X,\O_X)$, der zu einem $(\Spec A, \O_{\Spec A})$ als lokal geringter
	Raum isomorph ist.
	
	Ein \emph{Schema} ist ein lokal geringter Raum $(X,\O_X)$, der eine
	offene Überdeckung durch affine Schemata besitzt, d.h.
	$X = \bigcup_{i\in I} U_i$ mit $U_i \subseteq X$ offen und 
	$(U_i, \O_X\rest{U_i})$ ist ein affines Schema.
\end{definition}

\begin{bemerkung}
	Beachte dabei: Ist $X$ ein topologischer Raum, $\F$ eine Garbe auf $X$,
	$U\subseteq X$ offen, so ist durch
	\[
		\F\rest U:\ V \mapsto \F\rest U (V) := \F(V)
	\]
	eine Garbe $\F\rest U$ auf $U$ definiert.
\end{bemerkung}

\begin{definition}[Morphismus von Schemata]
	Ein \emph{Morphismus von Schemata} ist ein
	Morphismus von lokal geringten Räumen
	\[
		(f,f\fis): (X, \O_X)  \to  (Y, \O_Y).
	\]
  mit $f:X \to Y$ stetig und $f^{\fis} : \O_Y \to f_*\O_X$ Garbenmorphismus
  auf $Y$ so dass $\O_{Y,f(x)}\to\O_{X,x}$ lokaler Ringhomomorphismus
\end{definition}

\begin{bemerkung}
	Man hat einen kontravarianten Funktor
	\[
		\funcdef{ \Ring & \to & \affSch \\
			A & \mapsto & (\Spec A, \O_{\Spec A}) \\
			A \xto{\varphi} B & \mapsto & 
			(f,f\fis): (\Spec B, \O_{\Spec B}) \to (\Spec A, \O_{\Spec A})}
	\]
	durch 
	\[
		f: \funcdef{\Spec B & \to & \Spec A\\
			\q & \mapsto & \varphi\inv(\q)},
	\]
	wobei die Stetigkeit hier klar ist, und
	\[
		f\fis: \O_{\Spec A} \to f_\ast \O_{\Spec B}.
	\]
	Letzterer ist für $g\in A$ gegeben durch
	\[
		f\fis_{D(g)}: \funcdef{\O_{\Spec A}(D(g)) = A_g & \to &  
			\big(f_\ast\O_{\Spec B}\big)(D(g))
			 \tikzmark{=} B_{\varphi(g)}\\
			 \frac{a}{g^n} & \mapsto & \frac{\varphi(a)}{\varphi(g)^n}}
	\]
	wobei wir \tikzarrow{mark above}{$\textcolor{lightgray}{\bullet}$} durch
	\[
		f\inv(D(g)) = \{\q\in \Spec B \mid f(\q) \in D(g) \}
			= \{\q\in \Spec B \mid \varphi\inv(\q) \not\ni g\}\\
			= \{\q \in \Spec B \mid \q \not\ni \varphi(g) \}
	\]
	erhalten.
	Diese Abbildung ist funktoriell und lokal, da für $\p\in \Spec A$
	\[
		f\fis_\p: \funcdef{ A_\p & \to & \O_{\Spec B, \q} \\
			\frac a \gamma & \mapsto & \frac{\varphi(a)}{\varphi(\gamma)}}
	\]
	für $\p = \varphi\inv(\q)$, $\gamma \notin \p$ 
	(also $\varphi(\gamma) \notin \q$) ein lokaler Ringhomomorphismus ist. 	
\end{bemerkung}
\pagebreak

% vim: set ft=tex :
