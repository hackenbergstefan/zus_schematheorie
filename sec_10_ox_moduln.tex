\section{$\O_X$-Moduln}
\newcommand{\OX}{$\O_X$-}

\subsection{$\O_X$-Moduln}
\begin{definition}[$\O_X$-Modul]
    Ein \emph{$\O_X$-Modul} (oder eine \emph{$\O_X$-Modulgarbe}) ist eine 
    Garbe $\M$ zusammen mit einer $\O_X(U)$-Modulstruktur auf $\M(U)$
    für jedes offene $U\osubset X$, so dass für 
    $V \osubset U \osubset X$ folgendes Diagramm kommutiert:
    \[\begin{tikzcd}
        \O_X(U) \times \M(U) \rar \dar{\cdot\rest V \times \cdot \rest V}
        & \M(U) \dar{\cdot \rest V}\\
        \O_X(V) \times \M(V) \rar & \M(V)
    \end{tikzcd}\]
    
    Ein \emph{Morphismus} $\M \to \M'$ von solchen ist ein Garbenmorphismus
    $\alpha: \M\to \M'$, so dass für jedes $U\osubset X$ 
    $\alpha(U):\M(U) \to \M'(U)$ $\O_X(U)$-linear ist.
\end{definition}


\begin{bemerkung}
    Man hat einige Konstruktionen aus der kommutativen Algebra auch für
    $\O_X$-Moduln, wie z.B.
    \begin{itemize}
      \item $\M \otimes_{\O_X} \M': U \mapsto \M(U) \otimes_{\O_X(U)} \M'(U)$.
      \item $\oplus_{i\in I} \M_i$ von $\O_X$-Moduln $\M_i$.
      \item Für $\alpha:\M\to \M'$ $\O_X$-Modul-Morphismus haben wir
        $\ker \alpha$ und $\im \alpha$,
        wobei Kern und Bild in $\Sh_X$ zu lesen sind.
    \end{itemize}
\end{bemerkung}


\begin{definition}[frei, lokal frei]
    Ein \OX Modul $\M$ heißt
    \begin{itemize}
      \item \emph{frei}, wenn es eine Menge $I$ und einen 
        \OX Modul-Isomorphismus 
        \[ \O_X^{(I)} := \bigoplus_{i\in I} \O_X \xto{\cong} \M\]
        gibt,
      \item \emph{lokal frei} oder \emph{Vektorbündel von Rang $r$}, 
        wenn es zu jedem $x\in X$ ein $x \in U\osubset X$ und einen
        $\O_U$-Modul-Isomorphismus
        \[ \O_U^r \xto{\cong} \M\rest U\]
        gibt.
    \end{itemize}
\end{definition}

\subsection{Exkurs: Vektorbündel in der Topologie}
Sei $X$ ein topologischer Raum. Dann ist ein $\R$-Vektorbündel vom Rang $r$
eine stetige Abbildung $\pi: E \to X$ mit einer $\R$-Vektorraumstruktur
auf $E_x := \pi\inv(\{x\})$ zusammen mit einem sog Bündelatlas,
bestehend aus Karten
\[ \psi_U: E\rest U := \pi\inv(U) \to U\times \R^r\]
mit $\pr_U \circ \psi_U = \pi\rest{\pi\inv(U)}$, d.h.
\[\begin{tikzcd}
    E\rest U = \pi\inv(U) \ar{rr}{\approx} 
    \drar{\pi}& &  U\times \R \dlar{\pr_U}\\
    & U &
\end{tikzcd}\]
kommutiert und die Karten sind
\begin{itemize}
  \item Homöomorphismen und so, dass
  \item $\psi_x: E_x \to \{x\} \times \R^r$ ein linearer Isomorphismus ist.
\end{itemize}

\paragraph{Wie verstehen wir das als Garbe von Moduln?}

Setze $\O_X := U \mapsto \O_X(U) := \{ f: U \to \R \mid f\text{ stetig}\}$,
also die Garbe der stetigen Funktionen. Dann ist $(X, \O_X)$ ein lokal
geringter Raum. Weiter haben wir $E \xto{\pi} X$ stetig.
Setze 
\[ \cal E : U \mapsto \cal E(U) := \{\sigma: U \to \pi\inv(U) \subseteq E \mid
    \sigma\text{ stetig, } \pi \circ \sigma = \id_U\}.\]
Dies ist eine Garbe. $\cal E$ ist sogar eine \OX-Modulgarbe:
Für $U\osubset X$ gilt
\[ \O_X(U) \times \cal E(U) \to \cal E(U),\ (f,\sigma) \mapsto f\cdot \sigma.\]
wobei
\[f\cdot \sigma : \funcdef{ U & \to & \pi\inv(U) \\
    x & \mapsto & \underbrace{f(x)}_{\in \R} \cdot 
    \underbrace{\sigma(x)}_{\in E_x}}
\]
und $E_x$ ein $\R$-Vektorraum ist.
 
Bleibt nur noch zu klären, wie die Bündelkarten 
$\psi_U: E\rest U = \pi\inv(U) \xto{\cong} U \times \R^r$ eingehen:
\[\begin{tikzcd}
    \pi\inv(U) \rar \drar[swap]{\pi} & U \times \R^r \dar[swap]{\pr_U} 
        & \lar[empty][description]{\ni} 
        (x,\alpha(x)) := (x, \pi_{\R^r} \circ \psi_U \circ \sigma(x)) \\
    & U \ular[bend left, mapsto]{\cal E(U) \ni \sigma}
        \uar[bend right,swap]{\psi_U \circ \sigma} & 
        x \lar[empty][description]{\ni} \uar[mapsto]
\end{tikzcd}\]

$\alpha: U \to \R^r$ ist eine stetige Abbildung, also
$\alpha \in \O_X(U)^r$. Weiter liefert $\psi_U$ einen 
$\O_X(U)$-Modul-Isomorphismus
\[
    \funcdef{ \cal E(U) & \xto{\cong} & \O_X(U)^r \\ 
        \sigma & \mapsto & \pr_{\R^r} \circ \psi_U \circ \sigma \\
        \psi_U\inv \circ (\id_U \times \alpha)  & \mapsfrom & \alpha.}
\]
Schränkt man auf $V \osubset U$ ein, ist dies verträglich. Also
\[ \cal E\rest U \cong \O_X(U)\]
als $\O_X\rest U$-Modulgarben.

\subsection{Quasi-Kohärenz}

\begin{definition}[quasi-kohärent]
    Eine $\O_X$-Modulgarbe $\M$ heißt \emph{quasi-kohärent}, wenn es
    zu jedem $x\in X$ ein $x \in U \osubset X$ und Mengen $I,J$ und
    eine exakte Sequenz von $\O_U$-Modulgarben
    \[\begin{tikzcd}
        \O_X\rest U^{(J)} \rar &\O_X\rest U^{(J)} \rar &
        \M_U \rar & 0
    \end{tikzcd}\]
    gibt.
\end{definition}

\begin{definition}[von seinen globalen Schnitten erzeugt]
    Ein \OX-Modul $\M$ wird \emph{von seinen globalen Schnitten erzeugt},
    wenn für jedes $x\in X$ der Morphismus von $\O_{X,x}$-Moduln
    \[ \M(X) \otimes_{\O_X(X)} \O_{X,x} \to \M_x\]
    surjektiv ist.
\end{definition}

Mit anderen Worten: Jeder Keim $m_x \in \M_x$ lässt sich schreiben als
\[ m_x = \sum_{\text{endl. viele }i} \lambda_i [\sigma_i]_x\]
für $\lambda_i \in \O_{X,x}$ und $\sigma_i \in \M(X)$.

Dies gilt nicht für $\O_X$ selbst; betrachte beispielsweise
$X = \C\P^1$ und $\O_X$ die Garbe der holomorphen Funktionen.

\begin{bemerkung}
    Es existiert ein surjektives 
    $\O_X\rest U^{(I)} \twoheadrightarrow \M\rest U$ genau dann, wenn
    $\M\rest U$ durch seine auf $U$ globalen Schnitte erzeugt wird.
    
    $\M$ ist quasi-kohärent genau dann, wenn
    $\M\rest U$ durch seine globalen Schnitte erzeugt wird und die Relationen
    (also $\ker(\O_X\rest U^{(I)}) \to \M)$) auch.  
\end{bemerkung}

\subsection{Quasikohärente Garben auf $\Spec A$}

\paragraph{Beachte folgende Konstruktion}
Ist $M$ ein $A$-Modul, so betrachte
\begin{itemize}
  \item für $f \in A$: $M_f = M \otimes_A A_f$ 
    als $A_f = \O_{\Spec A}(D(f))$-Modul.
  \item für $\p \in \Spec A$: $M_\p = M\otimes_A A_\p$
    als $\A_\p = \O_{\Spec A,\p}$-Modul.
\end{itemize}
Dies ist eine $\fr B$-Garbe für $\fr B = \{D(f)\mid f\in A\}$ der Basis 
der Topologie auf $\Spec A$. Dann folgt analog zu 
\thref{satz:spec a hat eindeutige ringgarbe} folgender Satz.

\begin{satz}
    \label{satz:a modul hat ein o spec a modulgarbe}
    Zu gegebenem $A$-Modul $M$ existiert (bis auf Isomorphie) genau eine
    $\O_{\Spec A}$-Modulgarbe $M^\sim$ auf $X = \Spec A$ mit
    \begin{align*}
        M^\sim (D(f)) &\cong M_f\\
        (M^\sim)_\p &\cong M_\p
    \end{align*} 
    Insbesondere ist $M^\sim(\Spec A) = M$.
\end{satz}


% 18.04.2013

\begin{satz}
    Der Funktor
    \[
        ^\sim: \funcdef{ \Moduln{A} & \to & \Moduln{\O_{\Spec A}} \\
            M & \mapsto & M^\sim\\
            (M \xto{\varphi} N) & \mapsto &  (M^\sim \xto{\sim} N^\sim)}
    \]
    ist exakt.
\end{satz}
\begin{proof}
    Es ist zu zeigen: Ist
    \[ M' \xto\alpha M \xto\beta M''\]
    eine exakte Sequenz in $\Moduln{A}$, so ist
    \[ (M')^\sim \xto{\alpha^\sim} M^\sim \xto{\beta^\sim} (M'')^\sim\]
    eine exakte Sequenz in $\Moduln{\O_{\Spec A}}$. Letzteres ist aber
    äquivalent dazu, dass
    \[ (M')_\p^\sim \xto{\alpha_\p^\sim} M_\p^\sim \xto{\beta_\p^\sim}
         (M'')_\p^\sim\]
    eine exakte Halmsequenz für alle $\p\in\Spec A$ ist.
    Dies ist aber klar, weil $\A_\p$ flach über $A$ ist 
    (\autocite[Example 9.1.1]{hartshorne1977algebraic} oder
    \autocite[Abschnitt 7 Satz 8]{bosch2009algebra}) und
    $M_\p^\sim = M_\p \cong M\otimes_A A_\p$. 
\end{proof}

\begin{korollar}
    \label{kor:m a modul dann m sim quaiskohaerent}
    Für einen $A$-Modul $M$ ist $M^\sim$ quasi-kohärent.
\end{korollar}
\begin{proof}
    Für $M$ hat man
    \[ A^{(J)} \to A^{(I)} \xto\varphi M \to 0.\]
    Nun wähle beispielsweise $I := M$ und $J := \ker\varphi$.
    Ferner ist 
    \[ (A^{(J)})^\sim = (\bigoplus_{j\in J})^\sim = \bigoplus_{j\in J}
        A^\sim = \bigoplus{j  \in J} \O_X = \O_X^{(J)}\]
    und da $^\sim$ exakt ist, folgt die Exaktheit von
    \[ \O_X^{(J)} \to \O_X^{(I)} \to M^\sim \to 0.\] 
\end{proof}

\begin{bemerkung}
    Sind $M$ und $N$ $A$-Moduln, so ist
    \[ (M\otimes_A N)^\sim = M^\sim \otimes_{\O_{\Spec A}} N^\sim.\]
\end{bemerkung}

\begin{satz}
    Sei $(X,\O_X)$ ein Schema. Dann ist eine $\O_X$-Modulgarbe $\M$
    genau dann quasi-kohärent, wenn für jede affin offene Teilmenge
    $U$ ein Isomorphismus
    \[ \M\rest U \cong (\M(U))^\sim\]
    existiert.
\end{satz}
\begin{proof}
    \begin{description mathquote}
    \item[\Leftarrow] Folgt aus \thref{kor:m a modul dann m sim quaiskohaerent}.
    \item[\Rightarrow]
        Aus nachstehenden Hilfslemmas haben wir
        für beliebiges $U = \Spec A \osubset X$ und $f\in A$ 
    \end{description mathquote}
\end{proof}

\pagebreak
